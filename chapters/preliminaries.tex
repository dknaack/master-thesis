\chapter{Preliminaries}

\section{Analysis of the Euclidean Algorithm}

The Euclidean algorithm can be seen as operating on a single rational number $x
= a/b \in \Q$ and returning the greatest common divisor between $a$ and $b$.

% TODO
\begin{Pseudocode}
while  do
  $(a, b) \gets (b \bmod a, a)$
end
\end{Pseudocode}

The size of the input is measured by the size of the denominator.
Since there might be multiple elements with the same denominator but different
numerators which take the same number of steps, we look at the element
with the smallest numerator.

\begin{lemma}
  For the Euclidean Algorithm, $F(n) / F(n+1)$ is the worst-case input,
  i.e. there exists no other input $a / b$ such that $b < F(n + 1)$ or
  $a < F(n)$ which requires more than $n$ steps.
\end{lemma}

\begin{proof}
  Let $X_n$ be the set of inputs, which require $n$ steps in the Euclidean Algorithm.
  For $n = 1$, we have $1/2$ which obviously requires one step of the algorithm.
  Assume $x' = \frac{a}{b}$ is the element with the smallest denominator, which
  requires $n$ steps.
  Then $x = \frac{b}{a+b}$ is the element with the smallest denominator,
  which requires $n + 1$ steps.

  First, $x$ requires $n + 1$ steps since applying the algorithm on this input
  yields $x'$.
  Next, we need to show that $x'$ is the element with the smallest denominator.
  For a contradiction, assume that there is an element $y = c/d \in X_{n+1}$
  with $d < a + b$ or $c < b$.
  Applying the algorithm on $y$ yields $(d \bmod c) / c$,
  from which it follows that $c \ge b$.
  Otherwise, it would contradict our assumption that $a/b$ is the element with
  smallest denominator.
  By the same argument $d \bmod c \ge a$ and therefore $d - c \ge a$.
  Putting both inequalities together leads to
  \[
    d < a + b \le d - c + c = d,
  \]
  which is a contradiction.
\end{proof}

\section{Continued Fraction}

\begin{definition}
  A continued fraction for a (potentially infinite) list of numbers $a_0, a_1,
  \dots \in \Z$ is defined as
  \begin{align*}
    [a_0; a_1, \dots] = a_0 + \frac{1}{[a_1; a_2, \dots]}
  \end{align*}
\end{definition}

\begin{definition}
  The $n$-th convergent of a continued fraction $[a_0; a_1, \dots]$
  is defined as the finite continued fraction $[a_0; a_1, \dots, a_n]$.
\end{definition}

\begin{lemma}
  For every $n \in \N$ there exists two polynomials $p_n$ and $q_n$ such that the
  $n$-th convergent of $[a_0; a_1, \dots]$ can be represented by the quotient
  of the two polynomials:
  \[
    [a_0; a_1, \dots, a_n] = \frac{p_n(a_0, a_1, \dots, a_n)}{q_n(a_0, a_1, \dots, a_n)}
  \]
\end{lemma}

\begin{proof}
  For $n = 0$, we have $p_0(a_0) = a_0$ and $q_0(a_0) = 1$.
  By induction, it follows that
  \[
    [a_0; a_1, \dots, a_n] = a_0 + \frac{1}{[a_1; a_2 \dots, a_n]} = a_0 + \frac{q_{n-1}(a_1, \dots, a_n)}{p_{n-1}(a_1, \dots, a_n)}.
  \]
  From this, we can derive the polynomials $p_n$ and $q_n$:
  \begin{align*}
    p_n(a_0, \dots, a_n) & = a_0 p_{n-1}(a_1, \dots, a_n) + p_{n-2}(a_2, \dots, a_n) \\
    q_n(a_0, \dots, a_n) & = p_{n-1}(a_1, \dots, a_n). \qedhere
  \end{align*}
\end{proof}

\begin{definition}
  The canonical structure of the $n$-th convergent $[a_0; a_1, \dots, a_n]$ is defined as
  \[
    [a_0; a_1, \dots, a_n] = \frac{p_n}{q_n} = \frac{p_n(a_0, \dots, a_n)}{p_{n-1}(a_1, \dots, a_n)}.
  \]
\end{definition}

\begin{lemma}
  The canonical structure of the continued fractions satisfies the recurrence
  \[
    \begin{matrix}
      p_n = a_n p_{n-1} + p_{n-2}, & p_{-1} = 1, & p_0 = a_0 \\
      q_n = a_n q_{n-1} + q_{n-2}, & q_{-1} = 0, & q_0 = 1
    \end{matrix}
  \]
\end{lemma}

\begin{lemma}
  For every positive $x \in \Q$, there are unique
  $a_0, a_1, \dots, a_n \in \N \setminus \{0\}$ with $a_n \ne 1$
  such that
  \[
    x = [a_0; a_1, \dots, a_n].
  \]
\end{lemma}

\begin{proof}

\end{proof}

\begin{lemma}
  If $[a_0; a_1, \dots, a_n]$ is the unique continued fraction for $a/b$,
  then the Euclidean algorithm requires $n$ steps on input $(a, b)$.
\end{lemma}

\begin{lemma}
  The smallest continued fraction which requires $n$ steps is $[1; 1, 1, \dots, 1]$.
\end{lemma}

\section{Generalized Euclidean Algorithm}

\begin{Pseudocode}
solve $Bx = c$
while $x$ is not integral do
  find $x_\ell$ which is not integral
  $c \gets B_\ell$
  $B_\ell \gets Bx \bmod Π(B)$
  solve $Bx = c$
end
\end{Pseudocode}

\begin{Pseudocode}
solve $Bx = c$
while $x$ is not integral do
  find $x_\ell$ which is not integral
  $c \gets B_\ell$
  $B_\ell \gets Bx \bmod Π(B)$

  for $i$ in $1..{\le} \ell$ do
    if $i = \ell$ then
      $x_i' \gets 1 / \{x_\ell\}$
    else
      $x_i' \gets \{x_i\} / \{x_\ell\}$
    end
  end
  $x \gets x'$
end
\end{Pseudocode}

\begin{align*}
  x_i' = \begin{cases}
    1 / x_l, & \text{ if } i = l, \\
    x_i / x_l & \text{ otherwise.}
  \end{cases}
\end{align*}

\section{Algebraic Numbers}

\begin{definition}
  A complex number $\alpha$ is an algebraic number if $\alpha$ is the root of some monic polynomial with rational coefficients
  \[
    p(\alpha) = 0, \quad p(x) = x^n + a_{n-1} x^{n-1} + \dots a_1 x + a_0, \quad a_0, a_1, \dots, a_{n-1} \in \Q.
  \]
\end{definition}

\begin{definition}[Ring]
  A ring is a set $R$ with two binary operations $+$ (addition) and $\cdot$ (multiplication)
  satisfying the following conditions:
  \begin{enumerate}
    \item $(R, +)$ is a commutative group.
    \item $(R, \cdot)$ is a monoid, i.e. $\cdot$ is associative and there exists an identity element $1 \in R$.
    \item Multiplication is distributive with respect to addition:
      \begin{align*}
        a \cdot (b + c) = (a \cdot b) + (a \cdot c), & \quad \text{for all } a, b, c \in R \\
        (b + c) \cdot a = (b \cdot a) + (c \cdot a), & \quad \text{for all } a, b, c \in R
      \end{align*}
  \end{enumerate}
  A ring is commutative if multiplication is commutative.
\end{definition}

\begin{definition}[Ideal]
  % TODO: Add note about left and right-sided ideals?
  Given a commutative ring $R$, an ideal $I$ is a subset of $R$ satisfying
  \begin{enumerate}
    \item $(I, +)$ is a subgroup of $(R, +)$
    \item For every $r \in R$ and every $x \in I$, the $r \cdot x \in I$.
  \end{enumerate}
\end{definition}

\begin{definition}[Quotient Ring]
  For an ideal $I$ over a ring $R$, the equivalence relation $\sim$ is defined as
  \[
    a \sim b \quad \text{if and only if} \quad a - b \in I.
  \]
  % TODO: Define the equivalence class of an element.
  The set of equivalence classes is denoted by $R/I$ and forms another ring with
  \begin{align*}
    (a + I) + (b + I) = (a + b) + I. \\
    (a + I) \cdot (b + I) = (a \cdot b) + I.
  \end{align*}
\end{definition}

\begin{example}
  Let $\Q[X]$ be the ring of polynomials in the variable $X$ with only rational
  coefficients and $I = (X^2 - X - 1)$ an ideal consisting of all multiples of $X^2 - X - 1$.
  Then, the operations of the quotient ring $\Q[X]/(X^2 - X - 1)$ are
  \begin{align*}
    (a_0 + a_1 X) + (b_0 + b_1 X)
    & = (a_0 + b_0) + (a_1 + b_1) X \\
    (a_0 + a_1 X) \cdot (b_0 + b_1 X)
    & = (a_0 b_0) + (a_1 b_0 + a_0 b_1) X + (a_1 b_1) X^2 \\
    & = (a_0 b_0 + a_1 b_1) + (a_1 b_0 + a_0 b_1 + a_1 b_1) X
  \end{align*}
\end{example}

The quotient ring $\Q[X]/(X^2 + X + 1)$ can also be seen as a extension field $\Q(\alpha)$ over $Q$,
where $\alpha$ is an algebraic number satisfying for the respective polynomial.
Hence, $\Q[X]/(X^2 + X + 1)$ forms a field.

\begin{definition}[Simple Extension]
  Given an algebraic number $\alpha$ of degree $d$,
  a simple extension of the rational numbers with $\alpha$ is the set
  \[
    \Q(\alpha) = \{\sum_{i=0}^d \alpha^i q_i \mid q_i \in \Q\}.
  \]
\end{definition}

