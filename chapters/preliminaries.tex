\chapter{Algebraic Number Theory}
\label{ch:preliminaries}

In this chapter, I recount many important definitions and theorem from
algebraic number theory, which will be used in the later chapters.
For a more comprehensive overview of this topic, see \cite{Dummit04}.

\section{Rings, Ideals, Integral Domains}

\begin{definition}
  A \emph{ring} is a set $R$ equipped with two binary operators, addition $+$
  and multiplication $·$, and two elements $0_R$ and $1_R$ of $R$, which satisfy
  the following properties:
  \begin{enumerate}
    \item \textbf{Associativity}: $a + (b + c) = (a + b) + c$ and $a · (b · c) = (a · b) · c$ for every $a, b, c ∈ R$.
    \item \textbf{Identity}: $0_R + a = a$ and $1_R · a = a$ for every $a ∈ R$.
    \item \textbf{Additive Inverse}: For every $a ∈ R$, there exists an element $b ∈ R$ such that $a + b = 0_R$.
    \item \textbf{Distributivity}: For every $a, b, c ∈ R$,
      \begin{align*}
        a + (b · c) = (a + b) · (a + c), \\
        a · (b + c) = (a · b) + (a · c).
      \end{align*}
    \item \textbf{Commutativity}: $a + b = b + a$ and $a · b = b · a$ for every $a, b ∈ R$.
  \end{enumerate}
\end{definition}

\begin{definition}
  An \emph{integral domain} is a ring $R$ in which $1 ≠ 0$ and for every $a, b ∈ R$,
  \[
    ab = 0 \implies a = 0 \text{ or } b = 0.
  \]
\end{definition}

\section{The Euclidean Algorithm and Euclidean Domains}

\begin{Pseudocode}
algorithm Euclidean(a, b)
  while $b ≠ 0$ do
    $b' ← a \bmod b$ // modulo
    $(a, b) ← (b', a)$ // exchange
  end
  return $a$
end
\end{Pseudocode}

\begin{itemize}
  \item \textbf{Modulo}: $b' ← a \bmod b$.
  \item \textbf{Exchange}: $(a, b) ← (b', a)$.
  \item \textbf{Termination}: $b = 0$
\end{itemize}

\begin{proposition}
  The ratio $b/a$ decreases by at least $1/2$ over two iterations.
\end{proposition}

\begin{proof}
  Suppose $b/a > 1/2$ in the first iteration.
  Over two iterations, we have
  \[
    \frac{b}{a} · \frac{a \bmod b}{b}
    ≤ \frac{b}{a} · \frac{a - b}{b}
    = 1 - \frac{b}{a}
    < 1 - \frac{1}{2}
    = \frac{1}{2}.
  \]
  Therefore, it decreases by at least $1/2$ over two iterations.
\end{proof}

\begin{definition}
  A \emph{Euclidean domain} is an integral domain $R$ with a function $f$,
  which maps any nonzero element of $R$ to a nonnegative integer,
  such that for every two elements $a$ and $b$ of $R$, there exist elements $q$ and $r$ of $R$ with
  \[
    a = qb + r \quad \text{ and } \quad r = 0 \text{ or } f(r) < f(b).
  \]
  The element $q$ is called the \emph{quotient} and $r$ is called the \emph{remainder}.
\end{definition}

\begin{example}
  % TODO: Polynomial rings as a more complex example
\end{example}

% TODO: Euclidean domains, irreducibility of polynomials

\section{Algebraic Numbers}

In this thesis, we will consider algebraic numbers which are roots of polynomials
\[
  p(x) = a_{d+1} x^{d+1} + a_d x^d + \dots + a_1 x + a_0
\]
with $aᵢ ∈ ℤ$ and $p$ is irreducible.

\begin{itemize}
  \item Quadratic fields as vector spaces
  \item Algebraic field extensions
  \item Monic polynomials, irreducible polynomials
  \item Rational expressions, polynomial rings
  \item the degree of a field extension $[K : L]$
  \item Finitely-generated field extensions
  \item Finite-degree field extensions
\end{itemize}

A rational expression of a number $α$ is a fraction of two polynomials, i.e.
\[
  \frac{p(α)}{q(α)} = \frac{pₙ α^n + \dots + p₁ α + p₀}{qₙ α^n + \dots + q₁ α + q₀}.
\]
The set of all rational expressions of an algebraic number $α$ is denoted as $ℚ(α)$.
Another formulation would be that $ℚ(α)$ is the smallest field containing $α$
and the rational numbers as a subfield.

\begin{theorem}
  The field $ℚ(α)$ is isomorphic to $ℚ[x]/⟨f(x)⟩$.
\end{theorem}

\begin{definition}
  Let $L/K$ be a field extension and $α ∈ L$.
  If $α$ satisfies
  \[
    α^n + c_{n-1} α^{n-1} + ⋯ + c_1 α + c_0 = 0
  \]
  with $c₀, c₁, …, c_{n-1} ∈ K$,
  then $α$ is called \emph{algebraic over $K$}.
  Furthermore, if every $α ∈ L$ is \emph{algebraic over $K$}, then $L$ is called
  algebraic over $K$ and $L/K$ is an \emph{algebraic field extension}.
\end{definition}

Although, elements of a field $L$ can only be algebraic over another field $K$,
we will only look at elements which are algebraic over $ℚ$.
In this thesis, an \emph{algebraic number} will mean any number $x ∈ ℝ$, which
is algebraic over $ℚ$.

\begin{definition}
  Let $L/K$ be a field extension.
  The degree $[L : K]$ is defined as the dimension of the vector space $L$ over $K$.
  A field extension is finite if its degree $[L : K]$ is finite.
\end{definition}

A basis for this vector space would be $(1, α, …, α^{n-1})$, for example.

If we fix a particular element $α ∈ L$, then multiplication with $α$ is a linear operation:
\[
  M_α : L → L, M_α(x) = α · x.
\]

\begin{definition}
  Let $L/K$ be an algebraic field extension.
  The norm of an element $α ∈ L$ is defined as the determinant of $M_α$.
\end{definition}

From this, it is straightforward to show that the norm is multiplicatively linear:
\[
  N(α · β) = N(α) · N(β).
\]
