\chapter{Preliminaries}
\label{ch:preliminaries}

This chapter introduces the foundational concepts needed throughout this thesis.
We begin with the most important topic first,
which is the Euclidean algorithm.
It serves as the basis of continued fractions and its generalization.
Through its analysis, it also leads to one of the first practical applications
of the Fibonacci numbers.
We proceed with the basics of algebraic number theory,
which is needed for the second part of Hermite's problem.
Finally, I give a brief introduction to lattice theory,
which is not only needed for the generalization of the Euclidean algorithm,
but also for the geometry of the continued fractions.

% ==============================================================================
\section{The Euclidean Algorithm}
% ==============================================================================

The input to the algorithm is a pair of integers $a, b$ with $a > b$ and the
goal is to find the greatest common divisor between the two numbers. The
algorithm works in two steps.
The first step is to calculate the remainder of $a$ with respect to $b$.
We find an integer $q$ for the quotient and a remainder $r < b$ such that
\[
  a = q b + r.
\]
The next step exchanges $a$ with $b$ and $b$ with $r$.
The algorithm continues with these two steps until $r = 0$.

\begin{lemma}
  Suppose $a = qb + r$, then $\gcd(a, b) = \gcd(b, r)$.
\end{lemma}

\begin{proof}
  Suppose $d = \gcd(b, r)$, then $b = b'd$ and $r = r'd$ for some $b', d' ∈ ℤ$.
  But then $d$ must divide $a$, since $a = qb + r = d(qb' + r')$.
  Thus, $d$ is a divisor of $a$ and we have $d ≤ \gcd(a, b)$.
  Next, suppose $d = \gcd(a, b)$.
  Because $d$ divides $a = qb + r$ and $b$,
  it must also divide $r$.
  Hence, $d ≤ \gcd(b, r)$.
  Combining both directions, we get the equality $\gcd(a, b) = \gcd(b, r)$.
\end{proof}

\begin{example}
  Let $a = 252$ and $b = 105$.
  The Euclidean algorithm proceeds as follows:
  \begin{align*}
    252 & = 2 · 105 + 42 \\
    105 & = 2 · 42 + 21 \\
    42 & = 2 · 21 + 0.
  \end{align*}
  Thus, $21$ is the greatest common divisor between $252$ and $105$.
\end{example}

Since $r < b$, the Euclidean algorithm must terminate at some point.
Therefore, it correctly calculates the greatest common divisor between the two
inputs $a$ and $b$.
In summary, the algorithm uses the following steps:
\begin{itemize}
  \item \textbf{Modulo}: $b' ← a \bmod b$.
  \item \textbf{Exchange}: $(a, b) ← (b', a)$.
  \item \textbf{Termination}: $b = 0$
\end{itemize}

The Euclidean algorithm not only works on integers,
but can be extended to any ring with a division with remainder operation.
Recall that a ring is a set $R$ equipped with two operations -- addition and
multiplication -- that behaves like the integers:
it forms an abelian group under addition and is closed under multiplication,
which is associative and distributes over addition.
A ring need not have multiplicative inverses or a multiplicative identity.
In this thesis, all rings will contain $1$ and be commutative unless stated
otherwise.

\begin{definition}
  An \emph{integral domain} is a ring $R$ in which $1 ≠ 0$ and for every $a, b ∈ R$,
  \[
    ab = 0 \implies a = 0 \text{ or } b = 0.
  \]
\end{definition}

An integral domain is a ring without zero divisors.
The Euclidean algorithm can be extended to any integral domain,
which defines an additional division with remainder operation.
Such a ring is called an Euclidean domain.

\begin{definition}
  A \emph{Euclidean domain} $(R, f)$ is an integral domain $R$ with a function $f$,
  which maps any nonzero element of $R$ to a nonnegative integer,
  such that for every two elements $a$ and $b$ of $R$, there exist elements $q$ and $r$ of $R$ with
  \[
    a = qb + r \quad \text{ and } \quad r = 0 \text{ or } f(r) < f(b).
  \]
  The element $q$ is called the \emph{quotient} and $r$ is called the \emph{remainder}.
  The function $f$ is called a \emph{Euclidean function}.
\end{definition}

\begin{example}
  % TODO: Real numbers using a different division with remainder
  Consider the real numbers with the Euclidean function $f(r) = r$.
  One possible division just uses the quotient $q = ab^{-1}$.
  However,
  another possible division operation would be
  \[
    a = \underbrace{\floor{\frac{a}{b}}}_q b + \underbrace{\left\{ \frac{a}{b} \right\} b}_r
  \]
\end{example}

\begin{example}
  Consider the ring $ℚ[x]$ of polynomials with rational coefficients.
  For any two polynomials $f(x), g(x) ∈ ℚ[x]$,
  we can find polynomials $q(x), r(x) ∈ ℚ[x]$ such that
  \[
    f(x) = q(x) g(x) + r(x)
    \quad
    \text{ and }
    \quad
    \deg(r) ≤ \deg(f).
  \]
  In this case, $\deg(f)$ is the Euclidean function and $q(x), r(x)$ are
  the quotient and remainder, respectively.
\end{example}

\begin{definition}
  A polynomial $f(x) ∈ ℚ[x]$ is \emph{irreducible},
  if there exists no polynomial $g(x)$ which divides $f(x)$.
\end{definition}

\section{Algebraic Numbers}

\begin{definition}
  A real number $α$ is said to be \emph{algebraic} if its the root of some polynomial
  with integer coefficients.
  If $α$ is not algebraic, then it is said to be \emph{transcendental}.
\end{definition}

The set of algebraic numbers includes many familiar constants, such as $\sqrt{2}$ or $\frac{1}{2}$,
and plays a central role in number theory.
Transcendental numbers, such as $\pi$ and $e$, are those which cannot be expressed
as roots of integer polynomials, and are far less well understood.
This thesis focuses only on algebraic numbers and, in particular, on their
representation using continued fractions and their generalizations.

\begin{lemma}
  If $α$ is algebraic, then there exists a unique monic polynomial $f_α(x) ∈ ℚ[x]$
  of smallest degree with $α$ as a root.
\end{lemma}

Every algebraic number is associated with a unique monic polynomial over the rational numbers
of minimal degree, known as its \emph{minimal polynomial}.
This polynomial is irreducible and captures essential information about the algebraic number.
Its degree also determines the size of the smallest number field containing the number.

\begin{definition}
  A field $K$ is an \emph{number field} if it is a finite extension of $ℚ$.
  The \emph{degree} of the number field is defined as the degree of the field
  extension $[K : ℚ]$, i.e. the dimension of $K$ as a vector space over $ℚ$.
\end{definition}

Number fields form the natural setting for studying algebraic numbers.
By adjoining an algebraic number to $ℚ$, one obtains a field that contains all
rational linear combinations of its powers, up to the degree of its minimal polynomial.
These fields have finite dimension over $ℚ$, and that dimension is referred to as their degree.

\begin{example}
  \hfill
  \begin{enumerate}
    \item The rational numbers form a number field and they have degree $1$.
    \item The quadratic field $ℚ(\sqrt{2})$ is a number field.
      Any element in $ ℚ(\sqrt{2})$ can be written as $a + b \sqrt{2}$ with $a,b ∈ ℚ$
      and it is therefore a number field with degree $[ℚ(\sqrt{2}) : ℚ] = 2$.
    \item The field $ℚ(\sqrt[3]{2})$ is a number field.
      It contains all elements of the form $a + b \sqrt[3]{2} + c \sqrt[3]{2^2}$ with $a,b,c ∈ ℚ$.
      Therefore, $1, \sqrt[3]{{2}}, \sqrt[3]{4}$ forms a basis for the vector space over $ℚ$
      and the field $ℚ(\sqrt[3]{2})$ has degree $[ℚ(\sqrt[3]{2}) : ℚ] = 3$.
  \end{enumerate}
\end{example}

An important construction associated with number fields is the \emph{field norm}.
It provides a way of assigning a rational number to each element of a number field,
and can be defined using linear algebra.
Multiplication by an element $α ∈ K$ can be viewed as a linear transformation on $K$,
and the norm is defined as the determinant of this transformation.
For example, consider the field $ℚ(\sqrt{2})$ with basis $\{1, \sqrt{2}\}$.
Let $α = 1 + \sqrt{2}$.
Then for any element $x = a + b \sqrt{2}$, we compute
\[
  αx = (1 + \sqrt{2})(a + b \sqrt{2}) = a + a \sqrt{2} + b \sqrt{2} + 2b = (a + 2b) + (a + b)\sqrt{2}.
\]
So in the basis $\{1, \sqrt{2}\}$, this corresponds to the matrix
\[
  M_α =
  \begin{pmatrix}
    1 & 2 \\
    1 & 1
  \end{pmatrix}.
\]
The norm $N_{K/ℚ}(α)$ is then the determinant of $M_α$, which is $1 \cdot 1 - 2 \cdot 1 = -1$.

\begin{definition}
  Let $K$ be a number field.
  The \emph{field norm} $N_{K/ℚ}(α)$ of an element $α$ of $K$ is defined as the
  determinant of its multiplication matrix $M_α$.
\end{definition}

An immediate consequence of the definition is multiplicativity:
the norm of a product is equal to the product of the norms.
This makes it a valuable tool for reasoning about algebraic equations
and for connecting the arithmetic of $K$ to that of $ℚ$.

\begin{lemma}
  For every $α, β ∈ K$, their product has the norm $N_{K/ℚ}(αβ) = N_{K/ℚ}(α) · N_{K/ℚ}(β)$.
\end{lemma}

% ==============================================================================
\section{Lattice Theory}
% ==============================================================================

One of the fundamental structures in linear algebra is the vector space.
Given a set of vectors $A = \{A_1, …, A_n\} ∈ ℝ^d$, the span of $A$ is the set
of all linear combinations of the basis vectors $A_i$.
A lattice is similarly defined over a set of vectors $A$, but instead of linear
combinations of vectors, a lattice consists of only integral linear
combinations.

\begin{definition}
  Given a matrix $A ∈ ℤ^{d × n}$, the \emph{lattice} over the basis $B$ is defined as
  \[
    \mathcal{L}(A) = \left\{\, A₁z₁ + \dots + A_n z_n \mid z_1, \dots, z_n ∈ ℤ^d \,\right\}.
  \]
  The \emph{rank} of $\mathcal{L}(A)$ is defined as the rank of the matrix $A$
  and the \emph{dimension} of the lattice $\mathcal{L}(A)$ is $d$.
  If $\mathrm{rank}(A) = d$, then $\mathcal{L}(A)$ is called a \emph{full rank}
  lattice and $A$ is a \emph{basis} for the lattice.
\end{definition}

\begin{definition}
  A matrix $U ∈ ℤ^{n×n}$ is \emph{unimodular} if $\det(U) = ±1$.
\end{definition}

\begin{lemma}
  If $U$ is unimodular, then $U^{-1}$ is unimodular.
\end{lemma}

\begin{lemma}
  Let $B$ and $B'$ be two bases.
  The lattices $\mathcal L(B)$ and $\mathcal L(B')$ are the same if and only if
  there exists a unimodular matrix $U ∈ ℤ^{n×n}$ such that $B' = BU$.
\end{lemma}

\begin{proof}
  If $\mathcal{L}(B) = \mathcal{L}(B')$, then $U = B^{-1} B'$ has determinant $\det(B^{-1} B') = $
  linear combination of the column vectors of $B$, so $B' = BU$ for some $U ∈ ℤ^{n×n}$.
  Since both $B$ and $B'$ are bases of the same full-rank lattice, they must have
  the same determinant up to sign.
  Thus, $\det(U) = ±1$, and $U$ is unimodular.

  If $B' = BU$ and $U$ is unimodular,
\end{proof}

\begin{figure}[tbp]
  \centering
  \includegraphics{build/lattice.pdf}
  \caption{
    A two-dimensional lattice $\mathcal L(B)$ with the basis vectors $B_1 = (2,
    1)$ and $B_2 = (1, 3)$. The fundamental parallelepiped $Π(B)$ is colored in
    {\color{cyan}cyan}.
  }
\end{figure}

\begin{definition}
  The \emph{fundamental parallelepiped} of a lattice $\mathcal{L}(B)$ with a basis $B ∈ ℤ^{d×d}$ is defined as
  \[
    Π(B) = \left\{\, B₁ x₁ + \dots + B_d x_d \mid x_1, \dots, x_d ∈ [0, 1) \,\right\}.
  \]
  The volume of this parallelepiped is
  \[
    \mathrm{vol}(Π(B)) = |\det(B)|.
  \]
\end{definition}

A useful fact about the fundamental parallelepiped of a lattice $\mathcal L(B)$
is that if $B$ is a square integer matrix, then the volume of the
parallelepiped $Π(B)$ and the number of integer points $ℤ^n$ contained in
$Π(B)$ is entirely determined by $\mathrm{det}(B)$, i.e.
\[
  \mathrm{vol}(Π(B)) = |Π(B) ∩ ℤ^n| = |\det(B)|.
\]
