\chapter{The Euclidean Algorithm}
\label{ch:euclidean}

\section{Analysis}

\begin{Pseudocode}
algorithm Euclidean(a, b)
  while $b ≠ 0$ do
    $b' ← a \bmod b$ // modulo
    $(a, b) ← (b', a)$ // exchange
  end
  return $a$
end
\end{Pseudocode}

\begin{itemize}
  \item \textbf{Modulo}: $b' ← a \bmod b$.
  \item \textbf{Exchange}: $(a, b) ← (b', a)$.
  \item \textbf{Termination}: $b = 0$
\end{itemize}

\begin{proposition}
  The ratio $b/a$ decreases by at least $1/2$ over two iterations.
\end{proposition}

\begin{proof}
  Suppose $b/a > 1/2$ in the first iteration.
  Over two iterations, we have
  \[
    \frac{b}{a} · \frac{a \bmod b}{b}
    ≤ \frac{b}{a} · \frac{a - b}{b}
    = 1 - \frac{b}{a}
    < 1 - \frac{1}{2}
    = \frac{1}{2}.
  \]
  Therefore, it decreases by at least $1/2$ over two iterations.
\end{proof}

\section{Fibonacci Numbers}

\begin{definition}
  The \emph{Fibonacci numbers} are defined as
  \begin{enumerate}
    \item $F(0) = F(1) = 1$.
    \item $F(n) = F(n - 1) + F(n - 2)$ for $n > 2$.
  \end{enumerate}
\end{definition}

\begin{proposition}[Lamé's Theorem \cite{Lame44}]
  The Euclidean algorithm requires at most $5h$ steps,
  where $h = \log_{10}(b)$.
\end{proposition}

\begin{proof}
  \todo[inline]{Using Fibonacci numbers.}
\end{proof}

\section{Extension to the Real Numbers using Continued Fractions}

What if the ratio $a/b$ perfectly equals the golden ratio?
In this case, the Euclidean algorithm no longer runs on two integer inputs,
since the golden ratio is irrational.
Therefore, we extend the algorithm to the real numbers.
Our input now consists of a single number $x ∈ ℝ$, which in the original
algorithm represents the ratio between the two inputs.
This allows us to run the algorithm on the golden ratio.

Given $x ∈ ℝ$, the algorithm proceeds just as the original.
We assume $x$ actually represents the fraction $x/1$,
then we find a number $a₀ ∈ ℕ$ times until $x - 1 · a₀ < 1$.
Once this is the case, we swap the inputs, i.e. we calculate the inverse fraction $1/(x - a)$.
Now, we subtract $x - a$ from $1$ until $1 - a₁ (x - a₀) < x - a₀$
and then calculate the inverse fraction again.
We repeat this process until $x = 0$.
Subtracting $a_i$ from the numerator such that it is smaller than the
denominator and then calculating the inverse.

\begin{example}
  Let $x = 13/5$.
  The algorithm proceeds as follows:
  \[
    \begin{array}{rclcrcl}
      13/5 & = & 2 & · & 1   & + & 3/5 \\
         1 & = & 1 & · & 3/5 & + & 2/5 \\
       3/5 & = & 1 & · & 2/5 & + & 1/5 \\
       2/5 & = & 2 & · & 1/5 & + & 0.
    \end{array}
  \]
\end{example}

\begin{example}
  Let $x$ be the golden ratio.
  The algorithm proceeds as follows:
  \[
    \begin{array}{rclcrcl}
      φ & = & 1 & · & 1   & + & φ - 1 \\
         1 & = & 1 & · & (φ - 1) & + & (2 - φ) \\
       φ - 1 & = & 1 & · & (2 - φ) & + & (2φ - 3) \\
       2 - φ & = & 1 & · & (2φ - 3) & + & (5 - 3φ) \\
       & \vdots &
    \end{array}
  \]
\end{example}


The real algorithm computes the coefficients of the continued fractions representation of $x$.
Continued fractions are fractions of the form
\[
  x = a_0 + \cfrac{1}{a_1 + \cfrac{1}{a_2 + \cfrac{1}{a_3 + \cfrac{1}{a_4 + \ddots}}}}.
\]
These fractions are denoted in the much more compact notation $x = [a₀; a₁, a₂, …]$.
Each continued fraction is usually defined of a sequence of integers $(a_n)_{n ≥ 0}$,
where all elements except the first are positive.
A continued fraction is finite, when its sequence $a_n$ is finite.
In this case, the continued fraction always represents a rational number.
