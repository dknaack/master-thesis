\chapter{The Euclidean Algorithm}

\section{Analysis}

\begin{Pseudocode}
algorithm Euclidean(a, b)
  while $b ≠ 0$ do
    $(a, b) ← (b, a \bmod b)$
  end

  return $a$
end
\end{Pseudocode}

\begin{proposition}
  The ratio $b/a$ decreases by at least $1/2$ over two iterations.
\end{proposition}

\begin{proof}
  Suppose $b/a > 1/2$ in the first iteration.
  Over two iterations, we have
  \[
    \frac{b}{a} · \frac{a \bmod b}{b}
    ≤ \frac{b}{a} · \frac{a - b}{b}
    = 1 - \frac{b}{a}
    < 1 - \frac{1}{2}
    = \frac{1}{2}.
  \]
  Therefore, it decreases by at least $1/2$ over two iterations.
\end{proof}

\section{Fibonacci Numbers}

\begin{definition}
  The \emph{Fibonacci numbers} are defined as
  \begin{enumerate}
    \item $F(0) = F(1) = 1$.
    \item $F(n) = F(n - 1) + F(n - 2)$ for $n > 2$.
  \end{enumerate}
\end{definition}

\begin{proposition}[Lamé's Theorem \cite{Lame1844}]
  The Euclidean algorithm requires at most $5h$ steps,
  where $h = \log_{10}(b)$.
\end{proposition}

\begin{proof}
  \todo[inline]{Using Fibonacci numbers.}
\end{proof}

\section{Continued Fractions}

Continued fractions are fractions of the form
\[
  a_0 + \cfrac{1}{a_1 + \cfrac{1}{a_2 + \cfrac{1}{a_3 + \cfrac{1}{a_4 + \ddots}}}},
\]

\begin{proposition}
  Let $a, b ∈ ℕ$.
  If $a/b$ can be represented as $[a₀; a₁, \dots, a_n]$, then the Euclidean
  algorithm requires exactly $n + 1$ steps given input $(a, b)$.
\end{proposition}

% TODO: Fix this proof, b'/a' is reversed
\begin{proof}
  By induction.
  For $n = 0$, we have $a = a_n · b$.
  Hence, $a \bmod b = 0$ and the algorithm stops after just one iteration.
  Suppose the algorithm requires $n + 1$ steps for inputs $a', b'$ with $a'/b' = [a₁; a₂ \dots, a_n]$.
  For $a, b$ with $a/b = [a₀; a₁, \dots, a_n]$, the algorithm calculates
  \[
    \frac{a \bmod b}{b} = \left\{a₀ + \frac{1}{[a₁; a₂, \dots, a_n]} \right\} = \frac{1}{[a₁; a₂, \dots, a_n]} = \frac{b'}{a'}
  \]
  By our assumption, the algorithm therefore requires $n + 1$ steps for $a, b$.
\end{proof}

\section{Periodicity and Quadratic Irrationals}

\begin{definition}
  A number $x ∈ ℝ$ is said to be a \emph{quadratic irrational} if $x$ is the root of some
  polynomial $p(x)$ over the rational numbers with degree $2$.
\end{definition}

\begin{proposition}
  A number $x ∈ ℝ$ is a quadratic irrational
  if and only if the Euclidean algorithm is periodic on input $x$.
\end{proposition}

\begin{proof}

\end{proof}
