\chapter{The Euclidean Algorithm}

\section{Analysis}

\begin{Pseudocode}
algorithm Euclidean(a, b)
  while $b ≠ 0$ do
    $(a, b) ← (b, a \bmod b)$
  end

  return $a$
end
\end{Pseudocode}

\begin{proposition}
  The ratio $b/a$ decreases by at least $1/2$ over two iterations.
\end{proposition}

\begin{proof}
  Suppose $b/a > 1/2$ in the first iteration.
  Over two iterations, we have
  \[
    \frac{b}{a} · \frac{a \bmod b}{b}
    ≤ \frac{b}{a} · \frac{a - b}{b}
    = 1 - \frac{b}{a}
    < 1 - \frac{1}{2}
    = \frac{1}{2}.
  \]
  Therefore, it decreases by at least $1/2$ over two iterations.
\end{proof}

\section{Fibonacci Numbers}

\begin{definition}
  The \emph{Fibonacci numbers} are defined as
  \begin{enumerate}
    \item $F(0) = F(1) = 1$.
    \item $F(n) = F(n - 1) + F(n - 2)$ for $n > 2$.
  \end{enumerate}
\end{definition}

\begin{proposition}[Lamé's Theorem \cite{Lame1844}]
  The Euclidean algorithm requires at most $5h$ steps,
  where $h = \log_{10}(b)$.
\end{proposition}

\begin{proof}
  \todo[inline]{Using Fibonacci numbers.}
\end{proof}

\section{Continued Fractions}

Continued fractions are fractions of the form
\[
  a_0 + \cfrac{1}{a_1 + \cfrac{1}{a_2 + \cfrac{1}{a_3 + \cfrac{1}{a_4 + \ddots}}}},
\]

\begin{proposition}
  Let $a, b ∈ ℕ$.
  If $a/b$ can be represented as $[a₀; a₁, \dots, a_n]$, then the Euclidean
  algorithm requires exactly $n + 1$ steps given input $(a, b)$.
\end{proposition}

% TODO: Fix this proof, b'/a' is reversed
\begin{proof}
  By induction.
  For $n = 0$, we have $a = a_n · b$.
  Hence, $a \bmod b = 0$ and the algorithm stops after just one iteration.
  Suppose the algorithm requires $n + 1$ steps for inputs $a', b'$ with $a'/b' = [a₁; a₂ \dots, a_n]$.
  For $a, b$ with $a/b = [a₀; a₁, \dots, a_n]$, the algorithm calculates
  \[
    \frac{a \bmod b}{b} = \left\{a₀ + \frac{1}{[a₁; a₂, \dots, a_n]} \right\} = \frac{1}{[a₁; a₂, \dots, a_n]} = \frac{b'}{a'}
  \]
  By our assumption, the algorithm therefore requires $n + 1$ steps for $a, b$.
  \todo[inline]{Finish this proof.}
\end{proof}

\begin{definition}
  The \emph{$n$-th convergent} of an infinite continued fraction $[a_0; a_1, \dots]$
  is defined as the finite continued fraction $[a_0; a_1, \dots, a_n]$.
\end{definition}

\begin{lemma}
  For every $n \in \N$ there exists two polynomials $p_n$ and $q_n$ such that the
  $n$-th convergent of $[a_0; a_1, \dots]$ can be represented by the quotient
  of the two polynomials:
  \[
    [a_0; a_1, \dots, a_n] = \frac{p_n(a_0, a_1, \dots, a_n)}{q_n(a_0, a_1, \dots, a_n)}
  \]
\end{lemma}

\begin{proof}
  For $n = 0$, we have $p_0(a_0) = a_0$ and $q_0(a_0) = 1$.
  By induction, it follows that
  \[
    [a_0; a_1, \dots, a_n] = a_0 + \frac{1}{[a_1; a_2 \dots, a_n]} = a_0 + \frac{q_{n-1}(a_1, \dots, a_n)}{p_{n-1}(a_1, \dots, a_n)}.
  \]
  From this, we can derive the polynomials $p_n$ and $q_n$:
  \begin{align*}
    p_n(a_0, \dots, a_n) & = a_0 p_{n-1}(a_1, \dots, a_n) + p_{n-2}(a_2, \dots, a_n) \\
    q_n(a_0, \dots, a_n) & = p_{n-1}(a_1, \dots, a_n). \qedhere
  \end{align*}
\end{proof}

\begin{definition}
  The canonical structure of the $n$-th convergent $[a_0; a_1, \dots, a_n]$ is defined as
  \[
    [a_0; a_1, \dots, a_n] = \frac{p_n}{q_n} = \frac{p_n(a_0, \dots, a_n)}{p_{n-1}(a_1, \dots, a_n)}.
  \]
\end{definition}

\begin{lemma}
  \label{lem:cf-1}
  Let $x = [a₀; a₁, …]$ and $xₖ = [aₖ; a_{k+1}, …]$, then
  \[
    x = \frac{p_k x_k + p_{k-1}}{q_k x_k + q_{k-1}}.
  \]
\end{lemma}

\begin{proof}
  We have $x = [a₀; a₁, …, a_{k-1}, x_k]$.
\end{proof}

\section{Periodicity and Quadratic Irrationals}

\begin{definition}
  A number $x ∈ ℝ$ is said to be a \emph{quadratic irrational} if $x$ is the root of some
  polynomial $p(x)$ over the rational numbers with degree $2$.
\end{definition}

The interesting connection between the quadratic irrationals and continued
fraction is that a continued fraction of a number $x ∈ ℝ$ is eventually
periodic if and only if $x$ is a quadratic irrational.
We begin with the first direction:

\begin{theorem}[Euler's Theorem]
  If the continued fraction representation of a number $x ∈ ℝ$ is eventually periodic,
  then $x$ is a quadratic irrational.
\end{theorem}

\begin{proof}
  Let $x$ be a continued fraction $[a₀; a₁, …]$ with a period of length $ℓ ≥ 1$ starting at index $k ≥ 0$,
  i.e. $x_k = [a_k; a_{k+1}, …] = [a_{k+ℓ}; a_{k+ℓ+1}, …] = x_{k+ℓ}$.
  By Lemma~\ref{lem:cf-1}, we have
  \[
    x
    = \frac{p_k x_k + p_{k-1}}{q_k x_k + q_{k-1}}
    = \frac{p_{k+ℓ} x_{k+ℓ} + p_{k+ℓ-1}}{q_{k+ℓ} x_{k+ℓ} + q_{k+ℓ-1}}
    = \frac{p_{k+ℓ} x_k + p_{k+ℓ-1}}{q_{k+ℓ} x_k + q_{k+ℓ-1}}
  \]
  Multiplying by the denominators on both sides results in the quadratic equation
  \[
    (q_{k+ℓ} x_k + q_{k+ℓ-1})(p_k x_k + p_{k-1}) - (q_k x_k + q_{k-1}) (p_{k+ℓ} x_k + p_{k+ℓ-1}) = 0.
  \]
  Hence, $x_k$ is a quadratic irrational and $x$ must be one, too.
\end{proof}

Let $x$ be a quadratic irrational with polynomial $p$.
Furthermore, let $x = [a₀; a₁, \dots]$ be its continued fraction and $x_k = [a_k; a_{k+1}, \dots]$.
For every $x_i$ we can find a polynomial $p_i$ with $x_i$ as its root.
We begin by expanding $x$ once as $a₀ + x₁^{-1}$.
By definition, $a₀ + x₁^{-1}$ is the root of $p$.
However, this gives us a new polynomial in $x₁$ (if we multiply by $x₁^2$):
\begin{align*}
  p₁(x₁) = x₁^2 p(a_0 + x_1^{-1}) = A x₁^2 (a₀ + x₁^{-1})^2 + B x₁^2 (a₀ + x₁^{-1}) + C x₁^2 = 0. \\
\end{align*}
Reordering the polynomial into the standard quadratic form results in
\begin{align*}
  p₁(x₁) = \underbrace{(A a₀^2 + B a₀ + C)}_{A₁} + \underbrace{(2A a₀ + B)}_{B₁} + \underbrace{A}_{C₁}.
\end{align*}

We can continue this process for every $i$.
In particular, the coefficients of each polynomial $p_{i+1}$ is calculated
using the following recurrence relation:
\begin{align*}
  A_{i+1} = A_i a_i^2 + B_i a_i + C_i, \qquad
  B_{i+1} = 2 A_i a_i + B_i, \qquad
  C_{i+1} = A_i.
\end{align*}

The goal is to show that these coefficients are bounded in some way.
It follows that there must be some triple which repeats and therefore the
representation must be periodic.
The first step towards this proof is bounding the discriminant of each
polynomial $pᵢ$.

\begin{lemma}
  \label{lem:same-disc}
  The polynomials $p₁, p₂, …$ have the same discriminant.
\end{lemma}

\begin{proof}
  Let $Δ$ be the discriminant of $p₁$.
  By induction, suppose $p_i$ has discriminant $Δ$.
  Then the polynomial $p_{i+1}$ has discriminant
  \begin{align*}
    \phantom{= {}} B_{i+1}^2 - 4 A_{i+1} C_{i+1}
    & = (2 A_i a_i + B_i)^2 - 4 (A_i a_i^2 + B_i a_i + C_i) A_i \\
    & = 4 A_i^2 a_i^2 + 4 A_i B_i a_i + B_i^2 - 4 A_i^2 a_i^2 - 4 B_i A_i a_i - 4 C_i A_i \\
    & = B_i^2 - 4 A_i C_i \\
    & = Δ. \qedhere
  \end{align*}
\end{proof}

In order for the continued fraction to be periodic, the coefficients $A_i, B_i, C_i$ must be bounded.
The only way this could not be the case is when $A_i C_i > 0$,
since then there would be infinitely many values for such that $B_i^2 - 4 A_i C_i$.
However, this is not the case.

\begin{lemma}
  \label{lem:infinite-neg}
  There exists infinitely many polynomials $p_i$ with $A_i C_i < 0$.
\end{lemma}

\begin{proof}
  Suppose $A_i C_i > 0$ for every $i$ after some point.
  We assume WLOG that $A_i > 0$ since the negation of $p_i$ has the same roots.
  Except in the first iteration, every $x_i$ is positive and therefore $B_i$ must be negative.
  But $B_i = 2 A_{i-1} a_{i-1} + B_{i-1}$ would be strictly increasing since
  $A_i$ and $a_i$ are both assumed to be positive.
  This is a contradiction and therefore there must be infinitely many $p_i$ with $A_i C_i < 0$.
\end{proof}

\begin{theorem}
  If $x$ is a quadratic irrational, then its continued fraction is eventually periodic.
\end{theorem}

\begin{proof}
  By Lemma~\ref{lem:same-disc}, every polynomial~$p_i$ has the same
  discriminant~$Δ = B_i^2 - 4 A_i C_i > 0$ independent of the index~$i$
  and by Lemma~\ref{lem:infinite-neg} there are infinitely many polynomials
  such that $B_i^2$ and $-4 A_i C_i$ are both positive.
  But there can only be finitely many positive integers which sum up to another positive integer.
  Hence, there must be some triple $(A_i, B_i, C_i)$ which repeats, i.e. there
  exists an integer $i ≠ j$ with $(A_i, B_i, C_i) = (A_j, B_j, C_j)$.
  These polynomials must have the same root $x_i = x_j$ and by construction of
  the continued fraction, we have $a_{i+ℓ} = a_{j+ℓ}$ for every $ℓ ≥ 0$.
\end{proof}
