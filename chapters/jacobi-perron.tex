\chapter{Jacobi-Perron Algorithm}

\section{Notes on the Book}

\begin{itemize}
  \item
    \textbf{Input}: Three numbers $u₀, v₀, w₀ ∈ ℝ$.
  \item
    \textbf{Output}:
    \begin{align*}
      u_{i + 1} &= v_i - ℓ_i u_i, \\
      v_{i + 1} &= v_i - m_i u_i, \\
      w_{i + 1} &= u_i,
    \end{align*}
  \item Non-homogenous form:
    \begin{align*}
      \frac{v_{i+1}}{u_{i+1}} & = \frac{{w_i}{u_i} - m_i}{\frac{v_i}{u_i} - ℓ_i}, \\
      \frac{w_{i+1}}{u_{i+1}} & = \frac{1}{\frac{v_i}{u_i} - ℓ_i},
    \end{align*}
\end{itemize}

The set $E_{n-1}$ denotes the $(n-1)$-dimensional Euclidean vector space of
$(n-1)$-tuples as real numbers in the book.
Essentially, $E_{n-1} = ℝ^n$.
\[
  a^{(k)} = (a₁^{(k)}, a₂^{(k)}, \dots, a_{n-1}^{(k)}) ∈ E_{n-1}
\]

\begin{definition}
  Let $f \colon ℝ^n → ℝ^n$ be a function such that for every $x ∈ ℝ^n$
  \[
    y = f(x) \text{ and } y_i ≠ x_i \text{ for every } i ≤ n.
  \]
  Then the transformation $T$ on any vector $x ∈ ℝ^n$ is defined as
  \[
    x T = (x₁ - y₁)^{-1} · (x₂ - y₂, \dots, x_{n-1} - y_{n-1}).
  \]
  The function $f$ is called a $T$-function or a function associated with $T$.
\end{definition}

\begin{example}
  Let $f(x) = 1/2 · x$, then
  \[
    x T = x₁^{-1} (x₂, \dots, x_n, 2).
  \]
\end{example}

\section{Periodic Cases}

\begin{theorem}
  The Jacobi-Perron algorithm is periodic for
  \[
    α = \sqrt[d]{(bc)^d + c} \text{ with } 1 ≤ c ≤ bc / (n - 2) \text{ and } n ≥ 3.
  \]
\end{theorem}

Let $f(a, k, m)$ be a function with
\[
  f(a, k, m) = ∑_{i=0}^k \binom{m + i - 1}{i} a^{k - i} (bc)^i
\]

\begin{lemma}
  Let $b$ and $c$ be numbers satisfying the conditions of the previous theorem.
  Then for $k ≤ n - 1$ and for $m ≤ n - k$,
  \[
    \floor{f(α, k, m)} = f(bc, k, m) = \binom{m + k}{k} (bc)^k,
  \]
\end{lemma}

\begin{proof}
\end{proof}
