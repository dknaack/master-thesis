\chapter{Bounds on the Decrease of the Determinant}

% ==============================================================================
\section{Choosing the Minimum Fractional Value}
% ==============================================================================

\begin{proposition}
  The determinant decreases by a factor of at least $1/2$ over two iterations
  using the minimum strategy.
\end{proposition}

\begin{proof}
  Suppose we do not achieve a decrease of $1/2$ in the first iteration, i.e. $\{x_ℓ\} > 1/2$.
  Then, over two iterations the determinant decreases by
  \[
    \{x₁\} \left\{\frac{1}{\{x₁\}}\right\} < 1 - \{x₁\} < \frac{1}{2}. \qedhere
  \]
\end{proof}

Unfortunately, the decrease cannot be improved beyond this point even in higher dimensions.
To see this, we construct an input which achieves a decrease very close to $1/2$.
For this input we choose $x₁$ to be very close to $1/2$ and we shove the rest of the variables very close to $1$.
The total decrease is then very close to $1/2$.

\begin{proposition}
  For every (sufficiently small) $ε > 0$,
  there exists an input $x ∈ ℝ^d$
  which achieves a decrease of exactly $1/2 - ε$ over two iterations.
\end{proposition}

\begin{proof}
  For $d = 1$, let $x₁ = 1/2 + ε$.
  Over two iterations, the determinant decreases by
  \[
    x₁ \left\{\frac{1}{x₁}\right\} = 1 - (1/2 + ε) = 1/2 - ε.
  \]


  For $d > 1$, let $x₁ = (1 + ε)/2$ and $xᵢ = 1 - ε/2$ for $i > 1$.
  Clearly, $x₁$ is chosen in the first iteration.
  In the second iteration, the choice does not matter as $\{xᵢ/x₁\}$ is the same for every $i > 1$.
  Hence, the total decrease over two iterations is
  \[
    x₁ \left\{\frac{xᵢ}{x₁}\right\}
    = xᵢ - x₁
    = 1 - \frac{ε}{2} - \frac{1 + ε}{2}
    = \frac{1}{2} - ε. \qedhere
  \]
\end{proof}


% ==============================================================================
\section{Choosing the Closest Pair of Values}
% ==============================================================================

The second strategy works as follows:
\begin{enumerate}
  \item Sort $x$ in increasing order.
  \item Find the index $ℓ$ which minimizes $\{x_{ℓ+1}/x_ℓ\}$.
    In the case that $ℓ = d$, let $x_{ℓ + 1} = 1$.
  \item Choose $ℓ$ in the first iteration and $ℓ + 1$ in the second iteration.
\end{enumerate}

To analyze the decrease of this strategy over two iterations,
we assume that each choice of our pivot yields the same decrease.
Hence, we have
\[
  x_1 = \frac{x_2}{x_1} - 1 = \frac{x_3}{x_2} - 1 = \dots = \frac{x_d}{x_{d-1}} = \frac{1}{x_d} - 1.
\]
Solving the equation for $x_d$ yields the following solution:
\[
  x_d^{d+1} + x_d - 1 = 0 \text{ and } x_i = x_d^{d+1-i} \text{ for } i < d.
\]

Let $ψ$ be the root of the polynomial.
The following lemma helps us calculate the fractional value of $1/ψ$.

\begin{lemma}
  For every $d ≥ 1$, we have $1/ψ < 2$.
\end{lemma}

\begin{proof}

\end{proof}

\begin{theorem}
  The determinant decreases by at least $ψ^{d+1}$ over two iterations with $d ≥ 2$.
\end{theorem}

\begin{proof}
  We assume WLOG that the vector is sorted in increasing order and
  $0 ≤ x_i ≤ 1$ for every $i ≤ d$.
  For a contradiction, assume the algorithm yields a smaller decrease than $ψ^{d+1}$ on input $x$.
  We must have $x_i > ψ^{d+1-i}$ for every $i = 1, \dots, d$.
  For the first value, we have $x₁ > ψ^d$, because otherwise we have a total decrease of
  \[
    x₁ \left\{ \frac{x₂}{x₁} \right\} ≤ ψ^{d-1} - ψ^d = ψ^{2d+1} < ψ^d.
  \]
  Suppose $x_i > ψ^{d+1-i}$ and $x_{i+1} ≤ ψ^{d-i}$.
  Then we can achieve a total decrease of
  \[
    x_i · \left\{ \frac{x_{i+1}}{x_i} \right\} ≤ x_{i+1} - x_i < ψ^{d-i} - ψ^{d+1-i} = ψ^{2d+1-i} < ψ^d.
  \]
  It follows that $x_i > ψ^{d+1-i}$ for every $i ≤ d$.
  But then
  \[
    x_d \left\{ \frac{1}{x_d} \right\} ≤ 1 - x_d < 1 - ψ = ψ^{d+1}.
  \]
  Hence, we achieve a decrease of at least $ψ^{d+1}$ over two iterations.
\end{proof}

The bound for this strategy is tight.
We can construct an input that moves arbitrarily close to this bound.
Formally, for every sufficiently small $ε > 0$, we can find an input $x$ which
achieves a decrease of at most $ψ^{d+1} - ε$.
The idea is to choose $x_d$ to be just over $ψ$ and all other variables as a multiple of $x_d$
such that we have to choose $x_d$.
But choosing $x_d$ only gives us a decrease of $ψ^{d+1} - ε$ in total.

% TODO: Explain what sufficiently small means
\begin{theorem}
  For every (sufficiently small) $ε > 0$,
  there exists an input $x ∈ ℝ^d$ with $d ≥ 2$,
  which achieves a decrease in the determinant of exactly $ψ^{d+1} - ε$ over two
  iterations.
\end{theorem}

\begin{proof}
  We choose $x_i = ψ^{d+1-i} (1 + ε)$ for $i = 1, \dots, d$.
  The strategy chooses between
  \[
    \left\{ \frac{x_{i+1}}{x_i} \right\}
    = \frac{ψ^{d-i} (1 + ε)}{ψ^{d+1-i} (1 + ε)} - 1 = \frac{1}{ψ} - 1,
    \text{ and }
    \left\{ \frac{1}{x_d} \right\}
    = \frac{1}{ψ + ε} - 1
  \]
  Clearly, the strategy chooses $x_d$ since its ratio $\{1/x_d\}$ is the smallest.
  Therefore, the total decrease over two iterations is
  \[
    (ψ + ε) \left\{ \frac{1}{ψ + ε} \right\} = 1 - ψ - ε = ψ^{d+1} - ε.
    \qedhere
  \]
\end{proof}
