\chapter{Periodic Representation of Cubic Irrationals}

For quadratic irrationals, we saw that the Euclidean algorithm can be used to
derive a periodic representation for the quadratic irrationals.
Naturally this leads to the question, whether the generalized Euclidean
algorithm can be used to represent cubic irrationals or irrationals of even
higher degree.
In the previous chapter, we have already seen that the algorithm is periodic
for a generalization of the golden ratios to a higher degree.
In this chapter, we will analyze the algorithm on its periodicity on cubic
irrationals.

% ==============================================================================
\section{Finding a Periodic Representation through Brute-Force}
% ==============================================================================

The generalized Euclidean algorithm allows us for one additional degree of
freedom.
We have already seen that in the case of Fibonacci numbers,
that different strategies in the choice of our pivot element will only allow
certain irrationals.
Therefore, to find a strategy for \emph{all} irrationals we simply try all strategies.

\begin{Pseudocode}[caption={The brute-force search algorithm for finding a periodic representation.}]
algorithm $\textsc{brute-force-search}(x, N)$
  for $L ∈ \{1, …, d\}^{≤ N}$ do
    $x₀ ← x$
    for $i ∈ \{1, …, |L|\}$ do
      $x_i ← \mathrm{pivot}_{L[i]}(x_{i-1})$
      if $x_i$ occurred twice then
        Find the index $j ≠ i$ where $x_j = x_i$
        $S ← L[1] \,…\, L[j]$
        $P ← L[j+1] \,…\, L[i]$
        return $(S, P)$
  return $ε$
\end{Pseudocode}

Given a real number $r ∈ ℝ$, we use the update rule from the generalized
algorithm to find its representation.
Of course, the update rule does not take a real number but a vector as input,
so we use the initial vector $x_i = q_i(r)$ for $i ≤ d$, where $q_i$ is an
arbitrary polynomial of degree $i$.
Most of the time we will use $q_i(r) = r^i$ as our polynomials.

Now that we have our input vector $x ∈ ℝ^d$, we need to find a possible
sequence of indices $ℓ_1, ℓ_2, …, ℓ_N$, where updating $x$ with those
indices leads to one vector occurring twice.
In order to find this sequence, we try every possible sequence
$ℓ_1, ℓ_2, … ∈ \{1, …, d\}^*$ until we find some duplicate vector.
In particular, we use a breadth-first search over the sequences.
We begin with all sequences of length $1$ and see if any vector occurs twice.
If not, then we continue with all sequences of length $2$ and check again if
any previous vector has occurred twice.
We continue this process indefinitely until we hopefully find a duplicate vector.

% Example?
\begin{example}
  Given $r = \sqrt[3]{2}$, we construct the input vector $x = \left(\sqrt[3]{2}, \sqrt[3]{4}\right)$.
  For this input, the algorithm finds the sequence $0\overline{10}$.

  % TODO: Include the whole example.
\end{example}

% Wait, how would we represent non-cubic irrationals? I guess we could include
% the whole tree. There should be a discussion about representing non-cubic
% irrationals and also that the end goal would be to find the actual strategy
% behind the brute-force search or at least one possible strategy, it might not
% have to be the optimal one which returns the result from the brute force
% search exactly.

% I think for this discussion it would be good to present the case for a
% minimum continued fraction or even the ternary continued fractions. This
% would sort of present the brute-force search more as a meta-proof on all
% Jacobi-Perron type algorithm which could be periodic.

If we do find a duplicate, then the representation for that number will be unambiguous.
However, if we don't, then it is unclear how we will represent that number.
One possible solution is to simply include everything, the whole tree.
The aim of this algorithm is not to find a representation in and of itself,
but to aid in finding one possible strategy for our goal.
In the end, we might only need to look at all sequences of length $d$ at every
iteration to find the optimal choice.
In the worst case, there would be no strategy we can follow to determine a
periodic representation for cubic irrationals.

% ==============================================================================
\section{Multi-Dimensional Continued Fractions (MDCF)}
% ==============================================================================

The multi-dimensional extension of the continued fractions is based on the
update rule. Recall that given a vector $x ∈ ℝ^d$ and index $ℓ ≤ d$, we update
the input vector over one iteration by
\[
  \begin{array}{lcrlcr}
    \displaystyle x_i' & = & \displaystyle \frac{x_i - a_i}{x_ℓ - a_ℓ}, &
    \displaystyle x_ℓ' & = & \displaystyle \frac{x_i - a_i}{x_ℓ - a_ℓ} \\[1em]
  \end{array}
\]
where $a = \floor{x}$ and $i ≠ ℓ$.
We can reverse this process by rearranging the equations
\[
  \begin{array}{lcrlcr}
    \displaystyle x_i & = & a_i + \displaystyle \frac{x_i'}{x_ℓ'}, &
    \displaystyle x_ℓ & = & a_ℓ + \displaystyle \frac{1}{x_ℓ'}
  \end{array}
\]
This allows us to calculate the previous vector $x$ from the next vector $x'$,
if we know the integer vector $a$.
Let $\mathrm{pivot}_ℓ^{-1}$ denote this inverse function.
We have
\[
  \mathrm{pivot}_ℓ(x) = x' \iff \mathrm{pivot}_ℓ^{-1}(a, x') = x.
\]

Independently of the Euclidean algorithm,
we can look at what happens when $a$ is not necessarily an integer vector.

\begin{definition}
  Given a sequence of $d$-dimensional vectors $\{rₙ\}_{n ≥ 0}$ and a sequence of
  indices $\{ℓₙ\}_{n ≥ 0}$, the \emph{multi-dimensional continued fraction} $[r₀; ℓ₁, r₁; …]$
  is defined as
  \[
    [r₀; ℓ₁, r₁; …] = \lim_{n → ∞} [r₀; ℓ₁, r₁; …; ℓₙ, rₙ],
  \]
  where the finite continued fractions are defined as
  \[
    [r₀] = r₀, \qquad [r₀; ℓ₁, r₁; …; ℓₙ, rₙ] = \mathrm{pivot}_{ℓ₁}^{-1}\big(r₀, [r₁; ℓ₂, r₂; …; ℓₙ, rₙ]\big).
  \]
\end{definition}

\begin{remark}~
  \begin{enumerate}
    \item
      The first item has no index.
      The index sequence $ℓₙ$ is always one shorter than the vector sequence $rₙ$.
    \item
      For the generalized Euclidean algorithm, the sequence $r_n$ consists solely of integer vectors.
      However, the MDCF can also be defined over rational or even real vectors.
      To differentiate the two, a sequence $rₙ$ will denote a sequence of real
      vectors and $aₙ$ will denote a sequence of integer vectors.
  \end{enumerate}
\end{remark}

The concepts from one-dimensional continued fractions naturally carry over to its
multi-dimensional counterpart.

\begin{definition}[Complete Quotient, Convergent, Periodicity]~
  Given a multi-dimensional continued fraction $[r₀; ℓ_1, r_1; ℓ_2, r_2; …]$,
  we define the following:
  \begin{itemize}
    \item The \emph{$k$-th complete quotient} is defined as $[rₖ; r_{k+1}, …]$.
    \item The \emph{$k$-th convergent} of $x$ is defined as the finite fraction $[r₀; ℓ₁, r₁; …; ℓ_k, r_k]$.
    \item The MDCF is \emph{eventually periodic} if there exists $k₀ ≥ 0$ and $ℓ ≥ 1$ such that $rₖ = r_{k+ℓ}$ for every $k ≥ k₀$.
      The MDCF is \emph{purely periodic} if $k₀ = 0$.
  \end{itemize}
\end{definition}

\section{Periodic MDCFs}

To show that any periodic MDCF contains elements of a field with degree $d + 1$
over the rationals, we proceed similarly to the one-dimensional continued
fraction.
We first show the multi-dimensional analogue of Lemma~\vref{lem:nesting}.

\begin{lemma}[Nesting]
  \label{lem:mdcf-nesting}
  Let $x ∈ ℝ^d$, then
  \[
    [a₀; ℓ₁, a₁; …; ℓₙ, aₙ; ℓ, x]
    = [a₀; ℓ₁, a₁; \cdots; ℓₙ, aₙ + \mathrm{pivot}_{ℓ}(x)]
  \]
\end{lemma}

\begin{proof}
  If $n = 0$, then by definition,
  \[
    [a₀; ℓ, x] = a₀ + \mathrm{pivot}_{ℓ}[x] = a₀ + \mathrm{pivot}_{ℓ}(x) = [a₀ + \mathrm{pivot}_ℓ(x)].
  \]
  Suppose the lemma holds for any $n ≥ 0$, then
  \begin{align*}
    [a₀; ℓ₁, a₁; …; ℓₙ, aₙ; ℓ, x]
    & = a₀ + \mathrm{pivot}_{ℓ₀}[a₁; …; ℓₙ, aₙ; ℓ, x] \\
    & = a₀ + \mathrm{pivot}_{ℓ₀}[a₁; …; ℓₙ, aₙ + \mathrm{pivot}_ℓ(x)] \\
    & = [a₀; ℓ₁, a₁; …; ℓₙ, aₙ + \mathrm{pivot}_ℓ(x)]. \qedhere
  \end{align*}
\end{proof}

In Lemma~\vref{lem:wallis}, we had to define two sequences $pₙ$ and $qₙ$ over
the sequence of the continued fraction $aₙ$.
Each sequence would only return a single scalar.
In total, we define four sequences:
A matrix sequence $P^{(n)}$, two vector sequences $Q^{(n)}$ and $p^{(n)}$ as well as a scalar sequence $q^{(n)}$.
Their are defined as follows:
\begin{align*}
  P_{ℓₙ}^{(n)} & = P^{(n-1)} a_n + p^{(n-1)}, & P_i^{(n)} & = P_i^{(n)}, & P^{(-1)}   & = I_d, \\
  Q_{ℓₙ}^{(n)} & = Q^{(n-1)} a_n + q^{(n-1)}, & Q_i^{(n)} & = Q_i^{(n)}, & Q^{(-1)}_j & = 0,   \\
  p^{(n)}      & = P_{ℓₙ}^{(n-1)},            &           &              & p^{(-1)}_j & = 0,   \\
  q^{(n)}      & = Q_{ℓₙ}^{(n-1)},            &           &              & q^{(-1)}   & = 1.
\end{align*}
where $i ≠ ℓ_n$.
What this sequence effectively does is reconstructing the lattice from an
initial solution vector $x ∈ ℝ^d$ and its coefficient vectors $a_n$.

\begin{lemma}[Wallis]
  \label{lem:mdcf-wallis}
  Let $x ∈ ℝ^d$, then
  \[
    [a₀; ℓ₁, a₁; …; ℓ_{n-1}, a_{n-1}; ℓ, x]
    = \frac{P x + p}{Q^T x + q}.
  \]
\end{lemma}

\begin{proof}
  If $n = 0$, then
  \[
    [x] = x = \frac{I_d x + 0}{0^T x + 1}.
  \]
  Suppose the lemma holds for $n ≥ 0$, then there exists a matrix $P$, vectors
  $Q, p$ and scalar $q$ such that
  \begin{align*}
    y & = [a₀; ℓ₁, a₁; …; ℓ_{n-1}, a_{n-1}; ℓ, x]                              \\
      & = [a₀; ℓ₁, a₁; …; ℓ_{n-1}, a_{n-1} + \mathrm{pivot}_ℓ(x)]              \\
      & = \frac{P (a + \mathrm{pivot}_ℓ(x)) + p}{Q^T (aₙ + \mathrm{pivot}_ℓ(x)) + q} \\
      & = \frac{x_ℓ}{x_ℓ} · \frac{P (a + \mathrm{pivot}_ℓ(x)) + p}{Q^T (aₙ + \mathrm{pivot}_ℓ(x)) + q}.
  \end{align*}
  The numerator has the following form:
  \begin{align*}
    x_ℓ (P (a + \mathrm{pivot}(x, ℓ)) + p)
    & = x_ℓ (P a + P_ℓ \frac{1}{x_ℓ} + \sum_{i ≠ ℓ} P_i \frac{x_i}{x_ℓ} + p) \\
    & = \underbrace{(P a + p)}_{P_ℓ'} x_ℓ + \sum_{i ≠ ℓ} \underbrace{P_i}_{P_i'} x_i + \underbrace{P_ℓ}_{p'} \\
    & = P' x + p'.
  \end{align*}
  Analogously, the denominator has the following form:
  \begin{align*}
    x_ℓ (Q^T (a + \mathrm{pivot}(x, ℓ)) + q)
    & = x_ℓ (Q^T a + Q_ℓ \frac{1}{x_ℓ} + \sum_{i ≠ ℓ} Q_i \frac{x_i}{x_ℓ} + q) \\
    & = \underbrace{(Q^T a + q)}_{Q_ℓ'} x_ℓ + \sum_{i ≠ ℓ} \underbrace{Q_i}_{Q_i'} x_i + \underbrace{Q_ℓ}_{q'} \\
    & = Q' x + q'.
  \end{align*}
  Hence,
  \begin{align*}
    y & = \frac{P(a + \mathrm{pivot}(x, ℓ)) + p}{Q(a + \mathrm{pivot}(x, ℓ)) + q} \\
      & = \frac{P' x + p'}{Q'^T x + q'}. \qedhere
  \end{align*}
\end{proof}

\begin{proposition}
  Given $r ∈ ℝ$, let $x = (r¹, r², …, r^d)$.
  If the MDCF representation of $x$ is purely periodic, then $[ℚ(r) : ℚ] ≤ d + 1$.
\end{proposition}

\begin{proof}
  Suppose the algorithm is purely periodic on input $x$ with period $ℓ$.
  Let $y$ be the $ℓ$-th complete quotient of $x$, then $x = y$.
  By Lemma~\ref{lem:mdcf-wallis},
  \[
    rⁱ = \frac{\sum_{j=1}^d P_{ij} rʲ + pᵢ}{\sum_{j=1}^d Qⱼ rʲ + qᵢ}, \text{ for every } i ≤ d.
  \]
  Multiplying both sides with the denominator results in the polynomial equation
  \[
    \sum_{j=1}^d (Q_j r^{i+j} - P_{ij} r^j) + r^i q_i - p_i = 0.
  \]
  For $i = 1$, the maximum degree of this polynomial is $d + 1$.
  Hence, $[ℚ(r) : ℚ] ≤ d + 1$.
\end{proof}

% ==============================================================================
\section{Homogeneous Coordinates}
% ==============================================================================

The pivot operation is quite cumbersome to write.
We'll always need to differentiate the cases $i = ℓ$ and $i ≠ ℓ$.
Homogeneous coordinates allow us to simplify the whole pivot operation as a single matrix.

Instead of an $d$-dimensional vector space, we project each vector into a $(d+1)$-dimensional vector space.
Each vector $x = (x₁, …, x_d) ∈ ℝ^d$ is extended by a new coordinate $x₀ = 1$.
We denote this vector as $\hat x = [1, x₁, …, x_d]$.
Two such vectors $\hat x, \hat y$ are considered equivalent if they are collinear.
Stated differently, each vector $[x₀, x₁, …, x_d]$ represents a line in the direction $(x₀, x₁, …, x_d)$.

% TODO: Explain mapping from and back to the original vector space.
\begin{center}
  \begin{tikzpicture}
    \matrix[
      column sep=2cm,
      nodes={text width=3cm, align=center},
    ] {
      \node (L0) {$\mathbb{R}^d$}; &
      \node (R0) {$\mathbb{RP}^{d+1}$}; \\
      \node (L1) {$(x₁, …, x_d)$}; &
      \node (R1) {$[1, x₁, …, x_d]$}; \\
      \node (L2) {$(x₁/x₀, …, x_d/x₀)$}; &
      \node (R2) {$[x₀, x₁, …, x_d]$}; \\
    };

    \draw[->] (L1) -- node[above] {} (R1);
    \draw[<-] (L2) -- node[above] {} (R2);
  \end{tikzpicture}
\end{center}

For example, consider the vector $[1, x₁, x₂]$ with $0 ≤ x₁, x₂ < 1$.
A pivot operation with $ℓ = 1$ would result in the vector $[1, 1/x₁, x₂/x₁]$.
This vector is equivalent to $[x₁, 1, x₂]$.
Therefore, we can reformulate this operation as a coordinate swap of $x_ℓ$ with
the new coordinate $x₀$:
\[
  \begin{bmatrix}
    0 & 1 & 0 \\
    1 & 0 & 0 \\
    0 & 0 & 1 \\
  \end{bmatrix}
  ·
  \begin{bmatrix} 1 \\ x₁ \\ x₂ \\ \end{bmatrix}
  =
  \begin{bmatrix} x₁ \\ 1 \\ x₂ \\ \end{bmatrix}.
\]
Similarly, subtracting the integer part of each value in $[1, x₁, x₂]$ is
equivalent to a series of skew operations:
\[
  \begin{bmatrix}
    1 & 0 & 0 \\
    -\floor{x₁} & 1 & 0 \\
    0 & 0 & 1 \\
  \end{bmatrix}
  ·
  \begin{bmatrix}
    1 & 0 & 0 \\
    0 & 1 & 0 \\
    -\floor{x₂} & 0 & 1 \\
  \end{bmatrix}
  ·
  \begin{bmatrix} 1 \\ x₁ \\ x₂ \\ \end{bmatrix}
  =
  \begin{bmatrix} 1 \\ x₁ - \floor{x₁} \\ x₂ - \floor{x₂} \\ \end{bmatrix}.
\]
Importantly, each of those matrices with
determinant $1$.
Therefore, the whole operation can be reversed by inverting the matrix.
This is the equivalent of the inverse pivot operation in homogeneous
coordinates.

We can also reformulate the MDCF as a series of matrix multiplications.
In the following $T(a)$ denotes the translation by a vector $a ∈ ℝ^d$
and the matrix $S(ℓ)$ denotes the swap of $x_\ell$ with $x_0$.
The MDCF can then be written as
\[
  [a₀] = \hat a₀, \qquad
  [a₀; ℓ₁, a₁; …; ℓₙ, aₙ] = T(a₀) · S(ℓ_1) · [a₁; ℓ_2, a_2; …; ℓ_n, a_n].
\]

This allows us to drastically simplify Lemma~\ref{lem:mdcf-wallis}.
Now, we only have a single matrix sequence $P_n$, defined as follows:
\begin{align*}
  P_n = P_{n-1} A_n, \qquad P_0 = I_{d+1}.
\end{align*}

\begin{lemma}
  \label{lem:mdcf-wallis'}
  Let $x ∈ ℝ^d$, then
  \[
    [a₀; ℓ₁, a₁; …; ℓ_{n-1}, a_{n-1}; ℓ, x] = P_{n-1} \hat x
  \]
\end{lemma}

\begin{proof}
  For $n = 0$, this is clearly true.
  Suppose the lemma is true for $n ≥ 0$, then
  \begin{align*}
    [a₀; ℓ₁, a₁; …; ℓ_n, a_n; ℓ, x]
    & = [a₀; ℓ₁, a₁; …; ℓ_n, a_n + \mathrm{pivot}_ℓ(x)] \\
    & = P_{n-1}
    \begin{pmatrix}
      a_n + \mathrm{pivot}_ℓ(x) \\ 1
    \end{pmatrix} \\
    & = P_{n-1} A_n \hat{x} \\
    & = P_n \hat{x} \qedhere
  \end{align*}
\end{proof}

\begin{theorem}
  Let $A ∈ ℤ^{d×d}, B ∈ ℤ^d$ and $C ∈ ℤ$.
  If $x^\top A x + B x + C = 0$ for some vector $x ∈ ℝ^d$,
  then the MDCF of $x$ is eventually periodic.
\end{theorem}

\begin{proof}

\end{proof}

\begin{corollary}
  If $r$ is the root of a polynomial of degree $2d$,
  then the MDCF of $x = (r^1, …, r^d)$ is eventually periodic.
\end{corollary}

% ==============================================================================
\section{MDCFs of Cubic Irrationals}
% ==============================================================================

\begin{conjecture}
  The brute-force algorithm for $d = 2$ is periodic if and only if $α$ is a cubic irrational.
\end{conjecture}

% ==============================================================================
\section{Comparison to the Jacobi-Perron Algorithm}
% ==============================================================================

Many generalizations to the Euclidean algorithm have been considered.
One of them was proposed by Jacobi to answer Hermite's question.
In his version, he computes the GCD of three numbers by successively dividing
the smallest number from the larger numbers.
This algorithm was later extended by Oskar Perron to arbitrarily many numbers.
The algorithm works as follows:

Given a list of positive integers $a₀, a₁, …, aₙ$, take the smallest number $a_ℓ$
and compute the remainder $a_i'$ resulting from the division of $a_i$ with $a_ℓ$.
The value $a_ℓ$ is kept until the next iteration, i.e. $a_ℓ' = a_ℓ$.
Continue this process until all but one value remains.
\begin{align*}
  a₀' = a₀ \bmod a_ℓ, a₁ = a₁ \bmod a_ℓ, …, a_ℓ' = a_ℓ, …, aₙ' = aₙ \bmod a_ℓ; \\
\end{align*}

Perron modified this algorithm for the purposes of his analysis.
The integers $a₁, …, aₙ$ are kept in a list.
We remove the first element from the list, calculate the remainders for each
remaining element and append the element to the end of the list.
One iteration in this modified version produces the values:
\begin{align*}
  a₀' = a₁ \bmod a₀, a₁' = a₂ \bmod a₀, …, a_{n-1}' = a_n \bmod a₀, aₙ = a₀. \\
\end{align*}
This process is repeated until the first element is zero.

Of course, the termination condition is not sufficient.
When this algorithm terminates, the remaining elements might not all be zero.
Therefore, we remove the first element from the list and continue with the
remaining list.

By allowing real numbers as inputs, this algorithm proceeds infinitely but
always converges to zero.

The Jacobi-Perron algorithm is actually a subset of the generalized Euclidean algorithm.
The generalized Euclidean algorithm is periodic for all real numbers where the Jacobi-Perron algorithm is periodic.
This comes from the fact, that the Jacobi-Perron algorithm is really the
generalized Euclidean algorithm with the specific sequence of pivots $L = \overline{12…d}$.
So in the $i$th iteration, we are choosing index $(i \bmod d) + 1$ as our pivot.

This has the convenient property that if the all inputs which are periodic for
the Jacobi-Perron algorithm must also be periodic for the brute-force algorithm.

\begin{theorem}
  The brute-force algorithm is periodic on input $(1, r, r^2)$ if
  \[
    r = \sqrt[n]{D + d}, \text{ where } d | D.
  \]
\end{theorem}

