\chapter{The Generalized Euclidean Algorithm}
\label{ch:generalized-euclidean}

For a real number, we used the Euclidean algorithm to construct its continued
fraction.
We saw that a continued fraction is periodic if and only if the number used in
its construction is a quadratic irrational.
Naturally, we can ask for a generalization of the Euclidean algorithm,
which can be used to construct higher-dimensional continued fractions
and potentially leads to an answer of Hermite's question.

In this chapter, we analyze one such possible candidate by Klein and Reuter~\cite{Klein24},
which is a generalization of the Euclidean algorithm.
While the Euclidean algorithm works with integers,
the generalized version works with vectors.
It solves the problem of lattice basis computation,
which can be seen as a multidimensional analogue of computing the greatest
common divisor.

% ==============================================================================
\section{Lattice Basis Computation}
% ==============================================================================

In a vector space, an over-defined set of vectors $A = \{A₁, …, Aₙ\}$ with $n > d$
can be easily reduced to a basis $B$ with exactly $d$ vectors by taking $d$
linearly independent vectors from the set $A$.
In a lattice, it is not as simple.
Taking only $d$ linearly independent vectors produces a different lattice than
the whole set, in general.
For example, consider the matrix
\[
  A = \begin{pmatrix}
    3 & 0 & 2 \\
    0 & 2 & 1 \\
  \end{pmatrix}.
\]
Taking any two columns $A_i, A_j$ of the matrix produces a linearly independent
set, but no set produces the same lattice as the whole matrix.
The missing vector $A_k$ cannot be reached by a simple integral combination of the
other two vectors $A_i, A_j$.
Instead, we have to find new basis vectors for this lattice.

% TODO: Note about the fact that we only care about one additional vector
% TODO: Shouldn't it be the rank instead of n > d?
The problem of finding a basis for a lattice is known as \emph{lattice basis computation}.
The input for this problem is a matrix $A ∈ ℤ^{d×n}$ with $n > d$ and the goal
is to find a basis which produces the same lattice, i.e. $\mathcal L(A) = \mathcal L(B)$.
The problem can also be seen as a multidimensional analogue of
computing the greatest common divisor from a given set of numbers.
Indeed, we can view a set of numbers $a₁, a₂, …, aₙ ∈ ℤ$ as the basis vectors
of the one-dimensional lattice
\[
  a₁ℤ + a₂ ℤ + \dots + aₙ ℤ.
\]
The greatest common divisor represents a reduced basis in this lattice.

Since lattice basis computation is such a fundamental operation, it has many
applications \cite{Ajtai96,Gentry08}.
Previous approaches for computing such bases either used a variant of the LLL
algorithm~\cite{Lenstra82} or computed the Hermite normal form of the
matrix~\cite{Storjohann96}.
The generalized Euclidean algorithm uses a fundamentally different approach,
which more closely resembles the classical Euclidean algorithm used in the
one-dimensional setting.

% ==============================================================================
\section{The Algorithm}
% ==============================================================================

\begin{Pseudocode}[
    float=tb,
    label={lst:generalized-euclidean},
    caption={The Generalized Euclidean Algorithm \cite{Klein24}.}]
solve $Bx = c$
while $x$ is not integral do
  find $x_ℓ$ which is not integral
  $c ← B_ℓ$
  $B_ℓ ← B\{x\}$
  solve $Bx = c$
end
\end{Pseudocode}

The algorithm is shown in Listing~\ref{lst:generalized-euclidean}.
It takes a matrix $B ∈ ℤ^{d × d}$ and a vector $c ∈ ℤ^d$ as input and finds a
new basis $B'$ such that $\mathcal L(B) = \mathcal L(B ∪ \{c\})$.
Its basic structure consists of two steps:
A modulo operation and an exchange operation.
The operations are similar to the Euclidean algorithm, where we first compute
the modulo $a \bmod b$ and then swap the two inputs.

First, we look at the modulo operation.
Suppose we have a basis $B$ with vectors $B₁, …, B_d$, any vector $a ∈ ℝ^d$ can
easily be represented as a linear combination $B₁x₁ + \dots + B_d x_d$ of the
basis vectors.
However, linear combinations are not integral combinations,
so the vector $a$ might not be in the lattice $\mathcal L(B)$.
But the vector $z = B₁ \floor{x₁} + \dots + B_d \floor{x_d}$
is a lattice point.
We could say that $z = B\floor{x}$ is the integer part of the vector.
The fractional part of $a$ would then be $r = a - z$.
The fractional part is actually contained in what is known as the \emph{fundamental parallelepiped} of the lattice.
% TODO: Fix this part

So in a lattice, each point $a ∈ ℝ^d$ can be represented as a combination of a lattice point $z
∈ \mathcal{L}(B)$ and a point in the fundamental parallelepiped $r ∈ Π(B)$.
Specifically,
given a solution $x$ to $Bx = a$,
\[
  a = B\floor{x} + B\{x\} = z + r.
\]
This is essentially a multidimensional division with remainder operation inside a lattice.
It allows us to define a modulo operation for the algorithm as follows:
\[
  a \pmod{Π(B)} := a - B\floor{B^{-1} a}.
\]
It remains to be shown that applying this modulo operation on $a$ does not
affect the underlying lattice $\mathcal L(B ∪ \{a\})$.

\begin{lemma}
  \label{lem:lattice-mod}
  Given a set of vectors $A = \{A_1, \dots, A_n\}$ with $A_i ∈ ℤ^d$
  and a linearly independent subset $B$,
  let $r = c \pmod{Π(B)}$ for any vector $c ∈ A \setminus B$.
  Then,
  \[
    \mathcal L(A) = \mathcal L((A \setminus \{c\}) ∪ \{r\}).
  \]
\end{lemma}

\begin{proof}
  By definition of the modulo operation $c'$ is an integral combination of $c'$ and $B$,
  specifically $c' = c - B \floor{B^{-1} c}$.
  Therefore, $c' ∈ \mathcal L(A)$ and $\mathcal L(A) ⊆ \mathcal L(A \setminus \{c\} ∪ \{c'\})$.
  Similarly, we can reach $c$ by the integral combination $c' + B \floor{B^{-1} c}$,
  so $c ∈ \mathcal L(A \setminus \{c\} ∪ \{c'\})$.
  Therefore applying the modulo operation does not change the lattice.
\end{proof}

With the modulo operation done, we turn our attention to the exchange operation.
On one side we have the basis $B$ and on the other we have the vector $c$.
So a swap of the two inputs will not work.
Instead, we swap $c$ with a single column $B_ℓ$ of the basis $B$,
where the specific index $ℓ$ depends on the result from the modulo operation.
For the modulo we solve a linear system $Bx = c$.
If one element $x_i$ is integral, then $c$ is already contained in the
sub-lattice $\mathcal L(B_i)$.
In the one-dimensional case, this means that $c$ is a multiple of $B_i$,
so it wouldn't make sense to swap with $B_i$.
Instead, we only exchange $c$ with a vector $B_ℓ$, where $x_ℓ$ is not integral
and we swap $B_ℓ$ with the remainder $c \pmod{Π(B)}$.

The exchange operation also leads directly to the stop condition.
The algorithm terminates when all the vector $x$ is entirely integral.
In this case, the vector $c$ is an integral combination of the basis vectors,
so we can leave it out of the basis.

\begin{lemma}
  \label{lem:termination}
  The algorithm terminates in at most $\det(B)$ iterations.
\end{lemma}

\begin{proof}
  By Cramer's rule, $\{x_ℓ\} = \frac{\det B'}{\det B}$
  and therefore the determinant decreases by a factor of $\{x_ℓ\} < 1$.
  Since the vectors in $B$ and $B'$ are integral, the determinant must also be integral.
  Therefore, the determinant decreases by at least one and in total it
  decreases at most $\det(B)$ iterations.
\end{proof}

In summary, the algorithm requires the following steps:
\begin{itemize}
  \item \textbf{Modulo}: $c \pmod{\Pi(B)} = c - B \floor{B^{-1} c}$.
  \item \textbf{Exchange}: $c = B_ℓ$
  \item \textbf{Termination}: $B^{-1} c ∈ ℤ^d$
\end{itemize}
and the previous two lemmas lead to the following theorem:

\begin{theorem}
  The generalized Euclidean algorithm solves the lattice basis reduction problem.
\end{theorem}

This solves the problem of computing a basis for an integer lattice.
The authors proceed to optimize the algorithm to achieve a better running time.
Of those optimization we need one, which removes the need to solve a linear
system in each iteration.

Instead of solving the system completely from new again,
we simply update the solution from the previous iteration.
Suppose $x = (x₁, …, x_d)$ is the solution of the current iteration,
then we calculate the next vector $x' = (x₁', …, x_d')$ using
\begin{align}
  \label{eq:update}
  x_i' =
  \begin{cases}
    \hphantom{-}\frac{1}{\{x_ℓ\}},  & \text{ if } i = ℓ, \\
    -\frac{\{x_i\}}{\{x_ℓ\}}, & \text{ if } i ≠ ℓ,
  \end{cases}
\end{align}
where $\{x_i\} := x_i - \floor{x_i}$.
This update rule follows from the fact that
\[
  B_ℓ \{x_ℓ\} + \sum_{i ≠ ℓ} B_i \{x_i\} = r
  \Leftrightarrow
  r - \sum_{i ≠ ℓ} B_i \{x_i\} = B_ℓ \{x_ℓ\}
  \Leftrightarrow
  r \frac{1}{\{x_ℓ\}} - \sum_{i ≠ ℓ} B_i \frac{\{x_i\}}{\{x_ℓ\}} = B_ℓ.
\]
With this rule,
we only need to solve the linear system at the start of the algorithm.
Afterwards we use the rule to update our solution vector.
Thus, this optimization reduces the running time significantly.

% ==============================================================================
\section{Extension to Real Lattices}
% ==============================================================================

The problem of lattice basis reduction only applies to integer lattices.
If we were to solve the problem for a real lattice,
we would have the problem that the generated lattice is no longer a discrete subgroup of $ℝ^n$.
For example, the lattice generated by the one-dimensional vectors $\sqrt{2}$ and $1$
has vectors which are infinitely close.
Thus, this integral combination cannot be considered as a typical lattice.
Nevertheless,
we can still run the generalized Euclidean algorithm on a real input.
We simply allow a matrix $B ∈ ℝ^{d×d}$ and a real vector $c ∈ ℝ^d$.
We use this for the first part of Hermite's question concerning a
representation of real numbers.

Lemma~\ref{lem:lattice-mod} still holds,
so the vectors are all integral combinations of the initial inputs.
However, the algorithm no longer terminates in $\det(B)$ iterations
as shown in Lemma~\ref{lem:termination}.
% TODO: I don't know if that is the real issue
The issue is that there may not exist any basis for the lattice generated by $B
∈ ℝ^{d×d}$ and $c ∈ ℝ^d$.
Consider the example of $\sqrt{2}$ and $1$ again.
There is simply no real number $r ∈ ℝ$, which solves $zr = 1$ and $zr =
\sqrt{2}$ for some $z ∈ ℤ$ at the same time.
Otherwise, $\sqrt{2}$ would have to be a rational number.
Thus, the algorithm could run indefinitely on a particular real input.

This is no issue for representing real numbers though.
For the representation,
we need to map each real number to a sequence of integers.
But the algorithm accepts a whole linear system,
not just a single number.
Hence, we need a way of mapping a real number to a linear system.
We do this in two steps:
First, we map the number to a solution vector $x$
and then we find a linear system,
which has $x$ as the solution.
The second part is easy,
since we can choose $B = I_d$ and $c = x$.

For the first part, we just have to choose a vector for every real number.
This can be realized as a function which maps any number $r ∈ ℝ$ to a vector $x ∈ ℝ^d$.
In this thesis, we will use the function
\[
  r ↦ (r, r^2, …, r^d).
\]
This function has the advantage that the elements of the vector are linearly independent,
if $r$ is an algebraic number with degree $≥ d$.
However, any function which keeps the elements linearly independent would work.
In general, we can choose polynomials $q_1(x), q_2(x), …, q_d(x)$ with rational
coefficients where $\deg(q_i) = i$ and
use the solution vector $x = (q_1(r), …, q_d(r))$.
But for this thesis, I will use the vector $(r, r^2, …, r^d)$ unless noted otherwise.

% TODO: Rewrite this whole section
% TODO: For the next chapter, we should remark that the entire execution only
% depends on the solution vector, not the original input matrix

% TODO: There should be some explanation on the fact that the solution vector
% kind of represents the ratio between the inputs just like in the Euclidean
% algorithm.

This covers a mapping from a real number to an input for the generalized Euclidean algorithm.
What is missing is the sequence of integers.
In the classical Euclidean algorithm,
we used the ratio $b^{-1}a$ between the two inputs $a, b ∈ ℤ$ to derive the sequence,
which were then used as coefficients in a continued fraction.
At each step, the integer part of $\floor{b^{-1} a}$ determined the specific
coefficient in the representation.
In a sense, we can use the solution vector $x = B^{-1} a$ as a ratio between the inputs.
The integer part $\floor{x}$ now defines a whole vector, which we use as
one part in the integer sequence.
This also gives us a representation of an entire vector not just a single number.
Indeed, if we do not use the mapping, then we can find a representation for any real vector.
Therefore, if $x, x', x'', …, x^{(n)}, …$ are the solution vectors at each iteration of the algorithm,
then we represent the original input vector $x$ as a list of integer vectors
$\left[\floor{x}, \floor{x'}, \floor{x''}, …, \floor{x^{(n)}}, …\right]$.

Since we only need the solution vector,
we can use the update rule from Equation~\ref{eq:update}
to calculate the next solution vector.
This removes the entire lattice from the equation
and we only need to focus on the solution vector itself.
This gives us a way to construct \emph{multidimensional continued fractions}.
In fact, for $d = 1$ the update rule matches the definition of continued fractions.

% TODO: Should we add a citation for Northshield and explain that continued
% fractions map positive to positive values which seems to be a fundamental
% requirement for the continued fractions to be periodic?

% I think a better wording would be, that the update rule makes the negation
% visible, which is not optimal. The update rule itself doesn't negate the
% variables, even without the update rule we would still have negated
% variables, since the update rule is just an improvement of the original
% algorithm.

\iffalse
\begin{figure}[tbp]
  \centering
  \includestandalone{figures/pivot-choice}
  \caption{
    Different choices for the remainder of vector $c$. The original algorithm
    always uses $r$ as the remainder, but the modified update rule would also consider $r'$.}
\end{figure}
\fi

\begin{Pseudocode}[
    float=tbp,
    label={lst:modified-generalized-euclidean},
    caption={
      The modified algorithm, where the solution $x$ remains entirely positive in each iterations.
      The linear system from the original algorithm is replaced by the update rule.
      The two negations ensure that the vectors represent the correct solution,
      when the modified update rule from Equation~\ref{eq:modified-update-rule} is used.
    }]
solve $Bx = c$
while $x$ is not integral do
  find $x_ℓ$ which is not integral
  $c ← B_ℓ$
  $B_ℓ ← -B\{x\}$
  $B ← -B$
  $x ← \mathrm{pivot}_ℓ(x)$
end
\end{Pseudocode}

The update rule would suffice for a representation of the real numbers.
However, this representation has some downsides.
The main problem is that the values inside the solution vector $x$ can become negative.
The rule flips the signs of all elements $x_i$ with $i ≠ ℓ$
and can therefore produce negative values.
For continued fractions, it is crucial that the complete quotients are positive \cite{Northshield11}
and for multidimensional continued fractions, this is equally important.
Without it, the structure of the algorithm becomes unstable and convergence may fail.
To avoid negative values, we slightly modify the rule such that it maps each $xᵢ$ to another positive value.
After we replace $B_ℓ$ with $c$, we flip the signs of all vectors $B_i$ with $i ≠ ℓ$.
This leads to the modified update rule, where the values $x_i$ for $i ≠ ℓ$ are
no longer negated:
\begin{align}
  \label{eq:modified-update-rule}
  x_i' =
  \begin{cases}
    \frac{1}{\{x_ℓ\}},  & \text{ if } i = ℓ, \\
    \frac{\{x_i\}}{\{x_ℓ\}} & \text{ otherwise.}
  \end{cases}
\end{align}
By $\mathrm{pivot}_ℓ(x) = x'$, we denote this modified update rule.
The modified algorithm can be seen in Listing~\ref{lst:modified-generalized-euclidean}.
In the algorithm, first $B_ℓ$ is flipped and then the whole matrix $B$ is flipped,
This is the same as only flipping the vectors $B_i$ for $i ≠ ℓ$.
The difference between the two algorithms is essentially that we propagate the
negation to the vectors $B_i$ instead of the elements $x_i'$.
Hence, $x$ is totally positive after one iteration of the algorithm.
We now have an easy way to represent real numbers,
except that this representation is not unique.

% TODO: We need to explain that this representation is not unique

\begin{example}
  % TODO: We should add an example here somewhere
  Consider $x = (13/5, 8/5)$.
  The vector has at least two different representations.
  It can be represented as either $[(1,1); (1, 0), (1, 0)]$ or $[(1,1); (0, 1), (0, 1)]$,
  since
  \begin{align*}
    (\underline{5/2}, 3/2) → (\underline{1/2}, 1) → (1, 0). \\
    (5/2, \underline{3/2}) → (1, \underline{1/2}) → () \\
    2/5
  \end{align*}
  Hence, the representation for $x$ is not unique.
\end{example}

To find a unique representation, all we need is a deterministic strategy,
that decides which element $x_ℓ$ to choose for pivoting.
For example, one possible strategy could be to choose the element with the
smallest fractional value.

In conclusion,
we represent a vector $x ∈ ℝ^d$ as a sequence of integer vectors by repeatedly
applying the $\mathrm{pivot}$ operation on it either indefinitely or until
$\{x\} = 0$.
The index $ℓ$ is given according to some strategy and at each step we store
$\floor{x}$ as one coefficient in the representation for $x$.
We turn this into a representation for a single number $r ∈ ℝ$ by mapping each
number to the vector $(r^1, r^2, …, r^d)$.
This solves the first part of Hermite's question.
In the next chapters, we will discuss the second part of Hermite's question,
whether the representation is periodic if and only if the number is a cubic
irrational, or algebraic number in general.

\iffalse
% ==============================================================================
\section{Comparison to the Jacobi-Perron Algorithm}
% ==============================================================================

When Jacobi \cite{Jacobi68} initially tried answering Hermite's question,
he developed his own generalization of the Euclidean algorithm.
His version computed the GCD of three numbers in a distinctly different way
than the Euclidean algorithm would.
However, he was unable to answer Hermite's question.
Later, Perron \cite{Perron07} extended Jacobi's algorithm to $n$ numbers for any $n ≥ 2$,
which is now known as the Jacobi-Perron algorithm.
He showed that when the algorithm is periodic, then the input consists of
algebraic numbers with degree $n$.
But even with his generalization, he was unable to show the other direction.

\begin{figure}[tbp]
  \centering
  \includestandalone{figures/jacobi-perron}
  \caption{
    One round of the Jacobi-Perron algorithm.
    The smallest number in the input is $7$,
    so it is used as the modulus.
    In the next iteration, $a_5$ is removed since it has remainder zero.
  }
  \label{fig:jacobi-perron}
\end{figure}

The input to the Jacobi-Perron algorithm is a list of numbers $a₁, …, aₙ$ and
it returns the GCD of these numbers.
The algorithm works in multiple rounds.
In each round, the algorithm takes the smallest number $a_ℓ$ and calculates the
remainder $aᵢ' = aᵢ \bmod a_ℓ$ for every $i ≠ ℓ$.
See Figure~\ref{fig:jacobi-perron} for an example of one round.
In the next round, the algorithm continues with the values $a₁', …, aₙ'$ where $a_ℓ' = a_ℓ$.
If any value $aᵢ$ is zero, then we remove the value from the input and continue
with the remaining numbers.
This process continues until all values have a remainder of zero.
The last modulus is the GCD for all numbers of the original input.

% TODO: Explain this better
The next step is extending the algorithm to real numbers,
which can simply be done by allowing real numbers in the input.
In this case, the numbers no longer reach zero, but approach it.
Therefore, we look at the ratios between the inputs starting with the ratios $x₁ = a₁/1, …, xₙ = aₙ/1$.
Given the ratios $x₁, …, x_d$ for the current iteration,
the ratios in the next iteration can be calculated by
\[
  x_i' = \frac{x_i}{x_ℓ}, \qquad x_ℓ' = \frac{1}{x_ℓ}.
\]
This is identical to the pivot operation!
The only difference is that the Jacobi-Perron prescribes the strategy:
We always take the smallest element in each iteration.

So the Jacobi-Perron algorithm is one possible strategy of the generalized
Euclidean algorithm.
This has the unintended benefit that cubic roots proven to be periodic for the
Jacobi-Perron algorithm~\cite{Bernstein64,Bernstein71} are automatically
periodic for the generalized Euclidean algorithm.
On the other hand,
the generalized Euclidean algorithm is more flexible than the Jacobi-Perron
algorithm, since we can choose different indices in each iteration.
This is particularly advantageous since computations for the Jacobi-Perron
seem to indicate that the algorithm is not periodic for some
cases.
Elsner and Hasse \cite{Elsner67} ran different variations of the algorithm on
the input $x = (\sqrt[3]{4}, \sqrt[3]{16})$ and after 80 iterations, they have
found no period.
\fi
