\chapter{Geometrical Interpretation of Continued Fractions}

\section{Quadratic Fields as a Vector Space}

\begin{itemize}
  \item Quadratic field extensions can be seen as a vector space over $ℚ$ with dimension $2$.
  \item A vector consists of a rational part and a multiplicative of the root part.
  \item Conjugate: Flip the sign before the root
  \item Norm: Given $a + b\sqrt{r}$, the norm is $N(a + b \sqrt{r}) = a^2 - r b^2$.
  \item Related to Pell's Equation
  \item Inverse: Conjugate divided by the norm
  \item Number of vectors is finite when considering the continued fraction,
    but not when considering the Euclidean algorithm
\end{itemize}

\section{Conic Sections}

\begin{itemize}
  \item
    Everything is in some way related to a cone
  \item
    Norm:
    Vertical cross section of a cone
  \item
    Projection from vector space to real numbers:
    Horizontal cross-section of a cone
  \item
    Parabola: Parallel cross-section of a cone
  \item
    In higher dimensions, we also get somewhat of a conic section.
    The problem is that this means the degree increases to $2d$ instead of $d+1$.
\end{itemize}

\section{Klein Polyhedrons}

\begin{itemize}
  \item Considers the convergents of a continued fraction as homogeneous coordinates,
    i.e. when we have $p / q$, then this corresponds to the vector $(p, q)$.
  \item Simplical cone: Given basis $B$, the simplical cone $C$ is defined as
    the set of all positive linear combinations.
  \item
    The odd and even convergents are each contained in two different simplical cones
    and connecting them forms the convex hull of the integer points contained in the cones.
  \item
    There exists a geometrical analogue of Lagrange's Theorem for Klein Polyhedron,
    which states that the following two statements are equivalent:
    \begin{itemize}
      \item
        There exists a non-identity operator $A ∈ \mathrm{SL}_2(ℤ)$ with
        distinct real eigenvalues under which the cone is invariant.
      \item
        The integer lengths of consecutive segments in the Klein polygon are periodic.
    \end{itemize}
  \item The generalization is much more complicated.
\end{itemize}
