\chapter{Hermite's Question}

\begin{problem}[Hermite's Question]
  Is there a representation of the real numbers as a sequence of natural
  numbers such that the sequence is eventually periodic if and only if the real
  number is a cubic irrational?
\end{problem}

More formally: Does there exists a function $f$ which maps any $α ∈ ℝ$ to an
infinite sequence $(a_n)_{n ∈ ℕ}$ of natural numbers such that $a_n$ is
eventually periodic if and only if $α$ is a cubic irrational?
The problem can very easily be generalized to higher degrees of algebraic numbers:
The representation is periodic if and only if $x$ is the root of a polynomial of degree $d + 1$.

\section{Comparison to the Jacobi-Perron Algorithm}

The algorithm in this form, where we always choose the same pivot,
is equivalent to the Jacobi-Perron algorithm, which was first proposed
by Jacobi \cite{Jacobi68} and later analyzed by Perron \cite{Perron07}.
In modern times, the algorithm was picked up again by Bernstein \cite{Bernstein06}.
Each of them has analyzed the algorithm in the context of Hermite's Question.

Charles Hermite initially posed a question to Jacobi, whether there exists a
representation of the real numbers as an infinite sequence of natural numbers
such that the sequence is eventually periodic if and only if the number is a
cubic irrational. The question remains unanswered with the latest attempt
\cite{Murru15} showing that there is an infinite sequence which is eventually
periodic for every cubic irrational.

\section{Finding a Periodic Representation through Brute-Force}

We use the generalized Euclidean algorithm to find a representation for $α ∈ ℝ$.
More specifically, we use the update rule from the algorithm to find the representation.
The update rule takes a vector and not a real number, so we use the vector $x_i = α^i$ for $i ≤ d$.

For our input $x ∈ ℝ^d$, we can build up a $d$-ary tree where $x$ is the root and any
two nodes of this tree $x', x''$ are connected if $\mathrm{pivot}(x, i) = x''$
for some $i$.
The brute-force algorithm performs a breadth-first search over this tree to
find a node which points to a higher node in the tree, i.e. the path forms a loop.
As our pivot sequence we use the path from the root to the this node.

\begin{example}
  The sequences for the prime numbers are:
  \begin{itemize}
    \item $2^{1/3}$: $0\overline{10}$.
    \item $3^{1/3}$: $01\overline{01}$.
    \item $5^{1/3}$: $0\overline{00111000}$.
    \item $7^{1/3}$: $0\overline{010100}$.
    \item $11^{1/3}$: $0\overline{1100}$.
    \item $13^{1/3}$: $00\overline{010000}$.
    \item $17^{1/3}$: $000\overline{11110000}$.
    \item Another choice for $5^{1/3}$ with shorter period: $00110\overline{101010}$.
  \end{itemize}
\end{example}

\begin{remark}
  The first number that has a leading $1$ as the pivot is $12$,
  which has the pivot sequence $1\overline{0111011101}$.
  However, there are other possible choices where it does not have a leading $0$.
\end{remark}

We can likely rule out alternating pivot sequences since this corresponds to
the Jacobi-Perron algorithm and for this algorithm it is conjectured
\cite{Karpenkov2024} that the algorithm is not periodic for $\sqrt[3]{4}$.
However, the brute-force algorithm is periodic for this input (see Table~\ref{table:cube-root-4}).

The choice of the initial input also matters.
We generally choose $x = (α, q(α))$, where $q$ is some polynomial of degree $2$.
But different choices of $q$ produce different sequences.
For example, $(\sqrt[3]{2}, \sqrt[3]{4})$ produces a different sequence of coefficients than $(\sqrt[3]{2}, \sqrt[3]{6})$.
Even though both inputs represent the same number $\sqrt[3]{2}$.
Choosing $q(α) = α^2 - α$ makes the sequence purely periodic for $\sqrt[3]{3}$ and $\sqrt[3]{4}$.

\begin{conjecture}
  The brute-force algorithm is periodic if and only if $α$ is a cubic irrational.
\end{conjecture}

\begin{lemma}
  \label{lem:consecutive-same}
  For every $k ∈ ℕ$, there exists a number $n ∈ ℕ$,
  such that $\sqrt[3]{n}$ and $\sqrt[3]{n + 1}$ have the same first $k$ terms in
  their pivot sequences.
\end{lemma}

\begin{corollary}
  Any strategy which only compares the fractional values does not produce the
  same pivot sequence as the brute-force algorithm.
\end{corollary}

\begin{corollary}
  The minimum strategy does not produce the same pivot sequence as the
  brute-force algorithm.
\end{corollary}

\iffalse
\begin{theorem}
  There exists no algorithm which given $x ∈ ℝ^d$ can determine the same
  strategy as the brute-force search using only $f(d)$ comparison, addition
  and subtraction operations, for some computable function $f \colon ℕ → ℕ$.
\end{theorem}

\begin{proof}
  Idea: A constant-number of additions and subtraction only compare the integer
  parts of each $x_i$ when the values are chosen close enough to an integer.
  Because the trees are the same for the first level, the output for both
  inputs must be the same.

  We choose $x₁ = f(d)$ and $x₂ = \sqrt{f(d) + 1}$ where $n^3 > f(d)$.
  In particular, $x₁$ is an integer and $x₂$ is between $x₁$ and the next integer.
  Suppose the algorithm compares $a₁ x₁$ with $a₂ x₂$ for $|a₁|, |a₂| ≤ f(d)$.
  \[
    a₂ x₂
    = a₂ \sqrt{(f(d))^2 + i + 1}
    ≤ a₂ \left( f(d) + \frac{1}{3 f(d)} \right)
    = a₂ · f(d) + \frac{a₂}{3 f(d)} \right)
  \]
  Furthermore, $a₂ x₂ ≥ a₂ · f(d)$.
  Therefore, the integer part of $a₂ x₂$ is equal to $a₂ x₁$,
  even if $a₂ = f(d)$.
  Hence, any algorithm can only compare the integer parts of $x₁$ and $x₂$.
\end{proof}
\fi

\begin{table}[t]
  \caption{Representation of $ψ = \sqrt[3]{4}$ using the brute-force search.}
  \label{table:cube-root-4}
  \centering
  \begin{tabular}{lllllll}
  \uzlhline
  \uzlemph{$\ell$} & \uzlemph{$x_1$} & \uzlemph{$x_2$} & \uzlemph{$x_1$} & \uzlemph{$x_2$} & \uzlemph{$a_1$} & \uzlemph{$a_2$} \\
	\hline
  $0$ & $\psi$ & $\psi^{2}$ & $1.5874$ & $2.51984$ & $1$ & $2$ \\
  $0$ & $\frac{1}{3} \psi^{2} + \frac{1}{3} \psi + \frac{1}{3}$ & $-\frac{1}{3} \psi^{2} + \frac{2}{3} \psi + \frac{2}{3}$ & $1.70241$ & $0.88499$ & $1$ & $0$ \\
  $1$ & $\frac{1}{4} \psi^{2} + \frac{1}{2} \psi$ & $\frac{1}{2} \psi^{2}$ & $1.42366$ & $1.25992$ & $1$ & $1$ \\
  $0$ & $\frac{1}{4} \psi^{2} + 1$ & $\frac{1}{2} \psi^{2} + \psi + 1$ & $1.62996$ & $3.84732$ & $1$ & $3$ \\
  $0$ & $\psi$ & $\psi^{2} - 2 \psi + 2$ & $1.5874$ & $1.34504$ & $1$ & $1$ \\
  $1$ & $\frac{1}{3} \psi^{2} + \frac{1}{3} \psi + \frac{1}{3}$ & $\psi - 1$ & $1.70241$ & $0.5874$ & $1$ & $0$ \\
  $1$ & $\frac{1}{3} \psi + \frac{2}{3}$ & $\frac{1}{3} \psi^{2} + \frac{1}{3} \psi + \frac{1}{3}$ & $1.1958$ & $1.70241$ & $1$ & $1$ \\
  $0$ & $\frac{1}{12} \psi^{2} - \frac{1}{6} \psi + \frac{1}{3}$ & $\frac{1}{4} \psi^{2} + \frac{1}{2} \psi$ & $0.27875$ & $1.42366$ & $0$ & $1$ \\
  $0$ & $\psi + 2$ & $\psi^{2} - 1$ & $3.5874$ & $1.51984$ & $3$ & $1$ \\
  $0$ & $\frac{1}{3} \psi^{2} + \frac{1}{3} \psi + \frac{1}{3}$ & $-\frac{1}{3} \psi^{2} + \frac{2}{3} \psi + \frac{2}{3}$ & $1.70241$ & $0.88499$ & $1$ & $0$ \\
  \uzlhline
\end{tabular}

\end{table}

\begin{table}[t]
  \caption{Period Length of the first $28$ numbers.}
  \centering
  \begin{tabular}{ll}
  \uzlhline
  \uzlemph{$n^3$} & \uzlemph{Period Length of $n^3$} \\
  \hline
  2 & 2 \\
  3 & 2 \\
  4 & 8 \\
  5 & 8 \\
  6 & 8 \\
  7 & 6 \\
  9 & 2 \\
  10 & 4 \\
  11 & 4 \\
  12 & 10 \\
  13 & 6 \\
  14 & 4 \\
  15 & 6 \\
  16 & 14 \\
  17 & 8 \\
  18 & 6 \\
  19 & 6 \\
  20 & 8 \\
  21 & 8 \\
  22 & 6 \\
  23 & 20 \\
  24 & 8 \\
  25 & 20 \\
  26 & 4 \\
  28 & 2 \\
  \uzlhline
\end{tabular}

\end{table}

\section{Multi-Dimensional Continued Fractions}

\begin{definition}
  Given a sequence of $d$-dimensional vectors $\{rₙ\}_{n ≥ 0}$ and a sequence of
  indices $\{ℓₙ\}_{n ≥ 0}$, the \emph{multi-dimensional fraction} $[r₀; ℓ₁, r₁; …]$ is defined as
  \[
    [r₀] = r₀, \qquad [r₀; ℓ₁, r₁; …; ℓₙ, rₙ] = r₀ + \mathrm{pivot}_{ℓ₁}[r₁; ℓ₂, r₂; …; ℓₙ, rₙ].
  \]
\end{definition}

\begin{remark}~
  \begin{enumerate}
    \item
      The first item has no index.
      The index sequence $ℓₙ$ is always one shorter than the vector sequence $rₙ$.
    \item
      For the generalized Euclidean algorithm, the sequence $r_n$ consists solely of integer vectors.
      However, the MDCF can also be defined over rational or even real vectors.
      To differentiate the two, a sequence $rₙ$ will denote a sequence of real
      vectors and $aₙ$ will denote a sequence of integer vectors.
  \end{enumerate}
\end{remark}

The concepts from one-dimensional continued fractions naturally carry over to its
multi-dimensional counterpart.

\begin{definition}[Complete Quotient, Convergent, Periodicity]~
  \begin{itemize}
    \item The \emph{$k$-th complete quotient} is defined as $[rₖ; r_{k+1}, …]$.
    \item The \emph{$k$-th convergent} of $x$ is defined as the finite fraction $[r₀; ℓ₁, r₁; …; ℓ_k, r_k]$.
    \item The MDCF is \emph{eventually periodic} if there exists $k₀ ≥ 0$ and $ℓ ≥ 1$ such that $rₖ = r_{k+ℓ}$ for every $k ≥ k₀$.
      The MDCF is \emph{purely periodic} if $k₀ = 0$.
  \end{itemize}
\end{definition}

To show that any periodic MDCF contains elements of a field with degree $d + 1$
over the rationals, we proceed similarly to the one-dimensional continued
fraction.
We first show the multi-dimensional analogue of Lemma~\ref{lem:nesting}.

\begin{lemma}[Nesting]
  \label{lem:mdcf-nesting}
  Let $x ∈ ℝ^d$, then
  \[
    [a₀; ℓ₁, a₁; …; ℓₙ, aₙ; ℓ, x]
    = [a₀; ℓ₁, a₁; \cdots; ℓₙ, aₙ + \mathrm{pivot}_{ℓ}(x)]
  \]
\end{lemma}

\begin{proof}
  If $n = 0$, then by definition,
  \[
    [a₀; ℓ, x] = a₀ + \mathrm{pivot}_{ℓ}[x] = a₀ + \mathrm{pivot}_{ℓ}(x) = [a₀ + \mathrm{pivot}_ℓ(x)].
  \]
  Suppose the lemma holds for any $n ≥ 0$, then
  \begin{align*}
    [a₀; ℓ₁, a₁; …; ℓₙ, aₙ; ℓ, x]
    & = a₀ + \mathrm{pivot}_{ℓ₀}[a₁; …; ℓₙ, aₙ; ℓ, x] \\
    & = a₀ + \mathrm{pivot}_{ℓ₀}[a₁; …; ℓₙ, aₙ + \mathrm{pivot}_ℓ(x)] \\
    & = [a₀; ℓ₁, a₁; …; ℓₙ, aₙ + \mathrm{pivot}_ℓ(x)]. \qedhere
  \end{align*}
\end{proof}

In Lemma~\vref{lem:wallis}, we had to define two sequences $pₙ$ and $qₙ$ over
the sequence of the continued fraction $aₙ$.
Each sequence would only return a single scalar.
This time we define two matrix sequences $P^{(n)}, Q^{(n)}$ over the sequences $aₙ$ and $ℓₙ$ as
\begin{align*}
  P_{ℓₙ}^{(n)} & = P_{n-1} a_n + p_{n-1}, & P_i^{(n)} & = P_i^{(n)}, & P^{(-1)}   & = I_d, \\
  Q_{ℓₙ}^{(n)} & = Q_{n-1}^T a_n + q_{n-1}, & Q_i^{(n)} & = Q_i^{(n)}, & Q^{(-1)}_j & = 0,   \\
  p^{(n)}      & = P_{ℓₙ}^{(n-1)},            &           &              & p^{(-1)}_j & = 0,   \\
  p^{(n)}      & = P_{ℓₙ}^{(n-1)},            &           &              & q^{(-1)}   & = 1.
\end{align*}
where $i ≠ ℓ_n$.
What this sequence effectively does is reconstructing the lattice from an
initial solution vector $x ∈ ℝ^d$ and its coefficient vectors $a_n$.

\begin{lemma}[Wallis]
  \label{lem:mdcf-wallis}
  Let $x ∈ ℝ^d$, then
  \[
    [a₀; ℓ₁, a₁; …; ℓ_{n-1}, a_{n-1}; ℓ, x]
    = \left(P^{(n-1)} x + p^{(n-1)}\right) \oslash \left(Q^{(n-1)} x + q^{(n-1)} \right),
  \]
  where $u \oslash v$ denotes the element-wise division of two vectors.
\end{lemma}

\begin{proof}
  If $n = 0$, then
  \[
    [x] = x = (I_d x + 0) \oslash (0 x + 1).
  \]
  Suppose the lemma holds for $n ≥ 0$, then there exists matrices $P$ and $Q$
  as well as vectors $p$ and $q$ such that
  \begin{align*}
    y & = [a₀; ℓ₁, a₁; …; ℓ_{n-1}, a_{n-1}; ℓ, x]                              \\
      & = [a₀; ℓ₁, a₁; …; ℓ_{n-1}, a_{n-1} + \mathrm{pivot}_ℓ(x)]              \\
      & = (P (a + \mathrm{pivot}(x, ℓ)) + p) \oslash (Q (aₙ + \mathrm{pivot}(x, ℓ)) + q) \\
  \end{align*}
  We extend each fraction in the vector $y$ by $x_ℓ/x_ℓ$.
  The numerator has the following form
  \begin{align*}
    x_ℓ (P (a + \mathrm{pivot}(x, ℓ)) + p)
    & = x_ℓ (P a + P_ℓ \frac{1}{x_ℓ} + \sum_{i ≠ ℓ} P_i \frac{x_i}{x_ℓ} + p) \\
    & = \underbrace{(P a + p)}_{P_ℓ'} x_ℓ + \sum_{i ≠ ℓ} \underbrace{P_i}_{P_i'} x_i + \underbrace{P_ℓ}_{p'} \\
    & = P' x + p'.
  \end{align*}
  The proof for the denominator is analogous.
  Hence,
  \begin{align*}
    y & = (P(a + \mathrm{pivot}(x, ℓ)) + p) \oslash (Q(a + \mathrm{pivot}(x, ℓ)) + q) \\
      & = (P' x + p') \oslash (Q' x + q'). \qedhere
  \end{align*}
\end{proof}

\begin{proposition}
  Given $r ∈ ℝ$, let $x = (r¹, r², …, r^d)$.
  If the MDCF of $x$ is purely periodic, then $[ℚ(r) : ℚ] ≤ d + 1$.
\end{proposition}

\begin{proof}
  Suppose the algorithm is purely periodic on input $x$ with period $ℓ$.
  Let $y$ be the $ℓ$-th complete quotient of $x$, then $x = y$.
  By Lemma~\ref{lem:mdcf-wallis}, there are matrices $P, Q$ and vectors $p, q$ such that
  \[
    rⁱ = \frac{\sum_{j=1}^d P_{ij} rʲ + pᵢ}{\sum_{j=1}^d Q_{ij} rʲ + qᵢ}, \text{ for every } i ≤ d.
  \]
  Multiplying both sides with the denominator results in the polynomial equation
  \[
    \sum_{j=1}^d (Q_{ij} r^{i+j} - P_{ij} r^j) + r^i q_i - p_i = 0.
  \]
  For $i = 1$, the maximum degree of this polynomial is $d + 1$.
  Hence, $[ℚ(r) : ℚ] ≤ d + 1$.
\end{proof}

\begin{lemma}
  \todo[inline]{Some lemma which simplifies the system down to degree $d$.}
\end{lemma}

\begin{theorem}
  Given $r ∈ ℝ$, let $xᵢ = r^i$ for every $i ≤ d$.
  If the brute-force algorithm is eventually periodic for $x = (x₁, …, x_d)$,
  then $[ℚ(r) : ℚ] ≤ d + 1$.
\end{theorem}

\begin{proof}
  \[
    \frac{\sum_{j=1}^d P_{ij} f_j(r) + pᵢ}{\sum_{j=1}^d Q_{ij} f_j(r) + qᵢ}
    = \frac{\sum_{j=1}^d P_{ij} f_j(r) + pᵢ}{\sum_{j=1}^d Q_{ij} f_j(r) + qᵢ}
  \]
\end{proof}
