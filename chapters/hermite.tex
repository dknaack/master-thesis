\chapter{Notes on Hermite's Question}

\begin{problem}[Hermite's Question]
  Is there a representation of the real numbers as a sequence of integers such
  that the sequence is eventually periodic if and only if the real number is a
  cubic irrational?
\end{problem}

More formally: Does there exists a function $f$ which maps any $α ∈ ℝ$ to an
infinite sequence $(a_n)_{n ∈ ℕ}$ of natural numbers such that $a_n$ is
eventually periodic if and only if $x$ is a cubic irrational?

For any $d$th irrational number $x ∈ ℝ$, we can find a unique irreducible polynomial $p(x) ∈ ℚ[x]$
which has $x$ as its root and is monic.
Furthermore, it must have $d$ distinct roots $α₁, α₂, \dots, α_d ∈ ℂ$,
where one of them is $x$.

\section{$\sin^2$ Algorithm}

Let $p$ be an irreducible cubic polynomial over $ℚ$ with integer coefficients.
The roots $α, β, γ$ are called \emph{conjugates}.

Given a linear basis $q₀, q₁, q₂$ over degree $2$ polynomials over $ℚ$,
the vectors
\[
  \begin{pmatrix}
    q₀(α) \\
    q₁(α) \\
    q₂(α) \\
  \end{pmatrix},
  \begin{pmatrix}
    q₀(β) \\
    q₁(β) \\
    q₂(β) \\
  \end{pmatrix},
  \begin{pmatrix}
    q₀(γ) \\
    q₁(γ) \\
    q₂(γ) \\
  \end{pmatrix}
\]
are called cubic conjugate vectors.

A triple of cubic conjugate vectors $a, b, c$ is called totally-real, if there
exists a matrix $B ∈ \text{SL}(3, ℤ)$ with irreducible characteristic
polynomial over $ℚ$ such that $a, b, c$ are eigenvectors of $B$ with distinct
eigenvalues.

\section{HAPD algorithm}

See \cite{Karpenkov2024}.
Given a monic irreducible polynomial $p(x)$ over $ℚ$ with integer coefficients and roots $α₁, \dots, α_d ∈ \overline{ℚ}$,
we can find algebraic vectors $v_1, \dots, v_d$ for each root $α_i$:
\[
  v_i = (q_0(α_i), q_1(α_i), q_2(α_i), \dots, q_{d-1}(α_i)),
\]
where the polynomials $q_i$ must have degree $i$, respectively.
The choice of $q_i$ does not matter as long as they have degree $i$.
Furthermore, we can find a rational matrix $C(p) ∈ ℚ^{d × d}$ such that $v_1, \dots, v_d$ are the eigenvectors of $M$.
Let $p(x) = x^d + a_{d-1} x^{d-1} + \dots + a₁ x^1 + a₀$, then
\[
  C(p) =
  \begin{pmatrix}
    0 & 0 & \dots & 0 & -a₀ \\
    1 & 0 & \dots & 0 & -a₁ \\
    0 & 1 & \dots & 0 & -a₂ \\
    \vdots & \vdots & \ddots & \vdots & \vdots \\
    0 & 0 & \dots & 1 & -a_{d-1} \\
  \end{pmatrix},
\]
By multiplying all elements of this matrix with the least common multiple of the denominators in $C(p)$,
we can turn it into an integer matrix.
The matrix $C(p)$ is also called the \emph{companion matrix} of $p$.

The matrix naturally gives rise to a linear recurrence of the form
\[
  F(n + d) = -a_{d-1} F(n + d - 1) - \dots - a_1 F(n + 1) - a_0 F(n).
\]
For example, when $d = 2$ and $a_0 = a_1 = -1$, the linear recurrence
corresponds to the Fibonacci sequence.

\begin{definition}
  The \emph{Markov-Davenport characteristic} of a vector $x ∈ ℝ^3$ with respect to vectors $u, v, w ∈ ℝ^3$
  is defined as
  \[
    \det
    \begin{pmatrix}
      x_1 & v_1 & w_1 \\
      x_2 & v_2 & w_2 \\
      x_3 & v_3 & w_3 \\
    \end{pmatrix}
    · \det
    \begin{pmatrix}
      u_1 & x_1 & w_1 \\
      u_2 & x_2 & w_2 \\
      u_3 & x_3 & w_3 \\
    \end{pmatrix}
    · \det
    \begin{pmatrix}
      u_1 & v_1 & x_1 \\
      u_2 & v_2 & x_2 \\
      u_3 & v_3 & x_3 \\
    \end{pmatrix}
  \]
\end{definition}

\section{Brute-Force Search}

\begin{table}[t]
  \caption{Representation of $\sqrt[3]{4}$ using the brute-force search.}
  \centering
  \begin{tabular}{lllllll}
  \uzlhline
  \uzlemph{$\ell$} & \uzlemph{$x_1$} & \uzlemph{$x_2$} & \uzlemph{$x_1$} & \uzlemph{$x_2$} & \uzlemph{$a_1$} & \uzlemph{$a_2$} \\
  \hline
  $0$ & $\psi$ & $\psi^{2}$ & $1.5874$ & $2.51984$ & $0$ & $1$ \\
  \hline
  \hline
  $0$ & $\frac{1}{4} \psi^{2}$ & $\psi - 1$ & $0.62996$ & $0.5874$ & $1$ & $0$ \\
  $0$ & $\psi - 1$ & $\psi^{2} - \psi$ & $0.5874$ & $0.93244$ & $1$ & $1$ \\
  $1$ & $\frac{1}{3} \psi^{2} + \frac{1}{3} \psi - \frac{2}{3}$ & $\psi - 1$ & $0.70241$ & $0.5874$ & $1$ & $1$ \\
  $0$ & $\frac{1}{3} \psi - \frac{1}{3}$ & $\frac{1}{3} \psi^{2} + \frac{1}{3} \psi - \frac{2}{3}$ & $0.1958$ & $0.70241$ & $5$ & $3$ \\
  $1$ & $\psi^{2} + \psi - 4$ & $\psi - 1$ & $0.10724$ & $0.5874$ & $0$ & $1$ \\
  $1$ & $-\frac{2}{3} \psi^{2} + \frac{1}{3} \psi + \frac{4}{3}$ & $\frac{1}{3} \psi^{2} + \frac{1}{3} \psi - \frac{2}{3}$ & $0.18257$ & $0.70241$ & $0$ & $1$ \\
  $1$ & $\frac{1}{2} \psi^{2} - 1$ & $\frac{1}{4} \psi^{2} + \frac{1}{2} \psi - 1$ & $0.25992$ & $0.42366$ & $0$ & $2$ \\
  $0$ & $-\frac{1}{5} \psi^{2} + \frac{1}{5} \psi + \frac{4}{5}$ & $\frac{2}{5} \psi^{2} + \frac{3}{5} \psi - \frac{8}{5}$ & $0.61351$ & $0.36038$ & $1$ & $0$ \\
  \uzlhline
\end{tabular}

\end{table}
