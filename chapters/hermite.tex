\chapter{Hermite's Question}

\begin{problem}[Hermite's Question]
  Is there a representation of the real numbers as a sequence of natural
  numbers such that the sequence is eventually periodic if and only if the real
  number is a cubic irrational?
\end{problem}

More formally: Does there exists a function $f$ which maps any $α ∈ ℝ$ to an
infinite sequence $(a_n)_{n ∈ ℕ}$ of natural numbers such that $a_n$ is
eventually periodic if and only if $α$ is a cubic irrational?
The problem can very easily be generalized to higher degrees of algebraic numbers:
The representation is periodic if and only if $x$ is the root of a polynomial of degree $d + 1$.

\section{Comparison to the Jacobi-Perron Algorithm}

The algorithm in this form, where we always choose the same pivot,
is equivalent to the Jacobi-Perron algorithm, which was first proposed
by Jacobi \cite{Jacobi68} and later analyzed by Perron \cite{Perron07}.
In modern times, the algorithm was picked up again by Bernstein \cite{Bernstein06}.
Each of them has analyzed the algorithm in the context of Hermite's Question.

Charles Hermite initially posed a question to Jacobi, whether there exists a
representation of the real numbers as an infinite sequence of natural numbers
such that the sequence is eventually periodic if and only if the number is a
cubic irrational. The question remains unanswered with the latest attempt
\cite{Murru15} showing that there is an infinite sequence which is eventually
periodic for every cubic irrational.

\section{Finding a Periodic Representation through Brute-Force}

We use the generalized Euclidean algorithm to find a representation for $α ∈ ℝ$.
More specifically, we use the update rule from the algorithm to find the representation.
The update rule takes a vector and not a real number, so we use the vector $x_i = α^i$ for $i ≤ d$.

For our input $x ∈ ℝ^d$, we can build up a $d$-ary tree where $x$ is the root and any
two nodes of this tree $x', x''$ are connected if $\mathrm{pivot}(x, i) = x''$
for some $i$.
The brute-force algorithm performs a breadth-first search over this tree to
find a node which points to a higher node in the tree, i.e. the path forms a loop.
As our pivot sequence we use the path from the root to the this node.

\begin{example}
  The sequences for the prime numbers are:
  \begin{itemize}
    \item $2^{1/3}$: $0\overline{10}$.
    \item $3^{1/3}$: $01\overline{01}$.
    \item $5^{1/3}$: $0\overline{00111000}$.
    \item $7^{1/3}$: $0\overline{010100}$.
    \item $11^{1/3}$: $0\overline{1100}$.
    \item $13^{1/3}$: $00\overline{010000}$.
    \item $17^{1/3}$: $000\overline{11110000}$.
    \item Another choice for $5^{1/3}$ with shorter period: $00110\overline{101010}$.
  \end{itemize}
\end{example}

\begin{remark}
  The first number that has a leading $1$ as the pivot is $12$,
  which has the pivot sequence $1\overline{0111011101}$.
  However, there are other possible choices where it does not have a leading $0$.
\end{remark}

We can likely rule out alternating pivot sequences since this corresponds to
the Jacobi-Perron algorithm and for this algorithm it is conjectured
\cite{Karpenkov2024} that the algorithm is not periodic for $\sqrt[3]{4}$.
However, the brute-force algorithm is periodic for this input (see Table~\ref{table:cube-root-4}).

The choice of the initial input also matters.
We generally choose $x = (α, q(α))$, where $q$ is some polynomial of degree $2$.
But different choices of $q$ produce different sequences.
For example, $(\sqrt[3]{2}, \sqrt[3]{4})$ produces a different sequence of coefficients than $(\sqrt[3]{2}, \sqrt[3]{6})$.
Even though both inputs represent the same number $\sqrt[3]{2}$.
Choosing $q(α) = α^2 - α$ makes the sequence purely periodic for $\sqrt[3]{3}$ and $\sqrt[3]{4}$.

\begin{conjecture}
  The brute-force algorithm is periodic if and only if $α$ is a cubic irrational.
\end{conjecture}

\begin{lemma}
  \label{lem:consecutive-same}
  For every $k ∈ ℕ$, there exists a number $n ∈ ℕ$,
  such that $\sqrt[3]{n}$ and $\sqrt[3]{n + 1}$ have the same first $k$ terms in
  their pivot sequences.
\end{lemma}

\begin{corollary}
  Any strategy which only compares the fractional values does not produce the
  same pivot sequence as the brute-force algorithm.
\end{corollary}

\begin{corollary}
  The minimum strategy does not produce the same pivot sequence as the
  brute-force algorithm.
\end{corollary}

\iffalse
\begin{theorem}
  There exists no algorithm which given $x ∈ ℝ^d$ can determine the same
  strategy as the brute-force search using only $f(d)$ comparison, addition
  and subtraction operations, for some computable function $f \colon ℕ → ℕ$.
\end{theorem}

\begin{proof}
  Idea: A constant-number of additions and subtraction only compare the integer
  parts of each $x_i$ when the values are chosen close enough to an integer.
  Because the trees are the same for the first level, the output for both
  inputs must be the same.

  We choose $x₁ = f(d)$ and $x₂ = \sqrt{f(d) + 1}$ where $n^3 > f(d)$.
  In particular, $x₁$ is an integer and $x₂$ is between $x₁$ and the next integer.
  Suppose the algorithm compares $a₁ x₁$ with $a₂ x₂$ for $|a₁|, |a₂| ≤ f(d)$.
  \[
    a₂ x₂
    = a₂ \sqrt{(f(d))^2 + i + 1}
    ≤ a₂ \left( f(d) + \frac{1}{3 f(d)} \right)
    = a₂ · f(d) + \frac{a₂}{3 f(d)} \right)
  \]
  Furthermore, $a₂ x₂ ≥ a₂ · f(d)$.
  Therefore, the integer part of $a₂ x₂$ is equal to $a₂ x₁$,
  even if $a₂ = f(d)$.
  Hence, any algorithm can only compare the integer parts of $x₁$ and $x₂$.
\end{proof}
\fi

\begin{table}[t]
  \caption{Representation of $ψ = \sqrt[3]{4}$ using the brute-force search.}
  \label{table:cube-root-4}
  \centering
  \begin{tabular}{lllllll}
  \uzlhline
  \uzlemph{$\ell$} & \uzlemph{$x_1$} & \uzlemph{$x_2$} & \uzlemph{$x_1$} & \uzlemph{$x_2$} & \uzlemph{$a_1$} & \uzlemph{$a_2$} \\
  \hline
  $0$ & $\psi$ & $\psi^{2}$ & $1.5874$ & $2.51984$ & $0$ & $1$ \\
  \hline
  \hline
  $0$ & $\frac{1}{4} \psi^{2}$ & $\psi - 1$ & $0.62996$ & $0.5874$ & $1$ & $0$ \\
  $0$ & $\psi - 1$ & $\psi^{2} - \psi$ & $0.5874$ & $0.93244$ & $1$ & $1$ \\
  $1$ & $\frac{1}{3} \psi^{2} + \frac{1}{3} \psi - \frac{2}{3}$ & $\psi - 1$ & $0.70241$ & $0.5874$ & $1$ & $1$ \\
  $0$ & $\frac{1}{3} \psi - \frac{1}{3}$ & $\frac{1}{3} \psi^{2} + \frac{1}{3} \psi - \frac{2}{3}$ & $0.1958$ & $0.70241$ & $5$ & $3$ \\
  $1$ & $\psi^{2} + \psi - 4$ & $\psi - 1$ & $0.10724$ & $0.5874$ & $0$ & $1$ \\
  $1$ & $-\frac{2}{3} \psi^{2} + \frac{1}{3} \psi + \frac{4}{3}$ & $\frac{1}{3} \psi^{2} + \frac{1}{3} \psi - \frac{2}{3}$ & $0.18257$ & $0.70241$ & $0$ & $1$ \\
  $1$ & $\frac{1}{2} \psi^{2} - 1$ & $\frac{1}{4} \psi^{2} + \frac{1}{2} \psi - 1$ & $0.25992$ & $0.42366$ & $0$ & $2$ \\
  $0$ & $-\frac{1}{5} \psi^{2} + \frac{1}{5} \psi + \frac{4}{5}$ & $\frac{2}{5} \psi^{2} + \frac{3}{5} \psi - \frac{8}{5}$ & $0.61351$ & $0.36038$ & $1$ & $0$ \\
  \uzlhline
\end{tabular}

\end{table}

\begin{table}[t]
  \caption{Period Length of the first $28$ numbers.}
  \centering
  \begin{tabular}{ll}
  \uzlhline
  \uzlemph{$n^3$} & \uzlemph{Period Length of $n^3$} \\
  \hline
  2 & 2 \\
  3 & 2 \\
  4 & 8 \\
  5 & 8 \\
  6 & 8 \\
  7 & 6 \\
  9 & 2 \\
  10 & 4 \\
  11 & 4 \\
  12 & 10 \\
  13 & 6 \\
  14 & 4 \\
  15 & 6 \\
  16 & 14 \\
  17 & 8 \\
  18 & 6 \\
  19 & 6 \\
  20 & 8 \\
  21 & 8 \\
  22 & 6 \\
  23 & 20 \\
  24 & 8 \\
  25 & 20 \\
  26 & 4 \\
  28 & 2 \\
  \uzlhline
\end{tabular}

\end{table}

\section{Notes on Algebraic Numbers}

For any $d$th irrational number $x ∈ ℝ$, we can find a unique irreducible polynomial $p(x) ∈ ℚ[x]$
which has $x$ as its root and is \emph{monic}.
Furthermore, it must have $d$ distinct roots $α₁, α₂, \dots, α_d ∈ ℂ$,
where one of them is $x$.

\section{Notes on the $\sin^2$ Algorithm}

Let $p$ be an irreducible cubic polynomial over $ℚ$ with integer coefficients.
The roots $α, β, γ$ are called \emph{conjugates}.

Given a linear basis $q₀, q₁, q₂$ over degree $2$ polynomials over $ℚ$,
the vectors
\[
  \begin{pmatrix}
    q₀(α) \\
    q₁(α) \\
    q₂(α) \\
  \end{pmatrix},
  \begin{pmatrix}
    q₀(β) \\
    q₁(β) \\
    q₂(β) \\
  \end{pmatrix},
  \begin{pmatrix}
    q₀(γ) \\
    q₁(γ) \\
    q₂(γ) \\
  \end{pmatrix}
\]
are called cubic conjugate vectors.

A triple of cubic conjugate vectors $a, b, c$ is called totally-real, if there
exists a matrix $B ∈ \text{SL}(3, ℤ)$ with irreducible characteristic
polynomial over $ℚ$ such that $a, b, c$ are eigenvectors of $B$ with distinct
eigenvalues.

\section{Notes on the HAPD algorithm}

See \cite{Karpenkov2024}.
Given a monic irreducible polynomial $p(x)$ over $ℚ$ with integer coefficients and roots $α₁, \dots, α_d ∈ \overline{ℚ}$,
we can find algebraic vectors $v_1, \dots, v_d$ for each root $α_i$:
\[
  v_i = (q_0(α_i), q_1(α_i), q_2(α_i), \dots, q_{d-1}(α_i)),
\]
where the polynomials $q_i$ must have degree $i$, respectively.
The choice of $q_i$ does not matter as long as they have degree $i$.
Furthermore, we can find a rational matrix $C(p) ∈ ℚ^{d × d}$ such that $v_1, \dots, v_d$ are the eigenvectors of $M$.
Let $p(x) = x^d + a_{d-1} x^{d-1} + \dots + a₁ x^1 + a₀$, then
\[
  C(p) =
  \begin{pmatrix}
    0 & 0 & \dots & 0 & -a₀ \\
    1 & 0 & \dots & 0 & -a₁ \\
    0 & 1 & \dots & 0 & -a₂ \\
    \vdots & \vdots & \ddots & \vdots & \vdots \\
    0 & 0 & \dots & 1 & -a_{d-1} \\
  \end{pmatrix},
\]
By multiplying all elements of this matrix with the least common multiple of the denominators in $C(p)$,
we can turn it into an integer matrix.
The matrix $C(p)$ is also called the \emph{companion matrix} of $p$.

The matrix naturally gives rise to a linear recurrence of the form
\[
  F(n + d) = -a_{d-1} F(n + d - 1) - \dots - a_1 F(n + 1) - a_0 F(n).
\]
For example, when $d = 2$ and $a_0 = a_1 = -1$, the linear recurrence
corresponds to the Fibonacci sequence.

\begin{definition}
  The \emph{Markov-Davenport characteristic} of a vector $x ∈ ℝ^3$ with respect to vectors $u, v, w ∈ ℝ^3$
  is defined as
  \[
    \det
    \begin{pmatrix}
      x_1 & v_1 & w_1 \\
      x_2 & v_2 & w_2 \\
      x_3 & v_3 & w_3 \\
    \end{pmatrix}
    · \det
    \begin{pmatrix}
      u_1 & x_1 & w_1 \\
      u_2 & x_2 & w_2 \\
      u_3 & x_3 & w_3 \\
    \end{pmatrix}
    · \det
    \begin{pmatrix}
      u_1 & v_1 & x_1 \\
      u_2 & v_2 & x_2 \\
      u_3 & v_3 & x_3 \\
    \end{pmatrix}
  \]
\end{definition}
