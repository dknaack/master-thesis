\chapter{Continued Fractions}
\label{ch:quadratic}

% TODO: Find a source for this chapter
We begin with Hermite's question for quadratic irrationals.
A \emph{quadratic irrational} is any real number
that is a root of a polynomial with degree $2$.
The main goal of this chapter is to show that the continued fraction expansion
of a number is periodic if and only if the number is a quadratic irrational.
Traditionally, this result is proven using the algebraic properties of the
continued fraction and the minimal polynomial of the quadratic irrational.
In contrast, this chapter takes a geometric approach based on Klein polygons.
These polygons visualize the continued fractions as points in a two-dimensional
integer lattice and reduce the proof to a simple geometric argument based on unimodular matrices.

% ==============================================================================
\section{Definition and Basic Properties}
\label{sec:cf-def}
% ==============================================================================

Although continued fractions are typically defined over a sequence of
positive integers, we extend the definition to allow for real numbers.
This will be useful for the theorems which follow.
A continued fraction will be denoted more compactly as $[a₀; a₁, …]$.

\begin{definition}
  \label{def:cont-frac}
  Given a sequence $(aₙ)_{n≥0}$ of real numbers, the finite continued
  fractions over this sequence are defined inductively as
  \[
    [a₀] = a₀, \qquad
    [a₀; a₁, …, aₙ] = a₀ + \frac{1}{[a₁; a₂, …, aₙ]}.
  \]
  The infinite continued fraction $[a₀; a₁, a₂, …]$ is then defined if the limit
  \[
    r = \lim_{n → ∞} [a₀; a₁, …, aₙ]
  \]
  exists.
\end{definition}

Using the definition,
we always have to begin at the first coefficient
and calculate the continued fraction of the tail.
However, this works in reverse, too.
We can begin at the end of the continued fraction
and merge the last two coefficients using the same formula as in the
definition.

\begin{lemma}
  \label{lem:cf-nesting}
  Let $a₀, a₁, …, aₙ, x ∈ ℝ$, then
  \[
    [a₀; a₁, …, aₙ, x] = [a₀; a₁, …, a_n + 1/x]
  \]
\end{lemma}

\begin{proof}
  If $n = 0$, then
  \[
    [a₀; x] = a₀ + \frac{1}{[x]} = a₀ + \frac{1}{x} = [a₀ + 1/x].
  \]
  Suppose the lemma is true for some $n ≥ 0$, then
  \begin{align*}
    [a₀; a₁, …, aₙ, x]
    & = a₀ + \frac{1}{[a₁; a₂, …, aₙ, x]} \\
    & = a₀ + \frac{1}{[a₁; a₂, …, aₙ + 1/x]} \\
    & = [a₀; a₁, …, aₙ, x]. \qedhere
  \end{align*}
\end{proof}

For an infinite continued fraction,
it is necessary to show that the finite continued fractions $[a₀; a₁, …, aₙ]$ are converging.
These continued fractions are called \emph{convergents},
specifically the \emph{$n$-th convergent} of $[a₀; a₁, …]$ is the finite
continued fraction $[a₀; a₁, …, aₙ]$.
If $[a₀; a₁, …]$ is a continued fraction with integer coefficients,
then each convergent produces a unique rational $p/q$.
This representation is called the \emph{canonical representation} of the convergent.
We calculate this representation inductively as follows:

The canonical representation of the first convergent is simply
\[
  [a₀] = a₀.
\]
Suppose that the $(n-1)$-th convergent of $[a₁; a₂, …]$ is $p'/q'$,
then the $n$-th convergent of the continued fractions $[a₀; a₁, …]$ is
\[
  [a₀; a₁, …, aₙ]
  = a₀ + \frac{1}{[a₁; a₂, …, aₙ]}
  = a₀ + \frac{q'}{p'}
  = \frac{p' a₀ + q'}{p'}.
\]
Thus, the canonical representation $p/q$ of $[a₀; a₁, …, aₙ]$ is defined as:
\begin{align*}
  p = p' a₀ + q', \quad q = p'.
\end{align*}

Of course, this requires the convergent of a different continued fraction.
Instead, we use a more direct way to calculate the canonical representation $pₙ/qₙ$
from the previous convergents $p_{n-1}/q_{n-1}$ and $p_{n-2}/q_{n-2}$
of the same continued fraction.
We use the following recurrence to calculate $p_n/q_n$:
\begin{align*}
  p_n & = p_{n-1} a_n + p_{n-2}, & p_{-1} & = 1, & p_{-2} & = 0, \\
  q_n & = q_{n-1} a_n + q_{n-2}, & q_{-1} & = 1, & q_{-2} & = 0.
\end{align*}

\begin{lemma}
  \label{lem:cf-wallis}
  Let $x ∈ ℝ$, then
  \[
    [a₀; a₁, …, a_{n-1}, x] = \frac{pₙ}{qₙ} = \frac{p_{n-1} x + p_{n-2}}{q_{n-1} x + q_{n-2}}.
  \]
\end{lemma}

\begin{proof}
  If $n = 0$, then
  \[
    [x] = x = \frac{1x + 0}{0x + 1} = \frac{p_{-1} x + p_{-2}}{q_{-1} x + q_{-2}}.
  \]
  Suppose, the lemma is true for $n ≥ 0$.
  By Lemma~\ref{lem:cf-nesting}, we have
  \begin{align*}
    [a₀; a₁, …, a_{n-1}, x]
    & = [a₀; a₁, …, a_{n-1} + 1/x].
  \end{align*}
  From the induction hypothesis, it follows that
  \begin{align*}
    [a₀; a₁, …, a_{n-1} + 1/x]
    & = \frac{p_{n - 2} (aₙ + 1/x) + p_{n-3}}{q_{n-2} (aₙ + 1/x) + q_{n-3}} \\
    & = \frac{x (p_{n-2} aₙ + p_{n-3}) + p_{n-2}}{x (q_{n-2} aₙ + q_{n-3}) + q_{n-2}} \\
    & = \frac{p_{n-1} x + p_{n-2}}{q_{n-1} x + q_{n-2}}. \qedhere
  \end{align*}
\end{proof}

Thus, we can use this simple recurrence to calculate the convergent of a
continued fraction.
This is unique if we cannot reduce the fraction $pₙ/qₙ$ any further.
In other words, $pₙ$ and $qₙ$ have a greatest common divisor of $1$.
By Bezout's identity, $pₙ$ and $qₙ$ have a greatest common divisor of $1$
if and only if there are integers $a, b$ such that $apₙ + bqₙ = 1$.
Choosing $a = q_{n-1}$ and $b = p_{n-1}$ solves the identity
as shown by the following lemma.

\begin{lemma}
  \label{lem:cf-det}
  For every $n ≥ 0$, we have $p_{n-1} q_{n-2} - q_{n-1} p_{n-2} = (-1)^n$.
\end{lemma}

\begin{proof}
  For $n = 0$, we have
  \[
    p_{-1} q_{-2} - q_{-1} p_{-2} = 1 - 0 = 1.
  \]
  Suppose that the lemma holds for $n ≥ 0$.
  By Lemma~\ref{lem:cf-wallis},
  \begin{align*}
    p_{n-1} q_{n-2} - q_{n-1} p_{n-2}
    & = (p_{n-2} a_{n-1} + p_{n-3}) q_{n-2} - (q_{n-2} a_{n-1} + q_{n-3}) p_{n-2} \\
    & = p_{n-3} q_{n-2} - q_{n-3} p_{n-2}.
  \intertext{By the induction hypothesis,}
    p_{n-1} q_{n-2} - q_{n-1} p_{n-2}
    & = -(p_{n-2} q_{n-3} - q_{n-2} p_{n-3})
      = -(-1)^{n-1}
      = (-1)^n. \qedhere
  \end{align*}
\end{proof}

% ==============================================================================
\section{Construction Using the Euclidean Algorithm}
\label{sec:cf-construction}
% ==============================================================================

% TODO: Should we jump out of the gate with this problem?
The definition only tells us how to calculate the value of a continued fraction,
not how to construct a continued fraction.
However, the first part of Hermite's question requires an explicit representation of every real number.
Thus, we need to construct a unique continued fraction $[a₀; a₁, …]$ for every
real number $x$ such that $x = [a₀; a₁, …]$.

So far, we have allowed continued fractions with real coefficients.
For the construction, we allow only integer coefficients.
More specifically, we allow only continued fractions $[a₀; a₁, a₂, …]$
where the first coefficient $a₀$ can be any integer but the remaining ones have to be positive.
These constraints do not guarantee a unique representation alone.
Consider $x = 3/2$, with the current requirements there are two possible representations:
\[
  x = [1; 1, 1] = 1 + \cfrac{1}{1 + \cfrac{1}{1}} \qquad \text{ or } \qquad x = [1; 2] = 1 + \cfrac{1}{2}.
\]
The issue is that we can always split the last coefficient in the continued fraction.
In general, if $x = [a₀; a₁, …, aₙ]$, then also $x = [a₀; a₁, …, aₙ - 1, 1]$.
Therefore, we additionally require that in a finite continued fraction the last value is never $1$.

We begin with the representation of rational numbers.
We use the Euclidean algorithm to construct the continued fraction for a number $x ∈ ℚ$.
More specifically, if $x = p/q$, then we run the algorithm with the input pair $(p, q)$.
At each step, we keep track of the quotient,
which will be used as a coefficient in the continued fraction.
So if we calculate $p = a₀q + r$ in the first division step, then the quotient $a₀$ will be
the first coefficient in the continued fraction $[a₀; a₁, …, aₙ]$ for $p/q$.
This is also the reason why we allow $a₀$ to be negative.
If the fraction $p/q$ is negative, then $a₀$ is negative, too.

\begin{example}
  \label{ex:euclidean-cf}
  Consider $x = 13/5$.
  The Euclidean algorithm computes
  \begin{align*}
    13 & = 2 · 5 + 3 \\
     5 & = 1 · 3 + 2 \\
     3 & = 1 · 2 + 1 \\
     2 & = 2 · 1 + 0.
  \end{align*}
  The quotient in each line correspond directly to the continued fraction of $13/5$:
  \[
    \frac{13}{5}
    = [2; 1, 1, 2]
    = 2 + \cfrac{1}{1 + \cfrac{1}{1 + \cfrac{1}{2}}}
    = 2 + \cfrac{1}{1 + \cfrac{2}{3}}
    = 2 + \cfrac{3}{5}
    = \frac{13}{5}.
  \]
\end{example}

\begin{lemma}
  \label{lem:cf-rat}
  Every rational number has a finite continued fraction.
\end{lemma}

\begin{proof}
  Let $p/q$ be the reduced fraction of a rational number, i.e. $\gcd(p, q) = 1$.
  We proceed by induction over the number of steps when running the
  Euclidean algorithm on $(p, q)$.
  Suppose only one step is required, then $p = a₀ q + 0$.
  Because $\gcd(p, q) = 1$, the fraction $p/q$ must be an integer.
  Hence, the continued fraction $[a₀]$ represents $p/q$ correctly.

  Next, suppose that there exists a valid continued fraction for any rational
  number which requires $n$ steps in the Euclidean algorithm.
  Let $p = a₀ q + r$ be the first step of the Euclidean algorithm
  and suppose $(p, q)$ requires $n+1$ steps.
  Then, $(q, r)$ requires $n$ steps and by induction there exists a continued
  fraction $[a₁; a₂, …, aₙ]$ for $q/r$.
  Using this fact, we can construct the continued fraction for $p/q$, since
  \[
    \frac{p}{q}
    = a₀ + \frac{r}{q}
    = a₀ + \frac{1}{\frac{r}{q}}
    = a₀ + \frac{1}{[a₁; a₂, …, aₙ]}
    = [a₀; a₁, …, aₙ].
  \]
  Therefore, $[a₀; a₁, …, aₙ]$ is a valid continued fraction for $p/q$.
\end{proof}

So for every rational number, we have a finite continued fraction.
It remains to be shown that there exists a continued fraction for every irrational number.
However, we cannot use the Euclidean algorithm again.
In order to run the Euclidean algorithm for an irrational number $x$,
we would have to find two integers $p, q$ such that $p/q = x$
and that is by definition not possible.
Therefore, we cannot use the Euclidean algorithm directly to construct a
continued fraction.

Instead, we take a different approach.
For a given number $x₀ ∈ ℝ$,
we begin the algorithm with the input $(x₀, 1)$
and we use the division from Example~\vref{ex:real-divmod}, i.e.
we split $x$ into its integer and fractional part:
\[
  x = \floor{x} · 1 + \{x\}
\]
In the next iteration, we multiply the pair $(1, \{x\})$ by $1/\{x\}$
such that we have $(1/\{x\}, 1)$.
Then, the next input pair is $(x₁, 1) = (1/\{x₀\}, 1)$
and  we repeat the division step:
\[
  \frac{1}{\{x\}} = \floor{\frac{1}{\{x\}}} + \left\{ \frac{1}{\{x\}} \right\}.
\]
We continue this process until $xₙ = 0$.
At each step, we take the integer part $aₙ = \floor{xₙ}$ as one coefficient in
the continued fraction $[a₀; a₁, …]$.
The values $xₙ$ are also called \emph{complete quotients} of $x$.

\begin{example}
  \label{ex:cf-rat}
  Consider $x = 13/5$, again.
  The new algorithm computes
  \begin{align*}
    \frac{13}{5} = 2 + \frac{3}{5}
    \quad \Rightarrow \quad
    \frac{5}{3} = 1 + \frac{2}{3}
    \quad \Rightarrow \quad
    \frac{3}{2} = 1 + \frac{1}{2}
    \quad \Rightarrow \quad
    \frac{2}{1} = 2 + 0.
  \end{align*}
  Thus, we get the same continued fraction $[2; 1, 1, 2]$ for $13/5$
  as in Example~\ref{ex:euclidean-cf}.
\end{example}

\begin{example}
  \label{ex:cf-irrat}
  Consider $x = \sqrt{2}$, which has an integer part of $1$.
  The first iteration of the algorithm results in
  \[
    \sqrt{2} = 1 + (\sqrt{2} - 1).
  \]
  The inverse of $\sqrt{2} - 1$ is $\sqrt{2} + 1$,
  since $(\sqrt{2} - 1)(\sqrt{2} + 1) = 2 - 1 = 1$.
  Thus, the next input pair is $(\sqrt{2} + 1, 1)$ and the algorithm computes
  \[
    \sqrt{2} = 2 + (\sqrt{2} - 1).
  \]
  After this step,
  the algorithm cycles.
  Therefore, the continued fraction for $\sqrt{2}$ is $[1; 2, 2, …]$.
\end{example}

% TODO: Explain that we need to show that it converges
The next step is to show that the constructed continued fraction actually
produces a correct representation for the number.
By definition, a number $x ∈ ℝ$ is represented by a continued fraction $[a₀;
a₁, …]$ if and only if the limit
\[
  x = \lim_{n → ∞} [a₀; a₁, …, aₙ]
\]
exists.
In other words, the convergents must approach the number $x$ as $n$ increases.
Thus, the next step is to prove that convergents actually approach the number $x$.

Lemma~\ref{lem:cf-det} already gives us a hint towards the convergence of
continued fractions, because dividing the equation by $q_n q_{n-1}$ results in
\[
  \frac{p_n}{q_n} - \frac{p_{n-1}}{q_{n-1}} = \frac{(-1)^{n+1}}{q_n q_{n-1}}.
\]
Since $aₙ > 0$ for $n ≥ 1$, the denominators $q_n$ and $q_{n-1}$ are always
increasing after some point.
Consequently, the distance between consecutive convergents is decreasing.
But we need to show that the convergents actually approach the represented number.
For the proof, we use the fact that the algorithm always guarantees $xₙ - aₙ < 1$.

\begin{lemma}
  \label{lem:cf-approx}
  Let $x ∈ ℝ$ and let $[a₀; a₁, …]$ be its continued fraction expansion.
  If $pₙ/qₙ$ is the $n$-th convergent, then
  \[
    \left| x - \frac{pₙ}{qₙ} \right| < \frac{1}{qₙ^2}.
  \]
\end{lemma}

% TODO: Have we proven that (p_{n-1} q_n - p_n q_{n-1}) = (-1)^n yet?
\begin{proof}
  Let $x_n = [a_n; a_{n+1}, …]$.
  Then, $x = [a₀; a₁, …, a_{n-1}, x_n]$ and $a_n = \floor{x_n}$.
  Using Lemma~\ref{lem:cf-wallis},
  we can represent $x$ as well as $p_n/q_n$ using the previous convergents
  $p_{n-1}/q_{n-1}$ and $p_{n-2}/q_{n-2}$ together with the coefficients $x_n$
  and $a_n$, respectively.
  Hence,
  \begin{align*}
    \left| x - \frac{pₙ}{qₙ} \right|
    & = \left| \frac{x_n p_{n-1} + p_{n-2}}{x_n q_{n-1} + q_{n-2}} - \frac{a_n p_{n-1} + p_{n-2}}{a_n q_{n-1} + q_{n-2}} \right| \\
    & = \left| \frac{(x_n p_{n-1} + p_{n-2})(a_n q_{n-1} + q_{n-2}) - (x_n q_{n-1} + q_{n-2})(a_n p_{n-1} + p_{n-2})}{(x_n q_{n-1} + q_{n-2})(a_n q_{n-1} + q_{n-2})} \right| \\
    & = \left| \frac{(p_{n-1} q_{n-2} - q_{n-1} p_{n-2})(x_n - a_n)}{((x_n - a_n) q_{n-1} + q_n) q_n} \right|.
  \end{align*}
  From Lemma~\ref{lem:cf-det} it follows that
  \begin{align*}
    \left| x - \frac{pₙ}{qₙ} \right|
    & = \Biggl| \frac{(-1)^{n+1} \overbrace{(x_n - a_n)}^{≤ 1}}{q_n^2 + \underbrace{(x_n - a_n)}_{≥ 0} q_{n-1} q_n} \Biggr| < \frac{1}{q_n^2}. \qedhere
  \end{align*}
\end{proof}

% TODO: Mention that we can also prove the other direction
We now have a continued fraction for both rational number and irrational numbers.
What remains to be shown is that the continued fractions are unique.
This gives us a unique representation for every real number $x ∈ ℝ$ by a
continued fraction $[a₀; a₁, …]$.

\begin{theorem}
  \label{thm:irrat-cf}
  Every real number $x$ has a unique continued fraction.
\end{theorem}

\begin{proof}
  From the previous considerations, it follows that for every number $x ∈ ℝ$,
  there exists a continued fraction $[a₀; a₁, …]$ such that $[a₀; a₁, …] = x$.
  Because the continued fraction $[a₀; a₁, …]$ is made up of integers and the
  ones after $a₀$ are all positive, its integer part is entirely determined by
  the first coefficient $a₀$ and its fractional part is determined by $[0; a₁, a₂, …]$.
  Although, there is one special case of $[0; 1] = 1$,
  this cannot be constructed using the algorithm.
  The algorithm would have to begin with a fractional part of $1$, which is impossible.

  Now, suppose there is a different continued fraction $[b₀; b₁, …]$ with $[b₀; b₁, …] = x$.
  Because $[0; a₁, a₂, …]$ and $[0; b₁, b₂, …]$ both lie between $0$ and $1$,
  the continued fractions must share the same first coefficient $a₀ = b₀$.
  Otherwise, they would not have the same integer part and represent different numbers.
  By induction, suppose that the first $n ≥ 0$ terms are equal.
  Then, the continued fractions $[a_{n+1}; a_{n+2}, …]$ and $[b_{n+1}; b_{n+2}, …]$ must be equal.
  But by the same argument, we have $a_{n+1} = b_{n+1}$.
  Therefore, the continued fraction $[a₀; a₁, …]$ is unique.
\end{proof}

This concludes the first part for Hermite's question.
We now have a representation of the real numbers using continued fractions.
The second part is about periodic continued fractions and quadratic irrationals.
But before we look into that, we will look into the geometry behind the continued fractions
to understand them better.
The geometrical view will also be useful for the second part of Hermite's question.

% ==============================================================================
\section{Geometrical Interpretation Based on Klein Polygons}
\label{sec:cf-geometry}
% ==============================================================================

The geometry of continued fraction was first analyzed by Klein \cite{Klein95}.
He viewed the convergents $p_n/q_n$ of a continued fraction $[a₀; a₁, …]$ not
as rational numbers on a one-dimensional number line,
but as a convergent \emph{vector} $bₙ = (pₙ, qₙ)$ in the two-dimensional integer lattice $ℤ^2$.
Klein has shown many geometrical analogues of important theorems for continued fraction
using these vectors.
However, we will focus on a small subset, which will be useful for the
continued fractions of quadratic irrationals.

\begin{figure}[tb]
  \centering
  \includestandalone{figures/klein-polygon}
  \caption{
    The convergents $pₙ/qₙ$ of $\sqrt{2}$ as vectors $bₙ = (pₙ, qₙ)^⊤$.
    The even and odd convergents form two different polygonal chains which
    approach the irrational line given by $x/y = \sqrt{2}$.
  }
  \label{fig:klein-polygon}
\end{figure}

As an example, Figure~\ref{fig:klein-polygon} shows the convergent vectors of $\sqrt{2}$.
The figure clearly shows that there are two distinct polygonal chains:
One formed by the even convergents and the other formed by the odd convergents.
These chains both appear to approach a line where $x/y = \sqrt{2}$.
The phenomenon can be explained by the convergence as proven in Lemma~\ref{lem:cf-approx}.
If each convergent $(pₙ, qₙ)$ is scaled down by its second coordinate,
then the resulting vector is $(pₙ/qₙ, 1)$.
This vector lies within a distance of at most $1/qₙ^2$ from the vector $(\sqrt{2}, 1)$.
Consequently, scaling the vector back up results in a maximum distance of $1/qₙ$,
which decreases as $n$ increases.
Thus, convergence in this geometrical interpretation means that the vectors are
approaching a line spanned by the vector $(x, 1)$.

Convergence is one property that can be interpreted geometrically.
The definition of continued fractions itself admits a geometrical interpretation.
Suppose we have the vector $\tilde b_{n-1} = (\tilde p_{n-1}, \tilde q_{n-1})^⊤$
for the continued fraction $[a₁; a₂, …, aₙ]$.
Then,
\[
  \frac{pₙ}{qₙ}
  = \frac{a₀ \tilde p_{n-1} + \tilde q_{n-1}}{\tilde p_{n-1}}
  \iff
  \binom{pₙ}{qₙ}
  = \binom{a₀ \tilde p_{n-1} + \tilde q_{n-1}}{\tilde p_{n-1}}
\]
What is more important is that this operation is linear.
We can represent it as the product of two matrices $R$ and $S$.
The first matrix $R$ calculates the reciprocal by swapping the two coordinates.
The second matrix $S$ skews the vector by its second coordinate.
These matrices are defined as follows:
\[
  S^a =
  \begin{pmatrix}
    1 & 1 \\
    0 & 1 \\
  \end{pmatrix}^a
  =
  \begin{pmatrix}
    1 & a \\
    0 & 1 \\
  \end{pmatrix},
  \quad
  R =
  \begin{pmatrix}
    0 & 1 \\
    1 & 0 \\
  \end{pmatrix}.
\]
The vector $b_n$ can then be calculated according to
\[
  \binom{pₙ}{qₙ}
  = S^{a₀} R \binom{\tilde p_{n-1}}{q_{n-1}}
  = S^{a₀} \binom{\tilde q_{n-1}}{\tilde p_{n-1}}
  = \binom{a₀ \tilde p_{n-1} + \tilde q_{n-1}}{\tilde p_{n-1}}.
\]

For the continued fractions, we quickly moved from the definition to Lemma~\ref{lem:cf-wallis},
which tells us how to calculate the convergent $pₙ/qₙ$ using $p_{n-1}/q_{n-1}$ and $p_{n-2}/q_{n-2}$.
Naturally, we can extend this to our geometrical interpretation, too.
In fact, it is just a linear operation:
\begin{align*}
  b_n =
  \begin{pmatrix}
    p_n \\ q_n
  \end{pmatrix}
  =
  \begin{pmatrix}
    p_{n-1} a_n + p_{n-2} \\ q_{n-1} a_n + q_{n-2}
  \end{pmatrix}
  =
  \begin{pmatrix}
    p_{n-1} \\ q_{n-1}
  \end{pmatrix}
  a_n
  +
  \begin{pmatrix}
    p_{n-2} \\ q_{n-2}
  \end{pmatrix}
  = b_{n-1} a_n + b_{n-2}.
\end{align*}

This also explains the choice of the coefficient $aₙ$.
If we replace $aₙ$ in the equation with a variable $t ≥ 0$,
then $b_{n-1} t + b_{n-2}$ forms a ray starting at $b_{n-2}$ and going in the
direction of $b_{n-1}$.
This ray must intersect the line generated by $(x, 1)$ at some point.
In fact, it intersects the line at $t = xₙ$, since
\[
  λ
  \begin{pmatrix}
    x \\
    1 \\
  \end{pmatrix}
  =
  b_{n-1} x_n + b_{n-2}
  \iff
  \frac{λ x}{λ} = x = \frac{p_{n-1} x_n + p_{n-2}}{q_{n-1} x_n + q_{n-2}}.
\]
Since $aₙ = \floor{xₙ}$, we choose $aₙ$ as the last integer point before the intersection point.

For the first new lemma, we return to what we saw in
Figure~\ref{fig:klein-polygon} and show that even and odd convergent lie on
different sides of the line spanned by $(x, 1)$.
In terms of rational numbers,
this means that the convergents alternate between being less than $x$ and being
greater than $x$.

\begin{lemma}
  \label{lem:klein-conv}
  Even and odd convergent lie on different sides of the line spanned by $(x, 1)$.
\end{lemma}

\begin{proof}
  We proceed via induction on $n$.
  For the first two vectors $b_{-2}$ and $b_{-1}$, this is obviously true.
  Now suppose, that $b_{n-1}$ and $b_{n-2}$ lie on different sides.
  The next vector $b_n$ can be calculated according to
  \[
    b_n = b_{n-1} a_n + b_{n-2}.
  \]
  Furthermore, we can find an intersection with the line spanned $(x, 1)$ using
  the complete quotient $xₙ$ of $x$:
  \[
    λ
    \begin{pmatrix}
      x \\
      1 \\
    \end{pmatrix}
    = b_{n-1} xₙ + b_{n-2}.
  \]
  By the induction hypothesis, the vector $b_{n-2}$ is on a different side than $b_{n-1}$.
  However, $a_n$ is smaller than $x_n$, so $b_n = b_{n-1} a_n + b_{n-2}$ never crosses the line.
  Therefore, $b_n$ is on the same side as $b_{n-2}$.
\end{proof}

If we combine the previous two vectors in a matrix $B_n = \begin{pmatrix}
  b_{n-1} & b_{n-2} \\
\end{pmatrix}$,
then we can calculate the next matrix by multiplying with the matrices $S$ and $R$, since
\[
  B_n S^{a_n} R =
  \begin{pmatrix}
    p_{n-1} & p_{n-2} \\
    q_{n-1} & q_{n-2} \\
  \end{pmatrix}
  \begin{pmatrix}
    a_n & 1 \\
    1   & 0 \\
  \end{pmatrix}
  =
  \begin{pmatrix}
    p_{n-1} a_n + p_{n-2} & p_{n-1} \\
    q_{n-1} a_n + q_{n-2} & q_{n-1} \\
  \end{pmatrix}
  =
  B_n.
\]
Because $\det(S^a) = \det(S)^a = 1$ and $\det(R) = -1$, it is straightforward to
see that $B_n$ always has determinant $±1$.
Geometrically, this means that there is no integer point inside the
parallelogram spanned by $b_{n-1}$ and $b_{n-2}$.
This is the geometrical analogue of Lemma~\ref{lem:cf-det},
because $\det(B_n) = p_{n-1} q_{n-2} - p_{n-2} q_{n-1}$.

\begin{lemma}
  \label{lem:klein-empty}
  Between the polygonal chains of the even and odd convergent vectors,
  there exists no nonzero integer point.
\end{lemma}

\begin{proof}
  We can cover the area between the chains
  using parallelograms made up of the points $b_n$, $b_{n-1}, b_{n-2}$, and a suitable fourth point.
  Each parallelogram contains exactly
  \begin{align*}
    \left|\det\begin{pmatrix}
      b_n - b_{n-2} & b_{n-1} - b_{n-2}
    \end{pmatrix}\right|
    & = \left|\det\begin{pmatrix}
      a_n b_{n-1} & b_{n-1} - b_{n-2}
    \end{pmatrix}\right| \\
    & = \left|\det\begin{pmatrix}
      a_n b_{n-1} & -b_{n-2}
    \end{pmatrix}\right| \\
    & = a_n
  \end{align*}
  integer points.
  However, these points all lie on the line between $b_n$ and $b_{n-2}$.
  Therefore, the area between the chains cannot contain any integer point.
\end{proof}

\begin{definition}
  \label{def:klein-polygon}
  Let $v₁, v₂ ∈ ℝ^2$ be two linearly independent vectors.
  Consider the cone $C$ in $ℝ^2$ bounded by the two rays of $v₁, v₂$. That is
  \[
    C = \{\, λ₁ v₁ + λ₂ v₂ \mid λ₁, λ₂ ≥ 0 \,\}.
  \]
  The \emph{Klein polygon} $K$ spanned by $v₁, v₂$ is defined as the convex hull
  of all nonzero integer points inside the cone, i.e.
  \[
    K = \mathrm{conv}(C ∩ ℤ^2 \setminus \{(0, 0)\}).
  \]
  The \emph{boundary} of a Klein polygon is denoted as $Π(K)$.
\end{definition}

The even and odd convergent vectors are part of two such Klein polygons.
The even convergents are part of the Klein polygon $K_0$ spanned by $(0, 1)$ and $(x, 1)$,
and the odd convergents are part of the Klein polygon $K_1$ spanned by $(x, 1)$ and $(1, 0)$.
This is shown in the following theorem.

\begin{theorem}
  The Klein polygons $K_0$ and $K_1$ are equal to the
  convex hull of the even and odd convergents, respectively.
\end{theorem}

\begin{proof}
  Suppose there is a point in the Klein polygons,
  which is not in the convex hull of either the even or odd convergents.
  This means that the point must lie between the two polygonal chains,
  which is impossible by Lemma~\ref{lem:klein-empty}.
  Now suppose there is a point in the convex hull of the convergents,
  but not in one of the Klein polygons.
  Then, there must be one convergent outside the Klein polygon,
  which is closer to the line $(x, 1)$.
  However, this integer point is still inside the cone,
  so it is also contained in the Klein polygon.
\end{proof}

\begin{corollary}
  The vertices of $K_0$ and $K_1$ are equal to the
  even and odd convergents, respectively.
\end{corollary}

% ==============================================================================
\section{Quadratic Irrationals and Periodicity}
\label{sec:cf-quadratic}
% ==============================================================================

% TODO: Write introduction for this section
This section is about the second part of Hermite's question for quadratic
irrationals, i.e. whether the continued fraction of a number is periodic if and only
if the number is a quadratic irrational.
Formally, we call a continued fraction $[a₀; a₁, …]$ \emph{periodic}
if there exists a starting index $k₀ ≥ 0$ and a period $ℓ ≥ 1$ such that $aₖ = a_{k+ℓ}$ for every $k ≥ k₀$.
A continued fraction is called \emph{purely periodic} if $k₀ = 0$,
i.e. the period starts immediately.
For a periodic continued fraction starting at $K$ with length $ℓ$,
we will denote it as $[a₀; a₁, …, a_{k₀-1}, \overline{a_{k₀}, …, a_{k₀+ℓ}}]$.
This is similar to how in decimal notation, we denote a period with a bar over the digits,
e.g. $1/3 = 0.\overline{3}$.
In a continued fraction, we similarly denote the period with a line over the
coefficients that are infinitely repeated.
For example, $\sqrt{2} = [1; \overline{2}]$.

For the proof, we have to show two directions.
The first is that every periodic continued fraction is a quadratic irrational
and the second is that every quadratic irrational has a periodic continued fraction.
We begin with the former and we will use the geometry from the previous section
in the proof.

\begin{theorem}
  If the continued fraction of $x ∈ ℝ$ is periodic, then $x$ is a quadratic irrational.
\end{theorem}

\begin{proof}
  Let $x$ be a purely periodic continued fraction $[a₀; a₁, …]$.
  Then, there exists a complete quotient such that $x = x_ℓ = [a_ℓ; a_{ℓ+1}, …]$ for some $ℓ ≥ 1$.
  For each pair of convergent vectors $b_{n-1}, b_{n-2}$, we can use the
  complete quotient $x_ℓ$ to find an intersection point with the line $(x, 1)$
  and since $x_ℓ = x$, we can also use $x$ itself.
  Stated differently, there exists a scalar $λ ∈ ℝ$ such that
  \[
    B_ℓ
    \begin{pmatrix}
      x \\
      1 \\
    \end{pmatrix}
    =
    b_{ℓ-1} x + b_{ℓ-2}
    = λ
    \begin{pmatrix}
      x \\
      1 \\
    \end{pmatrix}
  \]
  Therefore, $(x, 1)^⊤$ is an eigenvector of the matrix $B_ℓ$
  and $λ$ is the eigenvalue.
  Because $B$ is a 2-by-2 matrix,
  the eigenvalue can only be a quadratic irrational.
  The eigenvector is a solution to the linear system $(B_ℓ - λ I) (x, 1)^⊤ = 0$,
  where the coefficients come from the field $ℚ(λ)$.
  Therefore, $x$ must be a quadratic irrational.

  We proceed with the eventually-periodic case.
  Suppose $x$ is periodic only after some index $k ≥ 0$,
  then $xₖ = [a_k; a_{k+1}, …]$ is a quadratic irrational.
  By Lemma~\ref{lem:cf-wallis},
  \[
    x = \frac{p_{k-1} x_k + p_{k-2}}{q_{k-1} x_k + q_{k-2}}.
  \]
  Because $x$ is a rational expression of $x_k$ with integer coefficients,
  it lives in the same field as $x_k$.
  Therefore, $x$ is a quadratic irrational, too.
\end{proof}

% TODO: Figure for shift of Klein polygon
The converse was originally proven by Lagrange \cite{Lagrange70}.
Here, a proof by Korkina \cite{Korkina96} is presented,
which uses the geometrical interpretation of Klein to show periodicity.
The idea is that for quadratic irrationals,
there always exists a unimodular matrix $U$ which shifts the corner points of a
Klein polygon along its boundary and preserves volume.
Because the volume between $b_n$ and $b_{n-2}$ directly corresponds to
the coefficients $a_0, a_1, a_2, …$ of the continued fraction,
they must repeat at some point.

\begin{figure}[tb]
  \centering
  \includestandalone{figures/full-klein-polygon}
  \caption{
    The four Klein polygons spanned by $±(\sqrt{2}, 1)$ and $±(-\sqrt{2}, 1)$.
  }
  \label{fig:full-klein-polygon}
\end{figure}

For the proof, we not only consider the Klein polygon in the positive quadrant,
but in all four quadrants.
Instead of the vectors $(1, 0)$ and $(0, 1)$, we use the conjugate
$\overline{x}$ and the vector $(\overline{x}, 1)$.
There are now four different Klein polygons each spanned by $±(x, 1)$
and $±(-\overline{x}, 1)$.
An example is shown in Figure~\ref{fig:full-klein-polygon}.
The idea behind the proof is that the matrix $U$ does not change any of the four
Klein polygons, but it is not the identity and therefore it must move the
points along the boundary.

\begin{theorem}
  If $x$ is a quadratic irrational,
  then its continued fraction is periodic.
\end{theorem}

\begin{proof}
  Let $\overline{x}$ denote the conjugate of $x$.
  By Theorem~\ref{thm:unimodular-algebraic}, we can find a unimodular matrix $U$ for $x$,
  which has $(x, 1)^⊤$ and $(\overline{x}, 1)^⊤$ as eigenvectors.
  Applying $U$ on the Klein polygon, we have
  \begin{align*}
    UK
    = U \mathrm{conv}(C ∩ ℤ^2 \setminus \{(0, 0)\})
    = \mathrm{conv}(UC ∩ Uℤ^2 \setminus \{(0, 0)\}).
  \end{align*}
  The matrix $U$ is unimodular, so by Lemma~\ref{lem:unimodular} we have $Uℤ^2 = ℤ^2$.
  Let $v ∈ C$.
  By definition of $C$,
  there exist coefficients $c₁, c₂ ≥ 0$ such that $v = c₁ (x, 1)^⊤ + c₂ (\overline{x}, 1)^⊤$.
  Because the vectors of $C$ are eigenvectors of $U$, we have
  \begin{align*}
    U v
    = c₁
    U\begin{pmatrix}
      x \\
      1 \\
    \end{pmatrix}
    +
    c₂ U\begin{pmatrix}
      x \\
      1 \\
    \end{pmatrix}
    = c₁ λ₁ \begin{pmatrix}
      x \\
      1 \\
    \end{pmatrix}
    + c₂ λ₂ \begin{pmatrix}
      x \\
      1 \\
    \end{pmatrix}
  \end{align*}
  Furthermore, the eigenvalues must be positive.
  If not, then we simply apply $U$ twice.
  Thus, $UC = C$ and $K$ must be invariant under $U$.

  Similarly, the boundary $Π(K)$ must also be invariant under this transformation.
  Because $U$ is a linear transformation which preserves the orientation of
  vectors and is not the identity, it must shift the integer points along the
  boundary.
  We assume that it shifts it in the positive direction,
  i.e. $B_{n+k} = U B_n$ for some $k ≥ 1$.
  If not, then we can choose $U^{-1}$ to shift them in the positive direction.
  Because $\det(U) = 1$, the matrix also preserves volume and
  \[
    a_{n+k}
    = \det\begin{pmatrix}
      b_{n+k} & b_{n+k-2}
    \end{pmatrix}
    = \det(U) \det\begin{pmatrix}
      b_n & b_{n-2}
    \end{pmatrix}
    = a_n.
  \]
  Hence, the continued fraction is periodic after some point.
\end{proof}

\begin{example}
  Consider the continued fraction $[1; \overline{2}]$ of $\sqrt{2}$.
  Even if we only know the first few terms of the continued fraction, i.e.
  $\sqrt{2} ≈ [1; 2, 2]$, we can still deduce that it must be periodic.
  We use the matrix representation to calculate the third convergent
  \[
    B_3 = B_2 · SSR = B_1 · SSR · SSR = SR · SSR · SSR = SRS · SRS · SR
  \]
  The first six matrices in this product are $U = SRSSRS = (SRS)^2$
  and
  \[
    U = (SRS)^2 =
    \left(
    \begin{pmatrix}
      1 & 1 \\
      0 & 1 \\
    \end{pmatrix}
    ·
    \begin{pmatrix}
      1 & 0 \\
      0 & 1 \\
    \end{pmatrix}
    ·
    \begin{pmatrix}
      1 & 1 \\
      0 & 1 \\
    \end{pmatrix}
    \right)^2
    =
    \begin{pmatrix}
      1 & 2 \\
      1 & 1 \\
    \end{pmatrix}^2
    =
    \begin{pmatrix}
      3 & 4 \\
      2 & 3 \\
    \end{pmatrix}
    .
  \]
  This matrix is the mutliplication matrix for $3 + 2\sqrt{2}$.
  In Example~\vref{ex:sqrt2-unit},
  we have already seen that this matrix has $(±\sqrt{2}, 1)$ as eigenvectors.
  Therefore, if $B_2$ and $B_3$ are convergent matrices,
  then so are $U^n B_2$ and $U^n B_3$ for all $n ≥ 1$.
  Thus, the continued fraction of $\sqrt{2}$ is periodic.
\end{example}
