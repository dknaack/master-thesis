\chapter{Periodic Representation of Quadratic Irrationals}

\begin{definition}
  A number $x ∈ ℝ$ is said to be a \emph{quadratic irrational} if $x$ is the root of some
  polynomial $p(x)$ over the rational numbers with degree $2$.
\end{definition}

\section{Continued Fractions}

\begin{definition}
  % TODO: Include definition for continued fractions
\end{definition}

\begin{definition}
  The \emph{$n$-th convergent} of an infinite continued fraction $[a₀; a₁, …]$ is
  defined as the finite continued fraction $[a₀; a₁, …, aₙ]$.
\end{definition}

For a continued fraction with only one entry, the convergent is simply $a₀/1$.
The convergent of longer continued fractions can be calculated inductively as follows:
\[
  [a₀; a₁, …, aₙ]
  = a₀ + \frac{1}{[a₁; a₂, …, aₙ]}
  = a₀ + \frac{\tilde q_{n-1}}{\tilde p_{n-1}}
  = \frac{\tilde p_{n-1} a₀ + \tilde q_{n-1}}{\tilde p_{n-1}},
\]
where $\tilde p_{n-1} / \tilde q_{n-1}$ is the convergent of $[a₁; a₂, …, aₙ]$.

\begin{example}
  % TODO: Add example for some continued fraction here
\end{example}

\section{Periodic Continued Fractions}

\begin{definition}
  A continued fraction $[a₀; a₁, …]$ is called \emph{eventually periodic}
  if there exists an index $k₀ ≥ 0$ and a period $ℓ ≥ 1$ such that $aₖ = a_{k+ℓ}$ for every $k ≥ k₀$.
  A continued fraction is called \emph{purely periodic} if $k₀ = 0$.
\end{definition}

\begin{example}
  Let $x = \sqrt{7}$.
  The algorithm proceeds as follows:
  \[
    \begin{array}{rclcrcl}
      \sqrt{7}      & = & 2 · 1               + (\sqrt{7} - 2)    & ≈ & 2.6458 & = & 2 · 1 + 0.6458      \\
      1             & = & 1 · (\sqrt{7} - 2)  + (3 - \sqrt{7})    & ≈ & 1      & = & 1 · 0.6458 + 0.3542 \\
      \sqrt{7} - 2  & = & 1 · (3 - \sqrt{7})  + (2\sqrt{7} - 5)   & ≈ & 0.6458 & = & 1 · 0.3542 + 0.2915 \\
      3 - \sqrt{7}  & = & 1 · (2\sqrt{7} - 5) + (8 - 3\sqrt{7})   & ≈ & 0.3542 & = & 1 · 0.2915 + 0.0627 \\
      2\sqrt{7} - 5 & = & 4 · (8 - 3\sqrt{7}) + (14\sqrt{7} - 37) & ≈ & 0.2915 & = & 4 · 0.0627 + 0.0407 \\
      & \vdots & & \vdots & & \vdots &
    \end{array}
  \]
\end{example}

The interesting connection between the quadratic irrationals and continued
fraction is that a continued fraction of a number $x ∈ ℝ$ is eventually
periodic if and only if $x$ is a quadratic irrational.
We begin with the first direction:

\begin{lemma}
  Let $x ∈ ℝ$, then
  \[
    [a₀; a₁, …, a_n, x] = [a₀; a₁, …, a_n + 1/x]
  \]
\end{lemma}

\begin{proof}
  \label{lem:nesting}
  If $n = 0$, then
  \[
    [a₀; x] = a₀ + \frac{1}{[x]} = a₀ + \frac{1}{x} = [a₀ + 1/x].
  \]
  Suppose the lemma is true for some $n ≥ 0$, then
  \begin{align*}
    [a₀; a₁, …, aₙ, x]
    & = a₀ + \frac{1}{[a₁; a₂, …, aₙ, x]} \\
    & = a₀ + \frac{1}{[a₁; a₂, …, aₙ + 1/x]} \\
    & = [a₀; a₁, …, aₙ, x]. \qedhere
  \end{align*}
\end{proof}

% TODO: Should we have this proof?
\begin{lemma}
  Given a continued fractions $[a₀; a₁, …, aₙ]$, its convergents $p_n/q_n$ satisfy the recurrence relation
  \begin{align*}
    pₙ & = p_{n-1} a_n + p_{n - 2}, & p_{-1} & = 1, & p_{-2} & = 0, \\
    qₙ & = q_{n-1} a_n + q_{n - 2}, & q_{-1} & = 0, & q_{-2} & = 1.
  \end{align*}
\end{lemma}

\begin{proof}
  For $n = 0$, we have
  \begin{align*}
    \frac{p₀}{q₀} = \frac{a₀}{1} = \frac{1 · a₀ + 0}{0 · a₀ + 1} = \frac{p_{-1} a₀ + p_{-2}}{q_{-1} a₀ + q_{-2}}.
  \end{align*}
  Suppose the lemma is true for $n ≥ 0$.
  By the previous lemma,
  \begin{align*}
    \frac{p_{n+1}}{q_{n+1}}
    = [a₀; a₁, …, a_n, a_{n+1}]
    = [a₀; a₁, …, a_n + 1/a_{n+1}].
  \end{align*}
  For this continued fraction, the first $n - 1$ convergents are the same as for $[a₀; a₁, …, aₙ]$.
  By our induction hypothesis,
  \begin{align*}
    \frac{p_{n+1}}{q_{n+1}}
    & = [a₀; a₁, …, a_n + 1/a_{n+1}] \\
    & = \frac{a_{n+1}}{a_{n+1}} · \frac{p_{n-1} \left(a_n + \frac{1}{a_{n+1}}\right) + p_{n-2}}{q_{n-1} \left(a_n + \frac{1}{a_{n+1}}\right) + q_{n-2}} \\
    & = \frac{p_n a_{n+1} + p_{n-1}}{q_n a_{n+1} + q_{n-1}}. \qedhere
  \end{align*}
\end{proof}

\begin{lemma}
  \label{lem:wallis}
  Let $x ∈ ℝ$, then
  \[
    [a₀; a₁, …, a_{n-1}, x] = \frac{p_{n-1} x + p_{n-2}}{q_{n-1} x + q_{n-2}}.
  \]
\end{lemma}

\begin{proof}
  If $n = 0$, then
  \[
    [x] = x = \frac{1x + 0}{0x + 1} = \frac{p_{-1} x + p_{-2}}{q_{-1} x + q_{-2}}.
  \]
  Suppose, the lemma is true for $n ≥ 0$.
  By Lemma~\ref{lem:nesting}, we have
  \begin{align*}
    [a₀; a₁, …, aₙ, x]
    & = [a₀; a₁, …, aₙ + 1/x].
  \end{align*}
  From the induction hypothesis, it follows that
  \begin{align*}
    [a₀; a₁, …, aₙ + 1/x]
    & = \frac{p_{n - 1} (aₙ + 1/x) + p_{n-2}}{q_{n-1} (aₙ + 1/x) + q_{n-2}} \\
    & = \frac{x (p_{n-1} aₙ + p_{n-2}) + p_{n-1}}{x (q_{n-1} aₙ + q_{n-2}) + q_{n-1}} \\
    & = \frac{x pₙ + p_{n-1}}{x pₙ + p_{n-1}}. \qedhere
  \end{align*}
\end{proof}

\begin{theorem}
  If the continued fraction representation of a number $x ∈ ℝ$ is eventually periodic,
  then $x$ is a quadratic irrational.
\end{theorem}

\begin{proof}
  Let $x$ be a continued fraction $[a₀; a₁, …]$ with a period of length $ℓ ≥ 1$
  starting at an index $k ≥ 0$,
  i.e. $x_k = [a_k; a_{k+1}, …] = [a_{k+ℓ}; a_{k+ℓ+1}, …] = x_{k+ℓ}$.
  By Lemma~\ref{lem:wallis}, we have
  \[
    x
    = \frac{p_k x_k + p_{k-1}}{q_k x_k + q_{k-1}}
    = \frac{p_{k+ℓ} x_{k+ℓ} + p_{k+ℓ-1}}{q_{k+ℓ} x_{k+ℓ} + q_{k+ℓ-1}}
    = \frac{p_{k+ℓ} x_k + p_{k+ℓ-1}}{q_{k+ℓ} x_k + q_{k+ℓ-1}}
  \]
  Multiplying by the denominators on both sides results in the quadratic equation
  \[
    (q_{k+ℓ} x_k + q_{k+ℓ-1})(p_k x_k + p_{k-1}) - (q_k x_k + q_{k-1}) (p_{k+ℓ} x_k + p_{k+ℓ-1}) = 0.
  \]
  Hence, $x_k$ is a quadratic irrational and $x$ must be one too since $x ∈ ℚ(x_k)$.
\end{proof}

\section{Continued Fractions of Quadratic Irrationals}

The converse was originally proven by Lagrange \cite{Lagrange70}.

\begin{figure}[tb]
  \centering
  \includestandalone{figures/critical-section}
  \caption{
    The hyperbolas each contain only elements with norm $±1$.
    The red area could contain an integral point, in which case this point
    would be closer than the initial point.
    The gray area cannot contain an integral point since the norm would be
    fractional.
  }
\end{figure}

\begin{figure}[tb]
  \centering
  \includestandalone{figures/parallelogram-cover}
  \caption{
    The parallelogram spanned by $B$ and $B'$ contains only $0$ as an integral point.
    The rest of the surface can be covered by identical parallelograms.
    Therefore, the surface can only contain an integral point at $B$.
  }
\end{figure}

\begin{align*}
  B_n =
  \begin{pmatrix}
    p_n \\ q_n
  \end{pmatrix}
  =
  \begin{pmatrix}
    p_{n-1} \\ q_{n-1}
  \end{pmatrix}
  a_n
  +
  \begin{pmatrix}
    p_{n-2} \\ q_{n-2}
  \end{pmatrix}
  = B_{n-1} a_n + B_{n-2}
\end{align*}

\begin{align*}
  T = \begin{pmatrix}
    1 & 1 \\
    0 & 1 \\
  \end{pmatrix},
  S = \begin{pmatrix}
    0 & 1 \\
    1 & 0 \\
  \end{pmatrix}
\end{align*}

\begin{align*}
  \det S_1 = \det\begin{pmatrix}
    1 & 1 \\
    0 & 1 \\
  \end{pmatrix} = 1,
  \det S = \det\begin{pmatrix}
    0 & 1 \\
    1 & 0 \\
  \end{pmatrix} = -1,
\end{align*}

\begin{align*}
  B_n & = T^{a₀} S T^{a₁} S T^{a₂} … S T^{aₙ} B_1 \\
      & = T^{a₀} \underbrace{S T^{a₁} S}_{S_2^{a_1}} T^{a₂} … S T^{aₙ} B_1
\end{align*}

\begin{align*}
  S_2^a = R S_1 \underbrace{R R}_{I} S_1 \underbrace{R R}_I S_1\, R … R\, S_1 R = R S_1^{a} R.
\end{align*}

\begin{definition}
  A point $(x, y)$ is a \emph{relative minimum} with respect to a line $x - α y = 0$,
  if for every point $(x', y')$ with $x' ≤ x$ \emph{or} $y' ≤ y$, we have
  \[
    |x - α y| ≤ |x' - α y'|.
  \]
\end{definition}

\begin{lemma}
  Every convergent is a relative minimum.
\end{lemma}

\begin{lemma}
  Every relative minimum is a convergent.
\end{lemma}

Pell's equation is defined as
\[
  x^2 - n y^2 = 1.
\]

\begin{lemma}
  There are integer points which satisfy Pell's equation.
\end{lemma}

\begin{lemma}
  Every integer point which satisfies Pell's equation is a relative minimum.
\end{lemma}

\begin{theorem}
  If $x$ is a quadratic irrational,
  then the continued fraction of $x$ is eventually periodic.
\end{theorem}

\begin{proof}

\end{proof}
