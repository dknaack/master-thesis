\chapter{Periodic Representation of Quadratic Irrationals}

We begin with the case of quadratic irrationals.
A quadratic irrational is any real number which is a root of a polynomial with degree $2$.
For quadratic irrationals, it is already well known that they can be
represented using periodic continued fractions,
where continued fractions are fractions of the form
\[
  a₀ + \cfrac{1}{a₁ + \cfrac{1}{a₂ + \cfrac{1}{⋱}}}.
\]
In this chapter we will prove that continued fractions are periodic if and only
if they represent a quadratic irrational.
Periodic continued fractions were first analyzed by Euler,
who has shown that such continued fractions must be quadratic irrationals.
Later, Lagrange \cite{Lagrange70} has proven the other direction, that
quadratic irrationals must always have periodic continued fractions.

\section{Continued Fractions}

\begin{definition}
  A continued fraction is defined as
  \[
    [r₀] = r₀, \qquad
    [r₀; r₁, …, rₙ] = r₀ + \frac{1}{[r₁; r₂, …, rₙ]}.
  \]
\end{definition}

\begin{definition}
  The \emph{$n$-th convergent} of an infinite continued fraction $[a₀; a₁, …]$ is
  defined as the finite continued fraction $[a₀; a₁, …, aₙ]$.
\end{definition}

For a continued fraction with only one entry, the convergent is simply $a₀/1$.
The convergent of longer continued fractions can be calculated inductively as follows:
\[
  [a₀; a₁, …, aₙ]
  = a₀ + \frac{1}{[a₁; a₂, …, aₙ]}
  = a₀ + \frac{\tilde q_{n-1}}{\tilde p_{n-1}}
  = \frac{\tilde p_{n-1} a₀ + \tilde q_{n-1}}{\tilde p_{n-1}},
\]
where $\tilde p_{n-1} / \tilde q_{n-1}$ is the convergent of $[a₁; a₂, …, aₙ]$.

\begin{example}
  % TODO: Add example for some continued fraction here
\end{example}

% ==============================================================================
\section{Periodic Continued Fractions}
% ==============================================================================

\begin{definition}
  A continued fraction $[a₀; a₁, …]$ is called \emph{eventually periodic}
  if there exists an index $k₀ ≥ 0$ and a period $ℓ ≥ 1$ such that $aₖ = a_{k+ℓ}$ for every $k ≥ k₀$.
  A continued fraction is called \emph{purely periodic} if $k₀ = 0$.
\end{definition}

\begin{example}
  Let $x = \sqrt{7}$.
  The algorithm proceeds as follows:
  \[
    \begin{array}{rclcrcl}
      \sqrt{7}      & = & 2 · 1               + (\sqrt{7} - 2)    & ≈ & 2.6458 & = & 2 · 1 + 0.6458      \\
      1             & = & 1 · (\sqrt{7} - 2)  + (3 - \sqrt{7})    & ≈ & 1      & = & 1 · 0.6458 + 0.3542 \\
      \sqrt{7} - 2  & = & 1 · (3 - \sqrt{7})  + (2\sqrt{7} - 5)   & ≈ & 0.6458 & = & 1 · 0.3542 + 0.2915 \\
      3 - \sqrt{7}  & = & 1 · (2\sqrt{7} - 5) + (8 - 3\sqrt{7})   & ≈ & 0.3542 & = & 1 · 0.2915 + 0.0627 \\
      2\sqrt{7} - 5 & = & 4 · (8 - 3\sqrt{7}) + (14\sqrt{7} - 37) & ≈ & 0.2915 & = & 4 · 0.0627 + 0.0407 \\
      & \vdots & & \vdots & & \vdots &
    \end{array}
  \]
\end{example}

The interesting connection between the quadratic irrationals and continued
fraction is that a continued fraction of a number $x ∈ ℝ$ is eventually
periodic if and only if $x$ is a quadratic irrational.
We begin with the first direction:

\begin{lemma}
  Let $x ∈ ℝ$, then
  \[
    [a₀; a₁, …, a_n, x] = [a₀; a₁, …, a_n + 1/x]
  \]
\end{lemma}

\begin{proof}
  \label{lem:nesting}
  If $n = 0$, then
  \[
    [a₀; x] = a₀ + \frac{1}{[x]} = a₀ + \frac{1}{x} = [a₀ + 1/x].
  \]
  Suppose the lemma is true for some $n ≥ 0$, then
  \begin{align*}
    [a₀; a₁, …, aₙ, x]
    & = a₀ + \frac{1}{[a₁; a₂, …, aₙ, x]} \\
    & = a₀ + \frac{1}{[a₁; a₂, …, aₙ + 1/x]} \\
    & = [a₀; a₁, …, aₙ, x]. \qedhere
  \end{align*}
\end{proof}

% TODO: Should we have this proof?
\begin{lemma}
  Given a continued fractions $[a₀; a₁, …, aₙ]$, its convergents $p_n/q_n$ satisfy the recurrence relation
  \begin{align*}
    pₙ & = p_{n-1} a_n + p_{n - 2}, & p_{-1} & = 1, & p_{-2} & = 0, \\
    qₙ & = q_{n-1} a_n + q_{n - 2}, & q_{-1} & = 0, & q_{-2} & = 1.
  \end{align*}
\end{lemma}

\begin{proof}
  For $n = 0$, we have
  \begin{align*}
    \frac{p₀}{q₀} = \frac{a₀}{1} = \frac{1 · a₀ + 0}{0 · a₀ + 1} = \frac{p_{-1} a₀ + p_{-2}}{q_{-1} a₀ + q_{-2}}.
  \end{align*}
  Suppose the lemma is true for $n ≥ 0$.
  By the previous lemma,
  \begin{align*}
    \frac{p_{n+1}}{q_{n+1}}
    = [a₀; a₁, …, a_n, a_{n+1}]
    = [a₀; a₁, …, a_n + 1/a_{n+1}].
  \end{align*}
  For this continued fraction, the first $n - 1$ convergents are the same as for $[a₀; a₁, …, aₙ]$.
  By our induction hypothesis,
  \begin{align*}
    \frac{p_{n+1}}{q_{n+1}}
    & = [a₀; a₁, …, a_n + 1/a_{n+1}] \\
    & = \frac{a_{n+1}}{a_{n+1}} · \frac{p_{n-1} \left(a_n + \frac{1}{a_{n+1}}\right) + p_{n-2}}{q_{n-1} \left(a_n + \frac{1}{a_{n+1}}\right) + q_{n-2}} \\
    & = \frac{p_n a_{n+1} + p_{n-1}}{q_n a_{n+1} + q_{n-1}}. \qedhere
  \end{align*}
\end{proof}

\begin{lemma}
  \label{lem:wallis}
  Let $x ∈ ℝ$, then
  \[
    [a₀; a₁, …, a_{n-1}, x] = \frac{p_{n-1} x + p_{n-2}}{q_{n-1} x + q_{n-2}}.
  \]
\end{lemma}

\begin{proof}
  If $n = 0$, then
  \[
    [x] = x = \frac{1x + 0}{0x + 1} = \frac{p_{-1} x + p_{-2}}{q_{-1} x + q_{-2}}.
  \]
  Suppose, the lemma is true for $n ≥ 0$.
  By Lemma~\ref{lem:nesting}, we have
  \begin{align*}
    [a₀; a₁, …, aₙ, x]
    & = [a₀; a₁, …, aₙ + 1/x].
  \end{align*}
  From the induction hypothesis, it follows that
  \begin{align*}
    [a₀; a₁, …, aₙ + 1/x]
    & = \frac{p_{n - 1} (aₙ + 1/x) + p_{n-2}}{q_{n-1} (aₙ + 1/x) + q_{n-2}} \\
    & = \frac{x (p_{n-1} aₙ + p_{n-2}) + p_{n-1}}{x (q_{n-1} aₙ + q_{n-2}) + q_{n-1}} \\
    & = \frac{x pₙ + p_{n-1}}{x pₙ + p_{n-1}}. \qedhere
  \end{align*}
\end{proof}

\begin{theorem}
  If the continued fraction representation of a number $x ∈ ℝ$ is eventually periodic,
  then $x$ is a quadratic irrational.
\end{theorem}

\begin{proof}
  Let $x$ be a continued fraction $[a₀; a₁, …]$ with a period of length $ℓ ≥ 1$
  starting at an index $k ≥ 0$,
  i.e. $x_k = [a_k; a_{k+1}, …] = [a_{k+ℓ}; a_{k+ℓ+1}, …] = x_{k+ℓ}$.
  By Lemma~\ref{lem:wallis}, we have
  \[
    x
    = \frac{p_k x_k + p_{k-1}}{q_k x_k + q_{k-1}}
    = \frac{p_{k+ℓ} x_{k+ℓ} + p_{k+ℓ-1}}{q_{k+ℓ} x_{k+ℓ} + q_{k+ℓ-1}}
    = \frac{p_{k+ℓ} x_k + p_{k+ℓ-1}}{q_{k+ℓ} x_k + q_{k+ℓ-1}}
  \]
  Multiplying by the denominators on both sides results in the quadratic equation
  \[
    (q_{k+ℓ} x_k + q_{k+ℓ-1})(p_k x_k + p_{k-1}) - (q_k x_k + q_{k-1}) (p_{k+ℓ} x_k + p_{k+ℓ-1}) = 0.
  \]
  Hence, $x_k$ is a quadratic irrational and $x$ must be one too since $x ∈ ℚ(x_k)$.
\end{proof}

% ==============================================================================
\section{Continued Fractions of Quadratic Irrationals}
% ==============================================================================

The converse was originally proven by Lagrange \cite{Lagrange70}.

\begin{lemma}
  \label{lem:cf-approx}
  Let $p_n/q_n$ be the convergents of $x ∈ ℝ$, then
  \[
    \left| x - \frac{pₙ}{qₙ} \right| < \frac{1}{qₙ^2}.
  \]
\end{lemma}

\begin{proof}
  \begin{align*}
    x - \frac{pₙ}{qₙ}
    & = \frac{r_n p_{n-1} + p_{n-2}}{r_n q_{n-1} + q_{n-2}} - \frac{a_n p_{n-1} + p_{n-2}}{a_n q_{n-1} + q_{n-2}} \\
    & = \frac{(r_n p_{n-1} + p_{n-2})(a_n q_{n-1} + q_{n-2}) - (r_n q_{n-1} + q_{n-2})(a_n p_{n-1} + p_{n-2})}{(r_n q_{n-1} + q_{n-2})(a_n q_{n-1} + q_{n-2})} \\
    & = \frac{(p_{n-1} q_{n-2} + q_{n-1} p_{n-2})(r_n - a_n)}{((r_n - a_n) q_{n-1} + q_n) q_n} \\
    & = \frac{(-1)^{n+1} (r_n - a_n)}{q_n^2 + (r_n - a_n) q_{n-1} q_n} < \frac{1}{q_n^2}. \qedhere
  \end{align*}
\end{proof}

\begin{theorem}
  If $x$ is a quadratic irrational,
  then its continued fraction is periodic.
\end{theorem}

\begin{proof}
  Let $x$ be a quadratic irrational which satisfies $ax^2 + bx + c = 0$.
  Furthermore, let $x = [a₀; a₁, a₂, …]$ and $x_n = [a_n; a_{n+1}, …]$ for every $n ≥ 0$.
  By Lemma~\ref{lem:wallis}, we must have
  \[
    a \left(\frac{x_n p_{n-1} + p_{n-2}}{x_n q_{n-1} + q_{n-2}}\right)^2
    + b \left(\frac{x_n p_{n-1} + p_{n-2}}{x_n q_{n-1} + q_{n-2}}\right)
    + c = 0.
  \]
  We rearrange the polynomial into the form
  \[
    A_n x_n^2 + B_n x_n + C_n = 0
  \]
  with
  \begin{align*}
    A_n & = a p_{n-1}^2 + b p_{n-1} q_{n-1} + c q_{n-1}^2, \\
    B_n & = 2a p_{n-1} p_{n-2} + b p_{n-1} q_{n-2} + b p_{n-2} q_{n-1} + 2c q_{n-1} q_{n-2}, \\
    C_n & = a p_{n-2}^2 + b p_{n-2} q_{n-2} + c q_{n-2}^2 = A_{n-1}.
  \end{align*}
  The goal is to show that $A_n, B_n, C_n$ are bounded independently of $n$.
  First, we note that the discriminant $D_n = B_n^2 - 4 A_n C_n$ is still $b^2 - 4ac$.
  If $A_n$ and $C_n$ are bounded, then $B_n$ is bounded.
  Furthermore, since $C_n = A_{n-1}$, it suffices to show that $A_n$ is bounded.

  By Lemma~\ref{lem:cf-approx}, we have
  \[
    \left| x - \frac{p_{n-1}}{q_{n-1}} \right| < \frac{1}{q_{n-1}^2}
    \Rightarrow
    p_{n-1} = q_{n-1} x + \frac{δ_{n-1}}{q_{n-1}}, \quad \text{ for some } |δ_{n-1}| < 1,
  \]
  from which it follows that
  \begin{align*}
    A_n & = a p_{n-1}^2 + b p_{n-1} q_{n-1} + c q_{n-1}^2, \\
    & = a \left( q_{n-1} x + \frac{δ_{n-1}}{q_{n-1}} \right)^2 + b \left( q_{n-1} x + \frac{δ_{n-1}}{q_{n-1}} \right) q_{n-1} + c q_{n-1}^2 \\
    & = \underbrace{(ax^2 + bx + c)}_{ = 0} q_{n-1}^2 + (2ax + b) δ_{n-1} + a \frac{δ_{n-1}^2}{q_{n-1}^2}.
  \end{align*}
  This allows us to bound $A_n$ by
  \begin{align*}
    |A_n| ≤ 2|a||x| + |b| + |a| = (2 |x| + 1) |a| + |b|.
  \end{align*}
  Because $C_n = A_{n-1}$ and the discriminant $D_n$ stays the same, we can
  bound $B_n$ by
  \begin{align*}
    B_n^2 ≤ D_k + 4 |A_n| |C_n| ≤ b^2 + 4 |a| |c| + ((2 |x| + 1)|a| + |b|)^2.
  \end{align*}

  We have shown that for every polynomial the coefficients $A_n, B_n, C_n$ are all bounded,
  therefore there must be some triple $(A_n, B_n, C_n)$ which repeats.
  But the construction of the continued fraction is the same for the same polynomials.
  Hence, the continued fraction of $x$ must be eventually periodic.
\end{proof}

% ==============================================================================
\section{The Geometry behind Continued Fractions}
% ==============================================================================

With the prospect of the generalized Euclidean algorithm, I present a geometric
proof based on Klein \cite{Klein95}.
The basis behind this proof is that instead of viewing the convergents $p_n/q_n$ as
fractions on a 1-dimensional number line,
we view them as two dimensional integer vectors $B_n$, where the first coordinate
represents the numerator $p$ and the second coordinate the denominator $q$.

From Lemma~\ref{lem:wallis},
we directly know that the vectors $B_n$ can be calculated as follows:
\begin{align*}
  B_n =
  \begin{pmatrix}
    p_n \\ q_n
  \end{pmatrix}
  =
  \begin{pmatrix}
    p_{n-1} \\ q_{n-1}
  \end{pmatrix}
  a_n
  +
  \begin{pmatrix}
    p_{n-2} \\ q_{n-2}
  \end{pmatrix}
  = B_{n-1} a_n + B_{n-2}
\end{align*}

\begin{figure}[tb]
  \centering
  \includestandalone{figures/critical-section}
  \caption{
    The hyperbolas each contain only elements with norm $±1$.
    The red area could contain an integral point, in which case this point
    would be closer than the initial point.
    The gray area cannot contain an integral point since the norm would be
    fractional.
  }
\end{figure}

\begin{figure}[tb]
  \centering
  \includestandalone{figures/parallelogram-cover}
  \caption{
    The parallelogram spanned by $B$ and $B'$ contains only $0$ as an integral point.
    The rest of the surface can be covered by identical parallelograms.
    Therefore, the surface can only contain an integral point at $B$.
  }
\end{figure}

We can also begin with a real number $x$ and find its continued fraction using
the vector representation.
We map the vector $x$ to its homogeneous version $\hat x = (x, 1)$.
To find the continued fraction for $x$,
we subtract the integer part $a = \floor{x}$ and invert the fraction.
In the vector representation, this amounts to a translation $T$ of the vector in
the first coordinate and a swap $S$ of the coordinates.
These operations are done with the two matrices:
\begin{align*}
  T_a = \begin{pmatrix}
    1 & a \\
    0 & 1 \\
  \end{pmatrix},
  S = \begin{pmatrix}
    0 & 1 \\
    1 & 0 \\
  \end{pmatrix}
\end{align*}

Importantly, both operations are unimodular; they do not change the underlying lattice.

\begin{align*}
  B_n & = T^{a₀} S T^{a₁} S T^{a₂} … S T^{aₙ} B_1 \\
  & = T^{a₀} \underbrace{S T^{a₁} S}_{S_2^{a_1}} T^{a₂} … S T^{aₙ} B_1
\end{align*}

\begin{align*}
  S_2^a = R S_1 \underbrace{R R}_{I} S_1 \underbrace{R R}_I S_1\, R … R\, S_1 R = R S_1^{a} R.
\end{align*}

\begin{definition}
  A point $(x, y)$ is a \emph{relative minimum} with respect to a line $x - α y = 0$,
  if for every point $(x', y')$ with $x' ≤ x$ \emph{or} $y' ≤ y$, we have
  \[
    |x - α y| ≤ |x' - α y'|.
  \]
\end{definition}

\begin{lemma}
  Every convergent is a relative minimum.
\end{lemma}

\begin{lemma}
  Every relative minimum is a convergent.
\end{lemma}

If we can find an infinite sequence of points $A₁, A₂, …$
such that for every $n ≥ 0$, $A_n = U^n A_1$ for some unimodular matrix $U$
and $A_n$ is a relative minimum,
then the continued fraction must be periodic.
To find these numbers, we turn to Pell's equation.
Pell's equation is defined as
\[
  x^2 - n y^2 = 1.
\]
A trivial solution is $x = 1, y = 0$.
However, we are interested in non-trivial solutions and the following lemma
shows that the equation does have such solutions.

\begin{lemma}
  There are non-trivial integral points which satisfy Pell's equation.
\end{lemma}

\begin{proof}

\end{proof}

\begin{lemma}
  Every integral point of Pell's equation is a relative minimum.
\end{lemma}

\begin{proof}

\end{proof}

\begin{theorem}
  If $x$ is a quadratic irrational,
  then the continued fraction of $x$ is eventually periodic.
\end{theorem}

\begin{proof}
  By the previous lemma, all units $ε_n$ are relative minima.
  There exists a matrix $U$ with unit determinant such that $ε_{n+1} = U^n ε_1$.
  Therefore, the transformation by the continued fractions must also be the same between every unit.
  With the continued fractions, we can only use the positive matrices $T$ and
  $S$ to reach another positive matrix $U$.
  There can only be finitely many possible multiplications to reach $U$.
  Therefore, the coefficients $a₁, a₂, a₃, …$ must repeat at some point.
\end{proof}
