\chapter{Periodic Representation of Quadratic Irrationals}
\label{ch:quadratic}

% TODO: Find a source for this chapter
We begin with the case of quadratic irrationals.
A quadratic irrational is any real number which is a root of a polynomial with degree $2$.
For quadratic irrationals, it is already well known that they can be
represented using periodic continued fractions,
where continued fractions are fractions of the form
\[
  a₀ + \cfrac{1}{a₁ + \cfrac{1}{a₂ + \cfrac{1}{⋱}}}
\]
with $a₀ ∈ ℤ$ and $a₁, a₂, … ∈ ℤ_{> 0}$.
They are periodic when the sequence $a₀, a₁, a₂, …$ eventually repeats.
In this chapter we will prove that the continued fractions of a number are
periodic if and only if the number is a quadratic irrational.
Periodic continued fractions were first analyzed by Euler,
who has shown that such continued fractions must be quadratic irrationals.
Later, Lagrange \cite{Lagrange70} has proven the other direction, that
quadratic irrationals must always have periodic continued fractions.
For the proof, we require many properties from the continued fractions.
Specifically, we will show that continued fractions are always the best
rational approximation of a real number.
The continued fractions can also be interpreted geometrically.
This was initially proposed by Klein \cite{Klein95},
who viewed the continued fractions as points of a lattice.
A similar geometrical interpretation will be used in a later chapter for the
generalization of the continued fractions to higher dimensions.

% ==============================================================================
\section{Continued Fractions}
% ==============================================================================

We begin with a more precise definition of what a continued fraction is and in
particular how infinite continued fractions are defined.
Although continued fractions are typically defined over a sequence of positive integers,
we extend the definition to allow for real numbers.
This will be useful for the theorems which follow.
In general, $r₀, r₁, …$ will denote a sequence of real numbers
and $a₀, a₁, …$ will denote a sequence of positive integers with the exception
of the first entry $a₀$, which can be any integer.

\begin{definition}
  Given a sequence $\{r_n\}_{n ≥ 0}$ of real numbers, the finite continued
  fractions over this sequence are defined inductively as
  \[
    [r₀] = r₀, \qquad
    [r₀; r₁, …, rₙ] = r₀ + \frac{1}{[r₁; r₂, …, rₙ]}.
  \]
  The infinite continued fraction $[r₀; r₁, r₂, …]$ is then defined if the limit
  \[
    r = \lim_{n → ∞} [r₀; r₁, …, rₙ]
  \]
  exists.
  The fraction $[r₀; r₁, …, rₖ]$ is also called the \emph{$k$-th convergent} of $[r₀; r₁, …, r_n]$.
\end{definition}

If the sequence consists of positive integers,
then each convergent $[a₀; a₁, …, aₙ]$ has a unique rational value $pₙ/qₙ$.
This value can be calculated as follows:
For a convergent with only one entry, the convergent is simply $[a₀] = a₀ = a₀/1$.
The convergent of a longer fraction can be calculated by the following:
\[
  [a₀; a₁, …, aₙ]
  = a₀ + \frac{1}{[a₁; a₂, …, aₙ]}
  = a₀ + \frac{\tilde q_{n-1}}{\tilde p_{n-1}}
  = \frac{\tilde p_{n-1} a₀ + \tilde q_{n-1}}{\tilde p_{n-1}},
\]
where $\tilde p_{n-1} / \tilde q_{n-1}$ is the convergent of $[a₁; a₂, …, aₙ]$.
Of course, this requires calculating a convergent of a different continued
fraction.
Instead, we can use a more direct way to calculate the convergent $pₙ/qₙ$ from
the previous convergents $p_{n-1}/q_{n-1}$ and $p_{n-2}/q_{n-2}$ of the same
continued fractions.
To prove this, we will first need the following lemma,
which shows that we can also start from the back of a continued fraction.

\begin{lemma}
  \label{lem:cf-nesting}
  Let $r₀, r₁, …, r_n, x ∈ ℝ$, then
  \[
    [r₀; r₁, …, r_n, x] = [r₀; r₁, …, r_n + 1/x]
  \]
\end{lemma}

\begin{proof}
  If $n = 0$, then
  \[
    [r₀; x] = r₀ + \frac{1}{[x]} = r₀ + \frac{1}{x} = [r₀ + 1/x].
  \]
  Suppose the lemma is true for some $n ≥ 0$, then
  \begin{align*}
    [r₀; r₁, …, rₙ, x]
    & = r₀ + \frac{1}{[r₁; r₂, …, rₙ, x]} \\
    & = r₀ + \frac{1}{[r₁; r₂, …, rₙ + 1/x]} \\
    & = [r₀; a₁, …, rₙ, x]. \qedhere
  \end{align*}
\end{proof}

We can use the following recurrence relation to calculate the convergent $p_n/q_n$:
\begin{align*}
  p_n & = p_{n-1} r_n + p_{n-2}, & p_{-1} & = 1, & p_{-2} & = 0, \\
  q_n & = q_{n-1} r_n + q_{n-2}, & q_{-1} & = 1, & q_{-2} & = 0.
\end{align*}
The correctness is shown by the following lemma.

\begin{lemma}
  \label{lem:cf-wallis}
  Let $x ∈ ℝ$, then
  \[
    [a₀; a₁, …, a_{n-1}, x] = \frac{pₙ}{qₙ} = \frac{p_{n-1} x + p_{n-2}}{q_{n-1} x + q_{n-2}}.
  \]
\end{lemma}

\begin{proof}
  If $n = 0$, then
  \[
    [x] = x = \frac{1x + 0}{0x + 1} = \frac{p_{-1} x + p_{-2}}{q_{-1} x + q_{-2}}.
  \]
  Suppose, the lemma is true for $n ≥ 0$.
  By Lemma~\ref{lem:cf-nesting}, we have
  \begin{align*}
    [a₀; a₁, …, aₙ, x]
    & = [a₀; a₁, …, aₙ + 1/x].
  \end{align*}
  From the induction hypothesis, it follows that
  \begin{align*}
    [a₀; a₁, …, aₙ + 1/x]
    & = \frac{p_{n - 1} (aₙ + 1/x) + p_{n-2}}{q_{n-1} (aₙ + 1/x) + q_{n-2}} \\
    & = \frac{x (p_{n-1} aₙ + p_{n-2}) + p_{n-1}}{x (q_{n-1} aₙ + q_{n-2}) + q_{n-1}} \\
    & = \frac{pₙ x + p_{n-1}}{qₙ x + q_{n-1}}. \qedhere
  \end{align*}
\end{proof}

% ==============================================================================
\section{Construction of Continued Fractions}
% ==============================================================================

% TODO: Should we jump out of the gate with this problem?
The first part of Hermite's question requires a representation of the real numbers.
This means we have to find a unique continued fraction $[r₀; r₁, …]$ for every real number $x$.
Although the definition of a continued fraction allows real numbers,
for the representation we restrict ourselves to continued fractions which only
contain integers.
More specifically, we will only look at continued fractions $[a₀; a₁, a₂, …]$
where $a₀$ can be any integer but $aᵢ$ has to be a positive integer for every $i > 0$.
However, these constraints do not guarantee a unique representation alone.
Consider $x = 3/2$, with the current requirements there are two possible representations:
\[
  x = [1; 1, 1] = 1 + \cfrac{1}{1 + \cfrac{1}{1}} \qquad \text{ or } \qquad x = [1; 2] = 1 + \cfrac{1}{2}.
\]
The issue is that we can always split the last coefficient in the continued fraction.
In general, if $x = [a₀; a₁, …, aₙ]$, then also $x = [a₀; a₁, …, aₙ - 1, 1]$.
Therefore, we additionally require that in a finite continued fraction the last value is never $1$.

We begin with the representation of rational numbers.
We use the Euclidean algorithm to construct the continued fraction for a number $x ∈ ℚ$,
More specifically, if $x = p/q$, then we run the algorithm with the input pair $(p, q)$.
At each step, we keep track of the quotient,
which will be used as a coefficient in the continued fraction.
So if $p = a₀q + r$ is the first division step, then the quotient $a₀$ will be
the first coefficient in the continued fraction $[a₀; a₁, …, aₙ]$ of $p/q$.

\begin{example}
  Consider $x = 13/5$.
  The Euclidean algorithm computes
  \begin{align*}
    13 & = 2 · 5 + 3 \\
     5 & = 1 · 3 + 2 \\
     3 & = 1 · 2 + 1 \\
     2 & = 2 · 1 + 0.
  \end{align*}
  The quotient in each line correspond directly to the continued fraction of $13/5$:
  \[
    \frac{13}{5}
    = [2; 1, 1, 2]
    = 2 + \cfrac{1}{1 + \cfrac{1}{1 + \cfrac{1}{2}}}
    = 2 + \cfrac{1}{1 + \cfrac{2}{3}}
    = 2 + \cfrac{3}{5}
    = \frac{13}{5}.
  \]
\end{example}

\begin{lemma}
  \label{lem:cf-rat}
  Every rational number has a finite continued fraction.
\end{lemma}

\begin{proof}
  We begin by constructing a continued fraction for every rational number.
  Let $p/q$ be a reduced fraction, i.e. $\gcd(p, q) = 1$.
  We proceed by induction over the number of steps when running the
  Euclidean algorithm on $(p, q)$.
  Suppose only one step is required, then $p = a₀ q + 0$.
  Because $p/q$ is a reduced fraction, we have $q = 1$ and $p/q$ is an integer.
  Hence, $[a₀] = a₀ = p$ is a valid continued fraction of $p/q$.

  For $n ≥ 0$, let $p = a₀ q + r$ be the first step of the Euclidean algorithm.
  By induction, suppose that $q/r$ has a finite continued fraction $[a₁; a₂, …, aₙ]$.
  Then,
  \[
    \frac{p}{q}
    = a₀ + \frac{r}{q}
    = a₀ + \frac{1}{\frac{r}{q}}
    = a₀ + \frac{1}{[a₁; a₂, …, aₙ]}
    = [a₀; a₁, …, aₙ].
  \]
  Therefore, there exists a finite continued fraction for every rational number $p/q$.
\end{proof}

We cannot directly use the Euclidean algorithm for irrational numbers.
It is not clear how we would split an irrational number $x$ into two inputs $(p, q)$ for the Euclidean algorithm.
The solution is to look at the ratio between the two inputs in the Euclidean algorithm.
Consider $x = 13/5$ again.
In the first iteration, the ratio is $13/5 = 2 + 3/5$
and in the second iteration the ratio is $5/3$.
So what we are doing is removing the integer part of the fraction $p/q$ and
then continuing with the inverse of the remainder, i.e. $q/r$.
The extension to the real numbers is now straightforward.
Instead of only allowing rational numbers as ratios,
we can use any real number.
This leads to the following algorithm to construct a continued fraction
$[a₀; a₁, …]$ for a given number $x ∈ ℝ$:

\begin{enumerate}
  \item Set $x_0 = x$ and $n = 0$.
  \item Define $a_n = \floor{x_n}$.
  \item Calculate $x_{n+1} = \frac{1}{x_n - a_n}$.
  \item If $x_{n+1} ≠ 0$, increment $n$ and go to step 2.
\end{enumerate}

The terms $x_n$ are also called the \emph{$n$-th complete quotient} of $x$.
The result is a continued fraction $[a₀; a₁, …]$ for each number $x ∈ ℝ$.
Because the continued fraction $[a₀; a₁, …]$ is made up of integers and the
ones after $a₀$ are all positive, its integer part is entirely determined by
the first coefficient $a₀$ and its fractional part is determined by $[a₁; a₂, …]$.

\begin{figure}[tb]
  \centering
  \includestandalone{figures/golden-rectangle}
  \caption{
    The golden rectangle with side lengths $1$ and $φ$.
    Cutting of one unit square from the rectangle gives another rectangle with lengths $φ - 1$ and $1$.
    The ratio between the longer and shorter side remains the same.
  }
  \label{fig:golden-rectangle}
\end{figure}

\begin{example}
  Consider the golden ratio $φ$.
  It is the positive real root of the polynomial $p(x) = x^2 - x - 1$,
  which means that it satisfies $φ = 1 + 1/φ$.
  Since it is positive, we have $1 < φ = 1 + 1/φ < 2$ and $\floor{φ} = 1$.
  This makes the construction of the continued fraction particularly simple since
  \[
    φ = 1 + \cfrac{1}{1 + \cfrac{1}{1 + \cfrac{1}{⋱}}}.
  \]
  So the golden ratio $φ$ is represented by $[1; \overline{1}]$.
\end{example}

For the golden ratio, there is also a geometric interpretation via the golden
rectangle as shown in Figure~\ref{fig:golden-rectangle}.
The golden rectangle has side lengths of $1$ and $φ$,
and the Euclidean algorithm tries to remove as many unit squares inside the
golden rectangle as possible.
In this case, only one square is removed
and we get a new rectangle with side lengths $1$ and $φ - 1$.
But the ratios between the longer and shorter sides remain the same since
\[
  φ^2 - φ - 1 = 0 ⇔ φ(φ - 1) = 1 ⇔ \frac{φ}{1} = \frac{1}{φ - 1}.
\]
Therefore, in the next rectangle only one square of length $φ - 1$ will be
removed again.

\begin{lemma}
  \label{lem:cf-det}
  $p_n q_{n-1} - q_n p_{n-1} = (-1)^{n+1}$ for $n ≥ -1$.
\end{lemma}

\begin{proof}
  For $n = -1$, we have
  \[
    p_{-1} q_{-2} - q_{-1} p_{-2} = 1 - 0 = 1.
  \]
  Suppose that the lemma holds for $n ≥ -1$, then
  \begin{align*}
    p_{n+1} q_n - q_{n+1} p_n
    & = (p_n a_{n+1} + p_{n-1}) q_n - (q_n a_{n+1} + q_{n-1}) p_n \\
    & = p_{n-1} q_n - q_{n-1} p_n \\
    & = (-1) (p_n q_{n-1} - q_n p_{n-1}) \\
    & = (-1)^{n+1}. \qedhere
  \end{align*}
\end{proof}

The lemma already gives us a hint towards the convergence of continued fractions,
because dividing the equation by $q_n q_{n-1}$ results in
\[
  \frac{p_n}{q_n} - \frac{p_{n-1}}{q_{n-1}} = \frac{(-1)^{n+1}}{q_n q_{n-1}}.
\]
Furthermore, $q_n$ and $q_{n-1}$ are always increasing.
Therefore, the distance between consecutive convergents is always decreasing
and the convergents themselves must approach some limit.
In the following lemma, we show that this limit is exactly the number we are trying to represent.

\begin{lemma}
  \label{lem:cf-approx}
  Let $x ∈ ℝ$.
  Suppose $[a₀; a₁, …]$ is the continued fraction constructed from $x$ with convergents $p_n/q_n$.
  Then,
  \[
    \left| x - \frac{pₙ}{qₙ} \right| < \frac{1}{qₙ^2}.
  \]
\end{lemma}

% TODO: Have we proven that (p_{n-1} q_n - p_n q_{n-1}) = (-1)^n yet?
\begin{proof}
  Let $x_n = [a_n; a_{n+1}, …]$.
  Then, $x = [a₀; a₁, …, a_{n-1}, x_n]$ and $a_n = \floor{x_n}$.
  Using Lemma~\ref{lem:cf-wallis},
  we can represent $x$ as well as $p_n/q_n$ using the previous convergents
  $p_{n-1}/q_{n-1}$ and $p_{n-2}/q_{n-2}$ together with the coefficients $x_n$
  and $a_n$, respectively.
  It follows that
  \begin{align*}
    & = \left| \frac{x_n p_{n-1} + p_{n-2}}{x_n q_{n-1} + q_{n-2}} - \frac{a_n p_{n-1} + p_{n-2}}{a_n q_{n-1} + q_{n-2}} \right| \\
    & = \left| \frac{(x_n p_{n-1} + p_{n-2})(a_n q_{n-1} + q_{n-2}) - (x_n q_{n-1} + q_{n-2})(a_n p_{n-1} + p_{n-2})}{(x_n q_{n-1} + q_{n-2})(a_n q_{n-1} + q_{n-2})} \right| \\
    & = \left| \frac{(p_{n-1} q_{n-2} - q_{n-1} p_{n-2})(x_n - a_n)}{((x_n - a_n) q_{n-1} + q_n) q_n} \right|.
  \end{align*}
  Applying Lemma~\ref{lem:cf-det} completes the proof:
  \begin{align*}
    \left| x - \frac{pₙ}{qₙ} \right|
    & = \Biggl| \frac{(-1)^{n+1} \overbrace{(x_n - a_n)}^{≤ 1}}{q_n^2 + \underbrace{(x_n - a_n)}_{≥ 0} q_{n-1} q_n} \Biggr| < \frac{1}{q_n^2}. \qedhere
  \end{align*}
\end{proof}

We now have a continued fraction for both rational number and irrational numbers.
What remains to be shown is that the continued fractions are unique.
This would give us a representation for every real number $x ∈ ℝ$ using a
continued fraction $[a₀; a₁, …]$, which is constructed using the Euclidean
algorithm.

\begin{theorem}
  \label{thm:irrat-cf}
  Every real number $x$ has a unique continued fraction.
\end{theorem}

\begin{proof}
  From the previous considerations, it follows that for every number $x ∈ ℝ$,
  there exists a continued fraction $[a₀; a₁, …]$ such that $[a₀; a₁, …] = x$.
  Suppose there is a different continued fraction $[b₀; b₁, …]$ with $[b₀; b₁, …] = x$.
  Because $[0; a₁, a₂, …]$ and $[0; b₁, b₂, …]$ both lie between $0$ and $1$,
  the continued fractions must share the same first coefficient $a₀ = b₀$.
  Otherwise, they would not have the same integer part and represent different numbers.
  By induction, suppose that the first $n ≥ 0$ terms are equal.
  Then, the continued fractions $[a_{n+1}; a_{n+2}, …]$ and $[b_{n+1}; b_{n+2}, …]$ must be equal.
  But by the same argument, we have $a_{n+1} = b_{n+1}$.
  Therefore, the continued fraction $[a₀; a₁, …]$ is unique.
\end{proof}

% ==============================================================================
\section{The Geometry behind Continued Fractions}
% ==============================================================================

With the prospect of the generalized Euclidean algorithm, I present a geometric
proof based on Klein \cite{Klein95}.
The basis behind his interpretation is that instead of viewing the convergents
$p_n/q_n$ as fractions on a 1-dimensional number line,
we view them as two dimensional integer vectors $b_n$, where the first
coordinate represents the numerator $p_n$ and the second coordinate the
denominator $q_n$.
There are now two different ways to calculate the convergent vector $b_n$.

The first follows directly from the definition of the continued fractions.
Suppose we have the vector $\tilde b_{n-1} = (\tilde p_{n-1}, \tilde q_{n-1})^⊤$
for the continued fraction $[a₁; a₂, …, aₙ]$.
To calculate the convergent of $[a₀; a₁, …, aₙ]$,
we would calculate the reciprocal of $\tilde p_{n-1} / \tilde q_{n-1}$ first
and then add the integer part $a₀$ to it.
The equivalent of calculating the reciprocal in the vector space is swapping the two coordinates
since $\tilde b_{n-1} = (\tilde p_{n-1}, \tilde q_{n-1})^⊤$ is the vector for $\tilde p_{n-1} / \tilde q_{n-1}$
and therefore $(\tilde q_{n-1}, \tilde p_{n-1})^⊤$ must be the vector for $\tilde q_{n-1} / \tilde p_{n-1}$.
The second part is adding some integer part $a₀$ to the reciprocal.
This is equivalent to adding $a₀ q$ to the first coordinates
or skewing the vector using linear transformation $S$.
In fact, the reciprocal can also be done by a matrix transformation $R$.
These matrices are defined as follows:
\[
  S^a =
  \begin{pmatrix}
    1 & 1 \\
    0 & 1 \\
  \end{pmatrix}^a
  =
  \begin{pmatrix}
    1 & a \\
    0 & 1 \\
  \end{pmatrix},
  \quad
  R =
  \begin{pmatrix}
    0 & 1 \\
    1 & 0 \\
  \end{pmatrix}.
\]
The vector $b_n$ can then be calculated using
\[
  b_n = S^{a_0} R \tilde b_{n-1}.
\]

The second way to calculate the convergent vectors follows from Lemma~\ref{lem:cf-wallis}.
We already know that the vectors $b_n$ can be calculated as follows:
\begin{align*}
  b_n =
  \begin{pmatrix}
    p_n \\ q_n
  \end{pmatrix}
  =
  \begin{pmatrix}
    p_{n-1} a_n + p_{n-2} \\ q_{n-1} a_n + q_{n-2}
  \end{pmatrix}
  =
  \begin{pmatrix}
    p_{n-1} \\ q_{n-1}
  \end{pmatrix}
  a_n
  +
  \begin{pmatrix}
    p_{n-2} \\ q_{n-2}
  \end{pmatrix}
  = b_{n-1} a_n + b_{n-2}.
\end{align*}
We combine the previous two vectors in a matrix $B_n = \begin{pmatrix}
  b_{n-1} & b_{n-2} \\
\end{pmatrix}$.
We can calculate the next matrix by multiplying with the matrices $S$ and $R$:
\[
  B_n S^{a_n} R =
  \begin{pmatrix}
    p_{n-1} & p_{n-2} \\
    q_{n-1} & q_{n-2} \\
  \end{pmatrix}
  \begin{pmatrix}
    a_n & 1 \\
    1   & 0 \\
  \end{pmatrix}
  =
  \begin{pmatrix}
    p_{n-1} a_n + p_{n-2} & p_{n-1} \\
    q_{n-1} a_n + q_{n-2} & q_{n-1} \\
  \end{pmatrix}
  =
  B_n.
\]
Since $\det(S^a) = \det(S)^a = 1$ and $\det(R) = -1$, it is straightforward to
see that $B_n$ always has determinant $±1$.
But this was proven already by Lemma~\ref{lem:cf-det}.

\begin{figure}[tb]
  \centering
  \includestandalone{figures/klein-polygon}
  \caption{
    A Klein polygon for $\sqrt{2}$.
  }
  \label{fig:klein-polygon}
\end{figure}

This can also be interpreted geometrically, as shown in Figure~\ref{fig:klein-polygon}.
We begin with the unit vectors $b_{-1} = (1, 0)$ and $b_{-2} = (0, 1)$.
To calculate the next vector, we start at the point $b_{-2} = (0, 1)$ and draw a line using the vector $b_{-1} = (1, 0)$.
This line intersects the irrational line $(1, α)$ at some point
and the next vector $b_0$ is the integer lattice point on the line before the
intersection.
We continue with $b_{-1}$ and draw a line using the vector $b_0$.
The next vector $b_2$ is the integer point before the intersection with $(α, 1)$, again.

The even and odd vectors each form a polygonal chain which gets closer and
closer to the irrational line.
In his original lecture notes, Klein showed that the vectors are indeed the
closest points to the irrational line $α$ and furthermore he has shown the converse:
The closest point are either convergents or they lie on a line between two
convergents.

\begin{definition}
  A point $(p, q) ∈ ℤ^2 \setminus \{(0, 0)\}$ is a \emph{relative minimum} with
  respect to an irrational line $(α, 1)$, if for every point $(p', q') ∈ ℤ^2$
  with $p' ≤ p$ \emph{or} $q ≤ q'$, we have
  \[
    |q α - p| ≤ |q' α - p'|.
  \]
\end{definition}

\begin{figure}[tb]
  \centering
  \includestandalone{figures/parallelogram-cover}
  \caption{
    The parallelogram spanned by $B$ and $B'$ contains only $0$ as an integral point.
    The rest of the surface can be covered by identical parallelograms.
    Therefore, the surface can only contain an integral point at $B$.
  }
\end{figure}

\begin{theorem}
  \label{thm:conv-is-relmin}
  Every convergent is a relative minimum.
\end{theorem}

\begin{proof}
  For the first two convergents, this is obviously true.
  By induction, suppose that the convergent $b_{n-1}$ and $b_n$ are relative minima.
  Reflecting $b_{n+1}$ across the line $α$ produces two parallel lines:
  One going through $b_{n+1}$ and one going through the reflected point.
  Any point which is closer must lie in the cylinder between the two lines.
  Next, we construct a parallelogram from the origin and the points $B_{n-1}, B_n$ and $B_{n+1}$.
  There are two cases for this parallelogram:
  But the parallelogram only contains
  \begin{align*}
    |\det \begin{pmatrix}
      a B_n & B_{n-1} \\
    \end{pmatrix}|
    & = a_{n+1} |\det \begin{pmatrix}
      B_n & B_{n-1} \\
    \end{pmatrix}| \\
    & = a_{n+1}
  \end{align*}
  integer lattice points and all of these point are on the line between $B_{n-1}$ and $B_{n+1}$.
  Therefore, $B_{n+1}$ is also a relative minimum.
\end{proof}

% TODO: Technically, they can also be a lattice point between two convergents.
\begin{theorem}
  \label{thm:relmin-is-conv}
  Every relative minimum is a convergent.
\end{theorem}

\begin{proof}

\end{proof}

% ==============================================================================
\section{Continued Fractions of Quadratic Irrationals}
% ==============================================================================

% TODO: Write introduction for this section
This section is about the second part of Hermite's question for quadratic
irrationals, i.e. whether the continued fraction of a number is periodic if and only
if the number is a quadratic irrational.
Formally, we call a continued fraction $[a₀; a₁, …]$ \emph{periodic}
if there exists a starting index $K ≥ 0$ and a period $ℓ ≥ 1$ such that $aₖ = a_{k+ℓ}$ for every $k ≥ K$.
A continued fraction is called \emph{purely periodic} if $K = 0$,
i.e. the period starts immediately.
For a periodic continued fraction starting at $K$ with length $ℓ$,
we will denote it as $[a₀; a₁, …, a_{K-1}, \overline{a_K, …, a_{K+ℓ}}]$.
This is similar to how in decimal notation, we denote a period with a bar over the digits,
e.g. $1/3 = 0.\overline{3}$.
In a continued fraction, we similarly denote the period with a line over the
coefficients that are infinitely repeated.

\begin{example}
  We construct the continued fraction for $x = \sqrt{2}$.
  Because the root lies between $1$ and $2$, the first term must be $a₀ = 1$.
  The next complete quotient is
  \[
    x₁ = \frac{1}{\sqrt{2} - 1} = \frac{\sqrt{2} + 1}{(\sqrt{2} - 1)(\sqrt{2} + 1)} = \sqrt{2} + 1.
  \]
  It lies between $2$ and $3$,
  so the next term is $a₁ = 2$.
  However, the next complete quotient is the same, since
  \[
    x₂ = \frac{1}{\sqrt{2} - 1} = \sqrt{2} + 1 = x₁.
  \]
  Therefore, terms will repeat and the resulting continued fraction for $\sqrt{2}$ is $[1; \overline{2}]$.
\end{example}

For the proof, we have to show two directions.
The first is that every periodic continued fraction is a quadratic irrational
and the second is that every quadratic irrational has a periodic continued fraction.
We begin with the former and we will use the geometry from the previous section
in the proof.

\begin{theorem}
  If the continued fraction of $x ∈ ℝ$ is periodic, then $x$ is a quadratic irrational.
\end{theorem}

\begin{proof}
  Let $x$ be a continued fraction $[a₀; a₁, …]$, which is purely periodic.
  Then, there exists a complete quotient such that $x = x_ℓ = [a_ℓ; a_{ℓ+1}, …]$ for some $ℓ ≥ 1$.
  For each pair of convergent vectors $b_{n-1}, b_{n-2}$,
  we can use the complete quotient $x_ℓ$ to find an intersection point with the
  line $(x, 1)$ and since $x_ℓ = x$, we can also use $x$.
  Stated differently, there exists a $λ ∈ ℝ$ such that
  \[
    B_ℓ
    \begin{pmatrix}
      x \\
      1 \\
    \end{pmatrix}
    =
    b_{ℓ-1} x + b_{ℓ-2}
    = λ
    \begin{pmatrix}
      x \\
      1 \\
    \end{pmatrix}
  \]
  Therefore, $(x, 1)^⊤$ is an eigenvector of the matrix $B_ℓ$
  and $λ$ is the eigenvalue.
  Because $B$ is a 2-by-2 matrix,
  the eigenvalue can only be a quadratic irrational.
  The eigenvector is a solution to the linear system $(B_ℓ - λ I_2) (x, 1) = 0$,
  where the coefficients come from the field $ℚ(λ)$.
  Therefore, $x$ must be a quadratic irrational.

  We proceed with the eventually-periodic case
  where some index $k ≥ 0$ exists after which the period begins.
  By Lemma~\ref{lem:cf-wallis},
  \[
    x = \frac{p_{k-1} x_k + p_{k-2}}{q_{k-1} x_k + q_{k-2}}.
  \]
  Because $x$ is a rational expression of $x_k$,
  it lives in the same field as $x_k$.
  The continued fraction for $x_k$ is purely periodic, so $x_k$ is a quadratic irrational.
  But then $x$ is one, too.
\end{proof}

% TODO: Figure for shift of Klein polygon
The converse was originally proven by Lagrange \cite{Lagrange70}.
Here, a proof by Korkina \cite{Korkina96} is presented,
which uses the geometrical interpretation of Klein to show periodicity.
The idea is that for quadratic irrationals,
there always exists a matrix $U$ which shifts the boundary of a Klein polygon
in the positive direction and preserves volume.
Because the volume between two vectors $b_n, b_{n-2}$ directly corresponds to
the coefficients $a_0, a_1, …$, they must repeat at some point.

The first task is to show that such a matrix actually exists.
We show this by interpreting the convergent vectors as members of the
quadratic field $ℚ(x)$.
The number $x$ is a quadratic irrational, so $ℚ(x)$ is isomorphic to a
two-dimensional vector space.
Each vector $b_n$ then has an associated multiplication matrix $M_n$,
where determinant of this matrix is the algebraic norm $N(b_n)$ of the vectors.
We show that this norm is bounded independently of $n$.

\begin{lemma}
  The norm $N(p_n + q_n x)$ is bounded by some constant $c$.
\end{lemma}

\begin{proof}
  Suppose $x$ satisfies $Ax^2 + Bx + C = 0$ with $A, B, C ∈ ℤ$.
  Then, The norm for any element $p + qx ∈ ℚ(x)$ is
  \[
    N(p + qx) = Ap^2 + Bpq + Cq^2.
  \]
  By Lemma~\ref{lem:cf-approx}, there exists some $δₙ ∈ ℝ$ with $|δₙ| < 1$
  such that $pₙ = qₙ x + \frac{δₙ}{qₙ}$.
  Therefore, we can bound the norm by
  \begin{align*}
    |N(p_n + q_n x)|
    & = |A pₙ^2 + B pₙ qₙ + C qₙ^2| \\
    & = \left|A \left(qₙ x + \frac{δₙ}{qₙ} \right)^2 + B \left(qₙ x + \frac{δₙ}{qₙ} \right) + C qₙ^2\right| \\
    & = \left|\underbrace{(Ax^2 + Bx + C)}_{= 0} qₙ^2 + 2 A δₙ x + B \frac{δₙ}{qₙ} + A \frac{δₙ^2}{qₙ^2} \right| \\
    & \le (2|x| + 1)|A| + |B|. \qedhere
  \end{align*}
\end{proof}

From the lemma, it follows that there must be two different vectors $b_n, b_m$,
which have the same norm.

\begin{corollary}
  For every quadratic irrational $x$, there exists an integer matrix $U ≠ I_2$
  with determinant $1$, which has eigenvectors $(x, 1)$ and $(\overline{x}, 1)$.
\end{corollary}

\begin{example}
  \label{ex:sqrt2-unit}
  Consider $x = \sqrt{2}$.
  The fundamental unit is $3 + 2\sqrt{2}$ since $3^2 - 2 · 2^2 = 1$.
  It has the multiplication matrix
  \[
    U = \begin{pmatrix}
      3 & 4 \\
      2 & 3 \\
    \end{pmatrix},
  \]
  which has determinant $1$ by definition.
  Furthermore,
  \[
    \begin{pmatrix}
      3 & 4 \\
      2 & 3 \\
    \end{pmatrix}
    \begin{pmatrix}
      \sqrt{2} \\ 1 \\
    \end{pmatrix}
    =
    \begin{pmatrix}
      4 + 3\sqrt{2} \\
      3 + 2\sqrt{2} \\
    \end{pmatrix}
    =
    (3 + 2\sqrt{2})
    \begin{pmatrix}
      \sqrt{2} \\ 1 \\
    \end{pmatrix}
  \]
  and
  \[
    \begin{pmatrix}
      3 & 4 \\
      2 & 3 \\
    \end{pmatrix}
    \begin{pmatrix}
      -\sqrt{2} \\ 1 \\
    \end{pmatrix}
    =
    \begin{pmatrix}
      4 - 3\sqrt{2} \\
      3 - 2\sqrt{2} \\
    \end{pmatrix}
    =
    (3 - 2\sqrt{2})
    \begin{pmatrix}
      -\sqrt{2} \\ 1 \\
    \end{pmatrix}.
  \]
  Therefore, it has $(\sqrt{2}, 1)$ and $(-\sqrt{2}, 1)$ as eigenvectors.
\end{example}

\begin{theorem}
  If $x$ is a quadratic irrational,
  then its continued fraction is periodic.
\end{theorem}

\begin{proof}
  For any quadratic irrational $x$ with conjugate $\overline{x}$, we can find a
  non-identity matrix $U ∈ ℤ^{2×2}$ with $\det(U) = 1$ which has $(1, x)$ and
  $(1, \overline{x})$ as eigenvectors of this matrix.
  Because the Klein polygon $K$ is the set of integer points in the cone
  spanned by these eigenvectors, it must be invariant under this transformation.
  Similarly, the boundary $Π(K)$ is also invariant under this transformation.
  However, $A$ is not the identity, so the matrix $A$ must shift the points along the boundary.
  We assume that it shifts it in the positive direction,
  i.e. $B_{n+k} = A B_n$ for some $k ≥ 1$.
  If not, then we can choose $A^{-1}$ to shift them in the positive direction.
  Because $\det(A) = 1$, the matrix $A$ preserves volume, so
  \[
    a_{n+k}
    = \det\begin{pmatrix}
      b_{n+k} & b_{n+k-2}
    \end{pmatrix}
    = \det(A) \det\begin{pmatrix}
      b_n & b_{n-2}
    \end{pmatrix}
    = a_n.
  \]
  Hence, the continued fraction is periodic after some point.
\end{proof}

% TODO: Define T_a
\begin{example}
  Consider the matrix $U$ from Example~\ref{ex:sqrt2-unit}.
  The matrix has $(\sqrt{2}, 1)$ and $(-\sqrt{2}, 1)$ as eigenvectors
  and it has determinant $1$.
  Furthermore, the matrix shifts each convergents by two,
  because
  \[
    B_n = B_{n-2} S^2 R S^2 R
    \qquad
    \text{ and }
    \qquad
    S^2 R S^2 R =
    \begin{pmatrix}
      1 & 2 \\
      1 & 1 \\
    \end{pmatrix}^2
    =
    \begin{pmatrix}
      3 & 4 \\
      2 & 3 \\
    \end{pmatrix}.
  \]
  So, $B_{n+2} = U B_n$ for every $n ≥ 1$.
\end{example}

%===============================================================================
%begin{old proof}
\iffalse
The converse was originally proven by Lagrange \cite{Lagrange70}.
We construct the continued fraction for a particular quadratic irrational using
the Euclidean algorithm.
When running the Euclidean algorithm,
the only possible numbers it can produce must be of the form
\[
  p + \sqrt{D} q
\]
since the Euclidean algorithm can only subtract a constant factor of either an
integer or the quadratic irrational.
These numbers lie in a special field called a quadratic field.
A quadratic field, denoted as $ℚ(\sqrt{D})$, is a field extension of the rational numbers.
It is the smallest field that contains $\Q$ as a subfield and $\sqrt{D}$.
This means that it must contain all numbers $p + \sqrt{D} q$ with $p, q ∈ ℚ$.

A different interpretation of the field extension would be as a two-dimensional
vector space over the rational numbers.
An element $p + \sqrt{D} q ∈ ℚ(\sqrt{D})$ would be seen as the vector $(p, q) ∈ ℚ^2$.
Addition of two elements would correspond to vector addition
and multiplication with a rational number would correspond to a scalar multiplication.
Multiplication of two elements $a, b ∈ ℚ(\sqrt{D})$ is more complicated.
If we fix one element $a = a₀ + \sqrt{D} a₁$, then multiplication is a linear operation:
\[
  (a₀ + \sqrt{D} a₁)(b₀ + \sqrt{D} b₁) = (a₀ b₀ + D a₁ b₁) + \sqrt{D} (a₀ b₁ + a₁ b₀)
\]
which is equivalent to
\[
  \begin{pmatrix}
    a₀ & D a₁ \\
    a₁ & a₀ \\
  \end{pmatrix}
  \begin{pmatrix}
    b₀ \\
    b₁ \\
  \end{pmatrix}
  =
  \begin{pmatrix}
    a₀ b₀ + D a₁ b₁ \\
    a₀ b₁ + a₁ b₀ \\
  \end{pmatrix}.
\]

\begin{definition}
  The norm $N(α)$ of an number $α$ in a quadratic field $ℚ(\sqrt{D})$ is defined as
  the determinant of the multiplication matrix of $α$.
\end{definition}

The norm is an instance of Pell's equation.

\begin{example}
  Consider $x = \sqrt{2}$, which has the following convergents:
  \[
    \binom{0}{1}, \binom{1}{0}, \binom{1}{1}, \binom{3}{2}, \binom{7}{5}, \binom{17}{12}, \binom{41}{29}, …
  \]
  If the norm is bounded, then there must be two elements with the same bound
  and indeed the second and fourth elements have the same bound:
  \[
    N(1 + 0\sqrt{2}) = 1 = 3^2 - 2 · 2^2 = N(3 + 2\sqrt{2}).
  \]
  So $(3 + 2\sqrt{2})(1 + 0\sqrt{2})$ is a convergent,
  but then $(3 + 2\sqrt{2})^n(1 + 0\sqrt{2})$ must also be a convergent for every $n ≥ 0$.
  For example, we have for $n = 2$,
  \[
    (3 + 2\sqrt{2})^2(1 + 0\sqrt{2})
    = 9 + 12\sqrt{2} + 8
    = 17 + 12\sqrt{2}
  \]
  and $(17, 12)$ is in fact one of the convergents.
  There are other elements with the same norm.
  For example, $(1, 1)$ and $(7, 5)$ have the same bound:
  \[
    N(1 + \sqrt{2}) = 1^2 - 2 · 1^2 = -1 = 49 - 50 = 7^2 - 2 · 5^2 = N(7 + 5 \sqrt{2}).
  \]
  Furthermore, we can multiply $1 + \sqrt{2}$ with $3 + 2\sqrt{2}$ to get to $7 + 5\sqrt{2}$:
  \[
    (3 + 2 \sqrt{2}) (1 + \sqrt{2}) = 3 + 2\sqrt{2} + 3\sqrt{2} + 4 = 7 + 5 \sqrt{2}.
  \]
\end{example}

\begin{theorem}
  If $x$ is a quadratic irrational,
  then its continued fraction is periodic.
\end{theorem}

\begin{proof}
  By the previous lemma, the norm of any convergent is always bounded.
  Therefore, there must be at least two different convergents $Bᵢ$ and $Bⱼ$ with $N(Bᵢ) = N(Bⱼ)$.
  It follows that there exists some fundamental unit $ε ∈ ℚ(\sqrt{D})$ with $ε
  ≠ 1$ and $N(ε) = \det(M_ε) = 1$ such that
  \[
    M_ε Bᵢ = Bⱼ.
  \]
\end{proof}

Although the previous lemma already shows that continued fractions have a bounded norm,
we can go even further.
In fact, the convergent of a continued fraction reach all fundamental
units, i.e. a non-trivial vector with norm $±1$.
The proof for this follows from the theorems shown by Klein.
We show that every fundamental unit is a relative minimum,
from which it follows that every fundamental unit is also a convergent.

\begin{figure}[tb]
  \centering
  \includestandalone{figures/critical-section}
  \caption{
    The hyperbolas each contain only elements with norm $±1$.
    The red area could contain an integral point, in which case this point
    would be closer than the initial point.
    The gray area cannot contain an integral point since the norm would be
    fractional.
  }
  \label{fig:critical-section}
\end{figure}

\begin{proposition}
  Every fundamental unit is a relative minimum.
\end{proposition}

\begin{proof}
  Let $ε = (ε₀, ε₁)$ be a fundamental unit and let $δ = |ε₀ α - ε₁|$ denote the
  distance of $ε$ to the line $α$.
  Suppose there is a point $(x, y)$ which is closer than $ε$.
  The point cannot be on the same side of the line $α$ as $ε$,
  since any unit strays further away from the line
  and therefore also any element with larger norm strays further.
  Therefore, $(x, y)$ and $ε$ must lie on different sides.

  For the point $(x, y)$ to be closer, it must satisfy
  \[
    |x α - y| < |ε₀ α - ε₁| \text{ and } |N(x, y)| ≤ |N(ε)|.
  \]
  This area is also shown in Figure~\ref{fig:critical-section}
  and any lattice point which is closer must be inside this critical area.
  However, no such lattice point exists.
  The area is actually encompassed by the parallelogram induced by the
  multiplication matrix of $ε$.
  Since this multiplication matrix has determinant $±1$, there cannot be a
  lattice point inside the parallelogram.
\end{proof}

\begin{corollary}
  Every fundamental unit is a convergent.
\end{corollary}
\fi
%end{old proof}
%===============================================================================
