\chapter{Conclusion \& Outlook}
\label{ch:conclusion}

The main objective of this thesis
was to investigate Hermite's question
on a new representation of real numbers,
which is periodic if and only if the number is algebraic.
The basis for this representation has been a generalization of Euclidean
algorithm by \citeauthor{Klein24}.
Throughout this thesis,
we have seen that many concepts from the Euclidean algorithm naturally carry
over to its generalization, such as the Fibonacci numbers or continued
fractions.

For the Fibonacci numbers,
we saw that different strategies lead to different definitions of Fibonacci numbers.
The first example of such numbers was for the minimum strategy,
which chooses the smallest fractional value.
For this strategy, the $d$-bonacci numbers represent the worst case,
just like the Fibonacci numbers represent the worst case for the classical
Euclidean algorithm.
The second strategy improved upon this strategy by minimizing the ratio between two values over two iterations.
For this strategy, its golden ratio and the corresponding Fibonacci numbers
already came up when bounding the decrease of the determinant over just two iterations.
Although the golden ratio is not periodic under the strategy itself,
we have seen that the maximum strategy leads to periodicity.
These two roots were the first examples of periodic inputs for the generalized Euclidean algorithm.

Although Hermite's question remains unanswered,
this thesis provides a different perspective on Jacobi-Perron-type algorithms.
I have shown that many types of Jacobi-Perron algorithms are converging
under the condition that every index is used during the construction and that
the same indices are not too far apart.
This covers the first part of Hermite's question.
The second part of Hermite's question has been partially resolved in Section~\ref{sec:mcf-periodic},
where we have seen that any periodic MCF leads to an algebraic number.

What is missing is the second direction,
whether every algebraic number has a periodic representation.
Although in the scope of this thesis,
I was not able to show this,
the analysis in Chapter~\ref{ch:implementation}
strongly suggests that this is indeed the case.
In particular, the strategy by \citeauthor{Tamura09} has shown promise for
finding periodic representation of cubic and quartic irrationals.
The authors have already proven that their algorithm is periodic for certain
classes of cubic polynomials.
Future research into this algorithm could possibly lead to a periodic sequence
for cubic and quartic irrationals.

There are other directions, one can pursue for an answer of Hermite's question.
One promising direction is the generalization of Klein polygons to higher dimensions.
In Chapter~\ref{ch:quadratic},
we have seen a geometric proof of Lagrange's theorem,
which builds upon the notion of Klein polygons.
A generalization of Klein polygons does exist in higher dimensions
and they are known as Klein polyhedra.
Furthermore, there already exists a generalization of Lagrange's theorem by \citeauthor{German08}.
However, the missing link is the connection between the convergents and the
extreme points of such Klein polyhedra.
For a proof, one would have to show that the extreme points are equivalent to
the convergents of the MCFs.
Section~\ref{sec:mdcf-geometry} provides the necessary foundation for this.

% TODO: Add note about how approximation doesn't work well with the multi-dimensional continued fractions
