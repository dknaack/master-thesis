\chapter{Conclusion and Outlook}
\label{ch:conclusion}

The main objective of this thesis
was to investigate Hermite's question
on a new representation of real numbers,
which is periodic if and only if the number is algebraic.
The basis for this representation has been a generalization of the Euclidean
algorithm by \citeauthor{Klein24}.
Specifically, I used a small subroutine from the algorithm,
which updates the previous solution vector instead of solving a linear system
at each iteration.
This subroutine served as the basis for the multidimensional continued fractions.
Although these were not enough to fully solve Hermite's question,
we saw that many concepts from the Euclidean algorithm naturally carry
over to its generalization.

Beginning with the Fibonacci numbers,
we saw that different strategies led to different definitions of Fibonacci numbers.
The first example of such numbers was for the minimum strategy,
which chooses the smallest fractional value.
For this strategy, the $d$-bonacci numbers represent the worst case,
just like the Fibonacci numbers represent the worst case for the classical
Euclidean algorithm.
The second strategy improves upon the first by minimizing the ratio between two values over two iterations.
For this strategy, its golden ratio and the corresponding Fibonacci numbers
already came up when bounding the decrease of the determinant over just two iterations.
Although the golden ratio is not periodic under the strategy itself,
we saw that it becomes periodic under the maximum strategy.
These two roots were the first examples of periodic inputs for the generalized
Euclidean algorithm.

These examples led to the development of multidimensional continued fractions
as presented in Chapter~\ref{ch:mdcf}.
They are essentially the generalized Euclidean algorithm, but in reverse.
Since the algorithm allows us to choose a pivot index at each iteration,
there does not exist a unique multidimensional continued fraction for every vector.
However, this choice provides more flexibility.
Instead of focusing on a single algorithm, the multidimensional continued
fractions unify many Jacobi-Perron-type algorithms under a single framework.
Additionally, they share many properties with classical continued fractions.
For example, the linear recurrence for continued fractions proven in Lemma~\ref{lem:cf-wallis}
has an equivalent sequence proven in Lemma~\ref{lem:mdcf-wallis} for
multidimensional continued fractions.

The convergence of multidimensional continued fractions was established in Section~\ref{sec:mcf-convergence}.
The proof is considerably more complex than for continued fractions.
However, it still shows that the multidimensional continued fractions converge
under any algorithm, which selects all possible pivot indices and ensures these
indices not too far apart.
This covers algorithms like the classical Jacobi-Perron algorithm as well as
other algorithms that select the pivot indices based on a fixed list.
Most importantly, it establishes the first part of Hermite's question.
It shows that many algorithms yield a representation of real numbers
using the multidimensional continued fractions.

While the first part of Hermite's question has been solved,
the second part is only partially solved.
Whereas the first part only asks of a representation of real numbers,
the second part states that the representation should be periodic if and only
if the number is a cubic irrational (or algebraic number, in general).
The first direction has been proven in Section~\ref{sec:mcf-periodic}.
The proof is based on eigenvectors and shows that if the MCF of an irrational
vector $x$ is purely periodic, then $x$ is an eigenvector of the convergent
matrix.

Theorem~\ref{thm:unimodular-algebraic} suggests that we can always find a
unimodular matrix with $x$ as one of its eigenvectors,
thereby indicating that we should be able to find a periodic MCF.
However, I was not able to show this in the scope of this thesis.
Nevertheless, the analysis in Chapter~\ref{ch:implementation}
strongly supports the conjecture.
In particular, the algorithm proposed by \citeauthor{Tamura09} has shown promise for
finding periodic representations of cubic and quartic irrationals.
The authors have already proven that their algorithm is periodic for certain
classes of cubic polynomials.
Future research into this algorithm could potentially yield a periodic sequence
for all cubic and quartic irrationals.

In the case of continued fractions,
we were able to show that they are periodic only if the number is a quadratic irrational.
The proof uses Klein polygons, which are preserved under a specific unimodular transformation
and we can find such a transformation for every quadratic irrational.
We can find a similar transformation for all algebraic numbers, but for the
multidimensional continued fractions a formal connection to Klein polytopes has
yet to be established.
Section~\ref{sec:mdcf-geometry} already provides the necessary foundation for this.
Furthermore, a generalization of Lagrange's theorem has already been proven by
\citeauthor{German08}~\cite{German08}.

% TODO: Add note about how approximation doesn't work well with the multi-dimensional continued fractions
