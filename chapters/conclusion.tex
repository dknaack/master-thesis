\chapter{Conclusion \& Outlook}
\label{ch:conclusion}

With this thesis,
I have shown that many Jacobi-Perron-type algorithms are converging
and that for all of these algorithms,
if they become periodic for some input vector,
then that input must consist of algebraic numbers of degree $≤ d+1$.
This covers the first part of Hermite's question on the representation
and one direction of the second part.
However, what is missing is the second direction,
whether every algebraic number has a periodic representation.
Although I was not able to prove this,
the analysis in Chapter~\ref{ch:implementation} strongly suggests that this is
indeed the case.
In particular, the strategy by Tamura and Yasutomi has shown promise for
finding periodic representation of cubic and quartic irrationals.

Another possibility is the generalization of Klein polygons.
In Chapter~\ref{ch:quadratic},
we have seen a geometric proof of Lagrange's theorem,
which builds upon the notion of Klein polygons.
A generalization of Klein polygons does exist in higher dimensions
and they are known as Klein polyhedra.
Furthermore, there exists a generalization of Lagrange's theorem \cite{German08}.
However, the missing link is the connection between the convergents and the
extreme points of such Klein polyhedra.
For a proof, one would have to show that the extreme points are equivalent to
the convergents of a Klein polyhedra.
Section~\ref{sec:mdcf-geometry} already provides the necessary foundation for this.

% TODO: Add note about how approximation doesn't work well with the multi-dimensional continued fractions
