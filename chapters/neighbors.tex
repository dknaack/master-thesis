\chapter{Choosing the Closest Pair of Neighbors}

In this chapter, the following strategy is analyzed:

\begin{enumerate}
  \item Sort $x$ in increasing order.
  \item Find the index $ℓ$ which minimizes $\{x_{ℓ+1}/x_ℓ\}$.
  \item Choose $ℓ$ in the first iteration and $ℓ + 1$ in the second iteration.
\end{enumerate}

In the case that $ℓ = d$, we define $x_{ℓ + 1} := 1$.

\section{How Fast does the Determinant Decrease?}

To analyze the decrease of this strategy over two iterations,
we assume that each choice of our pivot yields the same decrease.
Hence, we have
\begin{equation}
  %\label{eq:equal-neighbor-distance}
  x_1 = \frac{x_2}{x_1} - 1 = \frac{x_3}{x_2} - 1 = \dots = \frac{x_d}{x_{d-1}} = \frac{1}{x_d} - 1.
\end{equation}
Solving the equation for $x_d$ yields the following solution:
\[
  x_d^{d+1} + x_d - 1 = 0 \text{ and } x_i = x_d^{d+1-i} \text{ for } i < d.
\]

The value $x_d$ is the real positive root of a polynomial $p(x)$
and the other values are chosen according to $x_d$.

The following lemma helps us calculate the fractional value of $1/ψ$.

\begin{lemma}
  For every $d ≥ 1$, we have $1/ψ < 2$.
\end{lemma}

\begin{proof}

\end{proof}

\begin{theorem}
  The determinant decreases by at least $ψ^{d+1}$ over two iterations with $d ≥ 2$.
\end{theorem}

\begin{proof}
  We assume WLOG that the vector is sorted in increasing order and
  $0 ≤ x_i ≤ 1$ for every $i ≤ d$.
  For a contradiction, assume the algorithm yields a smaller decrease than $ψ^{d+1}$ on input $x$.
  We must have $x_i > ψ^{d+1-i}$ for every $i = 1, \dots, d$.
  For the first value, we have $x₁ > ψ^d$, because otherwise we have a total decrease of
  \[
    x₁ \left\{ \frac{x₂}{x₁} \right\} ≤ ψ^{d-1} - ψ^d = ψ^{2d+1} < ψ^d.
  \]
  Suppose $x_i > ψ^{d+1-i}$ and $x_{i+1} ≤ ψ^{d-i}$.
  Then we can achieve a total decrease of
  \[
    x_i · \left\{ \frac{x_{i+1}}{x_i} \right\} ≤ x_{i+1} - x_i < ψ^{d-i} - ψ^{d+1-i} = ψ^{2d+1-i} < ψ^d.
  \]
  It follows that $x_i > ψ^{d+1-i}$ for every $i ≤ d$.
  But then
  \[
    x_d \left\{ \frac{1}{x_d} \right\} ≤ 1 - x_d < 1 - ψ = ψ^{d+1}.
  \]
  Hence, we achieve a decrease of at least $ψ^{d+1}$ over two iterations.
\end{proof}

The bound for this strategy is tight.
We can construct an input that moves arbitrarily close to this bound.
Formally, for every sufficiently small $ε > 0$, we can find an input $x$ which
achieves a decrease of at most $ψ^{d+1} - ε$.
The idea is to choose $x_d$ to be just over $ψ$ and all other variables as a multiple of $x_d$
such that we have to choose $x_d$.
But choosing $x_d$ only gives us a decrease of $ψ^{d+1} - ε$ in total.

% TODO: Explain what sufficiently small means
\begin{theorem}
  For every (sufficiently small) $ε > 0$,
  there exists an input $x ∈ ℝ^d$ with $d ≥ 2$,
  which achieves a decrease in the determinant of exactly $ψ^{d+1} - ε$ over two
  iterations.
\end{theorem}

\begin{proof}
  We choose $x_i = ψ^{d+1-i} (1 + ε)$ for $i = 1, \dots, d$.
  The strategy chooses between
  \[
    \left\{ \frac{x_{i+1}}{x_i} \right\}
    = \frac{ψ^{d-i} (1 + ε)}{ψ^{d+1-i} (1 + ε)} - 1 = \frac{1}{ψ} - 1,
    \text{ and }
    \left\{ \frac{1}{x_d} \right\}
    = \frac{1}{ψ + ε} - 1
  \]
  Clearly, the strategy chooses $x_d$ since its ratio $\{1/x_d\}$ is the smallest.
  Therefore, the total decrease over two iterations is
  \[
    (ψ + ε) \left\{ \frac{1}{ψ + ε} \right\} = 1 - ψ - ε = ψ^{d+1} - ε.
    \qedhere
  \]
\end{proof}
