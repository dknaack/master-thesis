\chapter{Experimental Analysis on Multi-Dimensional Continued Fractions}
\label{ch:implementation}

In the previous chapter, we have analyzed
In particular, we saw that most MDCFs converge,
and that any periodic MDCF consists of algebraic numbers.
The former solves the first part of Hermite's question, but the latter solves
only one direction of the second part.
The first part of this chapter will focus on the remaining direction:
whether every algebraic number admits a periodic MDCF.

We have already seen in Chapter~\ref{ch:fibonacci},
that the simplest periodic MDCFs can be considered a generalization of the
golden ratio.
The first part of this chapter extends this result by presenting further
examples of periodic MDCF for algebraic numbers.
These include a wide range of cube roots,
all of which appear to have periodic representations under one particular strategy.
While this does not amount to a proof, the evidence strongly supports
the possibility of a positive answer to the second part of Hermite’s question.

The second part of this chapter focuses on the approximation rate of MDCFs.
For ordinary continued fractions, the convergents are known to give
exceptionally good approximations to irrational numbers.
Here, we test whether MDCFs offer similar approximation behavior in higher dimensions.

\iffalse
% ==============================================================================
\section{Implementation details}
% ==============================================================================

% ==============================================================================
\begin{Python}[
    float=tbp,
    numbers=left,
    label={lst:bfs},
    caption={
      The implementation of the brute-force search for finding a periodic representation.
      The program iterates over all sequences with a maximum length of $N$
      until it finds a duplicate vector.
    }
  ]
def brute_force_search(x, N):
  d = len(x)
  indices = list(range(d))
  for n in range(N):
    for L in product(indices, repeat=n):
      y = x
      seen = {y: 0}
      for i in range(n):
        y = pivot(y, L[i])
        if y in seen:
          j = seen[y]
          start = L[:j]
          period = L[j:i+1]
          return start, period
        seen[y] = i + 1
\end{Python}
% ==============================================================================

% Brute-force search
The goal is to find a periodic MDCF for some algebraic number $α$.
For an MDCF, we require a sequence of indices $ℓ₁, ℓ₂, …$ which determine the
element to pivot with.
To find this sequence, different types of searches were constructed.

The program is an implementation of the generalized Euclidean algorithm
in Python and SageMath.
More specifically, only the pivot operation from the algorithm was implemented,
since this the actual part that is relevant for the construction of an MDCF.

The actual code for the search I implemented is shown in Listing~\ref{lst:bfs}.
The input to the algorithm is a vector $x$ containing algebraic numbers of
degree $≤ d+1$ and a maximum search depth $N$.
If possible, it outputs two index sequences of the start and period for a periodic MDCF,
which represents the original input vector.
To find this sequence, the algorithm uses a simple brute-force search over all
possible sequences $\{1,\dots,d\}^*$ with a maximum length of $N$.
We simply try every sequence of possible pivot indices in a breadth-first manner.
So we begin with all sequences of length $1$ and see if any vector occurs twice.
If not, then we continue with all sequences of length $2$ and check again if
any previous vector has occurred twice.
We continue this process indefinitely until we hopefully find a duplicate vector.
To keep track of duplicates, the dictionary \verb|seen| maps vectors to the index,
where they first occurred.

% ==============================================================================
\begin{Python}[
    float=tbp,
    numbers=left,
    caption={
      The implementation of the nondeterministic search.
      The search begins with the empty sequence and then queries the strategy
      for the next valid sequences.
      At the same time, it checks whether any vector has occurred twice
      and stops once it has found a duplicate.
    },
    label={lst:nondet-search},
  ]
def nondeterministic_search(x, N, strat):
  d = len(x)
  sequences = [[]]
  for n in range(N):
    new_sequences = []
    for L in sequences:
      y = x
      seen = {y: 0}
      for l in L:
        y = pivot(y, l)
        seen[y] = i + 1
      for l in strat(x, L):
        z = pivot(y, l)
        if z in seen:
          j = seen[z]
          start = L[:j]
          period = L[j:i+1]
          return start, period
        new_sequences.append(L + [l])
    sequences = new_sequences
\end{Python}
% ==============================================================================

% Nondeterministic search
The main goal of the brute-force search is to find a periodic representation for a cubic root at all.
The problem is that the search is quite expensive.
For a given maximum search depth, the search will take $O(d^n)$ steps,
so even for the lowest dimension $d = 2$, the search is already expensive.
The hope is that the sequences share something in common such that we can find
a more optimized search strategy, ideally one which can be decided without
trying every possible combination.
Therefore, two more types of searches were studied.
The first type is a deterministic search,
which only looks at one possible path in the tree.
For example, the minimum strategy would only look at one possible path.
The other type is a non-deterministic search,
which looks at multiple paths simultaneously.
For this type of search, the main point of interest was the approximation rate of the convergents
and whether convergents are the best rational approximations of the original input vector $x$,
i.e. whether they fulfill
\[
  \left|x_i - \frac{p_i^{(n)}}{q^{(n)}}\right| < \frac{1}{q^{(n)} \sqrt[d]{q^{(n)}}}, \text{ for every } i ≤ d
\]
at each step.
The idea would be that this could be one possible optimization to the brute-force search.
Instead of trying every possible candidate, we only choose paths which lead to good approximations.
For $d = 1$, this is already known by Lemma~\ref{lem:cf-approx},
but for higher dimensions it is not known whether the convergents are good
approximations, yet.

The strategy is given the initial input $x ∈ ℝ^d$ and a sequence of indices $L
∈ \{1, …, d\}^*$ and must return all valid indices which can be appended to the
current sequence to form a new sequence.
For the brute-force search, the strategy always outputs the entire set; any index is allowed.
The minimum strategy only outputs one index and its the one where $x^{(n)}$ has
the minimum fractional value, if $n$ is the length of the list $L$.
\[
  \texttt{strat} \colon ℝ^d → \mathcal P(\{1, …, d\}), x ↦ \{ℓ_1, …, ℓ_k\}.
\]

% Deterministic search
For the deterministic search, a strategy $s$ is a function $R^d → \{1, …, d\}$
which takes in the complete quotient $x^{(n)} ∈ ℝ^d$ of the input vector $x ∈ R^d$
and the number of iterations $n$.
The strategy outputs only a single index $ℓ$,
which is used to find the next complete quotient by $x^{(n+1)} =
\mathrm{pivot}_ℓ(x^{(n)})$.
For example, the minimum strategy would be defined as
\[
  \texttt{strat}(x, n) = \underset{\substack{ℓ ∈ \{1, …, d\} \\ \{x_ℓ\} ≠ 0}}{\text{arg min}} \{x_ℓ\}.
\]
The additional index $n$ is used for example in the Jacobi-Perron algorithm,
which chooses always the next index in the sequence.
So,
\[
  \texttt{strat}(x, n) = (n \bmod d) + 1.
\]

For the non-deterministic search, only the approximation criterion was tested,
but with different rates, i.e. whether for some constant $c ≥ 1$,
\[
  \left|x_i - \frac{p_i^{(n)}}{q^{(n)}}\right| < \frac{c}{q^{(n)} \sqrt[d]{q^{(n)}}}, \text{ for every } i ≤ d.
\]

In summary,
there are three different types of searches.
The first is the brute-force search, which is used to find a periodic MDCF at all.
The second is the nondeterministic search, which is used to see how well the
convergents approximate the original input vector.
The third is the deterministic search, which is used to compare the different
strategies.
\fi

\iffalse
% ==============================================================================
\section{Periodic MDCFs for Cube Roots}
% ==============================================================================

% TODO: Measure the times for each MDCF and list them!
For the first search, the cube roots $\sqrt[3]{2}$ to $\sqrt[3]{100}$ were tested.
The MDCFs for these roots is listed in Table~\ref{tbl:cubics}.
For cubic irrationals there are $O(2^n)$ possible sequences,
so the search is already quite expensive.
The search for the first 30 cube roots had a maximum search depth of $24$ and
this already took over two hours to complete.
Almost half of the time was spent on the root for $\sqrt[3]{29}$
and it did not find a representation for this root.
Instead, I used different strategies to find an MDCF for this root.

Regarding the representation of the roots,
there is no perceivable patterns between the roots and their length.
However, this is to be expected since continued fractions also follow no simple
pattern between square roots and their length.
What the MDCFs share in common is that their periods always have an even length
and the period always contains both indices;
that is, there is no periodic sequence consisting solely of one repeated index
(e.g., only ones or only twos).

Apart from the even length, there is one specific set of roots which have a
predetermined period length
and they are the roots with the shortest period, i.e. $\sqrt[3]{2},
\sqrt[3]{3}, \sqrt[3]{9}$ and $\sqrt[3]{28}$.
Out of all observed roots, they have the shortest period with only two indices.
The reason comes from a theorem proven by Bernstein \cite{Bernstein71}.
In his analysis of the Jacobi-Perron algorithm,
he has shown that Jacobi-Perron algorithm is periodic for any root of the form
% TODO: Is the period length always 1?
% TODO: Check the correct conditions. Specifically is c < D correct?
\[
  \sqrt[3]{D^3 + c}, \qquad \text{ where } c < D \text{ and } c|D.
\]
and that the period length is exactly $2$ in the case of cubic irrationals.
So other roots with a short periodic sequence are $\sqrt[3]{65}, \sqrt[3]{66}$,
for example.

\begin{table}[tbp]
  \caption{
    The shortest periodic index sequences for cube roots found using the
    brute-force search algorithm. The maximum search depth was set to $20$ and
    only the sequence for $29$ was not found. The roots for $8$ and $27$ are
    omitted since they are perfect cubes.}
  \label{tbl:cubics}
  \centering
  \begin{minipage}{0.48\textwidth}
\footnotesize
\begin{tabular}{ll}
\uzlhline
\uzlemph{$x$} & \uzlemph{MDCF} \\ \hline
$\sqrt[3]{2}$ & $\left[
\begin{matrix} 1 \\ 1 \\ \end{matrix}\,\,
\overline{
\begin{matrix} 0 \\ 1 \\ \end{matrix}\,\,
\begin{matrix} 2 \\ 1 \\ \end{matrix}\,\,
}\right]$ \\
$\sqrt[3]{3}$ & $\left[
\begin{matrix} 1 \\ 2 \\ \end{matrix}\,\,
\begin{matrix} 2 \\ 0 \\ \end{matrix}\,\,
\overline{
\begin{matrix} 1 \\ 5 \\ \end{matrix}\,\,
\begin{matrix} 2 \\ 1 \\ \end{matrix}\,\,
}\right]$ \\
$\sqrt[3]{4}$ & $\left[
\begin{matrix} 1 \\ 2 \\ \end{matrix}\,\,
\overline{
\begin{matrix} 1 \\ 0 \\ \end{matrix}\,\,
\begin{matrix} 1 \\ 1 \\ \end{matrix}\,\,
\begin{matrix} 1 \\ 3 \\ \end{matrix}\,\,
\begin{matrix} 1 \\ 1 \\ \end{matrix}\,\,
\begin{matrix} 1 \\ 0 \\ \end{matrix}\,\,
\begin{matrix} 1 \\ 1 \\ \end{matrix}\,\,
\begin{matrix} 0 \\ 1 \\ \end{matrix}\,\,
\begin{matrix} 3 \\ 1 \\ \end{matrix}\,\,
}\right]$ \\
$\sqrt[3]{5}$ & $\left[
\begin{matrix} 1 \\ 2 \\ \end{matrix}\,\,
\overline{
\begin{matrix} 1 \\ 1 \\ \end{matrix}\,\,
\begin{matrix} 2 \\ 0 \\ \end{matrix}\,\,
\begin{matrix} 2 \\ 1 \\ \end{matrix}\,\,
\begin{matrix} 0 \\ 1 \\ \end{matrix}\,\,
\begin{matrix} 0 \\ 1 \\ \end{matrix}\,\,
\begin{matrix} 0 \\ 1 \\ \end{matrix}\,\,
\begin{matrix} 1 \\ 0 \\ \end{matrix}\,\,
\begin{matrix} 3 \\ 0 \\ \end{matrix}\,\,
}\right]$ \\
$\sqrt[3]{6}$ & $\left[
\begin{matrix} 1 \\ 3 \\ \end{matrix}\,\,
\overline{
\begin{matrix} 1 \\ 0 \\ \end{matrix}\,\,
\begin{matrix} 4 \\ 1 \\ \end{matrix}\,\,
\begin{matrix} 2 \\ 1 \\ \end{matrix}\,\,
\begin{matrix} 0 \\ 2 \\ \end{matrix}\,\,
\begin{matrix} 0 \\ 1 \\ \end{matrix}\,\,
\begin{matrix} 0 \\ 1 \\ \end{matrix}\,\,
\begin{matrix} 1 \\ 0 \\ \end{matrix}\,\,
\begin{matrix} 3 \\ 1 \\ \end{matrix}\,\,
}\right]$ \\
$\sqrt[3]{7}$ & $\left[
\begin{matrix} 1 \\ 3 \\ \end{matrix}\,\,
\overline{
\begin{matrix} 1 \\ 0 \\ \end{matrix}\,\,
\begin{matrix} 10 \\ 7 \\ \end{matrix}\,\,
\begin{matrix} 0 \\ 1 \\ \end{matrix}\,\,
\begin{matrix} 1 \\ 0 \\ \end{matrix}\,\,
\begin{matrix} 0 \\ 1 \\ \end{matrix}\,\,
\begin{matrix} 4 \\ 0 \\ \end{matrix}\,\,
}\right]$ \\
$\sqrt[3]{9}$ & $\left[
\begin{matrix} 2 \\ 4 \\ \end{matrix}\,\,
\begin{matrix} 12 \\ 4 \\ \end{matrix}\,\,
\overline{
\begin{matrix} 6 \\ 12 \\ \end{matrix}\,\,
\begin{matrix} 12 \\ 6 \\ \end{matrix}\,\,
}\right]$ \\
$\sqrt[3]{10}$ & $\left[
\begin{matrix} 2 \\ 4 \\ \end{matrix}\,\,
\overline{
\begin{matrix} 6 \\ 4 \\ \end{matrix}\,\,
\begin{matrix} 3 \\ 6 \\ \end{matrix}\,\,
\begin{matrix} 0 \\ 2 \\ \end{matrix}\,\,
\begin{matrix} 6 \\ 0 \\ \end{matrix}\,\,
}\right]$ \\
$\sqrt[3]{11}$ & $\left[
\begin{matrix} 2 \\ 4 \\ \end{matrix}\,\,
\overline{
\begin{matrix} 4 \\ 4 \\ \end{matrix}\,\,
\begin{matrix} 2 \\ 4 \\ \end{matrix}\,\,
\begin{matrix} 0 \\ 2 \\ \end{matrix}\,\,
\begin{matrix} 6 \\ 0 \\ \end{matrix}\,\,
}\right]$ \\
$\sqrt[3]{12}$ & $\left[
\begin{matrix} 2 \\ 5 \\ \end{matrix}\,\,
\overline{
\begin{matrix} 1 \\ 4 \\ \end{matrix}\,\,
\begin{matrix} 5 \\ 0 \\ \end{matrix}\,\,
\begin{matrix} 0 \\ 1 \\ \end{matrix}\,\,
\begin{matrix} 0 \\ 2 \\ \end{matrix}\,\,
\begin{matrix} 0 \\ 2 \\ \end{matrix}\,\,
\begin{matrix} 3 \\ 0 \\ \end{matrix}\,\,
\begin{matrix} 0 \\ 1 \\ \end{matrix}\,\,
\begin{matrix} 2 \\ 2 \\ \end{matrix}\,\,
\begin{matrix} 0 \\ 2 \\ \end{matrix}\,\,
\begin{matrix} 6 \\ 1 \\ \end{matrix}\,\,
}\right]$ \\
$\sqrt[3]{13}$ & $\left[
\begin{matrix} 2 \\ 5 \\ \end{matrix}\,\,
\begin{matrix} 2 \\ 1 \\ \end{matrix}\,\,
\overline{
\begin{matrix} 1 \\ 0 \\ \end{matrix}\,\,
\begin{matrix} 5 \\ 3 \\ \end{matrix}\,\,
\begin{matrix} 1 \\ 3 \\ \end{matrix}\,\,
\begin{matrix} 1 \\ 0 \\ \end{matrix}\,\,
\begin{matrix} 3 \\ 2 \\ \end{matrix}\,\,
\begin{matrix} 2 \\ 0 \\ \end{matrix}\,\,
}\right]$ \\
$\sqrt[3]{14}$ & $\left[
\begin{matrix} 2 \\ 5 \\ \end{matrix}\,\,
\begin{matrix} 2 \\ 1 \\ \end{matrix}\,\,
\begin{matrix} 0 \\ 1 \\ \end{matrix}\,\,
\begin{matrix} 2 \\ 0 \\ \end{matrix}\,\,
\overline{
\begin{matrix} 3 \\ 15 \\ \end{matrix}\,\,
\begin{matrix} 2 \\ 1 \\ \end{matrix}\,\,
\begin{matrix} 0 \\ 1 \\ \end{matrix}\,\,
\begin{matrix} 1 \\ 1 \\ \end{matrix}\,\,
}\right]$ \\
$\sqrt[3]{15}$ & $\left[
\begin{matrix} 2 \\ 6 \\ \end{matrix}\,\,
\begin{matrix} 2 \\ 0 \\ \end{matrix}\,\,
\begin{matrix} 6 \\ 1 \\ \end{matrix}\,\,
\overline{
\begin{matrix} 4 \\ 4 \\ \end{matrix}\,\,
\begin{matrix} 6 \\ 4 \\ \end{matrix}\,\,
\begin{matrix} 2 \\ 0 \\ \end{matrix}\,\,
\begin{matrix} 0 \\ 18 \\ \end{matrix}\,\,
\begin{matrix} 2 \\ 0 \\ \end{matrix}\,\,
\begin{matrix} 6 \\ 0 \\ \end{matrix}\,\,
}\right]$ \\
$\sqrt[3]{16}$ & $\left[
\begin{matrix} 2 \\ 6 \\ \end{matrix}\,\,
\overline{
\begin{matrix} 1 \\ 0 \\ \end{matrix}\,\,
\begin{matrix} 1 \\ 0 \\ \end{matrix}\,\,
\begin{matrix} 0 \\ 1 \\ \end{matrix}\,\,
\begin{matrix} 8 \\ 3 \\ \end{matrix}\,\,
\begin{matrix} 1 \\ 0 \\ \end{matrix}\,\,
\begin{matrix} 0 \\ 2 \\ \end{matrix}\,\,
\begin{matrix} 1 \\ 1 \\ \end{matrix}\,\,
\begin{matrix} 1 \\ 0 \\ \end{matrix}\,\,
\begin{matrix} 2 \\ 2 \\ \end{matrix}\,\,
\begin{matrix} 3 \\ 1 \\ \end{matrix}\,\,
\begin{matrix} 0 \\ 1 \\ \end{matrix}\,\,
\begin{matrix} 1 \\ 0 \\ \end{matrix}\,\,
\begin{matrix} 1 \\ 1 \\ \end{matrix}\,\,
\begin{matrix} 5 \\ 4 \\ \end{matrix}\,\,
}\right]$ \\
\uzlhline
\end{tabular}
\end{minipage}
\begin{minipage}{0.48\textwidth}
\footnotesize
\begin{tabular}{ll}
\uzlhline
\uzlemph{$x$} & \uzlemph{MDCF} \\ \hline
$\sqrt[3]{17}$ & $\left[
\begin{matrix} 2 \\ 6 \\ \end{matrix}\,\,
\begin{matrix} 1 \\ 1 \\ \end{matrix}\,\,
\begin{matrix} 1 \\ 0 \\ \end{matrix}\,\,
\overline{
\begin{matrix} 3 \\ 0 \\ \end{matrix}\,\,
\begin{matrix} 0 \\ 3 \\ \end{matrix}\,\,
\begin{matrix} 0 \\ 1 \\ \end{matrix}\,\,
\begin{matrix} 0 \\ 1 \\ \end{matrix}\,\,
\begin{matrix} 0 \\ 4 \\ \end{matrix}\,\,
\begin{matrix} 3 \\ 3 \\ \end{matrix}\,\,
\begin{matrix} 1 \\ 0 \\ \end{matrix}\,\,
\begin{matrix} 1 \\ 1 \\ \end{matrix}\,\,
}\right]$ \\
$\sqrt[3]{18}$ & $\left[
\begin{matrix} 2 \\ 6 \\ \end{matrix}\,\,
\begin{matrix} 1 \\ 1 \\ \end{matrix}\,\,
\begin{matrix} 1 \\ 0 \\ \end{matrix}\,\,
\begin{matrix} 1 \\ 1 \\ \end{matrix}\,\,
\begin{matrix} 1 \\ 0 \\ \end{matrix}\,\,
\begin{matrix} 1 \\ 0 \\ \end{matrix}\,\,
\overline{
\begin{matrix} 5 \\ 17 \\ \end{matrix}\,\,
\begin{matrix} 1 \\ 0 \\ \end{matrix}\,\,
\begin{matrix} 1 \\ 0 \\ \end{matrix}\,\,
\begin{matrix} 1 \\ 0 \\ \end{matrix}\,\,
\begin{matrix} 1 \\ 0 \\ \end{matrix}\,\,
\begin{matrix} 1 \\ 1 \\ \end{matrix}\,\,
}\right]$ \\
$\sqrt[3]{19}$ & $\left[
\begin{matrix} 2 \\ 7 \\ \end{matrix}\,\,
\begin{matrix} 1 \\ 0 \\ \end{matrix}\,\,
\begin{matrix} 2 \\ 0 \\ \end{matrix}\,\,
\overline{
\begin{matrix} 0 \\ 2 \\ \end{matrix}\,\,
\begin{matrix} 0 \\ 1 \\ \end{matrix}\,\,
\begin{matrix} 0 \\ 3 \\ \end{matrix}\,\,
\begin{matrix} 5 \\ 0 \\ \end{matrix}\,\,
\begin{matrix} 1 \\ 0 \\ \end{matrix}\,\,
\begin{matrix} 2 \\ 1 \\ \end{matrix}\,\,
}\right]$ \\
$\sqrt[3]{20}$ & $\left[
\begin{matrix} 2 \\ 7 \\ \end{matrix}\,\,
\overline{
\begin{matrix} 1 \\ 0 \\ \end{matrix}\,\,
\begin{matrix} 2 \\ 1 \\ \end{matrix}\,\,
\begin{matrix} 1 \\ 3 \\ \end{matrix}\,\,
\begin{matrix} 1 \\ 2 \\ \end{matrix}\,\,
\begin{matrix} 1 \\ 0 \\ \end{matrix}\,\,
\begin{matrix} 1 \\ 0 \\ \end{matrix}\,\,
\begin{matrix} 1 \\ 2 \\ \end{matrix}\,\,
\begin{matrix} 6 \\ 3 \\ \end{matrix}\,\,
}\right]$ \\
$\sqrt[3]{21}$ & $\left[
\begin{matrix} 2 \\ 7 \\ \end{matrix}\,\,
\begin{matrix} 1 \\ 0 \\ \end{matrix}\,\,
\begin{matrix} 3 \\ 2 \\ \end{matrix}\,\,
\overline{
\begin{matrix} 6 \\ 3 \\ \end{matrix}\,\,
\begin{matrix} 1 \\ 1 \\ \end{matrix}\,\,
\begin{matrix} 0 \\ 1 \\ \end{matrix}\,\,
\begin{matrix} 2 \\ 2 \\ \end{matrix}\,\,
\begin{matrix} 1 \\ 0 \\ \end{matrix}\,\,
\begin{matrix} 0 \\ 22 \\ \end{matrix}\,\,
\begin{matrix} 1 \\ 0 \\ \end{matrix}\,\,
\begin{matrix} 3 \\ 0 \\ \end{matrix}\,\,
}\right]$ \\
$\sqrt[3]{22}$ & $\left[
\begin{matrix} 2 \\ 7 \\ \end{matrix}\,\,
\begin{matrix} 1 \\ 1 \\ \end{matrix}\,\,
\begin{matrix} 4 \\ 0 \\ \end{matrix}\,\,
\begin{matrix} 0 \\ 4 \\ \end{matrix}\,\,
\begin{matrix} 4 \\ 0 \\ \end{matrix}\,\,
\begin{matrix} 1 \\ 0 \\ \end{matrix}\,\,
\overline{
\begin{matrix} 3 \\ 20 \\ \end{matrix}\,\,
\begin{matrix} 1 \\ 0 \\ \end{matrix}\,\,
\begin{matrix} 4 \\ 2 \\ \end{matrix}\,\,
\begin{matrix} 0 \\ 3 \\ \end{matrix}\,\,
\begin{matrix} 5 \\ 1 \\ \end{matrix}\,\,
\begin{matrix} 1 \\ 1 \\ \end{matrix}\,\,
}\right]$ \\
$\sqrt[3]{23}$ & $\left[
\begin{matrix} 2 \\ 8 \\ \end{matrix}\,\,
\overline{
\begin{matrix} 1 \\ 0 \\ \end{matrix}\,\,
\begin{matrix} 5 \\ 0 \\ \end{matrix}\,\,
\begin{matrix} 2 \\ 1 \\ \end{matrix}\,\,
\begin{matrix} 1 \\ 2 \\ \end{matrix}\,\,
\begin{matrix} 4 \\ 2 \\ \end{matrix}\,\,
\begin{matrix} 1 \\ 1 \\ \end{matrix}\,\,
\begin{matrix} 3 \\ 3 \\ \end{matrix}\,\,
\begin{matrix} 0 \\ 1 \\ \end{matrix}\,\,
\begin{matrix} 10 \\ 1 \\ \end{matrix}\,\,
\begin{matrix} 0 \\ 1 \\ \end{matrix}\,\,
\begin{matrix} 4 \\ 11 \\ \end{matrix}\,\,
\begin{matrix} 1 \\ 1 \\ \end{matrix}\,\,
\begin{matrix} 5 \\ 1 \\ \end{matrix}\,\,
\begin{matrix} 5 \\ 1 \\ \end{matrix}\,\,
\begin{matrix} 2 \\ 5 \\ \end{matrix}\,\,
\begin{matrix} 6 \\ 2 \\ \end{matrix}\,\,
\begin{matrix} 0 \\ 1 \\ \end{matrix}\,\,
\begin{matrix} 1 \\ 0 \\ \end{matrix}\,\,
\begin{matrix} 0 \\ 2 \\ \end{matrix}\,\,
\begin{matrix} 7 \\ 1 \\ \end{matrix}\,\,
}\right]$ \\
$\sqrt[3]{24}$ & $\left[
\begin{matrix} 2 \\ 8 \\ \end{matrix}\,\,
\overline{
\begin{matrix} 1 \\ 0 \\ \end{matrix}\,\,
\begin{matrix} 7 \\ 2 \\ \end{matrix}\,\,
\begin{matrix} 1 \\ 1 \\ \end{matrix}\,\,
\begin{matrix} 2 \\ 5 \\ \end{matrix}\,\,
\begin{matrix} 1 \\ 0 \\ \end{matrix}\,\,
\begin{matrix} 0 \\ 1 \\ \end{matrix}\,\,
\begin{matrix} 0 \\ 2 \\ \end{matrix}\,\,
\begin{matrix} 6 \\ 4 \\ \end{matrix}\,\,
}\right]$ \\
$\sqrt[3]{25}$ & $\left[
\begin{matrix} 2 \\ 8 \\ \end{matrix}\,\,
\overline{
\begin{matrix} 1 \\ 1 \\ \end{matrix}\,\,
\begin{matrix} 1 \\ 1 \\ \end{matrix}\,\,
\begin{matrix} 2 \\ 4 \\ \end{matrix}\,\,
\begin{matrix} 0 \\ 1 \\ \end{matrix}\,\,
\begin{matrix} 2 \\ 0 \\ \end{matrix}\,\,
\begin{matrix} 3 \\ 4 \\ \end{matrix}\,\,
\begin{matrix} 1 \\ 0 \\ \end{matrix}\,\,
\begin{matrix} 12 \\ 1 \\ \end{matrix}\,\,
\begin{matrix} 6 \\ 3 \\ \end{matrix}\,\,
\begin{matrix} 4 \\ 2 \\ \end{matrix}\,\,
\begin{matrix} 3 \\ 4 \\ \end{matrix}\,\,
\begin{matrix} 22 \\ 6 \\ \end{matrix}\,\,
\begin{matrix} 0 \\ 1 \\ \end{matrix}\,\,
\begin{matrix} 3 \\ 11 \\ \end{matrix}\,\,
\begin{matrix} 1 \\ 0 \\ \end{matrix}\,\,
\begin{matrix} 12 \\ 2 \\ \end{matrix}\,\,
\begin{matrix} 0 \\ 1 \\ \end{matrix}\,\,
\begin{matrix} 1 \\ 0 \\ \end{matrix}\,\,
\begin{matrix} 0 \\ 2 \\ \end{matrix}\,\,
\begin{matrix} 7 \\ 1 \\ \end{matrix}\,\,
}\right]$ \\
$\sqrt[3]{26}$ & $\left[
\begin{matrix} 2 \\ 8 \\ \end{matrix}\,\,
\begin{matrix} 1 \\ 0 \\ \end{matrix}\,\,
\begin{matrix} 25 \\ 20 \\ \end{matrix}\,\,
\overline{
\begin{matrix} 1 \\ 1 \\ \end{matrix}\,\,
\begin{matrix} 8 \\ 17 \\ \end{matrix}\,\,
\begin{matrix} 1 \\ 0 \\ \end{matrix}\,\,
\begin{matrix} 25 \\ 18 \\ \end{matrix}\,\,
}\right]$ \\
$\sqrt[3]{28}$ & $\left[
\begin{matrix} 3 \\ 9 \\ \end{matrix}\,\,
\begin{matrix} 27 \\ 6 \\ \end{matrix}\,\,
\overline{
\begin{matrix} 9 \\ 27 \\ \end{matrix}\,\,
\begin{matrix} 27 \\ 9 \\ \end{matrix}\,\,
}\right]$ \\
$\sqrt[3]{30}$ & $\left[
\begin{matrix} 3 \\ 9 \\ \end{matrix}\,\,
\overline{
\begin{matrix} 9 \\ 6 \\ \end{matrix}\,\,
\begin{matrix} 3 \\ 9 \\ \end{matrix}\,\,
\begin{matrix} 0 \\ 3 \\ \end{matrix}\,\,
\begin{matrix} 9 \\ 0 \\ \end{matrix}\,\,
}\right]$ \\
$\sqrt[3]{31}$ & $\left[
\begin{matrix} 3 \\ 9 \\ \end{matrix}\,\,
\begin{matrix} 0 \\ 1 \\ \end{matrix}\,\,
\begin{matrix} 1 \\ 6 \\ \end{matrix}\,\,
\begin{matrix} 13 \\ 8 \\ \end{matrix}\,\,
\overline{
\begin{matrix} 1 \\ 0 \\ \end{matrix}\,\,
\begin{matrix} 5 \\ 9 \\ \end{matrix}\,\,
\begin{matrix} 0 \\ 7 \\ \end{matrix}\,\,
\begin{matrix} 29 \\ 2 \\ \end{matrix}\,\,
\begin{matrix} 2 \\ 7 \\ \end{matrix}\,\,
\begin{matrix} 14 \\ 1 \\ \end{matrix}\,\,
}\right]$ \\
$\sqrt[3]{32}$ & $\left[
\begin{matrix} 3 \\ 10 \\ \end{matrix}\,\,
\overline{
\begin{matrix} 2 \\ 12 \\ \end{matrix}\,\,
\begin{matrix} 4 \\ 2 \\ \end{matrix}\,\,
\begin{matrix} 1 \\ 1 \\ \end{matrix}\,\,
\begin{matrix} 2 \\ 44 \\ \end{matrix}\,\,
\begin{matrix} 1 \\ 0 \\ \end{matrix}\,\,
\begin{matrix} 4 \\ 1 \\ \end{matrix}\,\,
\begin{matrix} 6 \\ 1 \\ \end{matrix}\,\,
\begin{matrix} 6 \\ 4 \\ \end{matrix}\,\,
\begin{matrix} 0 \\ 2 \\ \end{matrix}\,\,
\begin{matrix} 2 \\ 0 \\ \end{matrix}\,\,
\begin{matrix} 2 \\ 1 \\ \end{matrix}\,\,
\begin{matrix} 8 \\ 9 \\ \end{matrix}\,\,
\begin{matrix} 3 \\ 3 \\ \end{matrix}\,\,
\begin{matrix} 9 \\ 1 \\ \end{matrix}\,\,
}\right]$ \\
\uzlhline
\end{tabular}
\end{minipage}

\end{table}

\begin{table}[tbp]
  \caption{Representation of $ψ = \sqrt[3]{4}$ using the brute-force search.}
  \label{table:cube-root-4}
  \centering
  \footnotesize
  \begin{tabular}{lllllll}
  \uzlhline
  \uzlemph{$\ell$} & \uzlemph{$x_1$} & \uzlemph{$x_2$} & \uzlemph{$x_1$} & \uzlemph{$x_2$} & \uzlemph{$a_1$} & \uzlemph{$a_2$} \\
  \hline
  $0$ & $\psi$ & $\psi^{2}$ & $1.5874$ & $2.51984$ & $0$ & $1$ \\
  \hline
  \hline
  $0$ & $\frac{1}{4} \psi^{2}$ & $\psi - 1$ & $0.62996$ & $0.5874$ & $1$ & $0$ \\
  $0$ & $\psi - 1$ & $\psi^{2} - \psi$ & $0.5874$ & $0.93244$ & $1$ & $1$ \\
  $1$ & $\frac{1}{3} \psi^{2} + \frac{1}{3} \psi - \frac{2}{3}$ & $\psi - 1$ & $0.70241$ & $0.5874$ & $1$ & $1$ \\
  $0$ & $\frac{1}{3} \psi - \frac{1}{3}$ & $\frac{1}{3} \psi^{2} + \frac{1}{3} \psi - \frac{2}{3}$ & $0.1958$ & $0.70241$ & $5$ & $3$ \\
  $1$ & $\psi^{2} + \psi - 4$ & $\psi - 1$ & $0.10724$ & $0.5874$ & $0$ & $1$ \\
  $1$ & $-\frac{2}{3} \psi^{2} + \frac{1}{3} \psi + \frac{4}{3}$ & $\frac{1}{3} \psi^{2} + \frac{1}{3} \psi - \frac{2}{3}$ & $0.18257$ & $0.70241$ & $0$ & $1$ \\
  $1$ & $\frac{1}{2} \psi^{2} - 1$ & $\frac{1}{4} \psi^{2} + \frac{1}{2} \psi - 1$ & $0.25992$ & $0.42366$ & $0$ & $2$ \\
  $0$ & $-\frac{1}{5} \psi^{2} + \frac{1}{5} \psi + \frac{4}{5}$ & $\frac{2}{5} \psi^{2} + \frac{3}{5} \psi - \frac{8}{5}$ & $0.61351$ & $0.36038$ & $1$ & $0$ \\
  \uzlhline
\end{tabular}

\end{table}

Finding periodic MDCFs for higher dimensions is even more difficult than two dimensions.
There are now $O(d^N)$ possible sequences with a maximum depth of $N$.
So each search is exponentially more expensive than the cubic case.
Some of the easier ones to find were again the roots identified by Bernstein.
\fi

% ==============================================================================
\section{Comparison of More Efficient Strategies}
% ==============================================================================

Considering the difficulty in finding higher-dimensional MDCFs,
a more efficient search strategy is needed.
When constructing the MDCF using the generalized Euclidean algorithm,
we always have a choice for the pivot element $x_ℓ$.
Ideally, we could just determine the pivot element solely from the vector $x$
without resorting to a search down the tree.
Therefore, I have compared different strategies which can be used for this purpose.
Each strategy receives the current vector $x$ as well as the number of steps
taken so far and outputs the next index to pivot with.
The search for a periodic MDCF then works just like the brute-force search:
If there is any vector which occured twice, then program terminates the search.

% ==============================================================================
\begin{Python}[
    float=tbp,
    numbers=left,
    caption={
      The implementation of the search for periodic MDCFs.
      The strategy \texttt{strat} outputs a single index $ℓ$, which is used
      for pivoting.
      The search stops once a duplicate vector $x$ has been found and the
      program returns the preperiod and period once found.
    },
    label={lst:det-search},
  ]
def search(x, N, strat):
  seen = {x: 0}
  for n in range(N):
    l = strat(x)
    x = pivot(x, l)
    if x in seen:
      j = seen[x]
      start = L[:j]
      period = L[j:i+1]
      return start, period
    else:
      seen[y] = i + 1
\end{Python}
% ==============================================================================

In summary, the following deterministic strategies were tried:
\begin{itemize}
  \item $\textbf{Min}, \textbf{Max}$: Choosing the minimum and maximum fractional value, respectively.
    These are the strategies which have been analyzed in Chapter~\ref{ch:fibonacci}.
  \item $\textbf{JPA}$: The Jacobi-Perron algorithm,
    which chooses indices in a fixed order.
    Specifically, it chooses the indices $1, 2, …, d$, and repeats this sequence indefinitely.
  \item $\textbf{JPA}'$: A modification of the Jacobi-Perron algorithm introduced by Podsypanin \cite{Podsypanin77}.
    Given a vector $x = (x₁, x₂)$ it chooses the index
    \[
      ℓ =
      \begin{cases}
        1, & \text{ if } x₁ > x₂, \\
        2, & \text{ if } x₁ < x₂.
      \end{cases}
    \]
    For higher dimensions, the algorithm chooses the largest element in $x$.
  \item $\textbf{TY}$:
    The algorithm of Tamura and Yasutomi \cite{Tamura09},
    which is based on the idea of the modified JPA.
    Given vector $x = (x₁, x₂)$, the algorithm chooses the index
    \[
      ℓ =
      \begin{cases}
        1, & \text{ if } \frac{x₁}{\sqrt{|N(x₁)|}} > \frac{x₂}{\sqrt{|N(x₂)|}}, \\
        2, & \text{ if } \frac{x₁}{\sqrt{|N(x₁)|}} < \frac{x₂}{\sqrt{|N(x₂)|}}.
      \end{cases}
    \]
    Again, the algorithm chooses the largest element in each iteration.
    However, it scales down each element by the square root of its norm.
    For higher dimensions, the algorithm chooses
    \[
      ℓ = \argmax_i \frac{x_i}{\sqrt[d+1]{|N(x_i)|}}.
    \]
  \item $\textbf{CC}$: Choosing the closest convergent.
    Out of the $d$ possible indices,
    we choose the one which produces the closest convergent $r^{(n)}$,
    which means that it minimizes the distance to the original input vector.
    This is measured either using the Euclidean norm $\|x - r^{(n)}\|_2$ or using the maximum norm $\|x - r^{(n)}\|_{\infty}$.
\end{itemize}
% TODO: Mention that the Tamura and Yasutomi algorithm was tested by the authors themselves for both cubic and quadratic cases.
% TODO: For how many steps did we run the construction?
The fixed strategies were the easiest to implement and therefore also the
fastest, so I ran them for a maximum number of $1000$ steps.
The other strategies had a little more overhead, so I only ran them for a
maximum of $100$ steps.

Table~\ref{tbl:comparison} lists the results for this section.
Out of all strategies, the best are clearly the one,
which chooses the minimum fractional value, and the fixed strategy, which
alternates between the first and second element.
Surprisingly, the worst strategy for constructing MDCFs seems to be the one,
which chooses the best convergent with either norm.
It was unable to construct a single MDCF for any cube root.
% Why was it unable to do so? Which indices did it choose? How well do the
% other strategies approximate the original input vector? Are they better or
% worse after some number of steps?

Although the minimum strategy was able to find the most MDCFs,
it was not able to find all MDCFs.
Some cube roots were only found by other strategies,
while others were only found by the minimum strategy.
In conclusion, no perfect strategy has been found.
This is not surprising,
since what this search is likely aiming for is a fundamental unit in a cubic field.
The problem of finding a fundamental unit is not trivial in higher degree fields. % TODO: cite
For cubic fields, usually an instance of the LLL algorithm is used.

% ==============================================================================
\section{Usage in Simultaneous Approximation}
% ==============================================================================

The continued fractions play an important role in Diophantine approximation,
where the goal is to approximate real numbers using rational numbers.
In Lemma~\vref{lem:cf-approx}, we have already seen that the convergents
$pₙ/qₙ$ of a continued fraction $x$ approximate the represented number $x$
particularly well.
More specifically, that the convergents satisfy the bound
\[
  \left|α - \frac{pₙ}{qₙ}\right| < \frac{1}{qₙ^2}.
\]
The Lemma also follows from Dirichlet's approximation theorem.
For its multidimensional counterpart,
the question then is whether they approximate the vector particularly well.
Approximating a vector instead of a single number is also known as simultaneous
Diophantine approximation.
We are now given an irrational vector $(α₁, …, α_d)$ and we have to find good
approximations $(p₁/q, …, p_d/q) ∈ ℚ^d$ for every number at once.
Using the simultaneous version of Dirichlet's approximation theorem \cite{Schmidt80},
one can show that there are infinitely many rational vectors $(p₁/q, …, p_d/q)$,
which satisfy
\begin{equation}
  \label{eq:sim-approx}
  \left|α_i - \frac{p_i}{q}\right| ≤ \frac{1}{q^{1 + 1/d}}
  \quad
  \text{ for every } i ∈ \{1, …, d\}.
\end{equation}
The idea would be that all convergents would satisfy this approximation bound.
However, if that is not the case, there is still a possibility that some path
during the construction satisfies the approximation bound.
From strongest to weakest, I have analyzed the following questions:
\begin{enumerate}
  \item Do all paths have convergents which satisfy the approximation bound?
  \item Is there a path where all convergents satisfy the approximation bound?
  \item Is there a path where infinitely many convergents satisfy the approximation bound?
\end{enumerate}

The algorithm for answering these question utilizes a breadth-first search over the tree.
It only expands a node $x ∈ ℝ^d$, if the convergent vector $r^{(n)}$ of this
path is a good approximation, i.e. satisfies the bound from
Equation~\ref{eq:sim-approx}.
Otherwise, the node is considered a leaf.
The search terminates either after a maximum number of steps is reached or if
there are no more nodes in the queue.
The latter of which would show that there are examples, where no path with the
given approximation bound exists.
This covers the first and second questions.
For the third question, I have implemented a brute-force search,
which simply tries all possible sequences to test for the approximation rate.
The goal would be to see how many convergents satisfy the approximation bound.

Figure~\ref{fig:results-approx} shows the results of this analysis.

% TODO: When repeating the period multiple times, does this actually keep the
% approximation bound intact or does it violate it after some point?
For most cubic roots from the previous test,
there are MDCFs which satisfy the approximation bound.
However, they are notably different MDCFs than those found in the brute-force
search.
So some cubic roots can have different MDCFs
and not all of them must necessarily be good simultaneous approximations.

The most surprising result of this analysis is that there are some cubic roots
for which this approximation rate is not possible at all times.
Perron already suggested that his algorithm does not satisfy this bound,
so a simple strategy ought to violate the bound at some point.
However, it turns out that no strategy can keep the approximation bound at each step.
In the search, only convergents which satisfy the approximation bound make it to the next round.
So it is possible that no convergents make it and for some cubic roots this is actually the case.
One example is $\sqrt[3]{5}$.
In this case, the number of convergents starts out growing but quickly drops
after only a few number of iterations.

Since not all cubic roots can be approximated well using MDCFs,
I weakened the bound to allow all convergents which satisfy
\[
  \left|x_i - \frac{p_i}{q}\right| < \frac{c}{q \, \sqrt[d]{q}} \qquad \text{ for every } i ≤ d,
\]
where $c$ is some constant independent of $n$.
The specific constant for each root is listed in Table~\ref{tbl:approx-const}.
% TODO: Make the actual table
\begin{table}[tbp]
  \centering
  \begin{tabular}{cc}
    \uzlhline
    Root & Constant \\
    \hline
    TODO & TODO \\
    \uzlhline
  \end{tabular}
  \caption{Approximation constants for the cubic roots.}
  \label{tbl:approx-const}
\end{table}

% TODO: Add plot for growth

% TODO: Add table for numbers which admit good approximations

% TODO: What about transcendental numbers?
