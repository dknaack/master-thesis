\chapter{Experimental Analysis on the Periodicity}

In this chapter, we consider the problem of finding a periodic MDCF for an algebraic number of degree $d+1$.
Finding a proof, that algebraic numbers of degree $≤ d+1$ always admit periodic MDCFs, is much more difficult.
We have already seen that the golden ratios are periodic, when running the
generalized Euclidean algorithm on them.
In this chapter, I present my empirical results on some cubic roots.
I have implemented the generalized Euclidean algorithm and used it to look for
cubic roots.
In this chapter, I present the results of my analysis.

\section{Experiment Setup}

Recall that Hermite's original question asks about the representation of a
single number $r ∈ ℝ$, but the generalized Euclidean algorithm expects a vector
as input.
So we first need a mapping from a real number $r$ to a real vector $x$.
One solution is to apply a set of $d$ polynomials $q₁, …, q_d$ on the
given number $r$ giving us the vector $x = (q₁(r), …, q_d(r))$.
Not all choices work equally well -- a trivial example, which won't work, would
be choosing constant polynomials, i.e. $qᵢ(r) = c_i$ for $c_i ∈ ℚ$.
Therefore, the polynomials should be chosen such that when $r$ is an algebraic
number of degree $≥d+1$, the elements of $x$ are linearly independent with
rational coefficients.
An easy way to achieve this is to choose polynomials with different degrees or
more specifically $\deg(q_i) = i$.
In my analysis, I have always chosen the polynomials $qᵢ(r) = r^i$.
Unless stated otherwise, the following analysis will always use this vector as the input.

\begin{Pseudocode}[float=tbp, label={lst:bfs}, caption={The brute-force search algorithm for finding a periodic representation.}]
algorithm $\textsc{brute-force-search}(x, N)$
  for $L ∈ \{1, …, d\}^{≤ N}$ do
    $x₀ ← x$
    for $i ∈ \{1, …, |L|\}$ do
      $x_i ← \mathrm{pivot}_{L[i]}(x_{i-1})$
      if $x_i$ occurred twice then
        Find the index $j ≠ i$ where $x_j = x_i$
        $S ← L[1] \,…\, L[j]$
        $P ← L[j+1] \,…\, L[i]$
        return $(S, P)$
  return $ε$
\end{Pseudocode}

Now that we have a vector $x$, we can begin the search.
The pseudocode for the search algorithm I implemented is shown in Listing~\ref{lst:bfs}.
The algorithm is a brute-force search over all possible sequences $\{1,\dots,d\}^*$.
We simply try every sequence of possible pivot indices in a breadth-first manner.
So we begin with all sequences of length $1$ and see if any vector occurs twice.
If not, then we continue with all sequences of length $2$ and check again if
any previous vector has occurred twice.
We continue this process indefinitely until we hopefully find a duplicate vector.

\section{Results for Cubic Irrationals}

For cubic irrationals there are $2^n$ possible sequences,
so the search is already quite expensive.
The search for the first 30 cube roots took roughly one hour to complete.

\begin{example}
  The sequences for the prime numbers are:
  \begin{itemize}
    \item $2^{1/3}$: $0\overline{10}$.
    \item $3^{1/3}$: $01\overline{01}$.
    \item $5^{1/3}$: $0\overline{00111000}$.
    \item $7^{1/3}$: $0\overline{010100}$.
    \item $11^{1/3}$: $0\overline{1100}$.
    \item $13^{1/3}$: $00\overline{010000}$.
    \item $17^{1/3}$: $000\overline{11110000}$.
    \item Another choice for $5^{1/3}$ with shorter period: $00110\overline{101010}$.
  \end{itemize}
\end{example}

\begin{remark}
  The first number that has a leading $1$ as the pivot is $12$,
  which has the pivot sequence $1\overline{0111011101}$.
  However, there are other possible choices where it does not have a leading $0$.
\end{remark}

\begin{table}[t]
  \caption{Representation of $ψ = \sqrt[3]{4}$ using the brute-force search.}
  \label{table:cube-root-4}
  \centering
  \begin{tabular}{lllllll}
  \uzlhline
  \uzlemph{$\ell$} & \uzlemph{$x_1$} & \uzlemph{$x_2$} & \uzlemph{$x_1$} & \uzlemph{$x_2$} & \uzlemph{$a_1$} & \uzlemph{$a_2$} \\
  \hline
  $0$ & $\psi$ & $\psi^{2}$ & $1.5874$ & $2.51984$ & $0$ & $1$ \\
  \hline
  \hline
  $0$ & $\frac{1}{4} \psi^{2}$ & $\psi - 1$ & $0.62996$ & $0.5874$ & $1$ & $0$ \\
  $0$ & $\psi - 1$ & $\psi^{2} - \psi$ & $0.5874$ & $0.93244$ & $1$ & $1$ \\
  $1$ & $\frac{1}{3} \psi^{2} + \frac{1}{3} \psi - \frac{2}{3}$ & $\psi - 1$ & $0.70241$ & $0.5874$ & $1$ & $1$ \\
  $0$ & $\frac{1}{3} \psi - \frac{1}{3}$ & $\frac{1}{3} \psi^{2} + \frac{1}{3} \psi - \frac{2}{3}$ & $0.1958$ & $0.70241$ & $5$ & $3$ \\
  $1$ & $\psi^{2} + \psi - 4$ & $\psi - 1$ & $0.10724$ & $0.5874$ & $0$ & $1$ \\
  $1$ & $-\frac{2}{3} \psi^{2} + \frac{1}{3} \psi + \frac{4}{3}$ & $\frac{1}{3} \psi^{2} + \frac{1}{3} \psi - \frac{2}{3}$ & $0.18257$ & $0.70241$ & $0$ & $1$ \\
  $1$ & $\frac{1}{2} \psi^{2} - 1$ & $\frac{1}{4} \psi^{2} + \frac{1}{2} \psi - 1$ & $0.25992$ & $0.42366$ & $0$ & $2$ \\
  $0$ & $-\frac{1}{5} \psi^{2} + \frac{1}{5} \psi + \frac{4}{5}$ & $\frac{2}{5} \psi^{2} + \frac{3}{5} \psi - \frac{8}{5}$ & $0.61351$ & $0.36038$ & $1$ & $0$ \\
  \uzlhline
\end{tabular}

\end{table}

\begin{table}[t]
  \caption{Period Length of the first $28$ numbers.}
  \centering
  \begin{tabular}{ll}
  \uzlhline
  \uzlemph{$n^3$} & \uzlemph{Period Length of $n^3$} \\
  \hline
  2 & 2 \\
  3 & 2 \\
  4 & 8 \\
  5 & 8 \\
  6 & 8 \\
  7 & 6 \\
  9 & 2 \\
  10 & 4 \\
  11 & 4 \\
  12 & 10 \\
  13 & 6 \\
  14 & 4 \\
  15 & 6 \\
  16 & 14 \\
  17 & 8 \\
  18 & 6 \\
  19 & 6 \\
  20 & 8 \\
  21 & 8 \\
  22 & 6 \\
  23 & 20 \\
  24 & 8 \\
  25 & 20 \\
  26 & 4 \\
  28 & 2 \\
  \uzlhline
\end{tabular}

\end{table}

\section{Results for Quartic and Higher-Degree Irrationals}

\section{Evaluation of More Efficient Search Strategies}

We can likely rule out alternating pivot sequences since this corresponds to
the Jacobi-Perron algorithm and for this algorithm it is conjectured
\cite{Karpenkov21} that the algorithm is not periodic for $\sqrt[3]{4}$.
However, the brute-force algorithm is periodic for this input (see Table~\ref{table:cube-root-4}).

The choice of the initial input also matters.
We generally choose $x = (α, q(α))$, where $q$ is some polynomial of degree $2$.
But different choices of $q$ produce different sequences.
For example, $(\sqrt[3]{2}, \sqrt[3]{4})$ produces a different sequence of coefficients than $(\sqrt[3]{2}, \sqrt[3]{6})$.
Even though both inputs represent the same number $\sqrt[3]{2}$.
Choosing $q(α) = α^2 - α$ makes the sequence purely periodic for $\sqrt[3]{3}$ and $\sqrt[3]{4}$.

The strategy is given the initial input $x ∈ ℝ^d$ and a sequence of indices $L ∈ \{1, …, d\}^*$ and must
return all valid indices which can be appended to the current sequence to form
a new sequence.
For the brute-force search, the strategy always outputs the entire set; any index is allowed.
The minimum strategy only outputs one index and its the one where $x^{(n)}$ has
the minimum fractional value, if $n$ is the length of the list $L$.
\[
  \text{next} \colon ℝ^d × \{1, …, d\}^* → \mathcal P(\{1, …, d\}), \text{next}(x, L) = \{ℓ_1, …, ℓ_k\}.
\]
