\chapter{Experimental Analysis of the Periodicity}

\section{Finding a Periodic Representation through Brute Force}

\section{Results for Cubic Irrationals}

\begin{example}
  The sequences for the prime numbers are:
  \begin{itemize}
    \item $2^{1/3}$: $0\overline{10}$.
    \item $3^{1/3}$: $01\overline{01}$.
    \item $5^{1/3}$: $0\overline{00111000}$.
    \item $7^{1/3}$: $0\overline{010100}$.
    \item $11^{1/3}$: $0\overline{1100}$.
    \item $13^{1/3}$: $00\overline{010000}$.
    \item $17^{1/3}$: $000\overline{11110000}$.
    \item Another choice for $5^{1/3}$ with shorter period: $00110\overline{101010}$.
  \end{itemize}
\end{example}

\begin{remark}
  The first number that has a leading $1$ as the pivot is $12$,
  which has the pivot sequence $1\overline{0111011101}$.
  However, there are other possible choices where it does not have a leading $0$.
\end{remark}

\begin{table}[t]
  \caption{Representation of $ψ = \sqrt[3]{4}$ using the brute-force search.}
  \label{table:cube-root-4}
  \centering
  \begin{tabular}{lllllll}
  \uzlhline
  \uzlemph{$\ell$} & \uzlemph{$x_1$} & \uzlemph{$x_2$} & \uzlemph{$x_1$} & \uzlemph{$x_2$} & \uzlemph{$a_1$} & \uzlemph{$a_2$} \\
  \hline
  $0$ & $\psi$ & $\psi^{2}$ & $1.5874$ & $2.51984$ & $0$ & $1$ \\
  \hline
  \hline
  $0$ & $\frac{1}{4} \psi^{2}$ & $\psi - 1$ & $0.62996$ & $0.5874$ & $1$ & $0$ \\
  $0$ & $\psi - 1$ & $\psi^{2} - \psi$ & $0.5874$ & $0.93244$ & $1$ & $1$ \\
  $1$ & $\frac{1}{3} \psi^{2} + \frac{1}{3} \psi - \frac{2}{3}$ & $\psi - 1$ & $0.70241$ & $0.5874$ & $1$ & $1$ \\
  $0$ & $\frac{1}{3} \psi - \frac{1}{3}$ & $\frac{1}{3} \psi^{2} + \frac{1}{3} \psi - \frac{2}{3}$ & $0.1958$ & $0.70241$ & $5$ & $3$ \\
  $1$ & $\psi^{2} + \psi - 4$ & $\psi - 1$ & $0.10724$ & $0.5874$ & $0$ & $1$ \\
  $1$ & $-\frac{2}{3} \psi^{2} + \frac{1}{3} \psi + \frac{4}{3}$ & $\frac{1}{3} \psi^{2} + \frac{1}{3} \psi - \frac{2}{3}$ & $0.18257$ & $0.70241$ & $0$ & $1$ \\
  $1$ & $\frac{1}{2} \psi^{2} - 1$ & $\frac{1}{4} \psi^{2} + \frac{1}{2} \psi - 1$ & $0.25992$ & $0.42366$ & $0$ & $2$ \\
  $0$ & $-\frac{1}{5} \psi^{2} + \frac{1}{5} \psi + \frac{4}{5}$ & $\frac{2}{5} \psi^{2} + \frac{3}{5} \psi - \frac{8}{5}$ & $0.61351$ & $0.36038$ & $1$ & $0$ \\
  \uzlhline
\end{tabular}

\end{table}

\begin{table}[t]
  \caption{Period Length of the first $28$ numbers.}
  \centering
  \begin{tabular}{ll}
  \uzlhline
  \uzlemph{$n^3$} & \uzlemph{Period Length of $n^3$} \\
  \hline
  2 & 2 \\
  3 & 2 \\
  4 & 8 \\
  5 & 8 \\
  6 & 8 \\
  7 & 6 \\
  9 & 2 \\
  10 & 4 \\
  11 & 4 \\
  12 & 10 \\
  13 & 6 \\
  14 & 4 \\
  15 & 6 \\
  16 & 14 \\
  17 & 8 \\
  18 & 6 \\
  19 & 6 \\
  20 & 8 \\
  21 & 8 \\
  22 & 6 \\
  23 & 20 \\
  24 & 8 \\
  25 & 20 \\
  26 & 4 \\
  28 & 2 \\
  \uzlhline
\end{tabular}

\end{table}

\section{Results for Quartic and Higher-Degree Irrationals}

\section{Evaluation of More Efficient Search Strategies}

We can likely rule out alternating pivot sequences since this corresponds to
the Jacobi-Perron algorithm and for this algorithm it is conjectured
\cite{Karpenkov21} that the algorithm is not periodic for $\sqrt[3]{4}$.
However, the brute-force algorithm is periodic for this input (see Table~\ref{table:cube-root-4}).

The choice of the initial input also matters.
We generally choose $x = (α, q(α))$, where $q$ is some polynomial of degree $2$.
But different choices of $q$ produce different sequences.
For example, $(\sqrt[3]{2}, \sqrt[3]{4})$ produces a different sequence of coefficients than $(\sqrt[3]{2}, \sqrt[3]{6})$.
Even though both inputs represent the same number $\sqrt[3]{2}$.
Choosing $q(α) = α^2 - α$ makes the sequence purely periodic for $\sqrt[3]{3}$ and $\sqrt[3]{4}$.
