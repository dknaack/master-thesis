\chapter{Experimental Analysis on the Periodicity}
\label{ch:implementation}

We have already seen in Chapter~\ref{ch:fibonacci},
that the simplest periodic MDCFs compose a special class of polynomials,
which can be interpreted as a generalization of the metallic ratios in higher
dimensions.
There is also a particular class of polynomials for the Jacobi-Perron algorithm,
which are periodic for that algorithm and in turn have periodic MDCFs.
However, this still leaves the question, whether all algebraic numbers of degree $d+1$
have a periodic $d$-dimensional continued fraction.
Within the scope of this thesis, I was unable to prove this question in
general, but I will nevertheless present my evidence for the conjecture.

% TODO: Rewrite this part
In this chapter, I present more examples of periodic MDCFs for common algebraic numbers.
These were all found using a special program, which implements part of the generalized Euclidean algorithm.
The chapter begins by outlining how the program was implemented and which search strategies were used.
I then present the results for the periodic MDCFs, followed by a discussion of
their significance and implications for further research.
The main focus of this analysis has been on cubic irrationals,
but there is a small discussion on algebraic numbers with higher degree at the end of the chapter.

\section{Methodology}

The goal is to find a periodic MDCF for some algebraic number $α$.
As previously mentioned, the representation for each algebraic number was found
using a special program.
This program is an implementation of the generalized Euclidean algorithm
in Python and SageMath.
Although Python is not the fastest language, the implementation is short.

The strategy is given the initial input $x ∈ ℝ^d$ and a sequence of indices $L ∈ \{1, …, d\}^*$ and must
return all valid indices which can be appended to the current sequence to form
a new sequence.
For the brute-force search, the strategy always outputs the entire set; any index is allowed.
The minimum strategy only outputs one index and its the one where $x^{(n)}$ has
the minimum fractional value, if $n$ is the length of the list $L$.
\[
  S \colon ℝ^d × \{1, …, d\}^* → \mathcal P(\{1, …, d\}), (x, L) ↦ \{ℓ_1, …, ℓ_k\}.
\]

The main goal of the brute-force search was to find a periodic representation for a cubic root at all.
The hope is that the sequences share something in common such that we can find
a more optimized search strategy, ideally one which can be decided without
trying every possible combination.
Therefore, two more types of searches were studied.
The first type is a deterministic search,
which only looks at one possible path in the tree.
For example, the minimum strategy would only look at one possible path.
The other type is a non-deterministic search,
which looks at multiple paths simultaneously.
For this type of search, the main point of interest was the approximation rate of the convergents
and whether convergents are the best rational approximations of the original input vector $x$,
i.e. whether they fulfill
\[
  |x_i - \frac{p_i^{(n)}}{q^{(n)}}| < \frac{1}{q^{(n)} \sqrt[d]{q^{(n)}}}, \text{ for every } i ≤ d
\]
at each step.
The idea would be that this could be one possible optimization to the brute-force search.
Instead of trying every possible candidate, we only choose paths which lead to good approximations.
For $d = 1$, this is already known by Lemma~\ref{lem:cf-approx},
but for higher dimensions it is not known whether the convergents are good
approximations, yet.

For the deterministic search, a strategy $s$ is a function $R^d → \{1, …, d\}$
which takes in the complete quotient $x^{(n)} ∈ ℝ^d$ of the input vector $x ∈ R^d$
and the number of iterations $n$.
The strategy outputs only a single index $ℓ$,
which is used to find the next complete quotient by $x^{(n+1)} =
\mathrm{pivot}_ℓ(x^{(n)})$.
For example, the minimum strategy would be defined as
\[
  s(x, n) = \underset{\substack{ℓ ∈ \{1, …, d\} \\ \{x_ℓ\} ≠ 0}}{\text{arg min}} \{x_ℓ\}.
\]
The additional index $n$ is used for example in the Jacobi-Perron algorithm,
which chooses always the next index in the sequence.
So,
\[
  s(x, n) = (n \bmod d) + 1.
\]

In summary, the following deterministic strategies were tried:
\begin{itemize}
  \item Minimum fractional value
  \item Maximum fractional value
  \item Strategies which are regular: $s(x, n) = ((n + k) \bmod d) + 1$ for all $k ≤ d$.
    This includes the Jacobi-Perron algorithm which sets $k = 0$.
  \item Some other paper, which decides the next candidate
    based on the algebraic norm of the underlying field. % TODO: Find the paper
\end{itemize}

For the non-deterministic search, only the approximation criterion was tested,
but with different rates, i.e. whether for some constant $c ≥ 1$,
\[
  |x_i - \frac{p_i^{(n)}}{q^{(n)}}| < \frac{c}{q^{(n)} \sqrt[d]{q^{(n)}}}, \text{ for every } i ≤ d.
\]

\begin{Pseudocode}[float=tbp, label={lst:bfs}, caption={The brute-force search algorithm for finding a periodic representation.}]
algorithm $\textsc{brute-force-search}(x, N)$
  for $L ∈ \{1, …, d\}^{≤ N}$ do
    $x₀ ← x$
    for $i ∈ \{1, …, |L|\}$ do
      $x_i ← \mathrm{pivot}_{L[i]}(x_{i-1})$
      if $x_i$ occurred twice then
        Find the index $j ≠ i$ where $x_j = x_i$
        $S ← L[1] \,…\, L[j]$
        $P ← L[j+1] \,…\, L[i]$
        return $(S, P)$
  return $ε$
\end{Pseudocode}

Now that we have a vector $x$, we can begin the search.
The pseudocode for the search algorithm I implemented is shown in Listing~\ref{lst:bfs}.
The algorithm is a brute-force search over all possible sequences $\{1,\dots,d\}^*$.
We simply try every sequence of possible pivot indices in a breadth-first manner.
So we begin with all sequences of length $1$ and see if any vector occurs twice.
If not, then we continue with all sequences of length $2$ and check again if
any previous vector has occurred twice.
We continue this process indefinitely until we hopefully find a duplicate vector.

\section{Results for Cubic Irrationals}

The index sequence for each root is listed in Table~\ref{tbl:cubics}.
For cubic irrationals there are $2^n$ possible sequences,
so the search is already quite expensive.
The search for the first 30 cube roots took over two hours to complete.
Almost half of the time was spent on the root for $\sqrt[3]{29}$.
For this case specifically, I was unable to find a periodic representation
after $20$ iterations. % TODO: Can we do more iterations?

At first glance, the table does not really indicate any patterns between the
number itself and the length of its sequence.
Between the roots, the period always has an even length
and furthermore the period always contains both indices.
There is no periodic sequence which consists of only ones or twos.

There is one more pattern that shows up.
There seems to be one set of numbers which has a very short sequence.
In particular, $\sqrt[3]{2}, \sqrt[3]{3}, \sqrt[3]{9}$ and $\sqrt[3]{28}$ have
the shortest period out of all roots.
That is because they have the special form of $\sqrt[3]{D^3 + 1}$ for any integer $D$.
In fact, Bernstein \cite{Bernstein71} has shown that the Jacobi-Perron
algorithm is periodic for any root of the form
% TODO: Is the period length always 1?
% TODO: Check the correct conditions. Specifically is c < D correct?
\[
  \sqrt[3]{D^3 + c} \text{ with } c < D \text{ and } c|D.
\]
So other roots with a short periodic sequence are $\sqrt[3]{65}, \sqrt[3]{66}$,
for example.

Although the Jacobi-Perron algorithm provides a solution for some of these
roots, it does not provide a solution for every root.
In particular, $\sqrt[3]{4}$ does not seem to have a periodic solution for the
Jacobi-Perron algorithm, but it does have one with the generalized Euclidean
algorithm.

\begin{table}[t]
  \caption{
    The shortest periodic index sequences for cube roots found using the
    brute-force search algorithm. The maximum search depth was set to $20$ and
    only the sequence for $29$ was not found. The roots for $8$ and $27$ are
    omitted since they are perfect cubes.}
  \label{tbl:cubics}
  \centering
  \begin{tabular}{ll}
  \uzlhline
  \uzlemph{$p(x)$} & \uzlemph{Index Sequence} \\
  \hline
  $\sqrt[3]{2}$ & $2\overline{12}$ \\
  $\sqrt[3]{3}$ & $12\overline{12}$ \\
  $\sqrt[3]{4}$ & $1\overline{12112211}$ \\
  $\sqrt[3]{5}$ & $1\overline{11222111}$ \\
  $\sqrt[3]{6}$ & $1\overline{11222111}$ \\
  $\sqrt[3]{7}$ & $1\overline{121211}$ \\
  $\sqrt[3]{9}$ & $12\overline{12}$ \\
  $\sqrt[3]{10}$ & $1\overline{2211}$ \\
  $\sqrt[3]{11}$ & $1\overline{2211}$ \\
  $\sqrt[3]{12}$ & $2\overline{1222122212}$ \\
  $\sqrt[3]{13}$ & $11\overline{121111}$ \\
  $\sqrt[3]{14}$ & $1212\overline{1212}$ \\
  $\sqrt[3]{15}$ & $112\overline{112112}$ \\
  $\sqrt[3]{16}$ & $1\overline{12112112121211}$ \\
  $\sqrt[3]{17}$ & $111\overline{22221111}$ \\
  $\sqrt[3]{18}$ & $111112\overline{111112}$ \\
  $\sqrt[3]{19}$ & $112\overline{221112}$ \\
  $\sqrt[3]{20}$ & $1\overline{12211211}$ \\
  $\sqrt[3]{21}$ & $111\overline{22112111}$ \\
  $\sqrt[3]{22}$ & $112112\overline{112112}$ \\
  $\sqrt[3]{23}$ & $1\overline{11211221222112121211}$ \\
  $\sqrt[3]{24}$ & $1\overline{11212211}$ \\
  $\sqrt[3]{25}$ & $2\overline{12212111121221121212}$ \\
  $\sqrt[3]{26}$ & $111\overline{2111}$ \\
  $\sqrt[3]{28}$ & $12\overline{12}$ \\
  $\sqrt[3]{29}$ & - \\
  $\sqrt[3]{30}$ & $1\overline{2211}$ \\
  $\sqrt[3]{31}$ & $2211\overline{221211}$ \\
  $\sqrt[3]{32}$ & $2\overline{11211112112212}$ \\
  \uzlhline
\end{tabular}

\end{table}

\begin{table}[t]
  \caption{Representation of $ψ = \sqrt[3]{4}$ using the brute-force search.}
  \label{table:cube-root-4}
  \centering
  \begin{tabular}{lllllll}
  \uzlhline
  \uzlemph{$\ell$} & \uzlemph{$x_1$} & \uzlemph{$x_2$} & \uzlemph{$x_1$} & \uzlemph{$x_2$} & \uzlemph{$a_1$} & \uzlemph{$a_2$} \\
  \hline
  $0$ & $\psi$ & $\psi^{2}$ & $1.5874$ & $2.51984$ & $0$ & $1$ \\
  \hline
  \hline
  $0$ & $\frac{1}{4} \psi^{2}$ & $\psi - 1$ & $0.62996$ & $0.5874$ & $1$ & $0$ \\
  $0$ & $\psi - 1$ & $\psi^{2} - \psi$ & $0.5874$ & $0.93244$ & $1$ & $1$ \\
  $1$ & $\frac{1}{3} \psi^{2} + \frac{1}{3} \psi - \frac{2}{3}$ & $\psi - 1$ & $0.70241$ & $0.5874$ & $1$ & $1$ \\
  $0$ & $\frac{1}{3} \psi - \frac{1}{3}$ & $\frac{1}{3} \psi^{2} + \frac{1}{3} \psi - \frac{2}{3}$ & $0.1958$ & $0.70241$ & $5$ & $3$ \\
  $1$ & $\psi^{2} + \psi - 4$ & $\psi - 1$ & $0.10724$ & $0.5874$ & $0$ & $1$ \\
  $1$ & $-\frac{2}{3} \psi^{2} + \frac{1}{3} \psi + \frac{4}{3}$ & $\frac{1}{3} \psi^{2} + \frac{1}{3} \psi - \frac{2}{3}$ & $0.18257$ & $0.70241$ & $0$ & $1$ \\
  $1$ & $\frac{1}{2} \psi^{2} - 1$ & $\frac{1}{4} \psi^{2} + \frac{1}{2} \psi - 1$ & $0.25992$ & $0.42366$ & $0$ & $2$ \\
  $0$ & $-\frac{1}{5} \psi^{2} + \frac{1}{5} \psi + \frac{4}{5}$ & $\frac{2}{5} \psi^{2} + \frac{3}{5} \psi - \frac{8}{5}$ & $0.61351$ & $0.36038$ & $1$ & $0$ \\
  \uzlhline
\end{tabular}

\end{table}

\section{Results for Higher Dimensions}

Finding periodic MDCFs for higher dimensions is even more difficult than two dimensions.
There are now $O(d^N)$ possible sequences with a maximum depth of $N$.
So each search is exponentially more expensive than the cubic case.

\section{Evaluation of More Efficient Search Strategies}

Considering the difficulty in finding higher-dimensional MDCFs,
a more efficient search strategy is needed.
The optimal strategy would be deterministic
and give only the next for any given real vector.
We could then follow the chain by evaluating it at every complete convergent to
see which index comes next.
However, even a non-deterministic strategy that eliminates possible candidates
from the search space would be better than brute force.
Therefore, I have analyzed different candidates, both deterministic and
non-deterministic, that could replace the brute-force search.

We can likely rule out alternating pivot sequences since this corresponds to
the Jacobi-Perron algorithm and for this algorithm it is conjectured
\cite{Karpenkov21} that the algorithm is not periodic for $\sqrt[3]{4}$.
However, the brute-force algorithm is periodic for this input (see Table~\ref{table:cube-root-4}).
