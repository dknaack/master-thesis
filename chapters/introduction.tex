\chapter{Introduction}
\label{ch:intro}

In 1839, Charles Hermite wrote a letter \cite{Hermite50} to Jacobi about the representation of real numbers.
He asked whether there exists a representation of the real numbers as a
sequence of integers, which is periodic if and only if the represented number
is a cubic irrational, i.e. the root of a cubic polynomial.
Although he posed the question almost two centuries ago,
the question remains unanswered until this day.

His motivation for the question comes from the representation of real numbers as continued fractions.
The standard way of representing numbers is through decimal notation.
A number is represented by its integer part followed by a potentially infinite
series of digits for the fractional part.
This representation leads to a trivial check, whether the represented number is
rational or not.
The decimal number is finite if and only if the original number is rational.
Continued fractions provide a similar check for quadratic irrationals.
They are fractions of the form
\[
  a₀ + \cfrac{1}{a₁ + \cfrac{1}{a₂ + \cfrac{1}{⋱}}},
\]
where $a₀, a₁, a₂, …$ are integers.
Euler was one of the first mathematician to analyze these fractions
and he showed that any periodic continued fraction is a quadratic irrational.
Later, Lagrange showed the opposite direction,
that every quadratic irrational has a periodic continued fraction.
In his letters,
Hermite was interested in a generalization of continued fraction,
which would be periodic for cubic irrationals.

The appeal of such a representation lies in the remarkable effectiveness of continued fractions.
Not only do they reveal algebraic properties of numbers through periodicity,
they also provide exceptionally good rational approximations.
Each truncation of a continued fraction gives a so-called \emph{convergent},
a rational number that closely approximates the original input.
In fact, these convergents are optimal:
no better approximation with a smaller or equal denominator exists.
For a generalization,
the hope is that it provides good approximations for multiple numbers at once.

Does there exist a representation of the real numbers as a sequence of
integers such that the representation is periodic if and only if the
represented number is a cubic irrational?
There are two parts to this question.
The first is the representation of real numbers as a sequence of integers.
Each finite subsequence of the representation should give us a rational number,
which approaches the represented number as the sequence grows larger.
The second part is about the periodicity of the integer sequence.
It should repeat after some point if and only if the represented is a cubic irrational.
A more general question would ask whether such a periodic sequence exists for
algebraic numbers of a fixed degree.

% ==============================================================================
\section{Contributions of this Thesis}
% ==============================================================================

This thesis extends previous work on a generalization of continued fractions.

This thesis introduces a lattice‐based generalization of the classical Euclidean
algorithm in higher dimensions and uses it to construct multidimensional continued
fractions (MDCFs) for arbitrary real vectors.  By modifying the division‐with‐remainder
step—taking integer parts and iterating on fractional parts in a vector setting—I
obtain a family of MDCF algorithms that extends and unifies all known JPA‐type methods.

A first contribution is a worst‐case analysis of the generalized Euclidean algorithm.
I show that, for inputs given by rational vectors of bit‐length $N$, the number of
division steps grows at most exponentially in $N$, and I derive a corresponding
sequence of “Fibonacci‐like” numbers and an associated golden‐ratio analogue.  This
sequence not only motivates the MDCF construction but also provides explicit bounds
on the running time of any lattice‐based Euclidean procedure.

The core theoretical results concern the MDCFs themselves.  Under the mild technical
assumption that each index in the continued‐fraction expansion is used infinitely
often, I prove convergence of the MDCF expansion for all real vectors.  Moreover,
I establish that any periodic MDCF expansion must encode an algebraic number of
degree at most $d$, where $d$ is the dimension of the construction.  Together,
these results offer a converse to Hermite’s question in one direction—periodicity
implies algebraicity—and classify all JPA‐type algorithms within a single unified
framework.

Finally, I present an implementation of the MDCF algorithms and compare several
construction strategies drawn from the literature.  By measuring approximation
rates and observing empirical periodicity for known algebraic inputs, I evaluate
the feasibility of each strategy in practice.  While the question of proving
periodicity for algebraic vectors remains open, these experiments suggest that
the lattice‐based approach is both efficient and promising for future study.

% ==============================================================================
\section{Related Work}
% ==============================================================================

Since Hermite originally posed his question to Jacobi, it was Jacobi who first attempted to answer it.
He developed an algorithm inspired by the Euclidean algorithm,
but extended it to operate on triples of numbers rather than pairs.
Using this algorithm, he sought to find a periodic representation for cubic irrationals
Jacobi showed that if his algorithm yields a periodic expansion,
then the input must be a cubic irrational.
However, he was unable to prove the converse -- that every cubic irrational leads to a periodic expansion in his algorithm.

Later, Oskar Perron generalized Jacobi's method to arbitrary dimensions \cite{Perron07},
resulting in what is now called the Jacobi–Perron Algorithm (JPA).
His algorithm was further analyzed by Bernstein \cite{Bernstein71},
who identified explicit classes of cubic irrationals for which the JPA
becomes periodic \cite{Bernstein64A, Bernstein65, Bernstein64B}.
However, numerical computations by Elsner and Hasse \cite{Elsner67} have shown
that the JPA seems to fail for certain cube roots,
casting doubt on whether the algorithm can actually solve Hermite’s problem.
As a result, numerous alternative algorithms have since been proposed
\cite{Assaf05, Hendy81, Schweiger00, Schweiger13}.

One such alternative is the family of bifurcating or ternary continued
fractions \cite{Gupta00},
which extend the classical continued fractions to two dimensions.
Using this representation, Murru has constructed periodic expansions for all
cubic irrationals \cite{Murru15}.
While this addresses one half of Hermite’s question --
showing that every cubic irrational can be represented periodically --
it does not provide a representation for arbitrary real numbers,
and thus does not resolve the full problem.

More recently, Karpenkov has proposed two new algorithms \cite{Karpenkov21, Karpenkov24}.
The first is called the $\sin^2$ and he has shown that this algorithm
becomes periodic for every totally real cubic irrational -- that is any roof of
a cubic polynomial with three real roots.
The second algorithm is called the HAPD algorithm \cite{Karpenkov24} and he
conjectures that this algorithm is periodic for all cubic irrationals,
although this conjecture remains unproven.

Such geometric viewpoints have a long history. Felix Klein famously interpreted
continued fractions as points on integer lattices \cite{Klein95}, offering
a visual and structural approach to understanding their behavior.
This interpretation underlies a geometric proof of Lagrange’s theorem, which will
also be presented in this thesis following the work of Korkina \cite{Korkina96}.
Arnol'd has suggested a generalization of this interpretation to higher dimensions \cite{Arnold98}
and it was conjectured that they satisfy an equivalent of Lagrange's theorem but in higher dimensions.
This was eventually proven by German \cite{German08}, further motivating the study of
multidimensional analogues.

Beyond continued fractions, related work has also explored alternative
generalizations of classical number-theoretic functions.
One notable direction is the extension of the Minkowski question-mark function
to two dimensions \cite{Beaver04}, which maps quadratic irrationals to rationals in a highly structured way.
Efforts to define higher-dimensional analogues of this function aim to mirror
the relationship between algebraic numbers and their representations,
offering yet another approach to answering Hermite’s question.

% ==============================================================================
\section{Structure of this Thesis}
% ==============================================================================

Chapter~\ref{ch:preliminaries} introduces the necessary background for this thesis,
which mainly includes algebraic number theory and the Euclidean algorithm.
Chapter~\ref{ch:quadratic} goes through the case of quadratic irrationals
and continued fractions.
Chapter~\ref{ch:generalized-euclidean} introduces the generalized Euclidean algorithm.
In Chapter~\ref{ch:fibonacci}, we analyze the generalized Euclidean algorithm for worst-case performance,
which results in a generalization of Fibonacci numbers and the golden ratio.
This golden ratio is the first case of a periodical representation of an algebraic number.
On the basis of this result, Chapter~\ref{ch:mdcf} generalizes the continued fractions to higher dimensions.
Chapter~\ref{ch:implementation} analyzes the second part of Hermite's problem,
whether multidimensional continued fractions of algebraic numbers are always periodic.
I present examples of such continued fractions for cubic irrationals and I
compare different strategies for constructing these continued fractions.
