\chapter{Introduction}
\label{ch:intro}

\begin{itemize}
  \item Motivation: Continued fractions for quadratic irrationals, Hermite's
    question as generalization of the problem.
  \item Application: Potentially for simultaneous Diophantine approximation
  \item How: Using a generalization of the Euclidean algorithm
\end{itemize}

Is there a representation of the real numbers as a sequence of integers such
that the sequence is eventually periodic if and only if the real number is a
cubic irrational?

\begin{problem}[Hermite's Problem]
  Given a real number $x ∈ ℝ$,
  find a sequence of natural numbers~$\{a_n\}_{n ≥ 0}$ such that
  \begin{enumerate}
    \item $\lim_{n → ∞} a_n = x$
    \item There exists an index $N ≥ 0$ and a period $ℓ ≥ 1$ such that $a_n = a_{n+ℓ}$
      for every $n ≥ N$ if and only if $x$ is a cubic irrational.
  \end{enumerate}
\end{problem}

% TODO: Fix definition of eventually periodic
More formally, we are looking for a function which maps any real number $r$ to
a sequence $(a_n)_{n \in \N}$ such that $\lim_{n → ∞} a_n = r$ and
$a_{k+n} = a_{\ell+n}$ for all $n \ge k$ and some $k \ne \ell$ if and only if there is a cubic polynomial $p$ such that $p(r) = 0$.
Importantly, the function must map \emph{all} real numbers to a sequence.
Nadir Murru \cite{Murru15} has already shown that there exists a mapping from all
cubic irrationals to eventually periodic sequences.
However, the author was unable to derive a sequence for all real numbers.

\section{Contributions of this Thesis}

\begin{itemize}
  \item Concepts from the Euclidean algorithm carry over to the generalized algorithm
  \item Multi-dimensional continued fractions
  \item Proof that the generalized Euclidean algorithm is periodic for a generalization of the metallic ratios
  \item Proof that periodic MDCFs are algebraic numbers of degree $≤ d+1$.
  \item Proof that MDCFs converge under certain conditions
  \item Analysis of the periodicity of cubic roots and other irrational numbers
\end{itemize}

\section{Related Work}

\begin{itemize}
  \item Initial algorithm developed by Jacobi \cite{Jacobi68} and later revised
    by Perron \cite{Perron07}. Overview in \cite{Bernstein71}.
  \item
    Periodical cases proven for the Jacobi-Perron algorithm \cite{Bernstein64}.
  \item
    A continued fractions based on Jacobi-Perron called Bifurcating continued
    fractions \cite{Gupta00}.
  \item
    Representation of cubic irrationals as periodic sequences, notably not a representation of all real numbers. \cite{Murru15}
  \item
    A periodic representation for totally real cubic irrationals \cite{Karpenkov24}.
    The same author has also proposed a similar algorithm which the author
    assumes to be periodic for all cubic irrationals \cite{Karpenkov21}.
\end{itemize}

\section{Structure of this Thesis}

\begin{enumerate}
  \item Preliminaries: Algebraic number theory and the Euclidean algorithm
  \item Quadratic irrationals
  \item Generalized Euclidean Algorithm: Introduction to lattice theory, lattice basis computation
  \item Generalization of the Fibonacci Numbers and their Golden Ratios
  \item Multi-Dimensional Continued Fraction
  \item Experimental Analysis: Periodic cubic roots, approximation rate for
    MDCFs and comparison of different ways of optimizing the algorithm.
\end{enumerate}
