\chapter{Introduction}
\label{ch:intro}

In 1839, Charles Hermite wrote a letter \cite{Hermite50} to Jacobi about the representation of real numbers.
He asked whether there exists a representation of the real numbers as a
sequence of integers, which is periodic if and only if the represented number
is a cubic irrational, i.e. the root of a cubic polynomial.
Although he posed the question almost two centuries ago,
it remains unanswered until this day.

His motivation for the question comes from the representation of real numbers as continued fractions.
The standard way to represent numbers is through decimal notation.
A number is represented by its integer part followed by a potentially infinite
series of digits for the fractional part.
This representation leads to a trivial check, whether the represented number is
rational or not.
The decimal number is finite if and only if the original number is rational.
Continued fractions provide a similar check for quadratic irrationals.
They are fractions of the form
\[
  a₀ + \cfrac{1}{a₁ + \cfrac{1}{a₂ + \cfrac{1}{⋱}}},
\]
where $a₀, a₁, a₂, …$ are integers.
Euler was one of the first mathematicians to analyze these fractions
and he showed that any periodic continued fraction is a quadratic irrational.
Later, Lagrange showed the opposite direction,
that every quadratic irrational has a periodic continued fraction.
In his letters,
Hermite was interested in a generalization of continued fractions,
which would be periodic for cubic irrationals.

The appeal of such a representation lies in the remarkable effectiveness of continued fractions.
Not only do they reveal algebraic properties of numbers through periodicity,
they also provide exceptionally good rational approximations.
Each truncation of a continued fraction gives a so-called \emph{convergent},
a rational number that closely approximates the original input.
In fact, these convergents are optimal:
no better approximation with a smaller or equal denominator exists.
For a generalization,
the hope is that it provides good approximations for multiple numbers at once.

Hermite's question can be easily generalized from cubic irrationals to any algebraic number:
Does there exist a representation of the real numbers as a sequence of
integers that is periodic if and only if the represented number is a algebraic number of degree $d$?
There are two parts to this question.
The first is the representation of real numbers as a sequence of integers.
Each finite subsequence of the representation should give us a rational number,
which approaches the represented number as the sequence grows larger.
The second part is about the periodicity of the integer sequence.
It should repeat after some point if and only if the represented is a cubic irrational.
A more general question would ask whether such a periodic sequence exists for
algebraic numbers of a fixed degree.

Since Hermite originally posed his question to Jacobi, it was Jacobi who first attempted to answer it.
He developed an algorithm inspired by the Euclidean algorithm,
which calculates the greatest common divisor of three numbers instead of two.
At each step,
the algorithm chooses the smallest number and uses it to divide the other two.
In the next triple, the other two numbers are replaced by their remainder.
This process is continued until the greatest common divisor is found.

Later, Oskar Perron generalized Jacobi's method to arbitrary dimensions \cite{Perron07},
resulting in what is now called the Jacobi–Perron Algorithm (JPA).
His algorithm is essentially a generalization of the Euclidean algorithm to $n$ numbers.
At each step, he still chooses the smallest element at each iteration.

% TODO: Should we begin with the Jacobi-Perron algorithm here?

% ==============================================================================
\section{Contributions of this Thesis}
% ==============================================================================

This thesis extends previous work on multidimensional continued fractions (MCFs).
The JPA usually only considers the smallest number in each iteration,
however there are algorithms based on the JPA which choose a different number,
like choosing the largest or choosing some number based on a cost function.
In this thesis, I present a whole class of MCFs,
which encompass all of these JPA-type algorithms.
The main contributions regarding the MCFs are as follows:
\begin{enumerate}
  \item \textbf{Convergence}:
    I establish sufficient conditions under which the proposed class of MCFs
    algorithms converges.
    This generalizes known convergence results for the JPA to a wider family of
    transformations and solves the first part of Hermite's question.
  \item \textbf{Algebraicity from Periodicity}:
    I prove that if a multidimensional continued fraction expansion becomes
    eventually periodic, then the original input vector consists of algebraic
    numbers.
    This result extends the classical correspondence between periodicity and
    quadratic irrationals in one dimension to higher-dimensional settings.
    This solves one direction of the second part to Hermite's question.
    It still leaves open the question whether algebraic numbers always lead to
    periodic MCFs.
\end{enumerate}

In addition to these theoretical results,
I have performed an experimental analysis on the MCFs.
The aim of this analysis was to see which strategies could be used
to construct periodic MCFs for algebraic numbers
and I provide a comparison of existing JPA-type algorithms,
which can be used for constructing MCFs of cubic and quartic irrationals.
One strategy shows particular promise in the construction,
since it has provided a periodic MCF for any cubic and quartic irrational I tested.

The second type of analysis was focused on the application of MCFs.
Since ordinary continued fractions can be used for the best rational approximations
of a single real numbers, the idea for MCFs would be that they provide the best
rational approximations of real vectors.
With my analysis, I show that not all convergents lead to good rational approximations.
I give one specific example where no MCF produces good rational
approximations of a vector at all times.

The basis of these MCFs is a generalization of the Euclidean algorithm from
Klein and Reuter \cite{Klein24}.
The initial aim was to analyze the algorithm on its worst-case performance
and whether there exists some generalization of Fibonacci numbers,
which represent the worst-case for the classical Euclidean algorithm.
As such, a side result of this thesis is a proof that such Fibonacci numbers do
exist at least for one strategy and, more importantly, that they represent the
worst-case for this strategy.
Using these numbers, I derive the multidimensional analogue of the golden
ratio, which can be seen as one of the simplest example of a periodic MCF.

% ==============================================================================
\section{Related Work}
% ==============================================================================

As previously mentioned,
one of the first algorithms studied for this problem is the JPA.
The algorithm was later analyzed by Bernstein \cite{Bernstein71},
who identified explicit classes of cubic irrationals for which the JPA
becomes periodic \cite{Bernstein64A, Bernstein65, Bernstein64B}.
However, numerical computations by Elsner and Hasse \cite{Elsner67} have shown
that the JPA seems to fail for certain cube roots,
casting doubt on whether the algorithm can actually solve Hermite’s problem.
As a result, numerous alternative algorithms have since been proposed
\cite{Assaf05, Hendy81, Schweiger00, Schweiger13}.

One such alternative is the family of bifurcating or ternary continued
fractions \cite{Gupta00},
which extend the classical continued fractions to two dimensions.
Using this representation, Murru has constructed periodic expansions for all
cubic irrationals \cite{Murru15}.
While this addresses one half of Hermite’s question --
showing that every cubic irrational can be represented periodically --
it does not provide a representation for arbitrary real numbers,
and thus does not resolve the full problem.

More recently, Karpenkov has proposed two new algorithms \cite{Karpenkov21, Karpenkov24}.
The first is called the $\sin^2$-algorithm and he has shown that this algorithm
becomes periodic for every totally real cubic irrational -- that is any roof of
a cubic polynomial with three real roots.
The second algorithm is called the HAPD algorithm \cite{Karpenkov24} and he
conjectures that this algorithm is periodic for all cubic irrationals,
although this conjecture remains unproven.

Such geometric viewpoints have a long history.
Felix Klein famously interpreted continued fractions as points on integer
lattices \cite{Klein95}, offering a visual and structural approach to
understanding their behavior.
This interpretation underlies a geometric proof of Lagrange’s theorem, which
will also be presented in this thesis following the work of
Korkina~\cite{Korkina96}.
Arnol'd has suggested a generalization of this interpretation to higher
dimensions \cite{Arnold98} and it was conjectured that they satisfy an
equivalent of Lagrange's theorem but in higher dimensions.
This was eventually proven by German \cite{German08}, further motivating the
study of multidimensional analogues.

Beyond continued fractions, there have been alternative
generalizations of classical number-theoretic functions.
One notable direction is the extension of the Minkowski question-mark function
to two dimensions \cite{Beaver04}, which maps quadratic irrationals to rationals in a highly structured way.
Efforts to define higher-dimensional analogues of this function aim to mirror
the relationship between algebraic numbers and their representations,
offering yet another approach to answering Hermite’s question.

% ==============================================================================
\section{Structure of this Thesis}
% ==============================================================================

Chapter~\ref{ch:preliminaries} introduces the necessary background for this thesis,
which mainly includes algebraic number theory and the Euclidean algorithm.
Chapter~\ref{ch:quadratic} goes through the case of quadratic irrationals
and continued fractions.
Chapter~\ref{ch:generalized-euclidean} introduces the generalized Euclidean algorithm.
In Chapter~\ref{ch:fibonacci}, we analyze the generalized Euclidean algorithm for worst-case performance,
which results in a generalization of Fibonacci numbers and the golden ratio.
This golden ratio is the first case of a periodical representation of an algebraic number.
On the basis of this result, Chapter~\ref{ch:mdcf} generalizes the continued fractions to higher dimensions.
Chapter~\ref{ch:implementation} analyzes the second part of Hermite's problem,
whether multidimensional continued fractions of algebraic numbers are always periodic.
I present examples of such continued fractions for cubic irrationals and I
compare different strategies for constructing these continued fractions.
