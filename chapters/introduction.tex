\chapter{Introduction}
\label{ch:intro}

In 1839, Charles Hermite wrote a letter to Jacobi~\cite{Hermite50} about the
representation of real numbers.
He asked whether there exists a representation of the real numbers as a
sequence of integers that is periodic if and only if the represented number
is a cubic irrational, i.e. the root of a cubic polynomial.
Although he posed the question nearly two centuries ago,
it remains unanswered to this day.

The standard way to represent numbers is through decimal notation.
A number is represented as a sequence of digits, which begins with the integer
part and is followed by a (potentially infinite) sequence of digits for the
fractional part.
If the decimal expansion of a number is finite, then the number is clearly rational.
Furthermore, if the decimal expansion is periodic, then the number is also rational.
The same behavior occurs for continued fractions and quadratic irrationals.
Continued fractions are fractions of the form
\[
  a₀ + \cfrac{1}{a₁ + \cfrac{1}{a₂ + \cfrac{1}{⋱}}},
\]
where $a₀, a₁, a₂, …$ are integers.
Every real number has a continued fraction expansion,
which can be constructed using the Euclidean algorithm.
If that continued fraction is periodic, then the number must be a quadratic irrational.
More importantly, the converse is also true:
If a number is a quadratic irrational,
then its continued fraction must be eventually periodic.
Naturally, we may ask whether we can extend this and
find a periodic representation for cubic irrationals.
However, no such representation exists as of yet.

The interest in a generalization to cubic irrationals
comes from the effectiveness of continued fractions in related fields.
The primary example is Diophantine approximation, where the goal is to
approximate real numbers using rational numbers as closely as possible.
It turns out that the best rational approximations are precisely those provided
by continued fractions.
A generalization of continued fractions could serve a similar role in the field
of simultaneous Diophantine approximation, where the goal is to approximate
multiple real numbers at once with a single rational vector.

Hermite's original question only applies to cubic irrationals,
but it can be easily generalized to any algebraic number:
Does there exist a representation of the real numbers as a sequence of integers
that is periodic if and only if the represented number is an algebraic number of
degree $d$?
There are two parts to this question.
The first is the representation of real numbers as a sequence of integers.
Each finite subsequence of the representation should give us a rational number,
which approaches the represented number as the sequence grows larger.
The second part is about the periodicity of the integer sequence.
In the original question, the sequence should repeat after some point if and
only if the represented number is a cubic irrational.
The general question asks whether a periodic sequence exists for any algebraic
number with a certain degree.

% ==============================================================================
\section{Background}
\label{sec:jacobi-perron}
% ==============================================================================

Since Hermite originally posed his question to Jacobi, it was he who first
attempted to answer it.
He developed an algorithm \cite{Jacobi68} inspired by the Euclidean algorithm,
which calculates the greatest common divisor of three numbers instead of two.
At each step,
the algorithm chooses the smallest number and uses it to divide the other two.
In the next triple, the other two numbers are replaced by their remainder.
This process is continued until the greatest common divisor is found.
Later, Oskar Perron generalized Jacobi's method to arbitrary dimensions \cite{Perron07},
resulting in what is now called the Jacobi–Perron Algorithm (JPA).
His algorithm is essentially a generalization of the Euclidean algorithm to $n$ numbers.
At each step, he still chooses the smallest element at each iteration.

Continued fractions are typically constructed using the Gauss transformation,
which is defined as
\begin{align*}
  T(x) = \frac{1}{x - \floor{x}}.
\end{align*}
This transformation is applied repeatedly until $T^m(x) = T^n(x)$ for $m ≠ n$,
in which case the continued fraction is periodic.
For example, $T(\sqrt{2}) = 1/(\sqrt{2} - 1) = \sqrt{2} + 1$
and $T^2(\sqrt{2}) = T(\sqrt{2} + 1) = \sqrt{2} + 1$.
Thus, the continued fraction of $\sqrt{2}$ is periodic after two iterations.

The Jacobi-Perron uses a similar transform,
which takes a vector $x = (x₁, …, x_d)$ as input and calculates
\begin{align*}
  T(x₁, …, x_d) =
  \left(
  \frac{x_2 - \floor{x_2}}{x_1 - \floor{x_1}},
  \frac{x_3 - \floor{x_3}}{x_1 - \floor{x_1}},
  …,
  \frac{x_d - \floor{x_d}}{x_1 - \floor{x_1}},
  \frac{1}{x_1 - \floor{x_1}}
  \right)
\end{align*}
Again, this transformation is applied repeatedly until $T^m(x) = T^n(x)$ for $m ≠ n$,
at which point it becomes periodic.
Perron showed that if the algorithm becomes periodic,
then each $x_i$ is an algebraic number with degree $d+1$.
However, he was not able to show the other direction.

In an attempt to find an algorithm, which solves both directions,
previous work often replaces this transformation with a different one,
which can be considered JPA-type algorithms.
For example, a different transformation could use the second smallest number.
They usually consider only one particular transformation and iterate this until a period has been found.
Despite numerous different transformations being proposed,
Hermite's question remains open.

% ==============================================================================
\section{Contributions of this Thesis}
\label{sec:contributions}
% ==============================================================================

% The other algorithms focus on a single path, whereby the only use their own
% transformation function to find a periodic representation.
Whereas previous algorithms typically only consider a single transformation,
I present a framework that is able to choose any one of these transformations.
This framework will be called \emph{multidimensional continued fractions} (MCF),
since it is based on a generalization of the Euclidean algorithm by Klein and
Reuter~\cite{Klein24}.

The main contribution of this thesis is a theoretical analysis of the
multidimensional continued fractions.
This includes two important theorems.
The first shows that many multidimensional continued fractions converge.
This was neglected by Jacobi at first, but it was eventually shown by Perron.
Since then, the main focus has returned to the second part
with the convergence often being implicitly assumed.
However, the convergence is crucial for a correct representation of the real numbers.
For this thesis, I give an explicit proof that many multidimensional continued
fractions converge.
This covers the first part of Hermite's question.
The second theorem partially solves the second part of Hermite's question.
It shows that every periodic multidimensional continued fraction leads to an
algebraic number, thus solving the first direction.
The converse still remains open.

Nevertheless, I have performed an experimental analysis on the MCFs,
which strongly suggests that every cubic irrational has a periodic MCF.
The aim of this analysis was to evaluate different strategies
for constructing periodic MCFs for algebraic numbers.
One of these strategies shows particular promise in the construction,
since it has provided a periodic MCF for every cubic and quartic irrational I tested.
In addition, I have evaluated the performance of MCFs in the field of
simultaneous Diophantine approximation.
Since ordinary continued fractions can be used for the best rational approximations
of a single real number, the idea for MCFs would be that they provide the best
rational approximations for real vectors.
With my analysis, I show that not all convergents lead to good rational approximations.
I give one specific example where no MCF produces good rational
approximations of a vector at all times.

The basis of these MCFs is a generalization of the Euclidean algorithm from
Klein and Reuter \cite{Klein24}.
The initial aim was to analyze the algorithm on its worst-case performance
and whether there exists some generalization of Fibonacci numbers,
which represent the worst-case for the classical Euclidean algorithm.
As such, a side result of this thesis is a proof that such Fibonacci numbers exist, at least for one strategy.
More importantly, I show that they represent the worst-case for this strategy.
Using these numbers, I derive the multidimensional analogue of the golden
ratio, which can be seen as one of the simplest examples of a periodic MCF.

In summary, there are three main contributions:
\begin{enumerate}
  \item A new class of multidimensional continued fractions, including a proof
    of their convergence and how they lead to algebraic numbers.
  \item An experimental analysis of the multidimensional continued fractions
    on cubic irrationals as well as their application in simultaneous
    Diophantine approximation.
  \item Worst-case bounds for the generalized Euclidean algorithm
    using higher-order Fibonacci numbers and multidimensional golden ratios.
\end{enumerate}

% ==============================================================================
\section{Related Work}
\label{sec:related-work}
% ==============================================================================

As previously mentioned,
one of the first algorithms studied for this problem is the JPA.
Initially developed by Jacobi and Perron,
the algorithm was analyzed again by Bernstein~\cite{Bernstein71},
who identified explicit classes of cubic irrationals which are periodic under
the JPA \cite{Bernstein64A, Bernstein65, Bernstein64B}.
However, numerical computations by Elsner and Hasse \cite{Elsner67} have shown
that the JPA seems to fail for certain cube roots,
casting doubt on whether the algorithm can actually solve Hermite’s problem.
As a result, numerous alternative algorithms have since been proposed
\cite{Assaf05, Hendy81, Schweiger13, Schweiger00}.

The most prominent examples include the subtractive algorithms by
Brun~\cite{Brun19} and Selmer~\cite{Selmer67}
as well as the fully subtractive algorithms introduced by Schweiger~\cite{Schweiger95}.
These have been analyzed for their application in Diophantine approximation
based on their ergodic properties~\cite{Schweiger04}.
However, they do not provide a full answer to Hermite's question.

There are also generalizations of continued fractions to two dimensions,
called bifurcating or ternary continued fractions \cite{Gupta00,Daus22}.
These have been used by Murru~\cite{Murru15} to construct periodic expansions
for all cubic irrationals.
While this addresses one half of Hermite’s question,
it does not provide a representation for arbitrary real numbers.
It only provides a representation of cubic irrationals
and thus, it does not provide a full answer to Hermite's question.

More recently, Karpenkov has proposed two new algorithms \cite{Karpenkov24, Karpenkov21}.
The first is called the $\sin^2$-algorithm and he has shown that this algorithm
becomes periodic for every totally real cubic irrational -- that is any roof of
a cubic polynomial with three real roots.
The second algorithm is called the HAPD algorithm \cite{Karpenkov24} and he
conjectures that this algorithm is periodic for all cubic irrationals,
although this conjecture remains unproven.

Apart from the algorithmic approach,
there has also been significant effort in a geometric generalization of
continued fractions.
Felix Klein famously interpreted the convergents of a continued fractions as
points on integer lattices \cite{Klein95}.
His interpretation has lead to a geometric proof of Lagrange’s theorem, which
will also be presented in this thesis following the work of Korkina~\cite{Korkina96}.
Arnol'd has suggested a generalization of this interpretation to higher
dimensions \cite{Arnold98} and it was conjectured that they satisfy a
multidimensional equivalent of the Lagrange's famous theorem,
which shows that every quadratic irrational has an eventually periodic
continued fraction.
This conjecture was eventually proven by German \cite{German08},
which indicates that there could be a connection between the multidimensional
continued fractions and algebraic numbers.

Beyond multidimensional continued fractions, there have been alternative
generalizations of classical number-theoretic functions.
One notable example is the Minkowski question-mark function $?(x)$,
which maps a quadratic irrational $x$ to a rational number.
Efforts to define higher-dimensional analogues of this function aim to mirror
the relationship between algebraic numbers and their representations.
There has been an extension of this function to two dimensions \cite{Beaver04},
but this has not solved Hermite's question, yet.

% ==============================================================================
\section{Structure of this Thesis}
\label{sec:structure}
% ==============================================================================

Chapter~\ref{ch:preliminaries} introduces the necessary background for this thesis,
which primarily includes algebraic number theory and lattice theory.
Chapter~\ref{ch:quadratic} examines the case of quadratic irrationals and continued fractions.
It presents a geometric interpretation of continued fractions based on Klein polygons,
which is used in the subsequent proof of Lagrange's theorem.
Chapter~\ref{ch:generalized-euclidean} introduces the generalized Euclidean algorithm.
In Chapter~\ref{ch:fibonacci}, the generalized Euclidean algorithm is analyzed
for worst-case performance under two different strategies.
Each strategy results in a different generalization of the Fibonacci numbers
and in its own definition of a golden ratio.
This golden ratio is the first case of a periodic representation of an algebraic number.
Building on this result, Chapter~\ref{ch:mdcf} generalizes continued fractions to higher dimensions
and presents the two main theoretical results of the thesis: convergence and periodicity.
Chapter~\ref{ch:implementation} analyzes the second part of Hermite's problem,
whether multidimensional continued fractions of algebraic numbers are always periodic.
In this chapter, examples of such continued fractions for cubic irrationals are presented,
and different strategies for constructing these continued fractions are compared.
