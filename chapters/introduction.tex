\chapter{Introduction}

\begin{problem}[Hermite's Question]
  Is there a representation of the real numbers as a sequence of integers such
  that the sequence is eventually periodic if and only if the real number is a
  cubic irrational?
\end{problem}

% TODO: Fix definition of eventually periodic
More formally, we are looking for a function which maps any real number $r$ to
a sequence $(a_n)_{n \in \N}$ such that $a_{k+n} = a_{\ell+n}$ for all $n \ge k$ and
some $k \ne \ell$ if and only if there is a cubic polynomial $p$ such that $p(r) = 0$.
Importantly, the function must map \emph{all} real numbers to a sequence.
Nadir Murru \cite{Murru15} has already shown that there exists a mapping from all
cubic irrationals to eventually periodic sequences.
However, the author was unable to derive a sequence for all real numbers.

\section{Contributions}

\section{Related Work}

\begin{itemize}
  \item Initial algorithm developed by Jacobi \cite{Jacobi68} and later revised
    by Perron \cite{Perron07}. Overview in \cite{Bernstein71}.
  \item
    Periodical cases proven for the Jacobi-Perron algorithm \cite{Bernstein64}.
  \item
    A continued fractions based on Jacobi-Perron called Bifurcating continued
    fractions \cite{Gupta00}.
  \item
    Representation of cubic irrationals. \cite{Murru15}
  \item
    A periodic representation for totally real cubic irrationals \cite{Karpenkov24}.
    The same author has also proposed a similar algorithm which the author
    assumes to be periodic for all cubic irrationals \cite{Karpenkov21}.
\end{itemize}

\section{Structure of this Thesis}
