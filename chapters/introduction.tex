\chapter{Introduction}
\label{ch:intro}

% Hook
In 1848, Charles Hermite wrote a question to Jacobi on the representation of real numbers.
He asked whether there exists a representation of the real numbers as a
sequence of integers, which is periodic if and only if the represented number
is a cubic irrational, i.e. the root of a cubic polynomial.
Although he posed the question almost two centuries ago,
the question remains unanswered until this day.

% Context
His motivation for the question comes from the representation of real numbers as continued fractions.
The standard way of representing numbers is through decimal notation.
A number is represented by its integer part followed by a potentially infinite
series of digits for the fractional part.
This representation leads to a trivial check, whether the represented number is
rational or not.
The decimal number is finite if and only if the original number is rational.
Continued fractions provide a similar check for quadratic irrationals.
They are fractions of the form
\[
  a₀ + \cfrac{1}{a₁ + \cfrac{1}{a₂ + \cfrac{1}{⋱}}},
\]
where $a₀, a₁, a₂, …$ are integers.
Euler was one of the first mathematician to analyze these fractions
and he showed that any periodic continued fraction is a quadratic irrational.
Later, Lagrange showed the opposite direction,
that every quadratic irrational has a periodic continued fraction.
In his letters,
Hermite was interested in a generalization of continued fraction,
which would be periodic for cubic irrationals.

% Why it's important -> Rational approximation -> Simultaneous approximation
The appeal of such a representation lies in the remarkable effectiveness of continued fractions.
Not only do they reveal algebraic properties of numbers through periodicity,
they also provide exceptionally good rational approximations.
Each truncation of a continued fraction gives a so-called \emph{convergent},
a rational number that closely approximates the original input.
In fact, these convergents are optimal:
no better approximation with a smaller or equal denominator exists.
For a generalization,
the hope is that it provides good approximations for multiple numbers at once.

\section{Contributions of this Thesis}

This thesis extends previous work on a generalization of continued fractions.

This thesis introduces a lattice‐based generalization of the classical Euclidean
algorithm in higher dimensions and uses it to construct multidimensional continued
fractions (MDCFs) for arbitrary real vectors.  By modifying the division‐with‐remainder
step—taking integer parts and iterating on fractional parts in a vector setting—I
obtain a family of MDCF algorithms that extends and unifies all known JPA‐type methods.

A first contribution is a worst‐case analysis of the generalized Euclidean algorithm.
I show that, for inputs given by rational vectors of bit‐length $N$, the number of
division steps grows at most exponentially in $N$, and I derive a corresponding
sequence of “Fibonacci‐like” numbers and an associated golden‐ratio analogue.  This
sequence not only motivates the MDCF construction but also provides explicit bounds
on the running time of any lattice‐based Euclidean procedure.

The core theoretical results concern the MDCFs themselves.  Under the mild technical
assumption that each index in the continued‐fraction expansion is used infinitely
often, I prove convergence of the MDCF expansion for all real vectors.  Moreover,
I establish that any periodic MDCF expansion must encode an algebraic number of
degree at most $d$, where $d$ is the dimension of the construction.  Together,
these results offer a converse to Hermite’s question in one direction—periodicity
implies algebraicity—and classify all JPA‐type algorithms within a single unified
framework.

Finally, I present an implementation of the MDCF algorithms and compare several
construction strategies drawn from the literature.  By measuring approximation
rates and observing empirical periodicity for known algebraic inputs, I evaluate
the feasibility of each strategy in practice.  While the question of proving
periodicity for algebraic vectors remains open, these experiments suggest that
the lattice‐based approach is both efficient and promising for future study.

\section{Related Work}

\begin{itemize}
  \item Initial algorithm developed by Jacobi \cite{Jacobi68} and later revised
    by Perron \cite{Perron07}. Overview in \cite{Bernstein71}.
  \item
    Periodical cases proven for the Jacobi-Perron algorithm \cite{Bernstein64}.
  \item
    A continued fractions based on Jacobi-Perron called Bifurcating continued
    fractions \cite{Gupta00}.
  \item
    Representation of cubic irrationals as periodic sequences, notably not a representation of all real numbers. \cite{Murru15}
  \item
    A periodic representation for totally real cubic irrationals \cite{Karpenkov24}.
    The same author has also proposed a similar algorithm which the author
    assumes to be periodic for all cubic irrationals \cite{Karpenkov21}.
\end{itemize}

\section{Structure of this Thesis}

\begin{enumerate}
  \item Preliminaries: Algebraic number theory and the Euclidean algorithm
  \item Quadratic irrationals
  \item Generalized Euclidean Algorithm: Introduction to lattice theory, lattice basis computation
  \item Generalization of the Fibonacci Numbers and their Golden Ratios
  \item Multi-Dimensional Continued Fraction
  \item Experimental Analysis: Periodic cubic roots, approximation rate for
    MDCFs and comparison of different ways of optimizing the algorithm.
\end{enumerate}
