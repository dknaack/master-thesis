\chapter{Introduction}
\label{ch:intro}

% Hook
In 1848, Charles Hermite wrote a question to Jacobi on the representation of real numbers.
He asked whether there exists a representation of the real numbers as a
sequence of integers, which is periodic if and only if the represented number
is a cubic irrational, i.e. the root of a cubic polynomial.
Although he posed the question almost two centuries ago,
the question remains unanswered until this day.

% Context
His motivation for the question comes from the representation of real numbers as continued fractions.
The standard way of representing numbers is through decimal notation.
A number is represented by its integer part followed by a potentially infinite
series of digits for the fractional part.
This representation leads to a trivial check, whether the represented number is
rational or not.
The decimal number is finite if and only if the original number is rational.
Continued fractions provide a similar check for quadratic irrationals.
They are fractions of the form
\[
  a₀ + \cfrac{1}{a₁ + \cfrac{1}{a₂ + \cfrac{1}{⋱}}},
\]
where $a₀, a₁, a₂, …$ are integers.
Euler was one of the first mathematician to analyze these fractions
and he showed that any periodic continued fraction is a quadratic irrational.
Later, Lagrange showed the opposite direction,
that every quadratic irrational has a periodic continued fraction.
In his letters,
Hermite was interested in a generalization of continued fraction,
which would be periodic for cubic irrationals.

% Why it's important -> Rational approximation -> Simultaneous approximation
The appeal of such a representation lies in the remarkable effectiveness of continued fractions.
Not only do they reveal algebraic properties of numbers through periodicity,
they also provide exceptionally good rational approximations.
Each truncation of a continued fraction gives a so-called \emph{convergent},
a rational number that closely approximates the original input.
In fact, these convergents are optimal:
no better approximation with a smaller or equal denominator exists.
For a generalization,
the hope is that it provides good approximations for multiple numbers at once.

\section{Contributions of this Thesis}

\begin{itemize}
  \item Concepts from the Euclidean algorithm carry over to the generalized algorithm
  \item Multi-dimensional continued fractions
  \item Proof that the generalized Euclidean algorithm is periodic for a generalization of the metallic ratios
  \item Proof that periodic MDCFs are algebraic numbers of degree $≤ d+1$.
  \item Proof that MDCFs converge under certain conditions
  \item Analysis of the periodicity of cubic roots and other irrational numbers
\end{itemize}

\section{Related Work}

\begin{itemize}
  \item Initial algorithm developed by Jacobi \cite{Jacobi68} and later revised
    by Perron \cite{Perron07}. Overview in \cite{Bernstein71}.
  \item
    Periodical cases proven for the Jacobi-Perron algorithm \cite{Bernstein64}.
  \item
    A continued fractions based on Jacobi-Perron called Bifurcating continued
    fractions \cite{Gupta00}.
  \item
    Representation of cubic irrationals as periodic sequences, notably not a representation of all real numbers. \cite{Murru15}
  \item
    A periodic representation for totally real cubic irrationals \cite{Karpenkov24}.
    The same author has also proposed a similar algorithm which the author
    assumes to be periodic for all cubic irrationals \cite{Karpenkov21}.
\end{itemize}

\section{Structure of this Thesis}

\begin{enumerate}
  \item Preliminaries: Algebraic number theory and the Euclidean algorithm
  \item Quadratic irrationals
  \item Generalized Euclidean Algorithm: Introduction to lattice theory, lattice basis computation
  \item Generalization of the Fibonacci Numbers and their Golden Ratios
  \item Multi-Dimensional Continued Fraction
  \item Experimental Analysis: Periodic cubic roots, approximation rate for
    MDCFs and comparison of different ways of optimizing the algorithm.
\end{enumerate}
