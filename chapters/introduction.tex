\chapter{Introduction}
\label{ch:intro}

% Hook
In 1848, Charles Hermite wrote a question to Jacobi on the representation of real numbers.
He asked whether there exists a representation of the real numbers as a
sequence of integers, which is periodic if and only if the represented number
is a cubic irrational, i.e. the root of a cubic polynomial.
Although he posed the question almost two centuries ago,
the question remains unanswered until this day.

% Context
His motivation for the question comes from the representation of real numbers as continued fractions.
The standard way of representing numbers is through decimal notation.
A number is represented by its integer part followed by a potentially infinite
series of digits for the fractional part.
This representation leads to a trivial check, whether the represented number is
rational or not.
The decimal number is finite if and only if the original number is rational.
Continued fractions provide a similar check for quadratic irrationals.
They are fractions of the form
\[
  a₀ + \cfrac{1}{a₁ + \cfrac{1}{a₂ + \cfrac{1}{⋱}}},
\]
where $a₀, a₁, a₂, …$ are integers.
Euler was one of the first mathematician to analyze these fractions
and he showed that any periodic continued fraction is a quadratic irrational.
Later, Lagrange showed the opposite direction,
that every quadratic irrational has a periodic continued fraction.
In his letters,
Hermite was interested in a generalization of continued fraction,
which would be periodic for cubic irrationals.

% Why it's important -> Rational approximation -> Simultaneous approximation
The appeal of such a representation lies in the remarkable effectiveness of continued fractions.
Not only do they reveal algebraic properties of numbers through periodicity,
they also provide exceptionally good rational approximations.
Each truncation of a continued fraction gives a so-called \emph{convergent},
a rational number that closely approximates the original input.
In fact, these convergents are optimal:
no better approximation with a smaller or equal denominator exists.
For a generalization,
the hope is that it provides good approximations for multiple numbers at once.

\section{Contributions of this Thesis}

This thesis extends previous work on a generalization of continued fractions.

This thesis introduces a lattice‐based generalization of the classical Euclidean
algorithm in higher dimensions and uses it to construct multidimensional continued
fractions (MDCFs) for arbitrary real vectors.  By modifying the division‐with‐remainder
step—taking integer parts and iterating on fractional parts in a vector setting—I
obtain a family of MDCF algorithms that extends and unifies all known JPA‐type methods.

A first contribution is a worst‐case analysis of the generalized Euclidean algorithm.
I show that, for inputs given by rational vectors of bit‐length $N$, the number of
division steps grows at most exponentially in $N$, and I derive a corresponding
sequence of “Fibonacci‐like” numbers and an associated golden‐ratio analogue.  This
sequence not only motivates the MDCF construction but also provides explicit bounds
on the running time of any lattice‐based Euclidean procedure.

The core theoretical results concern the MDCFs themselves.  Under the mild technical
assumption that each index in the continued‐fraction expansion is used infinitely
often, I prove convergence of the MDCF expansion for all real vectors.  Moreover,
I establish that any periodic MDCF expansion must encode an algebraic number of
degree at most $d$, where $d$ is the dimension of the construction.  Together,
these results offer a converse to Hermite’s question in one direction—periodicity
implies algebraicity—and classify all JPA‐type algorithms within a single unified
framework.

Finally, I present an implementation of the MDCF algorithms and compare several
construction strategies drawn from the literature.  By measuring approximation
rates and observing empirical periodicity for known algebraic inputs, I evaluate
the feasibility of each strategy in practice.  While the question of proving
periodicity for algebraic vectors remains open, these experiments suggest that
the lattice‐based approach is both efficient and promising for future study.

\section{Related Work}

Since the question was initially posed to Jacobi, he was also the first one to tackle the question.
He developed an algorithm which worked like the Euclidean algorithm but with three instead of two numbers.
Using this algorithm, he sought to find a periodic representation for cubic irrationals
and he was able to show that whenever his algorithm is periodic, then the input must be a cubic irrational.
However, he was unable to prove the other direction.

Perron extended Jacobi's work to arbitrarily many numbers \cite{Perron07}
and it is therefore commonly called the Jacobi-Perron Algorithm (JPA).
His algorithm was further analyzed by Bernstein \cite{Bernstein71},
who found an infinite class of cubic irrationals for which the algorithm
becomes periodic \cite{Bernstein65, Bernstein64A, Bernstein64B}.
Unfortunately, computations have indicated that the algorithm is not periodic
for certain cube roots \cite{Elsner67}
and the algorithm is no longer believed to be periodic for every cubic irrational.
Therefore, many different algorithms have been developed \cite{Schweiger00,Hendy81,Schweiger13}.

On the basis of the JPA,
there are so-called bifurcating or ternary continued fractions \cite{Gupta00},
which are one possible generalization of continued fractions to two dimensions.
Murru has used these to find a periodic representation for every cubic irrrational.
But this only solves the second part of Hermite's problem.
The first part, the representation for every real number, remains unanswered by his solution.

Recently, Karpenkov has proposed a different algorithm \cite{Karpenkov21}, which he calls the $\sin^2$-algorithm.
He has proven that his algorithm is periodic for every totally real cubic
irrational; that is any root of a cubic polynomial, where all roots are real.
He has proposed another algorithm, called the HAPD algorithm \cite{Karpenkov24}, which he
conjectures to be periodic for all cubic irrationals.
As of yet, his conjecture remains unproven.

Continued fractions have also been analyzed geometrically by Felix Klein \cite{Klein95}
as point in an integer lattice.
His interpretation can be used to geometrically prove Lagrange's theorem
and this proof based on the work of Korkina \cite{Korkina96} will also be
presented in this thesis.
This has lead to a generalization of Lagrange's theorem to higher dimensions \cite{German08}.

Apart from a generalization of continued fractions,
there has also been effort into other directions.
The notable example here is a generalization of the Minkowski question-mark function $?(x)$.

\section{Structure of this Thesis}

Chapter~\ref{ch:preliminaries} introduces the necessary background for this thesis,
which mainly includes algebraic number theory and the Euclidean algorithm.
Chapter~\ref{ch:quadratic} goes through the case of quadratic irrationals
and continued fractions.
Chapter~\ref{ch:generalized-euclidean} introduces the generalized Euclidean algorithm.
In Chapter~\ref{ch:fibonacci}, we analyze the generalized Euclidean algorithm for worst-case performance,
which results in a generalization of Fibonacci numbers and the golden ratio.
This golden ratio is the first case of a periodical representation of an algebraic number.
On the basis of this result, Chapter~\ref{ch:mdcf} generalizes the continued fractions to higher dimensions.
Chapter~\ref{ch:implementation} analyzes the second part of Hermite's problem,
whether multidimensional continued fractions of algebraic numbers are always periodic.
I present examples of such continued fractions for cubic irrationals and I
compare different strategies for constructing these continued fractions.
