\chapter{Multi-Dimensional Continued Fractions}
\label{ch:mdcf}

For quadratic irrationals, we used continued fractions to represent them as a
periodic sequence of integers.
The continued fractions were constructed using the Euclidean algorithm.
Naturally, we can generalize continued fractions to higher dimensions using the
generalized Euclidean algorithm.
This leads to a concept of multi-dimensional continued fractions, which could
potentially lead to an answer of Hermite's question.
Since it is based on the generalized Euclidean algorithm,
it extends previous generalizations of continued fractions based on the
Jacobi-Perron algorithm \cite{Gupta00}.

This chapter introduces the concept of MDCFs and discusses their potential as
an answer to Hermite's question.
We begin by defining what they are and deriving many properties similar to
continued fractions.
Subsequently, we look at numbers represented by periodic MDCFs and whether they
must be algebraic numbers of degree $d+1$.
Finally, we discuss the geometry behind the multi-dimensional continued fractions.
The other direction -- Whether algebraic numbers of degree $d+1$ admit periodic
MDCFs -- will be discussed in the next chapter.

% ==============================================================================
\section{Construction using the Generalized Euclidean Algorithm}
% ==============================================================================

To derive the continued fraction for a real number, we have used the Euclidean algorithm.
In particular, we looked at the ratio $a/b$ between the two inputs $a$ and $b$,
and we have used the integer part of that ratio as a coefficient in the
continued fraction.
Iterating the Euclidean algorithm then gave us a unique representation for
every real number.
The Euclidean algorithm can also be viewed as the generalized algorithm when $d = 1$.
In this case, the ratio $a/b$ represents the solution $x$ of the linear system $bx = a$.
So the integer part of the solution $x$ also represents a coefficient in the
continued fraction.
We can then use the pivot operation to iterate over this solution and gain the
rest of the coefficients.

The multi-dimensional continued fraction can now be easily derived from the
solution vector $x$ in higher dimensions.
Specifically, given a vector $x ∈ ℝ^d$, we take the integer part of each
element $\floor{x}$ as a coefficient of the multi-dimensional continued
fraction.
Then, we get the subsequent coefficients by iterating over this vector using
the pivot operation.
This gives us a top-down construction of MDCFs, but typically continued
fractions are defined in a bottom-up fashion.
For a bottom-up definition of the MDCFs, we simply reverse the pivot operation.
The inverse operation can be derived as follows:
Let $x ∈ ℝ^d$ and $a = \floor{x}$.
If $x' = \mathrm{pivot}_ℓ(x)$ for a given index $ℓ$, then
\[
  \begin{array}{lcrlcr}
    \displaystyle x_i' & = & \displaystyle \frac{x_i - a_i}{x_ℓ - a_ℓ}, &
    \displaystyle x_ℓ' & = & \displaystyle \frac{x_i - a_i}{x_ℓ - a_ℓ} \\[1em]
  \end{array}
\]
for all $i ≠ ℓ$.
For the inverse operation, we have to derive every element of $x$ from $x'$,
which can be done by the following equations:
\[
  \begin{array}{lcrlcr}
    \displaystyle x_i & = & a_i + \displaystyle \frac{x_i'}{x_ℓ'}, &
    \displaystyle x_ℓ & = & a_ℓ + \displaystyle \frac{1}{x_ℓ'}
  \end{array}
\]
This allows us to calculate the previous vector $x$ from the next vector $x'$,
if we know the integer vector $a$.
Let $\mathrm{pivot}_ℓ^{-1}$ denote this inverse function.
We have
\[
  \mathrm{pivot}_ℓ(x) = x' \iff \mathrm{pivot}_ℓ^{-1}(a, x') = x.
\]
This operation leads directly to a definition of MDCFs.

Although, we initially used integer vectors $a ∈ ℤ^d$ for the definition of the inverse operation,
we can also look at what happens when $a$ is not necessarily an integer vector.
For subsequent lemmas, we will need a definition of MDCFs which allows rational
or even real vectors as coefficients.

\begin{definition}
  Given a sequence of $d$-dimensional real vectors $\{r^{(n)}\}_{n ≥ 0}$ and a sequence of
  indices $\{ℓₙ\}_{n ≥ 1}$ between $1$ and $d$,
  the \emph{$d$-dimensional continued fraction} $[r^{(0)}; ℓ₁, r^{(1)}; …]$ is defined as
  \[
    [r^{(0)}; ℓ₁, r^{(1)}; …] = \lim_{n → ∞} [r^{(0)}; ℓ₁, r^{(1)}; …; ℓₙ, r^{(n)}],
  \]
  where the finite continued fractions $[r^{(0)}; ℓ₁, r^{(1)}; …; ℓₙ, r^{(n)}]$ are defined as
  \[
    [r^{(0)}] = r^{(0)},
    \qquad
    [r^{(0)}; ℓ₁, r^{(1)}; …; ℓₙ, r^{(n)}]
    = \mathrm{pivot}_{ℓ₁}^{-1}\big(r^{(0)}, [r^{(1)}; ℓ₂, r^{(2)}; …; ℓₙ, r^{(n)}]\big).
  \]
\end{definition}

For the representation to be correct, we require $r_{ℓₙ}^{(n)} ≠ 0$ for $n ≥ 1$.
This is similar to the continued fractions, where only the first value could be zero,
while all subsequent values had to be positive.
For the multi-dimensional counterpart we only require that the value chosen by
$ℓₙ$ is not zero.
The other values in the vector $r^{(n)}$ can assume any non-negative value.

The terminology from one-dimensional continued fractions naturally carry over to its
multi-dimensional counterpart.
Given an infinite MDCF representation~$[a₀; ℓ_1, a_1; …]$ of a vector $x ∈ ℝ^d$, we define the following:

\begin{itemize}
  \item The \emph{$k$-th convergent} of $x$ is the finite MDCF $[a₀; ℓ₁, a₁; …; ℓ_k, a_k]$.
  \item The \emph{$k$-th complete quotient} is the MDCF $[aₖ; a_{k+1}, …]$.
  \item The MDCF is \emph{eventually periodic} if there exists an index~$n₀ ≥ 0$
    and a period~$k ≥ 1$ such that $aₙ = a_{n+k}$ and $ℓₙ = ℓ_{n+k}$
    for every $n ≥ n₀$.
    The MDCF is \emph{purely periodic} if $n₀ = 0$.
\end{itemize}

For the analysis of MDCFs,
we begin with two lemmas, that are generalizations of
Lemma~\ref{lem:cf-nesting} and Lemma~\vref{lem:cf-wallis}.

\begin{lemma}
  \label{lem:mdcf-nesting}
  Let $x ∈ ℝ^d$, then
  \[
    [a₀; ℓ₁, a₁; …; ℓₙ, aₙ; ℓ, x]
    = [a₀; ℓ₁, a₁; \cdots; ℓₙ, \mathrm{pivot}_{ℓ}^{-1}(aₙ, x)]
  \]
\end{lemma}

\begin{proof}
  If $n = 0$, then by definition,
  \[
    [a₀; ℓ, x] = \mathrm{pivot}_{ℓ}^{-1}(a₀, [x]) = \mathrm{pivot}_{ℓ}^{-1}(a₀, x) = [\mathrm{pivot}_{ℓ}^{-1}(a₀, x)].
  \]
  Suppose the lemma holds for any $n ≥ 0$, then
  \begin{align*}
    [a₀; ℓ₁, a₁; …; ℓₙ, aₙ; ℓ, x]
    & = \mathrm{pivot}_{ℓ₁}^{-1}(a₀, [a₁; …; ℓₙ, aₙ; ℓ, x]) \\
    & = \mathrm{pivot}_{ℓ₁}^{-1}(a₀, [a₁; …; ℓₙ, \mathrm{pivot}_{ℓ}^{-1}(aₙ, x)] \\
    & = [a₀; ℓ₁, a₁; …; ℓₙ, \mathrm{pivot}_{ℓ}^{-1}(aₙ, x)]. \qedhere
  \end{align*}
\end{proof}

% TODO: Explain how to derive the sequences
The second lemma is more complicated.
In the lemma for continued fractions,
we defined the convergents of a continued fraction~$pₙ/qₙ$
using a fractional transformation based on the previous two
terms~$p_{n-1}/q_{n-1}$ and $p_{n-2}/q_{n-2}$.
For MDCFs, we can similarly derive a recursive formula to derive the values of
the convergent vector $(p₁/q, \dots, p_d/q)$ using the previous convergents.
Deriving the sequence is more involved than the one-dimensional case since
there is an additional index $ℓ$ at each step.

The sequence is essentially the generalized Euclidean algorithm, but in reverse.
Just like the MDCF is the application of the inverse pivot operation.
In the algorithm, we begin with a basis $B ∈ ℤ^{d×d}$ and a vector $c ∈ ℤ^d$.
First, we find a solution $x$ to $Bx = c$ and then we exchange $c$ with $B_ℓ$
and $B_ℓ$ with the remainder $B\{x\} = Bx - Ba_i$.
We continue this process until $x$ is entirely integral.
For the sequence, we reverse this process.
We begin with an already reduced basis $B'$ and a vector $c'$, which is an
integral combination of $B'$.
Then, we exchange $B'_{ℓ}$ with $c'$ and $c'$ with $B' a_i + B'_{ℓ}$.
We continue this process until $B x = c$.

In total, there are two types of sequences:
A sequence of $d+1$ vectors $P_0^{(n)}, P_1^{(n)}, …, P_d^{(n)}$ and a sequence
of $d+1$ scalars $Q_0^{(n)}, Q_1^{(n)}, …, Q_d^{(n)}$,
which represent the denominators of the convergents.
The sequences begin with
\begin{align*}
  P_0^{(-1)} & = 0, & P_i^{(-1)} & = e_i, \\
  Q_0^{(-1)} & = 1, & Q_i^{(-1)} & = 0
\end{align*}
and the sequences are updated according to the following recurrences:
\begin{align*}
  P_{ℓₙ}^{(n)} & = P_0^{(n-1)} + P_1^{(n-1)} a_1^{(n)} + ⋯ + P_d^{(n-1)} a_d^{(n)}, &
  P_i^{(n)} & = P_i^{(n-1)}, &
  P_0^{(n)} & = P_{ℓₙ}^{(n-1)} \\
  Q_{ℓₙ}^{(n)} & = Q_0^{(n-1)} + Q_1^{(n-1)} a_1^{(n)} + ⋯ + Q_d^{(n-1)} a_d^{(n)}, &
  Q_i^{(n)} & = Q_i^{(n-1)}, &
  Q_0^{(n)} & = Q_{ℓₙ}^{(n-1)},
\end{align*}
where $i ≠ ℓ_n$.
What this sequence effectively does is reconstructing the lattice from an
initial solution vector $x ∈ ℝ^d$ and its coefficient vectors $a_n$.

Modelling this back to the Euclidean algorithm,
the vectors $P_1^{(n)}, …, P_d^{(n)}$ represent the basis and $P_0^{(n)}$ is
the additional vector, which is reduced during a run of the algorithm.
This matches the behavior of the algorithm in reverse.
We begin with $P_0^{(n)}$ since in the Euclidean algorithm we would end with this vector.
Then, the formula calculates a new vector using an integral combination of the
old vectors and store it in $P_{ℓ_n}^{(n)}$.
This corresponds directly to the modulo and exchange operation of the original
Euclidean algorithm.
As the vectors $P_0^{(n)}, P_1^{(n)}, …, P_d^{(n)}$ would grow infinitely, the
scalars $Q_0^{(n)}, Q_1^{(n)}, …, Q_d^{(n)}$ scale the vectors back down such
that they converge to a limit as $n$ increases.

% TODO: Ensure that this is correct for the first index!
\begin{lemma}
  \label{lem:mdcf-wallis}
  Let $x ∈ ℝ^d$, then
  \[
    [a₀; ℓ₁, a₁; …; ℓ_{n-1}, a_{n-1}; ℓ, x]
    = \frac{P_0^{(n-1)} + P_1^{(n-1)} x_1 + ⋯ + P_d^{(n-1)} x_d}{Q_0^{(n-1)} + Q_1^{(n-1)} x_1 + ⋯ + Q_d^{(n-1)} x_d}.
  \]
\end{lemma}

\begin{proof}
  If $n = 0$, then
  \[
    [x] = x = \frac{I_d x + 0}{0^T x + 1}.
  \]
  Suppose the lemma holds for $n ≥ 0$.
  Then,
  \begin{align*}
    y & = [a₀; ℓ₁, a₁; …; ℓ_{n-1}, a_{n-1}; ℓ, x]                              \\
      & = [a₀; ℓ₁, a₁; …; ℓ_{n-1}, a_{n-1} + \mathrm{pivot}_ℓ(x)]              \\
      & = \frac{P^{(n-1)} (a + \mathrm{pivot}_ℓ(x)) + p^{(n-1)}}{Q^{(n-1)} (aₙ + \mathrm{pivot}_ℓ(x)) + q^{(n-1)}} \\
      & = \frac{x_ℓ}{x_ℓ} · \frac{P^{(n-1)} (a + \mathrm{pivot}_ℓ(x)) + p^{(n-1)}}{Q^{(n-1)} (aₙ + \mathrm{pivot}_ℓ(x)) + q^{(n-1)}}.
  \end{align*}
  The numerator has the following form:
  \begin{align*}
    & x_ℓ \left( P_0^{(n-2)} + P_ℓ^{(n-2)} \left( a_ℓ + \frac{1}{x_ℓ} \right) + \sum_{i ≠ ℓ} P_i^{(n-2)} \left(a_i + \frac{x_i}{x_ℓ} \right) \right) \\
    & = P_0^{(n-2)} x_ℓ + P_ℓ^{(n-2)} a_ℓ x_ℓ + P_ℓ^{(n-2)} + \sum_{i ≠ ℓ} P_i^{(n-2)} a_i x_ℓ + P_i^{(n-2)} x_i \\
    & = \underbrace{\left( P_0^{(n-2)} + P_ℓ^{(n-2)} a_ℓ + \sum_{i ≠ ℓ} P_i^{(n-2)} a_i \right)}_{P_ℓ^{(n-1)} a} x_ℓ
      + \underbrace{P_ℓ^{(n-2)}}_{P_0^{(n-1)}} x_ℓ
      + \sum_{i ≠ ℓ} \underbrace{P_i^{(n-1)}}_{P_i^{(n-1)}} x_i \\
    & = P_0^{(n-1)} + P_1^{(n-1)} a_1 + ⋯ + P_d^{(n-1)} a_d.
  \end{align*}
  Analogously, the denominator has the following form:
  \begin{align*}
    & x_ℓ \left( Q_0^{(n-2)} + Q_ℓ^{(n-2)} \left( a_ℓ + \frac{1}{x_ℓ} \right) + \sum_{i ≠ ℓ} Q_i^{(n-2)} \left(a_i + \frac{x_i}{x_ℓ} \right) \right) \\
    & = Q_0^{(n-2)} x_ℓ + Q_ℓ^{(n-2)} a_ℓ x_ℓ + Q_ℓ^{(n-2)} + \sum_{i ≠ ℓ} Q_i^{(n-2)} a_i x_ℓ + Q_i^{(n-2)} x_i \\
    & = \underbrace{\left( Q_0^{(n-2)} + Q_ℓ^{(n-2)} a_ℓ + \sum_{i ≠ ℓ} Q_i^{(n-2)} a_i \right)}_{Q_ℓ^{(n-1)} a} x_ℓ
      + \underbrace{Q_ℓ^{(n-2)}}_{Q_0^{(n-1)}} x_ℓ
      + \sum_{i ≠ ℓ} \underbrace{Q_i^{(n-1)}}_{Q_i^{(n-1)}} x_i \\
    & = Q_0^{(n-1)} + Q_1^{(n-1)} a_1 + ⋯ + Q_d^{(n-1)} a_d.
  \end{align*}
  Hence,
  \[
    [a₀; ℓ₁, a₁; …; ℓ_{n-1}, a_{n-1}; ℓ, x]
    = \frac{P_0^{(n-1)} + P_1^{(n-1)} x_1 + ⋯ + P_d^{(n-1)} x_d}{Q_0^{(n-1)} + Q_1^{(n-1)} x_1 + ⋯ + Q_d^{(n-1)} x_d}.
    \qedhere
  \]
\end{proof}

% ==============================================================================
\section{Convergence of Multi-Dimensional Continued Fractions}
% ==============================================================================

% TODO: The indices here are wrong...
In the following, we assume that we have an MDCF $[a^{(0)}; ℓ_1, a^{(1)}; ℓ_2, a^{(2)}; …]$
of an irrational vector $x = (x₁, …, x_d)$.
The convergent of this MDCF is denoted by
\[
  r^{(n)}
  = (r_1^{(n)}, …, r_d^{(n)})
  = \left( \frac{P_{1ℓ_n}^{(n)}}{Q_{ℓ_n}^{(n)}}, \dots, \frac{P_{dℓ_n}^{(n)}}{Q_{ℓ_n}^{(n)}} \right).
\]
The aim of this section is to show that the convergents actually live up to
their name and converge towards the vector $x$.
The proof is based on Perron's proof for the convergence of his
algorithm \cite{Perron07}.
For the proof we need two more requirements:
The first comes from Perron's original proof.
In his version, he required that there exists some constant $c > 0$ such that
\[
  0 < \frac{1}{a_{ℓ_{n-1}}^{(n)}} ≤ c \text{ and } 0 ≤ \frac{a_i^{(n)}}{a_{ℓ_{n-1}}^{(n)}} ≤ c \text{ for every } i ≠ ℓₙ.
\]
In addition to his requirement, we also need that the sequence $ℓ_n$
contains every index infinitely often.
This is already given in the Jacobi-Perron algorithm,
but not in the generalized Euclidean algorithm.
So we need this additional requirement for the convergence.

The vector $r^{(n)}$ is only one possible convergent,
the other vectors $P_i^{(n)}/Q_i^{(n)}$ with $i ≠ ℓ_n$ are also possible convergent vectors.
The first step for the proof is to show that each convergent $r^{(n)}$ is a
linear combination of the previous vectors $P_i^{(n-1)}/Q_i^{(n-1)}$.

\begin{lemma}
  \label{lem:conv-conv}
  There exists $λ₀, λ₁, …, λ_d ∈ [0, 1]$ with $λ₀ + λ₁ + ⋯ + λ_d = 1$ such that
  \[
    r_i^{(n)} = λ₀ \frac{P_{i0}^{(n-1)}}{Q_0^{(n-1)}} + λ₁ \frac{P_{id}^{(n-1)}}{Q_1^{(n-1)}} + ⋯ + λ_d \frac{P_{id}^{(n-1)}}{Q_1^{(n-1)}}.
  \]
\end{lemma}

\begin{proof}
  We choose $n ≥ 0$ such that every index in $\{1, …, d\}$ has occurred at least once.
  By Lemma~\ref{lem:mdcf-wallis}, we can calculate $P_{iℓ_n}^{(n)}$ using the previous values as follows:
  \begin{align*}
    P_{iℓ_n}^{(n)} = \sum_{k = 1}^d P_{ik}^{(n-1)} a_k^{(n)} + P_{i0}^{(n-1)} \\
  \end{align*}
  Dividing by $Q_{ℓ_n}^{(n)}$ gives us
  \begin{align*}
    \frac{P_{iℓ}^{(n)}}{Q_{ℓ}^{(n)}} = \sum_{k = 1}^d \frac{P_{ik}^{(n-1)}}{Q_ℓ^{(n)}} a_k^{(n)} + \frac{P_{i0}^{(n-1)}}{Q_ℓ^{(n)}}.
  \end{align*}
  Using $λ_k = a_k^{(n)} \frac{Q_k^{(n-1)}}{Q_ℓ^{(n)}}$ for $1 ≤ k ≤ d$ and $λ₀ = \frac{Q_0^{(n-1)}}{Q_ℓ^{(n)}}$,
  we can reformulate the convergent as the linear combination for this lemma.
  Because of the requirement that every index has occurred at least once,
  the denominators $Q_k^{(n-1)}$ of the linear combination cannot be zero.
  Therefore, this is a valid representation of the convergent $r_i^{(n)}$.
  For the coefficients $λ_i$, we require $λ₀ + λ₁ + ⋯ + λ_d = 1$ and $0 ≤ λᵢ ≤ 1$ for every index $i$.
  The first property follows from the definition of $Q_ℓ^{(n)}$:
  \[
    Q_ℓ^{(n)} = Q_0^{(n-1)} + Q_1^{(n-1)} a_1^{(n)} + ⋯ + Q_d^{(n-1)} a_d^{(n)}
  \]
  which is equivalent to
  \[
    1 = \frac{Q_0^{(n-1)}}{Q_ℓ^{(n)}} + \frac{Q_1^{(n-1)}}{Q_ℓ^{(n)}} a_1^{(n)} + ⋯ + \frac{Q_d^{(n-1)}}{Q_ℓ^{(n)}} a_d^{(n)} = λ₀ + λ₁ + ⋯ + λ_d.
  \]
  The second property follows from the fact that $a^{(n)}$ is non-negative.
  Therefore, $Q_ℓ^{(n)}$ is always less than or equal to $a_i^{(n)}
  Q_i^{(n-1)}$ for every index $i ≤ d$ and $λ_i$ is always bounded
  between $0$ and $1$.
\end{proof}

To prove the convergence, we will apply the squeeze theorem, again.
We use the two sequences $s^{(n)} = (s_1^{(n)}, …, s_d^{(n)})$ and $t^{(n)} = (t_1^{(n)}, …, t_d^{(n)})$ with
\[
  s_i^{(n)} = \min\left\{\frac{P_{i0}^{(n)}}{Q_0^{(n)}}, \frac{P_{i1}^{(n)}}{Q_1^{(n)}}, …, \frac{P_{id}^{(n)}}{Q_d^{(n)}}\right\}
\]
and
\[
  t_i^{(n)} = \max\left\{\frac{P_{i0}^{(n)}}{Q_0^{(n)}}, \frac{P_{i1}^{(n)}}{Q_1^{(n)}}, …, \frac{P_{id}^{(n)}}{Q_d^{(n)}}\right\}.
\]
The convergent $r^{(n)}$ clearly lies between the sequences $s^{(n)}$ and $t^{(n)}$.
So if $s^{(n)}$ and $t^{(n)}$ converge to the same limit, then $r^{(n)}$ also
converges to the same limit.
For the convergence of $s^{(n)}$ and $t^{(n)}$, we show that they are monotone
and bounded.
By the monotone-convergence theorem, they must converge.

\begin{lemma}
  The sequences $s_i^{(n)}$ and $t_i^{(n)}$ are monotone and bounded.
\end{lemma}

\begin{proof}
  We can bound the new convergent $r^{(n)}$ from below using $s^{(n-1)}$.
  For the previous iteration, we already have $P_{ij}^{(n-1)}/Q_j^{(n-1)} ≥ s_i^{(n-1)}$
  for every $0 ≤ j ≤ d$ by construction of the sequence~$s_i^{(n-1)}$.
  Using Lemma~\ref{lem:conv-conv},
  \begin{align*}
    \frac{P_{iℓ}^{(n)}}{Q_{ℓ}^{(n)}}
    & = λ₀ \frac{P_{i0}^{(n-1)}}{Q_0^{(n-1)}} + λ₁ \frac{P_{i1}^{(n-1)}}{Q_1^{(n-1)}} + ⋯ + λ_d \frac{P_{id}^{(n-1)}}{Q_d^{(n-1)}} \\
    & ≥ λ₀ s_i^{(n-1)} + λ₁ s_i^{(n-1)} + ⋯ + λ_d s_i^{(n-1)} = s_i^{(n-1)}.
  \end{align*}
  The bound for the other convergents $r_j^{(n)}$ with $j ≠ ℓ$ is even easier since
  \[
    \frac{P_{ij}^{(n)}}{Q_j^{(n)}} = \frac{P_{ij}^{(n-1)}}{Q_j^{(n-1)}} ≥ s_i^{(n-1)}.
  \]
  Using the same argument, we can bound each convergent from above by $t_i^{(n-1)}$.
  Therefore, in each iteration the convergent $r_i^{(n)}$ is
  bounded between the previous minimum~$s_i^{(n-1)}$ and the previous maximum~$t_i^{(n-1)}$,
  or equivalently $s_i^{(n-1)} ≤ s_i^{(n)} ≤ r_i^{(n)} ≤ t_i^{(n)} ≤ t_i^{(n-1)}$.
\end{proof}

The previous lemma only shows that the sequences are converging,
not that they are converging to the same limit.
To show the second part, we first need this crucial lemma
about the coefficients $λ₀, λ₁, …, λ_d$ in the linear combination from Lemma~\ref{lem:conv-conv}.
This is also the crucial property, that Perron requires for the convergence of his algorithm.

\begin{lemma}
  \label{lem:lambda-pos}
  There exists one index $k$ with $λ_k > 0$.
\end{lemma}

\begin{proof}
  We choose $k = ℓ_{n-1}$.
  Using the initial requirements for the coefficients $a_i^{(n)}$ and the fact
  that $Q_k^{(n-1)} ≥ Q_i^{(n-1)} = Q_i^{(n-2)}$ for every $i ≠ k$,
  it follows that
  \[
    \frac{λ_i}{λ_k} = \frac{a_i^{(n)}}{a_k^{(n)}} · \frac{Q_i^{(n-1)}}{Q_k^{(n-1)}} ≤ c · 1,
  \]
  or equivalently $λᵢ ≤ c λₖ$. From this, it follows that
  \[
    1 = λ₀ + λ₁ + ⋯ + λ_d ≤ λ_k (1 + dc).
  \]
  Therefore, $λₖ ≥ 1/(1 + dc) > 0$.
\end{proof}

\begin{lemma}
  \label{lem:min-max-conv}
  The minimum $s_i^{(n)}$ and maximum $t_i^{(n)}$ are converging to the same
  limit, if every index occurs infinitely often in $ℓ_n$.
\end{lemma}

\begin{proof}
  Let $s_i = \lim_{n → ∞} s_i^{(n)}$
  and $t_i = \lim_{n → ∞} t_i^{(n)}$.
  Suppose $s_i < t_i$.
  For the minimum~$s_i$, there must exist an $ε > 0$ and an index $N ≥ 0$ such
  that for every $n ≥ N$,
  \[
    \frac{P_{ij}^{(n)}}{Q_j^{(n)}} ≥ s_i^{(n)} > s_i - ε.
  \]
  Let $k = ℓ_{n-1}$.
  Then, $Q_i^{(n-1)} = Q_i^{(n-2)}$ and $Q_k^{(n-1)} = Q_0^{(n-2)} + Q_1^{(n-2)} a_1^{(n-1)} + ⋯ + Q_d^{(n-2)} a_d^{(n-1)}$.
  \begin{align*}
    r_i^{(n)}
    & = λ₀ \frac{P_{i0}^{(n-1)}}{Q_0^{(n-1)}} + λ₁ \frac{P_{i1}^{(n-1)}}{Q_1^{(n-1)}} + ⋯ + λ_d \frac{P_{id}^{(n-1)}}{Q_d^{(n-1)}} \\
    & > \lambda_k \frac{P_{ik}^{(n-1)}}{Q_k^{(n-1)}} + \sum_{i ≠ k} λ_i (s_i - ε) \\
    & = \lambda_k \frac{P_{ik}^{(n-1)}}{Q_k^{(n-1)}} + (1 - λ_k) (s_i - ε),
  \end{align*}
  or equivalently
  \begin{align*}
    r_i^{(n)} - s_i > λ_k \left( \frac{P_{ik}^{(n-1)}}{Q_k^{(n-1)}} - s_i \right) - ε.
  \end{align*}
  Lemma~\ref{lem:lambda-pos} tells us that there is a positive constant $M$
  that bounds $λ_k$ from below.
  It follows that
  \begin{align*}
    r_i^{(n)} - s_i > \frac{1}{M} \left( \frac{P_{ik}^{(n-1)}}{Q_k^{(n-1)}} - s_i \right) - ε
  \end{align*}
  for some constant $M > 0$.
  Advancing this inequality from $n$ to $n+1, n+2, …, n+k$ results in the new
  inequality
  \begin{align*}
    r_i^{(n+k)} - s_i > \frac{1}{M^k} \left( \frac{P_{ik}^{(n-1)}}{Q_k^{(n-1)}} - s_i \right) - ε\left( \frac{1}{M} + \frac{1}{M^2} + ⋯ + \frac{1}{M^{k-1}} \right).
  \end{align*}
  Since every index occurs infinitely often,
  there must be one index $n$ where $r_i^{(n)} ≤ t_i$
  and another index $n+k$ where $r_i^{(n)} ≥ s_i$.
  \begin{align*}
    0 > \frac{1}{M^k} \left( t_i - s_i \right) - ε\left( \frac{1}{M} + \frac{1}{M^2} + ⋯ + \frac{1}{M^{k-1}} \right).
  \end{align*}
  But $ε$ can be chosen arbitrarily small,
  which is a contradiction to the fact that $t_i - s_i > 0$.
  Therefore, $t_i$ and $s_i$ converge to the same limit.
\end{proof}

\begin{theorem}
  \label{thm:mdcf-conv}
  If for every $n ≥ 0$ there exists a constant $c ≥ 0$ such that
  \[
    0 < \frac{1}{a^{(n)}_{ℓₙ}} ≤ c, \qquad 0 ≤ \frac{a^{(n)}_i}{a^{(n)}_{ℓₙ}} ≤ c.
  \]
  for every $i ≠ ℓₙ$ and every index occurs infinitely often in the sequence $ℓₙ$,
  then the sequence $r^{(n)}$ converges to $x$.
\end{theorem}

\begin{proof}
  Let $x^{(n)}$ denote the $n$-th complete quotient of $x$.
  By Lemma~\ref{lem:mdcf-wallis}, we can represent each element in $x$ as
  \[
    x_i = \frac{P_{i0}^{(n-1)} + P_{i1}^{(n-1)} x_1^{(n)} + ⋯ + P_{id}^{(n-1)} x_d^{(n)}}{Q_{i0}^{(n-1)} + Q_{i1}^{(n-1)} x_1^{(n)} + ⋯ + Q_{id}^{(n-1)} x_d^{(n)}}.
  \]
  Using a similar argument as Lemma~\ref{lem:conv-conv}, we can represent this
  as a linear combination
  \[
    x_i = λ₀ \frac{P_{i0}^{(n-1)}}{Q_0^{(n-1)}}  + λ₁ \frac{P_{i1}^{(n-1)}}{Q_1^{(n-1)}} + λ_d \frac{P_{id}^{(n-1)}}{Q_d^{(n-1)}}.
  \]
  with $λ_i = x_i^{(n)} \frac{Q_i^{(n-1)}}{Q_ℓ^{(n)}}$.
  From the previous lemma, we know that $r^{(n)}$ converges to some limit $x'$.
  Since every index occurs infinitely often,
  we can find an index $m ≤ n - 1$ such that $ℓ_m = j$ and
  \[
    \frac{P_{ij}^{(n-1)}}{Q_j^{(n-1)}} = \frac{P_{ij}^{(m)}}{Q_j^{(m)}} = r_i^{(m)},
  \]
  or equivalently we can find an index where the term is a convergent.
  As $n$ increases there are infinitely many such convergents,
  so each term converges to $x_i'$.
  Thus,
  \[
    x_i
    = \lim_{n → ∞} λ₀ \frac{P_{i0}^{(n-1)}}{Q_0^{(n-1)}}  + λ₁ \frac{P_{i1}^{(n-1)}}{Q_1^{(n-1)}} + λ_d \frac{P_{id}^{(n-1)}}{Q_d^{(n-1)}}
    = λ₀ x_i' + λ₁ x_i' + ⋯ + λ_d x_i'
    = x_i'
  \]
  for every $i ≤ d$.
\end{proof}

One application of this theorem are periodic MDCFs.
Since they only have a finite set of a coefficients they easily satisfy the
first requirement of the theorem.

\begin{corollary}
  Periodic MDCFs converge if the period contains every index.
\end{corollary}

% ==============================================================================
\section{The Geometry behind Multi-Dimensional Continued Fractions}
% ==============================================================================

For the geometrical interpretation of continued fractions,
we represented each convergent $pₙ/qₙ$ as a 2D vector $(pₙ, qₙ)$.
This vector approaches an irrational line spanned by the vector $(1, α)$ where
$α$ is some irrational number.
For the generalization of this interpretation to multi-dimensional continued
fractions, we represent each convergent, which is a $d$-dimensional rational vector,
as a $(d+1)$-dimensional integer vector.
Specifically, given a vector $(p₁/q₁, …, p_d/q_d) ∈ ℚ^d$, first we find a
common denominator $(p₁'/q, …, p_d'/q_d)$ and then we map it to the vector
$(q, p₁, …, p_d) ∈ ℤ^{d+1}$.
We can go back from the integer vector to a rational vector by dividing each
coordinate with the first and removing the first coordinate from the integer
vector:
Given $(x₀, x₁, …, x_d) ∈ ℤ^{d+1}$, we map it back to $(x₁/x₀, …, x_d/x₀) ∈ ℚ^d$.

In this space, the representation for a particular convergent is not unique,
there can be multiple integer vectors representing the same convergent.
For example, if we have a vector $c ∈ ℤ^d$ for a convergent,
then we can multiply with some scalar $λ ∈ ℤ$ and get a new vector $c' = λ c$
which represents the same convergent.
This is because the scalar is eliminated when mapping back to the rational vector:
\[
  λ (c₀, c₁, …, c_d) ↦  \left(\frac{λ c₁}{λ c₀}, …, \frac{λ c_d}{λ c_0} \right).
\]
Therefore, two vectors $a, b ∈ ℤ^{d+1}$ are equivalent, denoted as $a ≡ b$,
if $a = λ b$ or $b = λ a$ for some $λ ∈ ℤ$.

% TODO: Are the convergents calculated by the recurrence relation in this
% section always minimal/irreducible?



% TODO: Explain mapping from and back to the original vector space.
\begin{center}
  \begin{tikzpicture}
    \matrix[
      column sep=2cm,
      nodes={text width=3cm, align=center},
    ] {
      \node (L0) {$\mathbb{R}^d$}; &
      \node (R0) {$\mathbb{RP}^{d+1}$}; \\
      \node (L1) {$(x₁, …, x_d)$}; &
      \node (R1) {$[1, x₁, …, x_d]$}; \\
      \node (L2) {$(x₁/x₀, …, x_d/x₀)$}; &
      \node (R2) {$[x₀, x₁, …, x_d]$}; \\
    };

    \draw[->] (L1) -- node[above] {} (R1);
    \draw[<-] (L2) -- node[above] {} (R2);
  \end{tikzpicture}
\end{center}

For example, consider the vector $[1, x₁, x₂]$ and suppose that $0 ≤ x₁, x₂ < 1$.
A pivot operation with $ℓ = 1$ would result in the vector $[1, 1/x₁, x₂/x₁]$.
This vector is equivalent to $[x₁, 1, x₂]$.
Therefore, we can reformulate this operation as a coordinate swap of $x_ℓ$ with
the new coordinate $x₀$:
\[
  \begin{bmatrix}
    0 & 1 & 0 \\
    1 & 0 & 0 \\
    0 & 0 & 1 \\
  \end{bmatrix}
  ·
  \begin{bmatrix} 1 \\ x₁ \\ x₂ \\ \end{bmatrix}
  =
  \begin{bmatrix} x₁ \\ 1 \\ x₂ \\ \end{bmatrix}
  =
  \begin{bmatrix} 1 \\ 1/x₁ \\ x₂/x₂ \\ \end{bmatrix}.
\]
Similarly, subtracting the integer part of each value in $[1, x₁, x₂]$ is
equivalent to a series of skew operations:
\[
  \begin{bmatrix}
    1 & 0 & 0 \\
    -\floor{x₁} & 1 & 0 \\
    0 & 0 & 1 \\
  \end{bmatrix}
  ·
  \begin{bmatrix}
    1 & 0 & 0 \\
    0 & 1 & 0 \\
    -\floor{x₂} & 0 & 1 \\
  \end{bmatrix}
  ·
  \begin{bmatrix} 1 \\ x₁ \\ x₂ \\ \end{bmatrix}
  =
  \begin{bmatrix} 1 \\ x₁ - \floor{x₁} \\ x₂ - \floor{x₂} \\ \end{bmatrix}.
\]
Importantly, each of those matrices has
determinant $±1$.
Therefore, the whole operation can be easily reversed by inverting the matrix.
This is the equivalent of the inverse pivot operation in homogeneous
coordinates.

We can also reformulate the MDCF as a series of matrix multiplications.
In the following $T(a)$ denotes the translation by a vector $a ∈ ℝ^d$
and the matrix $S(ℓ)$ denotes the swap of $x_\ell$ with $x_0$.
The MDCF can then be written as
\[
  [a₀] = \hat a₀, \qquad
  [a₀; ℓ₁, a₁; …; ℓₙ, aₙ] = T(a₀) · S(ℓ_1) · [a₁; ℓ_2, a_2; …; ℓ_n, a_n].
\]

This allows us to dramatically simplify Lemma~\ref{lem:mdcf-wallis}.
Instead of several matrix and vector sequences, we simplify it to a single matrix sequence $\{B^{(n)}\}_{n ≥ 0}$,
where one matrix $B^{(n)}$ consists of the column vectors $B₀^{(n)}, B₁^{(n)}, …, B_d^{(n)}$.
We begin with $B^{(0)} = I_d$ and update the matrix according to
\begin{align*}
  B_{ℓₙ}^{(n)} = B^{(n-1)} \hat a^{(n)},
  \qquad B_i^{(n)} = B_i^{(n-1)},
  \qquad B_0^{(n)} = B_{ℓₙ}^{(n-1)}.
\end{align*}
By construction, $B^{(n)}$ is the combined matrix of the sequences $P^{(n)}, Q^{(n)}, p^{(n)}$ and $q^{(n)}$:
\[
  B^{(n)} = \begin{pmatrix}
    p^{(n)} & P^{(n)} \\
    q^{(n)} & Q^{(n)} \\
  \end{pmatrix}.
\]

\begin{lemma}
  \label{lem:mdcf-wallis'}
  Let $x ∈ ℝ^d$, then
  \[
    [a^{(0)}; ℓ₁, a^{(1)}; …; ℓ_{n-1}, a^{(n-1)}; ℓ, x] ≡ B^{(n-1)} \hat x
  \]
\end{lemma}

% TODO
\begin{proof}
  For $n = 0$, this is clearly true.
  Suppose the lemma is true for $n ≥ 0$, then
  \begin{align*}
    [a^{(0)}; ℓ₁, a₁; …; ℓ_n, a_n; ℓ, x]
    & ≡ [a^{(0)}; ℓ₁, a₁; …; ℓ_n, a_n + \mathrm{pivot}_ℓ(x)] \\
    & ≡ B^{(n-1)}
    \begin{bmatrix}
      a^{(n)} + \mathrm{pivot}_ℓ(x) \\ 1
    \end{bmatrix} \\
    & ≡ B^{(n-1)} (\hat a^{(n)} + \hat{x}) \\
    & ≡ B^{(n)} \hat{x} \qedhere
  \end{align*}
\end{proof}

% ==============================================================================
\section{Periodic Multi-Dimensional Continued Fractions}
% ==============================================================================

The proof for periodic MDCFs is based on the same theorem for the Jacobi-Perron
algorithm, originally proven by Perron \cite{Perron07}.
We begin with the purely periodic case.
The idea behind this proof is that in a purely periodic MDCF of a vector $x ∈ ℝ^d$,
the vector itself is an eigenvector for one of the matrices $B^{(k)}$.
Furthermore, the elements of this eigenvector can only be algebraic numbers with degree $≤ d+1$.

\begin{lemma}
  \label{lem:mdcf-purely-periodic}
  If there exists a purely periodic MDCF for $x ∈ ℝ^d$,
  then $xᵢ$ is an algebraic number of degree $≤ d+1$ for every $i ≤ d$.
\end{lemma}

% TODO: Should we use x ≡ y or [x] = [y]?
\begin{proof}
  If the MDCF is purely periodic, then there is some $n ≥ 1$ such that $x = x^{(n)}$.
  By Lemma~\ref{lem:mdcf-wallis'},
  \[
    \hat x ≡ B^{(n)} \hat x^{(n)} ≡ B^{(n)} \hat x \iff λ₁ \hat x = λ₂ B^{(n)} \hat x \iff λ \hat x = B^{(n)} \hat x,
  \]
  for some $λ₁, λ₂ ∈ ℝ \setminus \{0\}$ with $λ = λ₁/λ₂$.
  Therefore, we are looking for an eigenvector $\hat x$ and an eigenvalue $λ$ of $B^{(n)}$.
  The characteristic polynomial $\det(B^{(n)} - λ I)$ can have a degree of at most $d+1$,
  therefore the eigenvalue $λ$ is an algebraic number of degree $d+1$.
  For the eigenvector $\hat x$, we have to find a nontrivial solution to the
  homogeneous linear system
  \[
    (B^{(n)} - λ I) \hat x = 0.
  \]
  Each coefficient in this linear system is either an integer or $λ$ and is
  therefore contained in the field $ℚ(λ)$.
  Hence, the elements of $\hat x$ must come from the same
  field $ℚ(λ)$ and they must be algebraic numbers with the same degree.

  Finally, the eigenvector $\hat x$ has to be projected back from homogeneous coordinates $x$.
  Since $xᵢ = \hat xᵢ / \hat x₀$ and the values $\hat xᵢ$ and $\hat x₀$ are members of the field $ℚ(λ)$,
  the projected value $xᵢ$ must also be contained in the same field.
  Therefore, each element in $x$ is an algebraic number of degree $≤ d+1$.
\end{proof}

\begin{theorem}
  \label{thm:mdcf-periodic}
  If there exists a periodic MDCF for a given vector $x ∈ ℝ^d$,
  then each $x_i$ is a root of some polynomial with degree $≤ d+1$ for every $i ≤ d$.
\end{theorem}

\begin{proof}
  Given such a MDCF for $x$, let $x^{(k)}$ denote the $k$-th complete quotient
  of this fraction.
  Suppose that the MDCF is periodic after $K ≥ 0$ with period $ℓ ≥ 0$, i.e.
  $x^{(k)} = x^{(k+ℓ)}$ for every $k ≥ K$.
  By Lemma~\ref{lem:mdcf-wallis'},
  \[
    \hat x = B^{(k)} \hat x^{(k)} = B^{(k+ℓ)} \hat x^{(k+ℓ)} = B^{(k+ℓ)} \hat x^{(k)}.
  \]
  Every element in $x$ can be represented as a linear rational expression of $x^{(k)}$:
  \[
    x_i = \frac{∑_{j=1}^d B_{ij}^{(k)} x_j^{(k)} + B_{i0}^{(k)}}{\sum_{j=1}^d B_{0j}^{(k)} x_j^{(k)} + B_{00}^{(k)}},
  \]
  % TODO: I think we should be more precise here since the elements could be
  % inside different fields. They are not, but this sentence does not indicate
  % that.
  From the previous lemma, it follows that the elements of $\hat x^{(k)}$
  are contained in some field $ℚ(λ)$, where $λ$ is an algebraic number of degree $≤ d+1$.
  Furthermore, $B^{(k)}$ consists solely of integers.
  Therefore, every element in $x$ is contained in the same field $ℚ(λ)$
  and they must also be an algebraic numbers of degree $≤ d+1$.
\end{proof}

% TODO: There should be a discussion about the potential for answering
% Hermite's question here...
