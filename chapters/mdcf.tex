\chapter{Periodic Representation of Higher Degree Irrationals}
\label{ch:mdcf}

For quadratic irrationals, we used continued fractions to represent them as a
periodic sequence of integers.
The continued fractions were constructed using the Euclidean algorithm.
Naturally, we can generalize continued fractions to higher dimensions using the
generalized Euclidean algorithm.
This leads to a concept of multi-dimensional continued fractions, which could
potentially lead to an answer of Hermite's question.
Since it is based on the generalized Euclidean algorithm,
it extends previous generalizations of continued fractions based on the
Jacobi-Perron algorithm \cite{Gupta00}.

This chapter introduces the concept of MDCFs and discusses their potential as
an answer to Hermite's question.
We begin by defining what they are and deriving many properties similar to
continued fractions.
Subsequently, we look at numbers represented by periodic MDCFs and whether they
must be algebraic numbers of degree $d+1$.
Finally, we discuss the geometry behind the multi-dimensional continued fractions.
The other direction -- Whether algebraic numbers of degree $d+1$ admit periodic
MDCFs -- will be discussed in the next chapter.

% ==============================================================================
\section{Multi-Dimensional Continued Fractions}
% ==============================================================================

To derive the continued fraction for a real number, we have used the Euclidean algorithm.
In particular, we looked at the ratio $a/b$ between the two inputs $a$ and $b$,
and we have used the integer part of that ratio as a coefficient in the
continued fraction.
Iterating the Euclidean algorithm then gave us a unique representation for
every real number.
The Euclidean algorithm can also be viewed as the generalized algorithm when $d = 1$.
In this case, the ratio $a/b$ represents the solution $x$ of the linear system $bx = a$.
So the integer part of the solution $x$ also represents a coefficient in the
continued fraction.
We can then use the pivot operation to iterate over this solution and gain the
rest of the coefficients.

The multi-dimensional continued fraction can now be easily derived from the
solution vector $x$ in higher dimensions.
Specifically, given a vector $x ∈ ℝ^d$, we take the integer part of each
element $\floor{x}$ as a coefficient of the multi-dimensional continued
fraction.
Then, we get the subsequent coefficients by iterating over this vector using
the pivot operation.
This gives us a top-down construction of MDCFs, but typically continued
fractions are defined in a bottom-up fashion.
For a bottom-up definition of the MDCFs, we simply reverse the pivot operation.
The inverse operation can be derived as follows:
Let $x ∈ ℝ^d$ and $a = \floor{x}$.
If $x' = \mathrm{pivot}_ℓ(x)$ for a given index $ℓ$, then
\[
  \begin{array}{lcrlcr}
    \displaystyle x_i' & = & \displaystyle \frac{x_i - a_i}{x_ℓ - a_ℓ}, &
    \displaystyle x_ℓ' & = & \displaystyle \frac{x_i - a_i}{x_ℓ - a_ℓ} \\[1em]
  \end{array}
\]
for all $i ≠ ℓ$.
For the inverse operation, we have to derive every element of $x$ from $x'$,
which can be done by the following equations:
\[
  \begin{array}{lcrlcr}
    \displaystyle x_i & = & a_i + \displaystyle \frac{x_i'}{x_ℓ'}, &
    \displaystyle x_ℓ & = & a_ℓ + \displaystyle \frac{1}{x_ℓ'}
  \end{array}
\]
This allows us to calculate the previous vector $x$ from the next vector $x'$,
if we know the integer vector $a$.
Let $\mathrm{pivot}_ℓ^{-1}$ denote this inverse function.
We have
\[
  \mathrm{pivot}_ℓ(x) = x' \iff \mathrm{pivot}_ℓ^{-1}(a, x') = x.
\]
This operation leads directly to a definition of MDCFs.

Although, we initially used integer vectors $a ∈ ℤ^d$ for the definition of the inverse operation,
we can also look at what happens when $a$ is not necessarily an integer vector.
For subsequent lemmas, we will need a definition of MDCFs which allows rational
or even real vectors as coefficients.

\begin{definition}
  Given a sequence of $d$-dimensional real vectors $\{r^{(n)}\}_{n ≥ 0}$ and a sequence of
  indices $\{ℓₙ\}_{n ≥ 1}$ between $1$ and $d$,
  the \emph{$d$-dimensional continued fraction} $[r^{(0)}; ℓ₁, r^{(1)}; …]$ is defined as
  \[
    [r^{(0)}; ℓ₁, r^{(1)}; …] = \lim_{n → ∞} [r^{(0)}; ℓ₁, r^{(1)}; …; ℓₙ, r^{(n)}],
  \]
  where the finite continued fractions $[r^{(0)}; ℓ₁, r^{(1)}; …; ℓₙ, r^{(n)}]$ are defined as
  \[
    [r^{(0)}] = r^{(0)},
    \qquad
    [r^{(0)}; ℓ₁, r^{(1)}; …; ℓₙ, r^{(n)}]
    = \mathrm{pivot}_{ℓ₁}^{-1}\big(r^{(0)}, [r^{(1)}; ℓ₂, r^{(2)}; …; ℓₙ, r^{(n)}]\big).
  \]
\end{definition}

For the representation to be correct, we require $r_{ℓₙ}^{(n)} ≠ 0$ for $n ≥ 1$.
This is similar to the continued fractions, where only the first value could be zero,
while all subsequent values had to be positive.
For the multi-dimensional counterpart we only require that the value chosen by
$ℓₙ$ is not zero.
The other values in the vector $r^{(n)}$ can assume any non-negative value.

The terminology from one-dimensional continued fractions naturally carry over to its
multi-dimensional counterpart.
Given an infinite MDCF representation~$[a₀; ℓ_1, a_1; …]$ of a vector $x ∈ ℝ^d$, we define the following:

\begin{itemize}
  \item The \emph{$k$-th convergent} of $x$ is the finite MDCF $[a₀; ℓ₁, a₁; …; ℓ_k, a_k]$.
  \item The \emph{$k$-th complete quotient} is the MDCF $[aₖ; a_{k+1}, …]$.
  \item The MDCF is \emph{eventually periodic} if there exists an index~$n₀ ≥ 0$
    and a period~$k ≥ 1$ such that $aₙ = a_{n+k}$ and $ℓₙ = ℓ_{n+k}$
    for every $n ≥ n₀$.
    The MDCF is \emph{purely periodic} if $n₀ = 0$.
\end{itemize}

For the analysis of MDCFs,
we begin with two lemmas, which are generalization of
Lemma~\ref{lem:cf-nesting} and Lemma~\vref{lem:cf-wallis}.

\begin{lemma}
  \label{lem:mdcf-nesting}
  Let $x ∈ ℝ^d$, then
  \[
    [a₀; ℓ₁, a₁; …; ℓₙ, aₙ; ℓ, x]
    = [a₀; ℓ₁, a₁; \cdots; ℓₙ, \mathrm{pivot}_{ℓ}^{-1}(aₙ, x)]
  \]
\end{lemma}

\begin{proof}
  If $n = 0$, then by definition,
  \[
    [a₀; ℓ, x] = \mathrm{pivot}_{ℓ}^{-1}(a₀, [x]) = \mathrm{pivot}_{ℓ}^{-1}(a₀, x) = [\mathrm{pivot}_{ℓ}^{-1}(a₀, x)].
  \]
  Suppose the lemma holds for any $n ≥ 0$, then
  \begin{align*}
    [a₀; ℓ₁, a₁; …; ℓₙ, aₙ; ℓ, x]
    & = \mathrm{pivot}_{ℓ₁}^{-1}(a₀, [a₁; …; ℓₙ, aₙ; ℓ, x]) \\
    & = \mathrm{pivot}_{ℓ₁}^{-1}(a₀, [a₁; …; ℓₙ, \mathrm{pivot}_{ℓ}^{-1}(aₙ, x)] \\
    & = [a₀; ℓ₁, a₁; …; ℓₙ, \mathrm{pivot}_{ℓ}^{-1}(aₙ, x)]. \qedhere
  \end{align*}
\end{proof}

% TODO: Explain how to derive the sequences
The second lemma is more complicated.
In the lemma for continued fractions,
we defined the convergents of a continued fraction~$pₙ/qₙ$
using a fractional transformation based on the previous two
terms~$p_{n-1}/q_{n-1}$ and $p_{n-2}/q_{n-2}$.
For MDCFs, we can similarly derive a recursive formula to derive the values of
the convergent vector $(p₁/q, \dots, p_d/q)$ using the previous convergents.
Deriving the sequence is more involved than the one-dimensional case since
there is an additional index $ℓ$ at each step.

The sequence is essentially the generalized Euclidean algorithm, but in reverse.
Just like the MDCF is the application of the inverse pivot operation.
In the algorithm, we begin with a basis $B ∈ ℤ^{d×d}$ and a vector $c ∈ ℤ^d$.
First, we find a solution $x$ to $Bx = c$ and then we exchange $c$ with $B_ℓ$
and $B_ℓ$ with the remainder $B\{x\} = Bx - Ba_i$.
We continue this process until $x$ is entirely integral.
For the sequence, we reverse this process.
We begin with an already reduced basis $B'$ and a vector $c'$, which is an
integral combination of $B'$.
Then, we exchange $B'_{ℓ}$ with $c'$ and $c'$ with $B' a_i + B'_{ℓ}$.
We continue this process until $B x = c$.

In total, we define two sequences:
A matrix sequence $\{P^{(n)}\}_{n ≥ 0}$ and a vector sequences $\{Q^{(n)}\}_{n ≥ 0}$.
Their are defined as follows:
\begin{align*}
  P_{ℓₙ}^{(n)} & = P^{(n-1)} a_n + p^{(n-1)}, & P_i^{(n)} & = P_i^{(n)}, & P^{(-1)}   & = I_d, \\
  Q_{ℓₙ}^{(n)} & = Q^{(n-1)} a_n + q^{(n-1)}, & Q_i^{(n)} & = Q_i^{(n)}, & Q^{(-1)}_j & = 0,   \\
  p^{(n)}      & = P_{ℓₙ}^{(n-1)},            &           &              & p^{(-1)}_j & = 0,   \\
  q^{(n)}      & = Q_{ℓₙ}^{(n-1)},            &           &              & q^{(-1)}   & = 1.
\end{align*}
where $i ≠ ℓ_n$.
What this sequence effectively does is reconstructing the lattice from an
initial solution vector $x ∈ ℝ^d$ and its coefficient vectors $a_n$.

% TODO: Ensure that this is correct for the first index!
\begin{lemma}
  \label{lem:mdcf-wallis}
  Let $x ∈ ℝ^d$, then
  \[
    [a₀; ℓ₁, a₁; …; ℓ_{n-1}, a_{n-1}; ℓ, x]
    = \frac{P_ℓ^{(n)}}{Q_ℓ^{(n)}} = \frac{P^{(n-1)} x + p^{(n-1)}}{Q^{(n-1)} x + q^{(n-1)}}.
  \]
\end{lemma}

\begin{proof}
  If $n = 0$, then
  \[
    [x] = x = \frac{I_d x + 0}{0^T x + 1}.
  \]
  Suppose the lemma holds for $n ≥ 0$.
  Then,
  \begin{align*}
    y & = [a₀; ℓ₁, a₁; …; ℓ_{n-1}, a_{n-1}; ℓ, x]                              \\
      & = [a₀; ℓ₁, a₁; …; ℓ_{n-1}, a_{n-1} + \mathrm{pivot}_ℓ(x)]              \\
      & = \frac{P^{(n-1)} (a + \mathrm{pivot}_ℓ(x)) + p^{(n-1)}}{Q^{(n-1)} (aₙ + \mathrm{pivot}_ℓ(x)) + q^{(n-1)}} \\
      & = \frac{x_ℓ}{x_ℓ} · \frac{P^{(n-1)} (a + \mathrm{pivot}_ℓ(x)) + p^{(n-1)}}{Q^{(n-1)} (aₙ + \mathrm{pivot}_ℓ(x)) + q^{(n-1)}}.
  \end{align*}
  The numerator has the following form:
  \begin{align*}
    & \hphantom{{} = {}} x_ℓ (P^{(n-1)} (a + \mathrm{pivot}(x, ℓ)) + p^{(n-1)}) \\
    & = x_ℓ (P^{(n-1)} a + P_ℓ^{(n-1)} \frac{1}{x_ℓ} + \sum_{i ≠ ℓ} P_i^{(n-1)} \frac{x_i}{x_ℓ} + p^{(n-1)}) \\
    & = \underbrace{(P^{(n-1)} a + p^{(n-1)})}_{P_ℓ^{(n)}} x_ℓ
        + \sum_{i ≠ ℓ} \underbrace{P_i^{(n-1)}}_{P_i^{(n)}} x_i
        + \underbrace{P_ℓ^{(n-1)}}_{p^{(n)}} \\
    & = P^{(n)} x + p^{(n)}.
  \end{align*}
  Analogously, the denominator has the following form:
  \begin{align*}
    & \hphantom{{} = {}} x_ℓ (Q^{(n-1)} (a + \mathrm{pivot}(x, ℓ)) + q^{(n-1)}) \\
    & = x_ℓ (Q^{(n-1)} a + Q_ℓ^{(n-1)} \frac{1}{x_ℓ} + \sum_{i ≠ ℓ} Q_i^{(n-1)} \frac{x_i}{x_ℓ} + q^{(n-1)}) \\
    & = \underbrace{(Q^{(n-1)} a + q^{(n-1)})}_{Q_ℓ^{(n)}} x_ℓ + \sum_{i ≠ ℓ} \underbrace{Q_i^{(n-1)}}_{Q_i^{(n)}} x_i + \underbrace{Q_ℓ^{(n-1)}}_{q^{(n)}} \\
    & = Q^{(n)} x + q^{(n)}.
  \end{align*}
  Hence,
  \[
    y
    = \frac{P^{(n-1)}(a + \mathrm{pivot}(x, ℓ)) + p^{(n-1)}}{Q^{(n-1)}(a + \mathrm{pivot}(x, ℓ)) + q^{(n-1)}}
    = \frac{P^{(n)} x + p^{(n)}}{Q^{(n)} x + q^{(n)}}. \qedhere
  \]
\end{proof}

\section{Periodic Multi-Dimensional Continued Fractions}

% TODO: This is always under the assumption that the convergents are converging

In this section I will show that if the MDCF of a vector $x ∈ ℝ^d$ is periodic,
then the elements of $x$ must be all algebraic numbers of degree $d+1$.
The proof requires two additional lemmas, which are multi-dimensional analogues
of Lemma~\ref{lem:nesting} and Lemma~\ref{lem:wallis}.
In the following, $aₙ$ will always denote a sequence of integers and $ℓₙ$ a
sequence of indices between $1$ and $d$.
We begin with a generalization of Lemma~\ref{lem:nesting}.


With these two lemmas proven, we can now turn our attention to periodic MDCFs.

\begin{proposition}
  Given $α ∈ ℝ$, let $x = (α¹, α², …, α^d)$.
  If the MDCF representation of $x$ is purely periodic, then $α$ is a root of a
  polynomial with degree $≤ d+1$.
\end{proposition}

\begin{proof}
  Suppose the algorithm is purely periodic on input $x$ with period $ℓ$.
  Let $y$ be the $ℓ$-th complete quotient of $x$, then $x = y$.
  By Lemma~\ref{lem:mdcf-wallis},
  \[
    αⁱ = \frac{\sum_{j=1}^d P_{ij} αʲ + pᵢ}{\sum_{j=1}^d Qⱼ αʲ + qᵢ}, \text{ for every } i ≤ d.
  \]
  Multiplying both sides with the denominator results in the polynomial equation
  \[
    \sum_{j=1}^d (Q_j α^{i+j} - P_{ij} α^j) + α^i q_i - p_i = 0.
  \]
  For $i = 1$, the maximum degree of this polynomial is $d + 1$.
\end{proof}

% ==============================================================================
\section{Convergence}
% ==============================================================================

\begin{figure}[tb]
  \centering
  \includestandalone{figures/representation}
  \caption{
    Representation of a periodic and non-periodic real number.
  }
\end{figure}

% TODO: Idea for sentence "Do the convergents fulfill their duty and provide convergence?"

For proof of convergence, we need further requirements for the MDCF:
\begin{itemize}
  \item The first $d$ indices in $ℓ_n$ are distinct; no index occurs twice at the start.
  \item Every index occurs infinitely often in the sequence $ℓₙ$.
  \item The values $1/a_{ℓₙ}^{(n)}$ and $a_i^{(n)}/a_{ℓₙ}^{(n)}$ are bounded
    between $0$ and some constant $c ≥ 0$ for every $n ≥ 0$.
    % TODO: Does $n$ really have to be greater than zero or should it be one?
\end{itemize}
In the following, we assume that we have an MDCF of a vector $α = (α₁, …, α_d)$
where the convergent of that MDCF is denoted by
\[
  r_j^{(n)} = (r_{1j}^{(n)}, …, r_{dj}^{(n)}) = \left(  \frac{P_{1j}^{(n)}}{Q_j^{(n)}}, \dots, \frac{P_{dj}^{(n)}}{Q_j^{(n)}} \right).
\]
Each column vector $P_i$ divided by $Q_i$ represents one possible convergent.
The goal is to show that each convergent approaches the same value $α_i$.
The proof is another application of the squeeze theorem:
We show that the two sequences $s_i^{(n)}$ and $t_i^{(n)}$ are converging.
They are defined as the minimum and maximum of $r_{i1}^{(n)}, …, r_{id}^{(n)}$, respectively.
Clearly, the values $r_{i1}^{(n)}, …, r_{id}^{(n)}$ are bounded between the two,
so it remains to show that the two sequences are converging to the same value.
For the proof, we show that the sequences are increasing and decreasing, respectively,
and that they converge to the same value.
It follows that the convergents also approach the same value.

\begin{lemma}
  The sequence $s_i^{(n)}$ is monotonically increasing and $t_i^{(n)}$ is decreasing.
\end{lemma}

\begin{proof}
  Let $ℓ = ℓ_n$ be the smallest index such that for each $k ∈ \{1, …, d\}$
  there exists a previous index $ℓ_m = k$ with $m < n$,
  i.e. $n$ is the first index at which point every index has occurred at least once.
  By Lemma~\ref{lem:mdcf-wallis},
  \begin{align*}
    P_{iℓ}^{(n)} = \sum_{k = 1}^d P_{ik}^{(n-1)} a_k^{(n)} + P_{i0}^{(n-1)} \\
    \iff
    \frac{P_{iℓ}^{(n)}}{Q_{ℓ}^{(n)}} = \sum_{k = 1}^d \frac{P_{ik}^{(n-1)}}{Q_ℓ^{(n)}} a_k^{(n)} + \frac{P_{i0}^{(n-1)}}{Q_ℓ^{(n)}}.
  \end{align*}
  Using $λ_k = \frac{a_k^{(n)} Q_k^{(n-1)}}{Q_ℓ^{(n)}}$ for $1 ≤ k ≤ d$ and $λ₀ = \frac{Q_0^{(n-1)}}{Q_ℓ^{(n)}}$,
  we can reformulate the convergent as a convex combination of the form
  \begin{align*}
    \frac{P_{iℓ}^{(n)}}{Q_{ℓ}^{(n)}} = \sum_{k = 1}^d λⱼ \frac{P_{ik}^{(n-1)}}{Q_k^{(n-1)}} + λ₀ \frac{P_{i0}^{(n-1)}}{Q_0^{(n-1)}}.
  \end{align*}
  Because of the requirement that every index has occurred at least once,
  $Q_k^{(n-1)} ≠ 0$ for every $k ≤ d$.
  By the definition of $Q_ℓ^{(n)}$, we have $λ₀ + λ₁ + \dots + λ_d = 1$.
  The sequence $t_i^{(n-1)}$ denotes the largest fraction and $s_i^{(n-1)}$ the smallest
  fraction of $\frac{P_{i0}^{(n-1)}}{Q_0^{(n-1)}}, …, \frac{P_{id}^{(n-1)}}{Q_d^{(n-1)}}$.
  Hence, we can bound the convergent by
  \[
    \frac{P_{iℓ}^{(n)}}{Q_{ℓ}^{(n)}} ≥ \sum_{k=1}^d λⱼ s_i^{(n-1)} + λ₀ s_i^{(n-1)} = s_i^{(n-1)}.
  \]
  and for the other convergents $r_j^{(n)}$ with $j ≠ ℓ$ the bound is even easier since
  \[
    \frac{P_{ij}^{(n)}}{Q_j^{(n)}} = \frac{P_{ij}^{(n-1)}}{Q_j^{(n-1)}} ≥ s_i^{(n-1)}.
  \]
  Similarly, we can bound the convergent from above by $t_i^{(n-1)}$.
  Therefore, in each iteration the fractions the values $\frac{P_{iℓ}^{(n)}}{Q_ℓ^{(n)}}$ are
  bounded between the previous minimum $s_i^{(n-1)}$ and the previous maximum
  $t_i^{(n-1)}$, or equivalently $s_i^{(n-1)} ≤ s_i^{(n)} ≤ t_i^{(n)} ≤ t_i^{(n-1)}$.
\end{proof}

The idea behind the proof is that we can reformulate each coordinate of a
convergent as a as a convex combination of all coordinates in the previous
convergent.
There must be two coefficients in the combination which are not zero
and therefore the next convergent moves inside the convex hull spanned by the previous convergent.
The convex hull shrinks at each step.

\begin{lemma}
  The minimum $s_i^{(n)}$ and maximum $t_i^{(n)}$ are converging to the same
  number $α_i$, if every index occurs infinitely often in $ℓ_n$.
\end{lemma}

\begin{proof}
  Let $s_i = \lim_{n → ∞} s_i^{(n)}$
  and $t_i = \lim_{n → ∞} t_i^{(n)}$.
  Suppose $s_i < t_i$.
  For $s_i$, there must exist an $ε > 0$ and an index $N$ such that for every $n ≥ N$,
  \[
    \frac{P_{ij}^{(n)}}{Q_j^{(n)}} ≥ s_i^{(n)} > s_i - ε.
  \]
  Hence,
  \[
    \frac{P_{iℓ}^{(n)}}{Q_j^{(n)}} > (λ₀ + λ₁ + \dots + λ_d) (s_i - ε).
    \qedhere
  \]
\end{proof}

% ==============================================================================
\section{The Geometry behind Multi-Dimensional Continued Fractions}
% ==============================================================================

For the geometrical interpretation of continued fractions,
we represented each convergent $pₙ/qₙ$ as a 2D vector $(pₙ, qₙ)$.
This vector approaches an irrational line spanned by the vector $(1, α)$ where
$α$ is some irrational number.
For the generalization of this interpretation to multi-dimensional continued
fractions, we represent each convergent, which is a $d$-dimensional rational vector,
as a $(d+1)$-dimensional integer vector.
Specifically, given a vector $(p₁/q₁, …, p_d/q_d) ∈ ℚ^d$, first we find a
common denominator $(p₁'/q, …, p_d'/q_d)$ and then we map it to the vector
$(q, p₁, …, p_d) ∈ ℤ^{d+1}$.
We can go back from the integer vector to a rational vector by dividing each
coordinate with the first and removing the first coordinate from the integer
vector:
Given $(x₀, x₁, …, x_d) ∈ ℤ^{d+1}$, we map it back to $(x₁/x₀, …, x_d/x₀) ∈ ℚ^d$.

In this space, the representation for a particular convergent is not unique,
there can be multiple integer vectors representing the same convergent.
For example, if we have a vector $c ∈ ℤ^d$ for a convergent,
then we can multiply with some scalar $λ ∈ ℤ$ and get a new vector $c' = λ c$
which represents the same convergent.
This is because the scalar is eliminated when mapping back to the rational vector:
\[
  λ (c₀, c₁, …, c_d) ↦  \left(\frac{λ c₁}{λ c₀}, …, \frac{λ c_d}{λ c_0} \right).
\]
Therefore, two vectors $a, b ∈ ℤ^{d+1}$ are equivalent, denoted as $a ≡ b$,
if $a = λ b$ or $b = λ a$ for some $λ ∈ ℤ$.

% TODO: Are the convergents calculated by the recurrence relation in this
% section always minimal/irreducible?



% TODO: Explain mapping from and back to the original vector space.
\begin{center}
  \begin{tikzpicture}
    \matrix[
      column sep=2cm,
      nodes={text width=3cm, align=center},
    ] {
      \node (L0) {$\mathbb{R}^d$}; &
      \node (R0) {$\mathbb{RP}^{d+1}$}; \\
      \node (L1) {$(x₁, …, x_d)$}; &
      \node (R1) {$[1, x₁, …, x_d]$}; \\
      \node (L2) {$(x₁/x₀, …, x_d/x₀)$}; &
      \node (R2) {$[x₀, x₁, …, x_d]$}; \\
    };

    \draw[->] (L1) -- node[above] {} (R1);
    \draw[<-] (L2) -- node[above] {} (R2);
  \end{tikzpicture}
\end{center}

For example, consider the vector $[1, x₁, x₂]$ and suppose that $0 ≤ x₁, x₂ < 1$.
A pivot operation with $ℓ = 1$ would result in the vector $[1, 1/x₁, x₂/x₁]$.
This vector is equivalent to $[x₁, 1, x₂]$.
Therefore, we can reformulate this operation as a coordinate swap of $x_ℓ$ with
the new coordinate $x₀$:
\[
  \begin{bmatrix}
    0 & 1 & 0 \\
    1 & 0 & 0 \\
    0 & 0 & 1 \\
  \end{bmatrix}
  ·
  \begin{bmatrix} 1 \\ x₁ \\ x₂ \\ \end{bmatrix}
  =
  \begin{bmatrix} x₁ \\ 1 \\ x₂ \\ \end{bmatrix}
  =
  \begin{bmatrix} 1 \\ 1/x₁ \\ x₂/x₂ \\ \end{bmatrix}.
\]
Similarly, subtracting the integer part of each value in $[1, x₁, x₂]$ is
equivalent to a series of skew operations:
\[
  \begin{bmatrix}
    1 & 0 & 0 \\
    -\floor{x₁} & 1 & 0 \\
    0 & 0 & 1 \\
  \end{bmatrix}
  ·
  \begin{bmatrix}
    1 & 0 & 0 \\
    0 & 1 & 0 \\
    -\floor{x₂} & 0 & 1 \\
  \end{bmatrix}
  ·
  \begin{bmatrix} 1 \\ x₁ \\ x₂ \\ \end{bmatrix}
  =
  \begin{bmatrix} 1 \\ x₁ - \floor{x₁} \\ x₂ - \floor{x₂} \\ \end{bmatrix}.
\]
Importantly, each of those matrices has
determinant $±1$.
Therefore, the whole operation can be easily reversed by inverting the matrix.
This is the equivalent of the inverse pivot operation in homogeneous
coordinates.

We can also reformulate the MDCF as a series of matrix multiplications.
In the following $T(a)$ denotes the translation by a vector $a ∈ ℝ^d$
and the matrix $S(ℓ)$ denotes the swap of $x_\ell$ with $x_0$.
The MDCF can then be written as
\[
  [a₀] = \hat a₀, \qquad
  [a₀; ℓ₁, a₁; …; ℓₙ, aₙ] = T(a₀) · S(ℓ_1) · [a₁; ℓ_2, a_2; …; ℓ_n, a_n].
\]

This allows us to dramatically simplify Lemma~\ref{lem:mdcf-wallis}.
Instead of several matrix and vector sequences, we simplify it to a single matrix sequence $\{B^{(n)}\}_{n ≥ 0}$,
where one matrix $B^{(n)}$ consists of the column vectors $B₀^{(n)}, B₁^{(n)}, …, B_d^{(n)}$.
We begin with $B^{(0)} = I_d$ and update the matrix according to
\begin{align*}
  B_{ℓₙ}^{(n)} = B^{(n-1)} \hat a^{(n)},
  \qquad B_i^{(n)} = B_i^{(n-1)},
  \qquad B_0^{(n)} = B_{ℓₙ}^{(n-1)}.
\end{align*}
By construction, $B^{(n)}$ is the combined matrix of the sequences $P^{(n)}, Q^{(n)}, p^{(n)}$ and $q^{(n)}$:
\[
  B^{(n)} = \begin{pmatrix}
    p^{(n)} & P^{(n)} \\
    q^{(n)} & Q^{(n)} \\
  \end{pmatrix}.
\]

\begin{lemma}
  \label{lem:mdcf-wallis'}
  Let $x ∈ ℝ^d$, then
  \[
    [a^{(0)}; ℓ₁, a^{(1)}; …; ℓ_{n-1}, a^{(n-1)}; ℓ, x] ≡ B^{(n-1)} \hat x
  \]
\end{lemma}

% TODO
\begin{proof}
  For $n = 0$, this is clearly true.
  Suppose the lemma is true for $n ≥ 0$, then
  \begin{align*}
    [a^{(0)}; ℓ₁, a₁; …; ℓ_n, a_n; ℓ, x]
    & ≡ [a^{(0)}; ℓ₁, a₁; …; ℓ_n, a_n + \mathrm{pivot}_ℓ(x)] \\
    & ≡ B^{(n-1)}
    \begin{bmatrix}
      a^{(n)} + \mathrm{pivot}_ℓ(x) \\ 1
    \end{bmatrix} \\
    & ≡ B^{(n-1)} (\hat a^{(n)} + \hat{x}) \\
    & ≡ B^{(n)} \hat{x} \qedhere
  \end{align*}
\end{proof}

\begin{lemma}
  \label{lem:mdcf-purely-periodic}
  If there exists a purely periodic MDCF for $x ∈ ℝ^d$,
  then $xᵢ$ is an algebraic number of degree $≤ d+1$ for every $i ≤ d$.
\end{lemma}

% TODO: Should we use x ≡ y or [x] = [y]?
\begin{proof}
  If the MDCF is purely periodic, then there is some $n ≥ 1$ such that $x = x^{(n)}$.
  By Lemma~\ref{lem:mdcf-wallis'},
  \[
    \hat x ≡ B^{(n)} \hat x^{(n)} ≡ B^{(n)} \hat x \iff λ₁ \hat x = λ₂ B^{(n)} \hat x \iff λ \hat x = B^{(n)} \hat x,
  \]
  for some $λ₁, λ₂ ∈ ℝ \setminus \{0\}$ with $λ = λ₁/λ₂$.
  Therefore, we are looking for an eigenvector $\hat x$ and an eigenvalue $λ$ of $B^{(n)}$.
  The characteristic polynomial $\det(B^{(n)} - λ I)$ can have a degree of at most $d+1$,
  therefore the eigenvalue $λ$ is an algebraic number of degree $d+1$.

  For the eigenvector $\hat x$, we have to find a nontrivial solution to the
  homogeneous linear system
  \[
    (B^{(n)} - λ I) \hat x = 0.
  \]
  Each coefficient in the linear system is either an algebraic
  number of degree $≤ d+1$ or an integer.
  Therefore, the eigenvectors solely consists of algebraic numbers which have a
  degree of $≤ d+1$.

  Finally, the eigenvector $\hat x$ has to be projected back from homogeneous coordinates $x$.
  But every $xᵢ = \hat xᵢ / \hat x₀$ is an algebraic number of degree $≤ d+1$,
  since $xᵢ$ is a rational expression of two such algebraic numbers.
\end{proof}

\begin{theorem}
  \label{thm:mdcf-periodic}
  If there exists a periodic MDCF for $x ∈ ℝ^d$,
  then each $x_i$ is a root of some polynomial with degree $≤ d+1$ for every $i ≤ d$.
\end{theorem}

\begin{proof}
  Let $x^{(k)}$ be the complete quotients of $x$.
  Suppose there exists a period $ℓ$ such that $x^{(k)} = x^{(k+ℓ)}$.
  By Lemma~\ref{lem:mdcf-wallis'},
  \[
    \hat x = B^{(k)} \hat x^{(k)} = B^{(k+ℓ)} \hat x^{(k+ℓ)} = B^{(k+ℓ)} \hat x^{(k)}.
  \]
  The algebraic numbers in $\hat x_k$ must have degree $≤ d+1$.
  Since every element in $x$ is a linear rational expression of the form
  \[
    x_i = \frac{∑_{j=1}^d B_{ij}^{(k)} x_j^{(k)} + B_{i0}^{(k)}}{\sum_{j=1}^d B_{0j}^{(k)} x_j^{(k)} + B_{00}^{(k)}},
  \]
  where $x_j^{(k)}$ is an algebraic number of degree $≤ d+1$ and the elements of $P^{(k)}$ are integers,
  every element in $x$ must also be an algebraic number of degree $≤ d+1$.
\end{proof}
