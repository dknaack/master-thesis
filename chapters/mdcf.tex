\chapter{Multidimensional Continued Fractions}
\label{ch:mdcf}

% TODO: This needs to be rewritten since we have already introduced a
% representation in the previous chapter on the generalized Euclidean algorithm
This chapter further investigates the properties of the multidimensional
continued fractions constructed in Section~\ref{sec:mcf-construction}.
We begin with a proper definition of multidimensional continued fractions
analogous to the definition of continued fractions.
This is followed by the two main results regarding multidimensional continued fraction.
The first is that they converge under certain conditions
and the second is that periodic MCFs always consist of algebraic numbers with degree $≤ d+1$.
For the latter, we have to analyze the geometry behind MCFs in the same
style as Klein did for continued fractions.
What is missing from this chapter is the other direction,
that MCFs containing certain algebraic numbers are always periodic.
This will be discussed in more detail in the next chapter.

% ==============================================================================
\section{Definition and Basic Properties}
\label{sec:mcf-definition}
% ==============================================================================

Previously, we have used the generalized Euclidean algorithm to construct multidimensional continued fractions.
However, this results in a top-down definition,
where we begin with a vector $x$ and derive its representation using the algorithm.
In contrast, the definition of continued fractions on page~\pageref{def:cont-frac} is bottom-up.
We begin with the continued fraction $[a₀]$ of a single coefficent
and then we inductively define the continued fraction $[a₀; a₁, …, aₙ]$
based on the smaller continued fraction $[a₁; a₂, …, aₙ]$.
Such a definition is needed, so that we can calculate the convergents of a
multidimensional continued fraction.

We can derive a bottom-up definition by reversing the pivot operation.
Let $x ∈ ℝ^d$ and $a = \floor{x}$.
Suppose that $x' = \mathrm{pivot}_ℓ(x)$ for a given index $ℓ ∈ \{1, …, d\}$.
The goal is to derive $x$ from $x'$.
By definition,
\[
  x_ℓ' = \frac{1}{x_ℓ - a_ℓ}
  \; \text{ and } \;
  x_i' = \frac{x_i - a_i}{x_ℓ - a_ℓ}
  \; \text{ for all } i ≠ ℓ.
\]
By rearranging these equations, we can calculate $x$ from $x'$ and $a$ using
\[
  x_ℓ = a_ℓ + \frac{1}{x_ℓ'}
  \; \text{ and } \;
  x_i = a_i + \frac{x_i'}{x_ℓ'}
  \; \text{ for all } i ≠ ℓ.
\]
Thus, we have an inverse function of the pivot operation.

The inverse function does not require the index $ℓ$ used in $x' = \mathrm{pivot}_ℓ(x)$.
The reason is that $x_ℓ' = 1/\{x_ℓ\}$ is always the largest element, since $\{x_i\} < 1$ for every element.
So we can derive $ℓ$ from the vector $x'$ by finding the index of the largest element.
Therefore, the inverse function $\mathrm{pivot}^{-1}$ only takes
a vector $x' ∈ ℝ^d$ and a vector $a ∈ ℤ^d$ as input.
It calculates the vector $x$ according to the previous equations.
In summary, we have
\[
  \mathrm{pivot}_ℓ(x) = x' \iff \mathrm{pivot}^{-1}(a, x') = x.
\]

Using this inverse operation we can directly derive the bottom-up definition for MCFs.
So far we have only used integer vectors $a ∈ ℤ^d$.
However, in the definition we will allow any real vector,
just like we did with continued fractions.
Again this is for subsequent theorems, where we will need rational or even real vectors coefficients.

\begin{definition}
  Let $(a^{(n)})_{n ≥ 0}$ be a sequence of $d$-dimensional real vectors.
  Then, the \emph{multidimensional continued fraction}~$[a^{(0)}; a^{(1)}, …, a^{(n)}]$ is defined inductively as
  \[
    [a^{(0)}] = a^{(0)},
    \qquad
    [a^{(0)}; a^{(1)}, …, a^{(n)}]
    = \mathrm{pivot}^{-1}\big(a^{(0)}, [a^{(1)}; a^{(2)}, …, a^{(n)}]\big).
  \]
\end{definition}

For the representation to be correct, we require $\max_i a_i^{(n)} ≠ 0$ for $n ≥ 1$.
This is similar to the continued fractions, where only the first value could be zero,
while all subsequent values had to be positive.
For the multidimensional counterpart we only require that the pivot element is not zero.
The other values in the coefficients $a^{(n)}$ can assume any non-negative value, including zero.

\begin{example}
  Consider the MCF $x = [(1,\, 2); (3,\, 2),\, (4,\, 5)]$.
  By definition,
  \begin{align*}
    x & = \mathrm{pivot}^{-1}\Bigl((1,\, 2),\, [(3,\, 2); (3,\, 5)]\Bigr) \\
      & = \mathrm{pivot}^{-1}\Bigl((1,\, 2),\, \mathrm{pivot}^{-1}\bigl((3,\, 2),\, [(3,\, 5)]\bigr)\Bigr) \\
      & = \mathrm{pivot}^{-1}\Bigl((1,\, 2),\, \mathrm{pivot}^{-1}\bigl((3,\, 2),\, (3,\, 5)\bigr)\Bigr).
  \end{align*}
  We start with the inner most pivot operation,
  where the largest element in $(4,\, 5)$ is $5$.
  Therefore, we must have pivoted with $ℓ = 2$.
  We invert the second element, divide the first element by $5$
  and we add the vector $(3,\, 2)$ to the result, which leads to
  \begin{align*}
    x & = \mathrm{pivot}^{-1}\bigl((1,\, 2),\, (3,\, 2) + (3/5,\, 1/5)\bigr)
        = \mathrm{pivot}^{-1}\bigl((1,\, 2),\, (18/5,\, 11/5)\bigr).
  \end{align*}
  In the next iteration, $18/5$ is the largest element.
  Therefore, we invert of the first element and divide the second element by $18/5$,
  resulting in
  \begin{align*}
    x & = (1,\, 2) + (5/18,\, 11/5 · 5/18) = (23/18,\, 41/18).
  \end{align*}
  Thus, $[(1,\, 2); (3,\, 2),\, (3,\, 5)] = (23/18,\, 41/18)$.
\end{example}

An infinite MCF will be defined in the next section.
Nevertheless, $[a^{(0)}; a^{(1)}, …, a^{(n)}]$ denotes the \emph{$n$-th convergent}
and $[a^{(n)}; a^{(n+1)}, …]$ denotes the \emph{$n$-th complete quotient} of the infinite MCF $[a^{(0)}; a^{(1)}, …]$.
For the analysis,
we begin with two lemmas that generalize Lemma~\ref{lem:cf-nesting} and Lemma~\vref{lem:cf-wallis}.
The first allows us to merge the last two coefficients of an MCF.
The second gives us a formula for calculating the convergents.

\begin{lemma}
  \label{lem:mcf-nesting}
  Let $x ∈ ℝ^d$, then
  \[
    [a^{(0)}; a^{(1)}, …, a^{(n)}, x]
    = [a^{(0)}; a^{(1)}, …, a^{(n-1)}, \mathrm{pivot}^{-1}(a^{(n)}, x)]
  \]
\end{lemma}

\begin{proof}
  If $n = 0$, then by definition,
  \[
    [a^{(0)}; x] = \mathrm{pivot}^{-1}(a^{(0)}, [x]) = \mathrm{pivot}^{-1}(a^{(0)}, x) = [\mathrm{pivot}^{-1}(a^{(0)}, x)].
  \]
  Suppose the lemma holds for any $n ≥ 0$, then
  \begin{align*}
    [a^{(0)}; a^{(1)}, …, a^{(n+1)}, x]
    & = \mathrm{pivot}^{-1}(a^{(0)}, [a^{(1)}; a^{(2)}, …, a^{(n+1)}, x]) \\
    & = \mathrm{pivot}^{-1}(a^{(0)}, [a^{(1)}; a^{(2)}, …, a^{(n)}, \mathrm{pivot}^{-1}(a^{(n)}, x)] \\
    & = [a^{(0)}; a^{(1)}, …, a^{(n)}, \mathrm{pivot}^{-1}(a^{(n+1)}, x)]. \qedhere
  \end{align*}
\end{proof}

% TODO: Explain how to derive the sequences
The second lemma is a linear recurrence for calculating the convergents.
For continued fractions,
we defined the convergents of a continued fraction~$pₙ/qₙ$
based on the previous two terms~$p_{n-1}/q_{n-1}$ and $p_{n-2}/q_{n-2}$.
For MCFs, we can similarly derive a recursive formula to derive the values of
the convergent vector $(p₁/q, \dots, p_d/q)$ using the previous convergents.
Deriving the sequence is more involved than the one-dimensional case,
since we have an additional pivot index $ℓ$ at each step.

There are two types of sequences:
A sequence of vectors $P_0^{(n)}, P_1^{(n)}, …, P_d^{(n)} ∈ ℤ^{d+1}$ and a sequence
of scalars $Q_0^{(n)}\!,\, Q_1^{(n)}\!,\, …, Q_d^{(n)} ∈ ℤ$.
The initial terms are
\[
  \begin{aligned}
    P_0^{(-1)} & = 0, & P_i^{(-1)} & = e_i, \\
    Q_0^{(-1)} & = 1, & Q_i^{(-1)} & = 0,
  \end{aligned}
\]
where $e_i$ is the $i$-th unit vector.
For the remaining terms,
Let $ℓ$ be the pivot index of $x^{(n)}$, i.e. $x^{(n)} = \mathrm{pivot}_ℓ(x^{(n-1)})$.
Then, the next terms in the sequences are:
\begin{equation}
  \label{eq:mcf-wallis}
  \begin{array}{r@{\;}c@{\;}l@{\;}c@{\;}l@{\;}c@{\;}c@{\;}c@{\;}l@{\qquad}r@{\;}c@{\;}l}
    P_ℓ^{(n)} &=& P_0^{(n-1)} &+& P_1^{(n-1)} a_1^{(n)} &+& ⋯ &+& P_d^{(n-1)} a_d^{(n)}, & P_0^{(n)} &=& P_ℓ^{(n-1)}, \\
    Q_ℓ^{(n)} &=& Q_0^{(n-1)} &+& Q_1^{(n-1)} a_1^{(n)} &+& ⋯ &+& Q_d^{(n-1)} a_d^{(n)}, & Q_0^{(n)} &=& Q_ℓ^{(n-1)}.
  \end{array}
\end{equation}
The terms $P_i^{(n)}$ and $Q_i^{(n)}$ with $i ≠ ℓ$ are carried over from the previous iteration,
i.e.
\[
  P_i^{(n)} = P_i^{(n-1)}, Q_i^{(n)} = Q_i^{(n-1)}.
\]

The idea behind these sequences is that they behave like the generalized
Euclidean algorithm, but in reverse.
We can consider the vectors $P_0^{(n)}, P_1^{(n)}, …, P_d^{(n)}$ as the basis,
which is reduced by the generalized algorithm.
It terminates when one vector is an integral combination
of the other vectors, because then the remainder is zero.
The sequence $P_0^{(n)}$ starts with the zero vector,
i.e. when the basis has been fully reduced.
Therefore, $P_0^{(n)}$ represents the remainder $c$ and the vectors $P_1^{(n)},
…, P_d^{(n)}$ represent the basis $B$.
Then, the formula calculates a new vector using an integral combination of the
old vectors and stores it in $P_ℓ^{(n)}$.
This corresponds directly to the modulo and exchange operation of the
generalized Euclidean algorithm.
As the vectors $P_0^{(n)}, P_1^{(n)}, …, P_d^{(n)}$ would grow infinitely, we
divide them by the scalars $Q_0^{(n)}, Q_1^{(n)}, …, Q_d^{(n)}$ to ensure that
they converge to a limit as $n$ increases.

% TODO: Ensure that this is correct for the first index!
\begin{lemma}
  \label{lem:mcf-wallis}
  Let $a^{(0)}, a^{(1)}, …, a^{(n-1)}, x ∈ ℝ^d$, then
  \[
    [a^{(0)}; a^{(1)}, …, a^{(n-1)}, x]
    = \frac{P_0^{(n-1)} + P_1^{(n-1)} x_1 + ⋯ + P_d^{(n-1)} x_d}{Q_0^{(n-1)} + Q_1^{(n-1)} x_1 + ⋯ + Q_d^{(n-1)} x_d}.
  \]
\end{lemma}

\begin{proof}
  We proceed via induction on $n$.
  If $n = 0$, then
  \[
    [x]
    = x
    = \frac{x}{1}
    = \frac{0 + e₁ x₁ + ⋯ + e_d x_d}{1 + 0 x₁ + ⋯ + 0 x_d}
    = \frac{P₀^{(0)} + P₁^{(0)} x₁ + ⋯ + P_d^{(0)} x_d}{Q_0^{(0)} + Q_1^{(0)} x₁ + ⋯ + Q_d^{(0)} x_d}.
  \]
  Suppose the lemma holds for $n ≥ 0$, then we show that it also holds for $n + 1$.
  From the previous lemma, it follows that
  \begin{align*}
    [a^{(0)}; a^{(1)}; …, a^{(n)}, x] & = [a^{(0)}; a^{(1)}, …, a^{(n-1)}, \mathrm{pivot}^{-1}(a^{(n)}, x)].
  \end{align*}
  Let $y = \mathrm{pivot}^{-1}(a^{(n)}, x)$ with $ℓ$ as the pivot index from $x$ to $y$.
  By the induction hypothesis,
  \begin{align*}
    [a^{(0)}; a^{(1)}; …, a^{(n)}, x]
    & = \frac{P_0^{(n-1)} + P_1^{(n-1)} y_1 + ⋯ + P_d^{(n-1)} y_d}{Q_0^{(n-1)} + Q_1^{(n-1)} y_1 + ⋯ + Q_d^{(n-1)} y_d} \\
    & = \frac{x_ℓ}{x_ℓ} · \frac{P_0^{(n-1)} + P_1^{(n-1)} y_1 + ⋯ + P_d^{(n-1)} y_d}{Q_0^{(n-1)} + Q_1^{(n-1)} y_1 + ⋯ + Q_d^{(n-1)} y_d},
  \end{align*}
  where the fraction is expanded with $x_ℓ/x_ℓ$ such that the numerator and denominator are each multiplied by $x_ℓ$.
  Let $P$ denote the numerator and let $Q$ denote the denominator of the expression.
  The numerator $P$ can then be simplified as follows:
  \begin{align*}
    P
    & = x_ℓ \left( P_0^{(n-1)} + P_ℓ^{(n-1)} y_ℓ + \sum_{i ∉ \{0,ℓ\}} P_i^{(n-1)} y_i \right) \\
    & = x_ℓ \left( P_0^{(n-1)} + P_ℓ^{(n-1)} \left( a_ℓ^{(n)} + \frac{1}{x_ℓ} \right) + \sum_{i ∉ \{0,ℓ\}} P_i^{(n-1)} \left(a_i^{(n)} + \frac{x_i}{x_ℓ} \right) \right) \\
    & = P_0^{(n-1)} x_ℓ + P_ℓ^{(n-1)} a_ℓ^{(n)} x_ℓ + P_ℓ^{(n-1)} + \sum_{i ∉ \{0,ℓ\}} P_i^{(n-1)} a_i^{(n)} x_ℓ + P_i^{(n-1)} x_i \\
    & = \underbrace{\left( P_0^{(n-1)} + P_ℓ^{(n-1)} a_ℓ^{(n)} + \sum_{i ∉ \{0,ℓ\}} P_i^{(n-1)} a_i^{(n)} \right)}_{P_ℓ^{(n)}} x_ℓ
      + \underbrace{P_ℓ^{(n-1)}}_{P_0^{(n)}}
      + \sum_{i ∉ \{0,ℓ\}} \underbrace{P_i^{(n-1)}}_{P_i^{(n)}} x_i \\
    & = P_0^{(n)} + P_1^{(n)} x_1 + ⋯ + P_d^{(n)} x_d.
  \end{align*}
  The simplification for the denominator $Q$ is identical.
  Therefore,
  \[
    [a^{(0)}; a^{(1)}, …, a^{(n)}, x]
    = \frac{P}{Q}
    = \frac{P_0^{(n)} + P_1^{(n)} x_1 + ⋯ + P_d^{(n)} x_d}{Q_0^{(n)} + Q_1^{(n)} x_1 + ⋯ + Q_d^{(n)} x_d}.
    \qedhere
  \]
\end{proof}

% ==============================================================================
\section{Infinite Multidimensional Continued Fractions and their Convergence}
\label{sec:mcf-convergence}
% ==============================================================================

% TODO: I don't like how we say that the requirement is from Perron, when it's
% not really from him, but just based on his version.
% TODO: The indices here are wrong...
So far in our analysis, we have implicitly assumed that the MCF $[a^{(0)}; a^{(1)}, a^{(2)} …]$
constructed using the generalized Euclidean algorithm always converges to the
original input vector $x ∈ ℝ^d$.
The aim of this section is to show that the convergents actually live up to
their name and converge towards the vector $x$ as $n$ increases.

\begin{definition}
  Let $(a^{(n)})_{n ≥ 0}$ be a sequence of $d$-dimensional real vectors
  and let
  \[
    r^{(n)} = (r_1^{(n)}, …, r_d^{(n)})^⊤ = [a^{(0)}; a^{(1)}, …, a^{(n)}].
  \]
  Then, the infinite multidimensional continued fraction $[a^{(0)}; a^{(1)}, …]$ is defined to be
  the vector $r = (r₁, …, r_d) ∈ ℝ^d$ if the limit
  \[
    r_i = \lim_{n → ∞} r_i^{(n)}
  \]
  exists.
\end{definition}

We assume that we have an infinite MCF $[a^{(0)}; a^{(1)}, …]$
with only integer vectors $a^{(n)}$.
The vector $r^{(n)}$ denotes the $n$-th \emph{convergent}
and $x^{(n)}$ denotes the $n$-th \emph{complete quotient} of the MCF $[a^{(0)}; a^{(1)}, …]$.
During the construction, we start with $x^{(0)}$ and use the pivot operation to
go from one complete quotient, $x^{(n-1)}$, to the next, $x^{(n)}$.
At each step, we have to define an index, that we use for pivoting.
If $x^{(n)} = \mathrm{pivot}_ℓ(x^{(n-1)})$ for some index $ℓ ∈ \{1, …, d\}$,
then $ℓ$ is called the \emph{pivot index} of the complete quotient $x^{(n)}$.

By Lemma~\ref{lem:mcf-wallis},
if $ℓ$ is the pivot index of the complete quotient $x^{(n)}$,
then the $n$-th convergent is simply
\[
  r^{(n)}
  = \frac{P_ℓ^{(n)}}{Q_ℓ^{(n)}}
  = \frac{P_0^{(n-1)} + P_1^{(n-1)} a_1^{(n)} + ⋯ + P_d^{(n-1)} a_d^{(n)}}{Q_0^{(n-1)} + Q_1^{(n-1)} a_1^{(n)} + ⋯ + Q_d^{(n-1)} a_d^{(n)}}.
\]
The vector $r^{(n)}$ is only one possible convergent out of several.
If $n$ is large enough, then we can consider the other vectors
$P_i^{(n)}/Q_i^{(n)}$ with $i ≠ ℓ$ as convergent vectors,
since $x_i^{(m)}$ must have been the largest element at some index $m < n$.
We will refer to $r^{(n)}$ as the \emph{primary convergent} and the others as
\emph{secondary convergents}.

The goal of this section is to show that the primary convergent $r^{(n)}$ approaches the vector $x$.
The proof is based on Perron's convergence proof for his algorithm \cite{Perron07}.
Based on this proof, an infinite MCF converges under the following conditions:
\begin{enumerate}
  \item
    For every $n ≥ 0$,
    \[
      0 < \frac{1}{a_ℓ^{(n)}} ≤ A
      \quad \text{ and } \quad
      0 ≤ \frac{a_i^{(n)}}{a_ℓ^{(n)}} ≤ A \quad \text{ for every } i ≠ ℓ,
    \]
    where $ℓ$ is the index of the largest element in the complete quotient $x^{(n)}$.
  \item
    During the construction of the MCF,
    every index $ℓ ∈ \{1, …, d\}$ is used infinitely often.
    Formally, for every $ℓ ∈ \{1, …, d\}$ and $N ≥ 0$,
    we can find an index $n ≥ N$ such that $ℓ$ is the pivot index of $x^{(n)}$.
  \item
    The distance between the same index is bounded by some constant $L$,
    i.e. for every index $n ≥ 0$ there exists an $m ∈ \{n+1, …, n+L\}$ such that
    the pivot indices of $x^{(m)}$ and $x^{(n)}$ are the same.
\end{enumerate}

The first condition is already satisfied by the generalized Euclidean algorithm,
if we set $A = 1$.
The integer part of every element in the complete quotient is determined by $a_i^{(n)}$,
Because $x^{(n)} = \mathrm{pivot}_ℓ(x^{(n-1)}$,
the value~$x_ℓ^{(n)}$ is the largest and therefore $a_ℓ^{(n)}$ must also be the
largest element in $a^{(n)}$.
Furthermore, $a_ℓ^{(n)}$ cannot be zero by definition.
Hence, this condition is satisfied for $A = 1$.
Nevertheless,
this condition is necessary when replacing the floor function with a different
function.
If one can guarantee that the new function satisfies the first condition, the
convergence will still hold.
The condition will only be needed in one place,
but in return it makes the proof more general.

The last two conditions are specifically for MCFs.
When updating the secondary convergents,
we exchange $P_0^{(n)} / Q_0^{(n)}$ with $P_ℓ^{(n)}/Q_ℓ^{(n)}$
and then replace  $P_ℓ^{(n)}/Q_ℓ^{(n)}$ with a new vector.
Thus, we only move one of the secondary convergents in each iteration.
However, the second condition guarantees that we move every secondary
convergent at some point
and the last condition guarantees that this must happen after at most $L$ iterations.
Therefore, every secondary convergent changes frequently.

\begin{example}
  Consider the MCF $x = [(2, 1); \overline{(2, 1)}]$,
  where the vector $(2, 1)$ repeats periodically.
  This MCF does not meet the requirements for this section.
  In this case, the first complete quotient is $x^{(1)}$, and the pivot index is $\ell = 1$,
  since the coefficient $a^{(1)} = (2, 1)$ determines the integer part
  of each coordinate in $x^{(1)}$, and $2$ is the largest entry.
  The following complete quotient $x^{(2)}$ only affects the fractional part
  of each coordinate, so the first coordinate continues to be the pivot.

  However, this violates the second requirement of the proof,
  which states that every index $\ell \in \{1, \dots, d\}$ must occur
  as a pivot index infinitely often.
  In this example, the pivot index is always $\ell = 1$,
  so the other index (in this case, $2$) is never selected.
  Therefore, the convergence proof presented in this section does not apply to this example.
\end{example}

Since we use the generalized Euclidean algorithm to construct an MCF,
we may assume that all vectors $a^{(n)}$ are positive integers,
including the initial coefficient $a^{(0)}$.
Although the first coefficient could be negative,
the second coefficient is always positive.
If we can prove convergence for the second complete quotient $x^{(1)} = [a^{(1)}; a^{(2)}, …]$,
then this implies the convergence for the whole MCF.
Thus, it suffices to show the convergence for the second complete quotient,
which is positive.

\begin{lemma}
  \label{lem:conv-conv}
  There exists a starting index $N ≥ 0$ such that for all $n ≥ N$,
  there are nonnegative coefficients $λ₀^{(n)}, λ₁^{(n)}, …, λ_d^{(n)}$
  satisfying $λ₀^{(n)} + λ₁^{(n)} + ⋯ + λ_d^{(n)} = 1$ and
  \[
    r^{(n)} = λ_0^{(n)} \frac{P_0^{(n-1)}}{Q_0^{(n-1)}}
            + λ_1^{(n)} \frac{P_1^{(n-1)}}{Q_1^{(n-1)}}
            + ⋯
            + λ_d^{(n)} \frac{P_d^{(n-1)}}{Q_d^{(n-1)}}.
  \]
\end{lemma}

\begin{proof}
  The index $N$ is chosen such that each index has been used at least once in the construction.
  In particular, all denominators $Q_i^{(n-1)}$ with $n ≥ N$ are not zero.
  Let $ℓ$ be the pivot index of $x^{(n)}$.
  According to Equation~\ref{eq:mcf-wallis}, the vector $P_ℓ^{(n)}$ can be calculated as follows:
  \begin{align*}
    P_ℓ^{(n)} & = P_0^{(n-1)} + P_1^{(n-1)} a_1^{(n)} + ⋯ + P_d^{(n-1)} a_d^{(n)},
  \end{align*}
  Dividing by $Q_ℓ^{(n)}$ results in
  \begin{align*}
    \frac{P_ℓ^{(n)}}{Q_ℓ^{(n)}}
    & = \frac{P_0^{(n-1)}}{Q_ℓ^{(n)}}
      + \frac{P_1^{(n-1)}}{Q_ℓ^{(n)}} a_1^{(n)}
      + ⋯
      + \frac{P_d^{(n-1)}}{Q_ℓ^{(n)}} a_d^{(n)},
  \end{align*}
  If we define the coefficients as $λ_k^{(n)} = a_k^{(n)} \frac{Q_k^{(n-1)}}{Q_ℓ^{(n)}}$ for $k ∈ \{1, …, d\}$
  and $λ₀^{(n)} = \frac{Q_0^{(n-1)}}{Q_ℓ^{(n)}}$,
  then we can reformulate the convergent $r^{(n)}$ as
  \[
      λ_0^{(n)} \frac{P_0^{(n-1)}}{Q_0^{(n-1)}}
    + λ_1^{(n)} \frac{P_1^{(n-1)}}{Q_1^{(n-1)}}
    + ⋯
    + λ_d^{(n)} \frac{P_d^{(n-1)}}{Q_d^{(n-1)}}.
  \]
  Because $N$ was chosen large enough such that every index has occurred at least once,
  the denominators $Q_i^{(n-1)}$ cannot be zero.
  Therefore, this is a well-defined representation of the convergent $r^{(n)}$.
  For the coefficients themselves, we require $λ₀ + λ₁ + ⋯ + λ_d = 1$ and $0 ≤ λᵢ ≤ 1$ for every index $i$.
  The first property follows from the definition of $Q_ℓ^{(n)}$:
  \[
    Q_ℓ^{(n)} = Q_0^{(n-1)} + Q_1^{(n-1)} a_1^{(n)} + ⋯ + Q_d^{(n-1)} a_d^{(n)}
  \]
  which is equivalent to
  \[
    1 = \frac{Q_0^{(n-1)}}{Q_ℓ^{(n)}} + \frac{Q_1^{(n-1)}}{Q_ℓ^{(n)}} a_1^{(n)} + ⋯ + \frac{Q_d^{(n-1)}}{Q_ℓ^{(n)}} a_d^{(n)} = λ₀^{(n)} + λ₁^{(n)} + ⋯ + λ_d^{(n)}.
  \]
  The second property follows from the fact that every vector $a^{(n)}$ is nonnegative.
  The denominator $Q_ℓ^{(n)}$ is the sum of the terms $a_i^{(n)} Q_i^{(n-1)}$
  Therefore, $Q_ℓ^{(n)} ≥ a_i^{(n)} Q_i^{(n-1)}$ and $λ_i$ is always bounded between $0$ and $1$.
\end{proof}

The lemma shows that the next convergent lies inside the convex hull of the secondary convergents.
We enclose the entire convex hull within an axis-aligned bounding box
and we show that this box converges to a single point,
thereby implying the convergence of the primary convergent $r^{(n)}$.
At each iteration, the bounding box is characterized by its minimum and maximum corner,
defined as $s^{(n)} = (s_1^{(n)}, …, s_d^{(n)})$ and $t^{(n)} = (t_1^{(n)}, …, t_d^{(n)})$ with
\[
  s_i^{(n)} = \min\left\{\frac{P_{i0}^{(n)}}{Q_0^{(n)}}, \frac{P_{i1}^{(n)}}{Q_1^{(n)}}, …, \frac{P_{id}^{(n)}}{Q_d^{(n)}}\right\}
\]
and
\[
  t_i^{(n)} = \max\left\{\frac{P_{i0}^{(n)}}{Q_0^{(n)}}, \frac{P_{i1}^{(n)}}{Q_1^{(n)}}, …, \frac{P_{id}^{(n)}}{Q_d^{(n)}}\right\},
\]
where $P_j^{(n)} = (P_{1j}^{(n)}, …, P_{dj}^{(n)})$.
The primary convergent $r^{(n)}$ lies inside the box,
since it is also one of the secondary convergent vectors
used in the definition of $s^{(n)}$ and $t^{(n)}$.
Therefore, if $s^{(n)}$ and $t^{(n)}$ converge to the same point,
then the sequence $r^{(n)}$ must converge to that point, too.

\begin{lemma}
  The sequences $s^{(n)}$ and $t^{(n)}$ converge.
\end{lemma}

\begin{proof}
  We show that the sequences are monotonic and bounded, which implies convergence.
  To show that the sequence is non-decreasing,
  it suffices to show that each secondary convergent at iteration $n$
  is at least the previous minimum $s_i^{(n)}$.
  We begin with the primary convergent.
  Let $ℓ$ be the pivot index in iteration $n$.
  From the previous iteration, we already have for every $j ≠ ℓ$:
  \[
    \frac{P_{ij}^{(n)}}{Q_j^{(n)}} = \frac{P_{ij}^{(n-1)}}{Q_j^{(n-1)}} ≥ s_i^{(n-1)}.
  \]
  We can bound the primary convergent $r^{(n)}$ using Lemma~\ref{lem:conv-conv}, since
  \begin{align*}
    \frac{P_{iℓ}^{(n)}}{Q_ℓ^{(n)}}
    & = λ₀^{(n)} \frac{P_{i0}^{(n-1)}}{Q_0^{(n-1)}} + λ₁^{(n)} \frac{P_{i1}^{(n-1)}}{Q_1^{(n-1)}} + ⋯ + λ_d^{(n)} \frac{P_{id}^{(n-1)}}{Q_d^{(n-1)}} \\
    & ≥ λ₀^{(n)} s_i^{(n-1)} + λ₁^{(n)} s_i^{(n-1)} + ⋯ + λ_d^{(n)} s_i^{(n-1)}
      = s_i^{(n-1)}.
  \end{align*}
  Thus, $s_i^{(n)} ≥ s_i^{(n-1)}$ and $s^{(n)}$ is non-decreasing.
  We can show $t_i^{(n)} ≤ t_i^{(n-1)}$ by bounding the convergents from above using a similar argument.
  Moreover, $s_i^{(n)} ≤ t_i^{(n)}$ for all $n ≥ 0$.
  Hence, both sequences converge.
\end{proof}

To show that the sequences $s^{(n)}$ and $t^{(n)}$ are actually
converging to the same limit, we first need the following crucial lemma about the
coefficients $λ₀^{(n)}, λ₁^{(n)}, …, λ_d^{(n)}$ from Lemma~\ref{lem:conv-conv}.
The lemma states that one of the coefficients is always greater than a constant
independent of $n$ and that this coefficient corresponds to the current pivot index.
From the recurrence for the convergents,
it follows that we always carry over the primary convergent $r^{(n-1)}$ as one
of the secondary convergent $P_0^{(n)}/Q_0^{(n)}$ in the next iteration.
Thus, the position of $r^{(n-1)}$ remains fixed,
but we can change the position of the other convergents.
However,
the region available for the next convergent $r^{(n)}$ is restricted by this lemma
such that the convex hull always shrinks by a constant amount.

\begin{lemma}
  \label{lem:lambda-pos}
  Let $ℓ$ be the pivot index in $x^{(n)}$.
  Then $λ_ℓ^{(n)} > 1/C$ for some integer $C > 0$.
\end{lemma}

% TODO: Show that it is not equal to 1!
\begin{proof}
  The ratio between the coefficients $λ_i^{(n)}$ and $λ_ℓ^{(n)}$ is
  \begin{equation}
    \label{eq:lambda-ratio}
    \frac{λ_i^{(n)}}{λ_ℓ^{(n)}}
    = \frac{a_i^{(n)}}{a_ℓ^{(n)}} · \frac{Q_i^{(n-1)}}{Q_ℓ^{(n-1)}}
    ≤ A · \frac{Q_i^{(n-1)}}{Q_ℓ^{(n-1)}}.
  \end{equation}
  If $Q_i^{(n-1)} < Q_ℓ^{(n-1)}$,
  then we can guarantee that $λ_ℓ^{(n)} ≥ 1/C$.
  For $Q_0^{(n-1)} = Q_ℓ^{(n-2)}$, this is straightforward.
  But for the other values the issue is that $Q_i^{(n-1)}$ might be larger than $Q_ℓ^{(n-1)}$,
  if $i$ is the pivot index, for example.

  We solve this issue using the initial requirements.
  Specifically, the third requirement,
  which states that after at most $L$ iterations,
  we must have used $ℓ$ as the pivot index again.
  Therefore, there must have been some index $m ∈ \{n - L, …, n\}$
  such that $ℓ = \argmax_i x_i^{(m)}$ and $Q_ℓ^{(n-1)} = Q_ℓ^{(m)}$.
  If $x^{(m)} = \mathrm{pivot}_ℓ(x^{(m-1)})$,
  then we must have $x_ℓ^{(m+1)} = 1/\{x_ℓ^{(m)}\} > 1$
  and therefore the integer part $a_ℓ^{(m+1)}$ must be at least $1$.
  Furthermore, if $k$ is the pivot index in the next iteration $m+1$,
  then
  \[
    Q_k^{(m+1)}
    = Q_0^{(m)} + Q_1^{(m)} a_1^{(m+1)} + ⋯ + Q_d^{(m)} a_d^{(m+1)}
    ≥ a_ℓ^{(m+1)} Q_ℓ^{(m)}
    ≥ \frac{1}{A} Q_ℓ^{(m)}.
  \]
  We can repeat this step for every pivot index from $m$ to $n$
  tracing the bound back from $Q_ℓ^{(m)}$ to $Q_i^{(n-1)}$.
  This path consists of at most $L$ steps.
  Thus, we can bound $Q_i^{(n-1)}$ from below by $Q_ℓ^{(n-1)}$ according to
  \[
    Q_i^{(n-1)} ≥ \frac{1}{A^L} Q_ℓ^{(m)} = \frac{1}{A^L} Q_ℓ^{(n-1)},
  \]
  which we can finally use to bound the ratio $Q_i^{(n-1)}/Q_ℓ^{(n-1)}$
  from Equation~\ref{eq:lambda-ratio}.
  The equation leads to the bound $λᵢ^{(n)} ≤ A^{L-1} λ_ℓ^{(n)}$,
  from which it follows that
  \[
    1 = λ₀^{(n)} + λ₁^{(n)} + ⋯ + λ_d^{(n)} ≤ λ_ℓ^{(n)} (1 + dA^{L-1}).
  \]
  Hence, setting $C = 1 + dA^{L-1} > 0$, we conclude that $λ_ℓ^{(n)} > 1/C$.
\end{proof}

\begin{figure}[tbp]
  \centering
  \includestandalone{figures/convergence}
  \caption{
    Illustration of the proof for the convergence of MCFs.
    The points $r^{(i)}, r^{(j)}$ and $r^{(k)}$ represent the convergents and $x$ is
    the vector, which is approximated by them.
    If $r^{(k)}$ is the convergent of the current iteration,
    then either $r^{(i)}$ or $r^{(j)}$ has to be moved in the next iteration.
    However, neither can go in the red portion, since $λ_k^{(n)} > 1/C$.
  }
  \label{fig:convergence}
\end{figure}

If the sequences $s^{(n)}$ and $t^{(n)}$ converge to the same limit,
then that means that the bounding box shrinks to a point.
Therefore, the convergent $r^{(n)}$, which is inside this box, converges to the
same point.
However, the sequences $s^{(n)}$ and $t^{(n)}$ could approach different limits, i.e.
\[
  s = \lim_{n → ∞} s_i^{(n)} < \lim_{n → ∞} t_i^{(n)} = t.
\]
Then, the box does not converge to a point but to another box delimited
by its corner points $s$ and $t$.
More importantly, this implies that the primary convergent may not converge.

The previous lemma contradicts this assumption.
The area of the convex hull always shrinks by a at least a constant proportion.
So if the sequences would approach different limits, then we can always find a
point at which all convergents step over the box and all of them are contained
in the supposed limit.
This is the main idea behind the following proof,
which is also illustrated in Figure~\ref{fig:convergence}.

\pagebreak
\begin{lemma}
  \label{lem:min-max-conv}
  The minimum $s_i^{(n)}$ and maximum $t_i^{(n)}$ are converging to the same
  limit.
\end{lemma}

\begin{proof}
  Let $s_i = \lim_{n → ∞} s_i^{(n)}$
  and $t_i = \lim_{n → ∞} t_i^{(n)}$ for every $i ≤ d$.
  Suppose $s_i < t_i$.
  For the minimum~$s_i$, there must exist an $ε > 0$ and an index $N ≥ 0$ such
  that for every $n ≥ N$,
  \[
    \frac{P_{ij}^{(n)}}{Q_j^{(n)}} ≥ s_i^{(n)} > s_i - ε.
  \]
  Let $ℓ$ be the index of the largest element in the previous complete quotient $x^{(n-1)}$.
  Since the limit $s_i - ε$ bounds the secondary convergents,
  we can use this to bound the primary convergent
  %$Q_i^{(n-1)} = Q_i^{(n-2)}$ and $Q_k^{(n-1)} = Q_0^{(n-2)} + Q_1^{(n-2)} a_1^{(n-1)} + ⋯ + Q_d^{(n-2)} a_d^{(n-1)}$. % TODO: Why was this here?
  \begin{align*}
    r_i^{(n)}
    & = λ₀^{(n)} \frac{P_{i0}^{(n-1)}}{Q_0^{(n-1)}} + λ₁^{(n)} \frac{P_{i1}^{(n-1)}}{Q_1^{(n-1)}} + ⋯ + λ_d^{(n)} \frac{P_{id}^{(n-1)}}{Q_d^{(n-1)}} \\
    & > λ_ℓ^{(n)} \frac{P_{iℓ}^{(n-1)}}{Q_ℓ^{(n-1)}} + \sum_{i ≠ ℓ} λ_i^{(n)} (s_i - ε) \\
    & = λ_ℓ^{(n)} r_i^{(n-1)} + (1 - λ_ℓ^{(n)}) (s_i - ε),
  \end{align*}
  or equivalently
  \begin{align*}
    r_i^{(n)} - s_i > λ_ℓ^{(n)} \left( r_i^{(n-1)} - s_i \right) - ε.
  \end{align*}
  Lemma~\ref{lem:lambda-pos} tells us that there is a positive constant $C$
  which bounds $λ_ℓ^{(n)}$ from below.
  It follows that
  \begin{align*}
    r_i^{(n)} - s_i > \frac{1}{C} \left( r_i^{(n-1)} - s_i \right) - ε
  \end{align*}
  for some constant $C > 0$.
  This inequality holds for all $n ≥ N$.
  Therefore, we can advance $n$ to some point $n + k$
  such that
  \begin{align*}
    r_i^{(n+k)} - s_i
    & > \frac{1}{C} \left( r_i^{(n+k-1)} - s_i \right) - ε \\
    & > \frac{1}{C} \left(\frac{1}{C} \left( r_i^{(n+k-2)} - s_i \right) - ε\right) - ε \\
    & = \frac{1}{C^2} \left(r_i^{(n+k-2)} - s_i \right) - ε\left(1 + \frac{1}{C}\right) \\
    & \, ⋮ \\
    & > \frac{1}{C^k} \left( r_i^{(n-1)} - s_i \right) - ε\left( \frac{1}{C} + \frac{1}{C^2} + ⋯ + \frac{1}{C^{k-1}} \right).
  \end{align*}
  Since every index occurs infinitely often,
  we can find one index $n$ where $r_i^{(n-1)} ≤ t_i$
  and another index $n+k$ where $r_i^{(n+k)} ≥ s_i$.
  Thus,
  \begin{align*}
    0 > \frac{1}{C^k} \left( t_i - s_i \right) - ε\left( \frac{1}{C} + \frac{1}{C^2} + ⋯ + \frac{1}{C^{k-1}} \right).
  \end{align*}
  But $ε$ can be chosen arbitrarily small,
  which is a contradiction to the fact that $t_i - s_i > 0$.
  Therefore, $t_i$ and $s_i$ converge to the same limit.
\end{proof}

\begin{theorem}
  \label{thm:mdcf-conv}
  The sequence $r^{(n)}$ converges to $x$.
\end{theorem}

\begin{proof}
  Let $x^{(n)}$ denote the $n$-th complete quotient of $x$.
  By Lemma~\ref{lem:mcf-wallis}, we can represent each element in $x$ as
  \[
    x_i = \frac{P_{i0}^{(n-1)} + P_{i1}^{(n-1)} x_1^{(n)} + ⋯ + P_{id}^{(n-1)} x_d^{(n)}}{Q_{i0}^{(n-1)} + Q_{i1}^{(n-1)} x_1^{(n)} + ⋯ + Q_{id}^{(n-1)} x_d^{(n)}}.
  \]
  Using a similar argument as in Lemma~\ref{lem:conv-conv}, we can represent this
  as a convex combination
  \begin{align*}
    x_i = μ₀^{(n)} \frac{P_{i0}^{(n-1)}}{Q_0^{(n-1)}}  + μ₁^{(n)} \frac{P_{i1}^{(n-1)}}{Q_1^{(n-1)}} + μ_d^{(n)} \frac{P_{id}^{(n-1)}}{Q_d^{(n-1)}}
  \intertext{where the coefficients $μ_i^{(n)}$ are defined as}
    μ_i^{(n)} = x_i^{(n)} \frac{Q_i^{(n-1)}}{Q_0^{(n-1)} + Q_d^{(n)} x_1^{(n)} + ⋯ + Q_d^{(n-1)} x_d^{(n)}}.
  \end{align*}
  From the previous lemma, we know that $r^{(n)}$ converges to some limit $x' ∈ ℝ^d$.
  Since every index occurs infinitely often,
  we can find an index $m ≤ n - 1$ such that the primary convergent $r_i^{(m)}$ is
  \[
    \frac{P_{ij}^{(n-1)}}{Q_j^{(n-1)}} = \frac{P_{ij}^{(m)}}{Q_j^{(m)}}.
  \]
  As $n$ increases there are infinitely many such convergents, so each term converges to $x_i'$.
  Thus, for each secondary convergent there are sufficiently small values $ε₀, ε₁, …, ε_d$ such that
  \begin{align*}
    x_i
    & = μ₀^{(n)} \frac{P_{i0}^{(n-1)}}{Q_0^{(n-1)}}  + μ₁^{(n)} \frac{P_{i1}^{(n-1)}}{Q_1^{(n-1)}} + μ_d^{(n)} \frac{P_{id}^{(n-1)}}{Q_d^{(n-1)}} \\
    & = μ₀^{(n)} (x_i' - ε₀) + μ₁^{(n)} (x_i' - ε₁) + ⋯ + μ_d^{(n)} (x_d' - ε_d) \\
    & = x_i' - (μ₀^{(n)} ε₀ + μ₁^{(n)} ε₁ + ⋯ + μ_d^{(n)} ε_d).
  \end{align*}
  However, as $n$ increases the values $εᵢ$ must become arbitrarily small
  and because the coefficients $μ₀^{(n)}, μ₁^{(n)}, …, μ_d^{(n)}$ all lie between $0$ and $1$,
  we can conclude that $x_i = x_i'$.
\end{proof}

One important example where this theorem applies is for periodic MCFs,
where the period contains every index as a pivot index.
Since the period has finite length,
the distance between the same pivot indices is constant.
Thus, they satisfy the requirements of this theorem.
Besides periodic MCFs, the theorem also includes other JPA-like algorithms.
For example, an algorithm could construct an MCF based on a predetermined list of indices $[ℓ₁, ℓ₂, …, ℓₘ]$,
which is repeated throughout the construction.
If this list contains every possible index,
then this theorem guarantees the convergence.

% ==============================================================================
\section{Geometrical Interpretation Based on Projective Spaces}
\label{sec:mdcf-geometry}
% ==============================================================================

% TODO: I'm not sold on the square brackets. With the list notation for the
% continued fraction, they're kind of ambiguous. Maybe we should just switch to
% regular parentheses instead.
In the geometrical interpretation of continued fractions,
each convergent $pₙ/qₙ$ is represented as a two-dimensional vector $(pₙ, qₙ)$.
These vectors approach an irrational line spanned by the vector $(1, α)$
where $α$ is some irrational number.
For the generalization to multidimensional continued fractions,
each convergent, which is a $d$-dimensional rational vector,
as a $(d+1)$-dimensional integer vector.
Specifically, given a convergent $r^{(n)} = (p₁/q₁, …, p_d/q_d) ∈ ℚ^d$,
we first find a common denominator $(p₁'/q, …, p_d'/q)$ and
then we map it to the vector $\hat r = (q, p₁', …, p_d') ∈ ℤ^{d+1}$.
Similarly, if we have a vector $(x₀, x₁, …, x_d) ∈ ℤ^{d+1}$,
then we map it back to $(x₁/x₀, …, x_d/x₀) ∈ ℚ^d$.

In this space, the representation for a particular convergent is not unique,
there can be multiple integer vectors representing the same convergent.
For example, if we have a vector $r ∈ ℤ^{d+1}$ for a convergent,
then we can multiply with some scalar $λ ∈ ℤ$ and get a new vector $r' = λ r$
which represents the same convergent.
This is because the scalar is eliminated when mapping it back to the rational vector:
\[
  λ (c₀, c₁, …, c_d)
  ↦ \left(\frac{λ c₁}{λ c₀}, …, \frac{λ c_d}{λ c_0} \right)
  = \left(\frac{c₁}{c₀}, …, \frac{c_d}{c_0} \right).
\]

We can extend this to any real vector in $ℝ^{d+1}$:
Two nonzero vectors $a, b ∈ ℝ^{d+1}$ are equivalent,
denoted as $a \sim b$, if $a = λ b$ for some scalar $λ ∈ ℝ$.
This relation then defines the equivalence class of an element $a ∈ ℝ^{d+1}$ as
\[
  [a] = \mathrm{span}(a) = \{\, λ a \mid λ ∈ ℝ, λ ≠ 0 \,\}.
\]
Formally, this is known as a real \emph{projective space}, denoted as $\mathbb{RP}^d$.
It is the set of equivalence classes in $ℝ^{d+1} \setminus \{0\}$ defined by the
equivalence relation $\sim$.
An element $x$ of this space is denoted as $[x₀, x₁, …, x_d]$,
where the square brackets indicate that this element is an equivalence
class.
In summary, we have the following mappings from $ℝ^d$ to $\mathbb{RP}^d$
and vice-versa:

\begin{center}
  \begin{tikzpicture}
    \matrix[
      column sep=2cm,
      nodes={text width=3cm, align=center},
    ] {
      \node (L0) {$\mathbb{R}^d$}; &
      \node (R0) {$\mathbb{RP}^d$}; \\
      \node (L1) {$(x₁, …, x_d)$}; &
      \node (R1) {$[1, x₁, …, x_d]$}; \\
      \node (L2) {$(x₁/x₀, …, x_d/x₀)$}; &
      \node (R2) {$[x₀, x₁, …, x_d]$}; \\
    };

    \draw[->] (L1) -- node[above] {} (R1);
    \draw[<-] (L2) -- node[above] {} (R2);
  \end{tikzpicture}
\end{center}

\begin{figure}[tbp]
  \centering
  \includestandalone{figures/projective-space}
  \caption{
    The convergents as vectors in a $d$-dimensional projective space.
    The ordinary convergents are projections at the $x₀ = 1$ plane.
  }
  \label{fig:projective-space}
\end{figure}

Next, we consider the pivot operation in a projective space.
Before we had to differentiate to cases to define the pivot operation,
now the pivot operation is just a linear operation on the projective coordinates.
For example, consider the vector $[1, x₁, x₂]$ and suppose that $0 ≤ x₁, x₂ < 1$.
A pivot operation with $ℓ = 1$ would result in the vector $[1, 1/x₁, x₂/x₁]$.
This vector is equivalent to $[x₁, 1, x₂]$.
Therefore, we can reformulate this operation as a coordinate swap of $x_ℓ$ with
the new coordinate $x₀$:
\[
  \begin{bmatrix}
    0 & 1 & 0 \\
    1 & 0 & 0 \\
    0 & 0 & 1 \\
  \end{bmatrix}
  ·
  \begin{bmatrix} 1 \\ x₁ \\ x₂ \\ \end{bmatrix}
  =
  \begin{bmatrix} x₁ \\ 1 \\ x₂ \\ \end{bmatrix}
  =
  \begin{bmatrix} 1 \\ 1/x₁ \\ x₂/x₂ \\ \end{bmatrix}.
\]
If $x₁$ and $x₂$ have a nonzero integer part,
then we first have to subtract this part from the vector $x$.
In the projective space, this is equivalent to a series of skew operations:
\[
  \begin{bmatrix}
    1 & 0 & 0 \\
    -\floor{x₁} & 1 & 0 \\
    0 & 0 & 1 \\
  \end{bmatrix}
  ·
  \begin{bmatrix}
    1 & 0 & 0 \\
    0 & 1 & 0 \\
    -\floor{x₂} & 0 & 1 \\
  \end{bmatrix}
  ·
  \begin{bmatrix} 1 \\ x₁ \\ x₂ \\ \end{bmatrix}
  =
  \begin{bmatrix} 1 \\ x₁ - \floor{x₁} \\ x₂ - \floor{x₂} \\ \end{bmatrix}.
\]
In general,
let $S(a)$ denote the skew matrix by a vector $a ∈ ℝ^d$,
i.e. the identity matrix with zeros in the first column swapped with $a$,
and let $R(ℓ)$ denote the permutation matrix,
which swaps $x_ℓ$ with $x_0$.
Then, we can define the pivot operation as
\[
  \mathrm{pivot}_ℓ(x) = R(ℓ) S(-\floor{x}) \hat x,
\]
where $\hat x = [1, x₁, …, x_d]$.

Importantly, we can invert each matrix.
For the matrix $S(a)$, we simply skew in the opposite direction
and the matrix $R(ℓ)$ is its own inverse, since swapping the same coordinate
twice yields the same vector.
Therefore, the whole operation can be easily reversed by inverting the matrix.
This is the equivalent of the inverse pivot operation in the projective space $\mathbb{RP}^d$.
Thus, we can reformulate MCFs as a series of matrix multiplications.
The projective definition of an MCF can be written as
\[
  [a^{(0)}] = \hat a^{(0)}, \qquad
  [a^{(0)}; a^{(1)}, …, a^{(n)}] = S(a₀) · R(ℓ) · [a^{(1)}; a^{(2)}, …, a^{(n)}],
\]
where $ℓ$ is the pivot index of $[a^{(1)}; a^{(2)}, …, a^{(n)}]$
and $\hat a^{(0)} = [1, a_1^{(0)}, …, a_d^{(0)}]$.

We can also use the projective space to dramatically simplify Lemma~\ref{lem:mcf-wallis}.
Instead of two different sequences $P_i^{(n)}$ and $Q_i^{(n)}$, we simplify it
to a single matrix sequence $(B^{(n)})_{n ≥ 0}$, which we call the
\emph{convergent matrix}.
Each $B^{(n)}$ consists of the column vectors $B₀^{(n)}, B₁^{(n)}, …, B_d^{(n)}$.
The sequence begins with $B^{(0)} = I_d$
and the remaining terms are calculated according to the recurrence
\begin{align*}
  B_ℓ^{(n)} = B^{(n-1)} \hat a^{(n)},
  \qquad B_i^{(n)} = B_i^{(n-1)},
  \qquad B_0^{(n)} = B_ℓ^{(n-1)},
\end{align*}
where $ℓ$ is the pivot index and $\hat a^{(n)} = [1, a_1^{(n)}, …, a_d^{(n)}]$.
By construction, the convergent matrix $B^{(n)}$ is the combined matrix of the original sequences
$P_i^{(n)}$ and $Q_i^{(n)}$ defined for Lemma~\vref{lem:mcf-wallis}:
\[
  B^{(n)} = \begin{bmatrix}
    Q_0^{(n)} & Q_1^{(n)} & ⋯ & Q_d^{(n)} \\
    P_0^{(n)} & P_1^{(n)} & ⋯ & P_d^{(n)} \\
  \end{bmatrix}.
\]
In fact, we can prove an equivalent statement from this lemma using the new sequence.

\begin{lemma}
  \label{lem:mcf-wallis'}
  Let $x ∈ ℝ^d$ and $\hat x = (1, x₁, …, x_d)$, then
  \[
    [r^{(0)}; r^{(1)}, …, r^{(n-1)}, x] \sim B^{(n-1)} \hat x.
  \]
\end{lemma}

% TODO
\begin{proof}
  By construction of $B^{(n-1)}$, we have
  \begin{align*}
    B^{(n-1)} \hat x
    & = B_0^{(n-1)} \hat x_0 + B_1^{(n-1)} \hat x_1 + ⋯ + B_d^{(n-1)} x_d \\
    & =
    \begin{bmatrix}
      P_0^{(n-1)} \\
      Q_0^{(n-1)} \\
    \end{bmatrix} \hat x_0
    + \begin{bmatrix}
      P_1^{(n-1)} \\
      Q_1^{(n-1)} \\
    \end{bmatrix} \hat x_1
    + ⋯ + \begin{bmatrix}
      P_d^{(n-1)} \\
      Q_d^{(n-1)} \\
    \end{bmatrix} \hat x_d \\
    & = \begin{bmatrix}
      P_0^{(n-1)} \hat x_0 + P_1^{(n-1)} \hat x_1 + ⋯ + P_d^{(n-1)} \hat x_d \\
      Q_0^{(n-1)} \hat x_0 + Q_1^{(n-1)} \hat x_1 + ⋯ + Q_d^{(n-1)} \hat x_d \\
    \end{bmatrix}.
  \end{align*}
  Projecting this vector back to $ℚ^d$ results exactly in the vector
  \[
    r^{(n)} = \frac{P_0^{(n-1)} + P_1^{(n-1)} x_1 + ⋯ + P_d^{(n-1)} x_d}{Q_0^{(n-1)} + Q_1^{(n-1)} x_1 + ⋯ + Q_d^{(n-1)} x_d}
  \]
  from
  Lemma~\ref{lem:mcf-wallis}.
\end{proof}

\iffalse
% TODO: Klein polyhedra?
Last but not least,
there also exists a generalization of Klein polygons to higher dimensions.
For three dimensions, they are known as Klein polyhedra
and in general they are known as Klein polytopes.

\begin{definition}
  Let $B = \{b₁, …, b_d\} ⊆ ℝ^d$ be a basis and let $C = \{\, λ₁ b₁ + ⋯ + λ_d b_d \mid λ_i ≥ 0 \,\}$.
  The \emph{Klein polytope} $K$ generated by $B$ is defined as
  \[
    K = \mathrm{conv}(C ∩ ℤ^d \setminus \{\symbf 0\}).
  \]
\end{definition}

The connection between Klein polytopes and the convergents is not clear to me, however.
As before, we can show that the area between the convergents is empty.
The idea for a Klein polyhedra is visualized in Figure~\ref{fig:klein-polytope}.
This time, we consider the parallelepiped between the secondary convergents of the
current iteration and the previous iteration.
The number of integer points inside this parallelepiped can be calculated using
the determinant between the convergents.
Although the parallelepiped is not empty,
all of its integer points must be its the boundary.
Therefore, the volume between the convergents is empty.

Similarly,
Equation~\ref{eq:??} already shows that the line lies inside the convex hull of
the secondary convergents
and the convergence shows that the area of the convergents decreases towards zero.
\fi

% ==============================================================================
\section{Algebraic Numbers and Periodicity}
\label{sec:mcf-periodic}
% ==============================================================================

The proof for periodic MCFs is based on the same theorem for the Jacobi-Perron
algorithm, originally proven by Perron \cite{Perron07}.
We begin with the purely periodic case.
The idea behind this proof is that in a purely periodic MCF of a vector $x ∈ ℝ^d$,
the vector itself is an eigenvector for one of the convergent matrices $B^{(k)}$.
Furthermore, the elements of this eigenvector can only be algebraic numbers with degree $≤ d+1$.

\begin{lemma}
  \label{lem:mcf-purely-periodic}
  If there exists a purely periodic MCF for $x ∈ ℝ^d$,
  then $[ℚ(x₁, …, x_d) : ℚ] ≤ d+1$.
\end{lemma}

% TODO: Should we use x ≡ y or [x] = [y]?
\begin{proof}
  If the MCF is purely periodic, then there is some index $n ≥ 1$ such that $x = x^{(n)}$.
  Let $\hat x = [1, x₁, …, x_d]$ and $\hat x^{(n)} = [1, x_1^{(n)}, …, x_d^{(n)}]$.
  By Lemma~\ref{lem:mcf-wallis'},
  \[
    \hat x \sim B^{(n)} \hat x^{(n)} \sim B^{(n)} \hat x \iff λ \hat x = B^{(n)} \hat x,
  \]
  for some nonzero $λ ∈ ℝ$.
  Therefore, we are looking for an eigenvector $\hat x$ and an eigenvalue $λ$ of $B^{(n)}$.
  The characteristic polynomial $\det(B^{(n)} - λ I)$ can have a degree of at most $d+1$,
  therefore the eigenvalue $λ$ is an algebraic number of degree $d+1$.
  For the eigenvector $\hat x$, we have to find a nontrivial solution to the
  homogeneous linear system
  \[
    (B^{(n)} - λ I) \hat x = 0.
  \]
  Each coefficient in this linear system is either an integer or $λ$ and is
  therefore contained in the field $ℚ(λ)$.
  Hence, we have $[ℚ(\hat x_0, \hat x_1, …, \hat x_d) : ℚ] ≤ d+1$.

  Finally, the eigenvector $\hat x$ has to be projected back from homogeneous coordinates $x$.
  Since $xᵢ = \hat xᵢ / \hat x₀$ is a rational expression and the values $\hat xᵢ$ and $\hat x₀$ are members of the same field $ℚ(λ)$,
  the projected value $xᵢ$ must also be contained in the same field.
  Therefore, each element in $x$ is an algebraic number
  and we have $[ℚ(x₁, …, x_d) : ℚ] ≤ d+1$.
\end{proof}

\begin{theorem}
  \label{thm:mdcf-periodic}
  If there exists a periodic MCF for $x ∈ ℝ^d$,
  then $[ℚ(x₁, …, x_d) : ℚ] ≤ d + 1$.
\end{theorem}

\begin{proof}
  Given such a MCF for $x$, let $x^{(k)}$ denote the $k$-th complete quotient
  of this fraction.
  Suppose that the MCF is periodic after $K ≥ 0$ with period $ℓ ≥ 0$, i.e.
  $x^{(k)} = x^{(k+ℓ)}$ for every $k ≥ K$.
  By Lemma~\ref{lem:mcf-wallis'}, $\hat x \sim B^{(k)} \hat x^{(k)}$,
  which means that very element in the projection $x$ can be represented
  as a rational expression of $x^{(k)}$:
  \[
    x_i = \frac{∑_{j=1}^d B_{ij}^{(k)} x_j^{(k)} + B_{i0}^{(k)}}{\sum_{j=1}^d B_{0j}^{(k)} x_j^{(k)} + B_{00}^{(k)}}.
  \]
  % TODO: I think we should be more precise here since the elements could be
  % inside different fields. They are not, but this sentence does not indicate
  % that.
  From Lemma~\ref{lem:mcf-purely-periodic},
  it follows that the elements of $\hat x^{(k)}$
  are contained in a field $ℚ(λ)$ with degree $[ℚ(λ) : ℚ] ≤ d+1$.
  Since $B^{(k)}$ consists solely of integers, every element in $x$ is contained in the same field $ℚ(λ)$.
  Therefore, they must also be algebraic numbers
  and we have $[ℚ(x₁, …, x_d) : ℚ] ≤ d+1$.
\end{proof}

This section has shown that a purely periodic multidimensional continued fraction must represent a vector of algebraic numbers.
More precisely, the components of the vector lie in a number field of degree at
most $d + 1$, where d is the dimension of the continued fraction.
This result establishes one direction of Hermite's question in higher dimensions.
The converse direction, however, remains open.

Although Theorem~\ref{thm:unimodular-algebraic} shows that for any algebraic
number $α$ there exists a unimodular matrix $U$ with eigenvector $(1, α, …, α^d)$,
it remains unclear whether the existence of this matrix implies a periodic MCF.
For two dimensions, we were able to show that the convergents and the vertices of a Klein polygon are equivalent.
Since $U$ preserves the Klein polygon, it implies that the continued fraction is periodic.
In higher dimensions the Klein polygon generalises to a Klein polytope.
While there exists a multidimensional analogue of Lagrange's theorem \cite{German08},
the connection between the convergents and the vertices of Klein polytopes is not known yet.
