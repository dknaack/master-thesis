\chapter{Higher-Order Fibonacci Sequences and their Golden Ratios}
\label{ch:fibonacci}

It is well known that the worst-case scenario for the classical Euclidean
algorithm involves consecutive Fibonacci numbers.
Naturally, we may ask whether an equivalent concept exists for the generalized
Euclidean algorithm.

Furthermore, the Euclidean algorithm is used to construct continued fractions;
when applied to Fibonacci numbers, it produces one of the simplest examples:
the golden ratio.
For higher-order Fibonacci numbers, we can similarly construct multidimensional
continued fractions, which may be considered generalizations of the golden
ratio.
Some of these are well known -- such as the supergolden ratio and the plastic
ratio -- both of which are roots of cubic polynomials.
These examples bring us a step closer to addressing Hermite’s question, as they
represent some of the simplest periodic multidimensional continued fractions.

While the classical algorithm has a single, well-defined worst-case input, the
generalized algorithm has multiple worst-case inputs depending on the pivot
strategy.
In this chapter, we consider two such strategies.
We begin with the locally optimal strategy, which selects the minimum
fractional value at each iteration.
In the following section, we refine this approach by minimizing the ratios
between the values.
Each of these strategies corresponds to a specific “golden ratio” for which the
algorithm becomes periodic.
They also have associated linear recurrences that can be seen as
generalizations of the Fibonacci sequence.
Ultimately, we derive a general definition of higher-order Fibonacci sequences
and demonstrate that the generalized Euclidean algorithm exhibits periodic
behavior when applied to their respective golden ratios.

% ==============================================================================
\section{The Classical Case: The Fibonacci Sequence and the Golden Ratio}
% ==============================================================================

We revisit the classical case of Fibonacci numbers and the golden ratio
to understand their relationship with the Euclidean algorithm.
Fibonacci numbers form a recursive sequence:
starting with the initial conditions $F(0) = 1$ and $F(1) = 1$,
each subsequent term is given by the formula
\[
  F(n) = F(n-1) + F(n-2).
\]
The first terms of this sequence are $1, 1, 2, 3, 5, 8, 13, 21, …$

Their relationship with the Euclidean algorithm becomes apparent when analyzing
the algorithm's runtime.
In fact, this analysis represents one of the earliest practical application of
the Fibonacci numbers.
Lamé \cite{Lame44} used the Fibonacci numbers to show that the number of
iterations in the Euclidean algorithm is bounded by the number of digits in the
input.
The proof is regarded by some as the beginning of computational
complexity theory.

\begin{lemma}
  \label{thm:lame}
  If the Euclidean algorithm requires at least $n$ steps for $(a, b) ∈ ℤ_{> 0}^2$ with $a < b$,
  then $a ≥ F(n+2)$ and $b ≥ F(n+1)$.
\end{lemma}

\begin{proof}
  For $n = 0$, we have $F(2) = 2$ and $F(1) = 1$
  and they are the smallest pair of positive integers which require one step of
  the Euclidean algorithm.
  In fact, they are the smallest pair satisfying $a < b$.
  So any pair, which takes only one step, must satisfy $a ≥ F(2)$ and $b ≥ F(1)$.

  Now, suppose that the statement holds for some $n ≥ 0$.
  Let $(a, b)$ be an input pair which requires $n+1$ steps.
  Then, $a = qb + r$ for some integers $q, r ≥ 1$.
  Since the pair $(b, r)$ requires $n$ steps, by the induction hypothesis,
  we have $b ≥ F(n+2)$ and $r ≥ F(n+1)$.
  Therefore
  \[
    a = qb + r ≥ F(n+2) + F(n+1) = F(n+3).
  \]
  Hence, the lemma also holds for $n+1$.
\end{proof}

The lemma can also be derived using the continued fraction for the ratios
between Fibonacci numbers $F(n+1)/F(n)$.
Given a finite continued fraction $[a₀; a₁, …, a_n]$, the Euclidean algorithm
requires $n$ steps for any input $(a, b)$ with the same ratio $a/b$ as the
continued fraction.
Since $[1; 1, …, 1] = F(n+1)/F(n)$ for some $n ≥ 0$, the Euclidean algorithm
takes at least $n$ steps for two consecutive Fibonacci numbers.
Furthermore, Fibonacci numbers are the smallest numbers which require $n$ steps
since we can only increase the coefficients of the continued fraction.

The continued fractions also highlights the connection between the Fibonacci
numbers and the golden ratio.
The convergents of the infinite continued fraction $[1; \overline{1}]$ are the ratios
between consecutive Fibonacci numbers $F(n+1)/F(n)$.
As $n → ∞$, these convergents approach the golden ratio $φ$.

\iffalse
\begin{example}
  Consider $a = 13$ and $b = 8$.
  The algorithm proceeds as follows:
  \[
    \begin{array}{rclcrcl}
      13/8 & = & 1 & · & 5/8 & + & 3/8 \\
       5/8 & = & 1 & · & 3/8 & + & 2/8 \\
       3/8 & = & 1 & · & 2/8 & + & 1/8 \\
       2/8 & = & 2 & · & 1/8 & + & 0.
    \end{array}
  \]
\end{example}
\fi

\begin{lemma}
  The ratios $\frac{F(n+1)}{F(n)}$ converge towards the golden ratio $φ$ as $n$ increases.
\end{lemma}

\begin{proof}
  First, we show that the ratios converge by relating them back to the
  continued fraction $[1; \overline{1}]$.
  By Lemma~\ref{lem:cf-wallis}, the convergents of the continued fraction are
  given by
  \[
    p_n = a_n p_{n-1} + p_{n-2}, \quad
    q_n = a_n q_{n-1} + q_{n-2}.
  \]
  Since the continued fraction consists solely of ones,
  the sequences $p_n$ and $F(n)$ are identical.
  The same is true for $q_n$, except that $q_{-1} = 0$ and $q_{-2} = 1$,
  so it lags one step behind $p_n$.
  Therefore, the convergents are precisely $F(n+1)/F(n)$.

  The convergence is already given by Lemma~\vref{lem:cf-approx},
  so it remains to be shown that the limit is in fact the golden ratio $φ$.
  Suppose the sequence converges to some different limit $φ'$
  \[
    φ' = \lim_{n → ∞} \frac{F(n+1)}{F(n)} = \lim_{n → ∞} 1 + \frac{F(n-1)}{F(n)} = 1 + \frac{1}{φ'}.
  \]
  Multiplying both sides by $φ'$ results in the defining polynomial of the golden ratio $φ$.
  Since the ratios $F(n+1)/F(n)$ are positive for all $n ≥ 0$, it follows that $φ' = φ$.
\end{proof}

\begin{corollary}
  \label{cor:fib-conv}
  The ratios $\frac{F(n+k)}{F(n)}$ converge towards $φ^k$ as $n$ increases.
\end{corollary}

\begin{theorem}
  The Euclidean algorithm requires at most $O(\log b)$ iterations.
\end{theorem}

\begin{proof}
  Suppose the algorithm requires $n$ iterations.
  By Lemma~\ref{thm:lame}, we have $b ≥ F(n+1)$.
  By Corollary~\ref{cor:fib-conv}, we also know that $F(n+1) ≥ φ^{n-1}$ at some point.
  Taking logarithms gives $\log b \geq (n - 1) \log φ$, so
  \[
    n \leq \frac{\log b}{\log φ} + 1 = O(\log b).
  \]
  Thus, the Euclidean algorithm requires at most $O(\log b)$ iterations.
\end{proof}

In summary,
the defining property of Fibonacci numbers with respect to the Euclidean
algorithm is that they represent the worst-case input.
At each step, they yield quotient of $1$, so they decrease slower than any
other pair of integers.
Additionally, their ratios approximate the golden ratio,
which can be used to bound the number of iterations.

% ==============================================================================
\section{A Higher-Order Fibonacci Sequence for the Minimum Strategy}
% ==============================================================================

% TODO: We should also discuss the thing about idempotence under the choice of the pivot.
% TODO: I would like a better explanation of what the goal of this section is

For the generalisation of Fibonacci numbers,
we must find a sequence that similarly yields a quotient of $1$ at each
iteration.
However, the problem is that no single definitive sequence can exist.
In the generalized Euclidean algorithm, we have an additional degree of freedom
in the choice of the pivot element.
As a result, multiple different notions of Fibonacci numbers can arise,
each corresponding to a different pivot strategy.

We begin by deriving the Fibonacci-like sequence for one particular strategy.
Recall that the determinant decreases by $\{x_ℓ\}$ in each iteration.
A natural first strategy is to choose the index $ℓ$ with the
smallest fractional value.
After all, this maximizes the decrease in the determinant over one iteration.
Locally, it yields the greatest reduction that we can achieve.
Therefore, we begin our analysis with this strategy.

% TODO: We should move this table
% TODO: Maybe it's better to show the numbers in the ratios, since we explain
% that they approach some golden ratio anyways
\begin{table}[tbp]
  \caption{The first 10 $d$-bonacci numbers for $d = 1, …, 5$ and their golden ratios.}
  \label{tbl:min-fibonacci}
  \centering
  \begin{tabular}{cc|cccccccccc}
\uzlhline
\uzlemph{$d$} & \uzlemph{$φ$} & \uzlemph{$F(0)$} & \uzlemph{$F(1)$} & \uzlemph{$F(2)$} & \uzlemph{$F(3)$} & \uzlemph{$F(4)$} & \uzlemph{$F(5)$} & \uzlemph{$F(6)$} & \uzlemph{$F(7)$} & \uzlemph{$F(8)$} & \uzlemph{$F(9)$} \\
\hline
$1$ & $1.61803$ & 1 & 2 & 3 & 5 & 8 & 13 & 21 & 34 & 55 & 89 \\
$2$ & $1.83929$ & 1 & 1 & 3 & 5 & 9 & 17 & 31 & 57 & 105 & 193 \\
$3$ & $1.92756$ & 1 & 1 & 1 & 4 & 7 & 13 & 25 & 49 & 94 & 181 \\
$4$ & $1.96595$ & 1 & 1 & 1 & 1 & 5 & 9 & 17 & 33 & 65 & 129 \\
$5$ & $1.98358$ & 1 & 1 & 1 & 1 & 1 & 6 & 11 & 21 & 41 & 81 \\
$6$ & $1.99198$ & 1 & 1 & 1 & 1 & 1 & 1 & 7 & 13 & 25 & 49 \\
\uzlhline
\end{tabular}

\end{table}

\begin{definition}
  The Fibonacci sequence for the minimum strategy is defined as
  \begin{enumerate}
    \item $F(0) = F(1) = ⋯ = F(d) = 1$.
    \item $F(n + 1) = F(n) + F(n - 1) + ⋯ + F(n - d)$.
  \end{enumerate}
\end{definition}

For $d = 2, 3$ and $4$, these sequqences are also referred to as the
Tribonacci, Tetranacci and Pentanacci numbers, respectively.
The first few terms are listed in Table~\ref{tbl:min-fibonacci}.
In general, they are known as the $d$-bonacci numbers
and were first explicitly studied by Miles~\cite{Miles60}.
Beyond the generalized Euclidean algorithm,
they have also been successfully applied in probability theory~\cite{Philippou82,Raab63,Schlegel94}.

For these numbers, we initialize the solution vector $x = (x₁, …, x_d)$ with
\[
  x_i = \frac{\sum_{k=0}^i F(n - i)}{F(n)}.
\]
Since
\[
  F(n) = F(n - 1) + ⋯ + F(n - d - 1) > F(n - 1) + ⋯ + F(n - i),
\]
each $x_i$ is bounded between $1$ and $2$.
Thus, each $x_i$ has a fractional value of $1$.
The smallest one is $x₁$, so we pivot with $ℓ = 1$ first.

Let $x' = \mathrm{pivot}_ℓ(x)$, then the first value is
\begin{align*}
  x₁' &
  = \frac{1}{\{x₁\}}
  = \frac{1}{x₁ - 1}
  = \frac{F(n)}{F(n - 1)}
  = \frac{F(n - 1) + ⋯ + F(n - d)}{F(n)}.
\end{align*}
This is equal to the value of $x_d$, if we had started with $F(n-1)$ instead of $F(n)$.

The other components of the input vector $x$ are updated as follows:
\begin{align*}
  xᵢ'
  & = \frac{\{xᵢ\}}{\{x₁\}} = \frac{xᵢ - 1}{x₁ - 1} \\
  & = \frac{F(n - 1) + ⋯ + F(n - i)}{F(n + 1)} · \frac{F(n)}{F(n - 1)} \\
  & = \frac{F(n - 1) + ⋯ + F(n - i)}{F(n - 1)}.
\end{align*}
Thus, in the next iteration, the value $x_i'$ has the same value as $x_{i-1}$,
if we had started with $F(n - 1)$.
Therefore, in the following step, the smallest fractional value must be $x_2$.
Repeating this process a total of $d$ times brings us back to the initial
configuration, except each term is shifted back by $d$ steps.
To reach the end, we need a total $n$ steps.

For the worst-case analysis, we begin with the following lemma, which shows
that the $\mathrm{pivot}$ operation from Equation~\ref{eq:modified-update-rule}
can be interpreted as a multi-dimensional division-with-remainder operation.
In this interpretation, we have a set of integers $p₁, …, p_d, q$ and divide
them by one particular element $p_ℓ$.

\begin{lemma}
  \label{lem:divmod}
  Let $x = \left(\frac{p₁}{q}, \frac{p₂}{q}, …, \frac{p_d}{q}\right)$
  and $x' = \left(\frac{p₁'}{q'}, \frac{p₂'}{q'}, …, \frac{p_d'}{q'}\right)$
  with $x' = \mathrm{pivot}_ℓ(x)$.
  Suppose $\gcd(q, p₁, …, p_d) = 1$ and $\gcd(q', p₁', …, p_d') = 1$.
  Then,
  \[
    q = p_ℓ', \qquad
    p_ℓ = a_ℓ p_ℓ' + q', \qquad
    p_i = a_i p_ℓ' + p_i',
  \]
  for every $i ∈ \{1, …, d\}$, where $a = \floor{x}$.
\end{lemma}

\begin{proof}
  By definition, we have
  \[
    x_ℓ'
    = \frac{p_ℓ'}{q'}
    = \frac{1}{\frac{p_ℓ}{q} - a_ℓ}
    = \frac{q}{p_ℓ - a_ℓ q}.
  \]
  Therefore, $q = p_ℓ'$ and $p_ℓ = a_ℓ q + q' = a_ℓ p_ℓ' + q'$.
  For the remaining elements, we have
  \[
    \frac{p_i'}{q'}
    = \frac{\frac{p_i}{q} - a_i}{\frac{p_ℓ}{q} - a_ℓ}
    = \frac{p_i - a_i q}{p_ℓ - a_ℓ q}.
  \]
  Therefore, $p_i = p_i' + a_i q = p_i' + a_i p_ℓ'$.
\end{proof}

To show that the $d$-bonacci numbers are the worst-case for the minimum
strategy, we first need to establish two conditions.
The first condition is that $x$ is strictly positive.
This can be ensured by simply ignoring the first iteration,
as in the second iteration, $x'$ will always be positive.

The second condition is that the values in $x$ must be distinct.
The reasoning behind this requirement is that one could easily check if there
are two elements $x_i, x_j$ with $x_i = x_j$.
Pivoting with either $ℓ = i$ or $ℓ = j$ removes the other element from the
vector, since the element will be zero in the next iteration.
Hence, this step will always be more efficient than pivoting a different element.

\begin{theorem}
  \label{thm:dbonacci}
  Let $x = (p₁/q, …, p_d/q) ∈ ℚ^d$.
  Suppose the algorithm requires at least $n ≥ 0$ steps for the vector $x$ when
  selecting the minimum fractional value at each step.
  Then, there exists a permutation~$σ$ such that
  \[
    q ≥ F(n+d), \qquad
    p_{σ(i)} ≥ \sum_{k = 0}^i F(n+d - k),
    \quad \text{ for every } i ≤ d.
  \]
\end{theorem}

\begin{proof}
  We proceed by induction on $n$.
  If we require $n = 0$ steps, then $x$ must already be integral, so
  each coordinate has no fractional part.
  The smallest pair of distinct integers is determined by the Fibonacci
  numbers.
  For all $i ≤ d$, the partial sum is
  \[
    \sum_{k = 0}^i F(d - k) = i + 1,
  \]
  because $F(k) = 1$ for $1 ≤ k ≤ d$.
  Furthermore,
  \[
    F(d+1) = F(0) + F(1) + ⋯ + F(d) = d+1.
  \]
  So setting $q = 1$ and ordering the $p_i$ from $2$ to $d+1$ gives us a
  permutation $σ$ such that the claim holds for any integral vector.

  Next, suppose that the claim holds for $n$ steps.
  For simplicity, let $n' = n + d$.
  Suppose that $x = (p₁/q, …, p_d/q)$ requires $n+1$ steps.
  Then, $x' = \mathrm{pivot}_ℓ(x)$ requires $n$ steps and by induction,
  there exists a permutation $σ$ such that
  \[
    q' ≥ F(n'), \qquad
    p_{σ(i)}' ≥ \sum_{k = 0}^i F(n' - k).
  \]
  In particular, $p_{σ(d)} = F(n' + 1)$.

  Assume without loss of generality that $σ(d)$ is the index of the largest
  element in $x'$.
  Then, we have to pivot with $ℓ = σ(d)$.
  Let $a = \floor{x}$, then $a_ℓ = \floor{1/\{x_ℓ\}} ≥ 1$.
  By Lemma~\ref{lem:divmod},
  \[
    {
      \everymath={\displaystyle}
      \renewcommand\arraystretch{2.1}
      \vspace{-1em}
      \begin{array}{r@{\;}c@{\;}l@{}l@{\;}c@{\;}l@{\;}c@{\,}c@{\,}c@{\;}l}
        q
          &=& &p_{σ(d)}'
          &≥& F(n' + 1), \\
        p_{σ(d)}
          &=& a_{σ(d)} &p_{σ(d)}' + q'
          &≥& F(n' + 1) &+& F(n')
          &=& \sum_{k = 0}^1 F(n' + 1 - k), \\
        p_{σ(i)}
          &=& a_{σ(d)} &p_{σ(d)}' + p_{σ(i)}'
          &≥& F(n' + 1) &+& \sum_{k = 0}^i F(n' - k)
          &=& \sum_{k = 0}^{i + 1} F(n' + 1 - k).
      \end{array}
    }
  \]

  To conclude, define a new permutation $σ'$ by letting $σ'(1) = σ(d)$ and
  $σ'(i) = σ(i - 1)$ for $2 ≤ i ≤ d$.
  Then the bound holds for each $i$ under $σ'$.
\end{proof}

For these Fibonacci numbers,
we can ask whether they also lead to a generalization of the golden ratio.
We can use this generalization to bound the runtime again.
The classical Fibonacci numbers approach the golden ratio, when looking at their ratios.
Therefore, we look at the ratios of consecutive $d$-bonacci numbers.
This limit can be considered a multi-dimensional generalization of the golden ratio.
For now, we will assume that these ratios converge.
A general proof of convergence will be shown in Section~\ref{sec:fib-conv}, at the end of the chapter.

\begin{theorem}
  If the ratios $\frac{F(n+1)}{F(n)}$ converge, then they approach $φ_d$,
  where $φ_d$ is the positive real root of the polynomial $p(x) = x^{d+1} - x^d - ⋯ - 1$.
\end{theorem}

\begin{proof}
  By assumption, the ratios are approach some limit $r$.
  Then,
  \[
    r
    = \lim_{n → ∞} \frac{F(n+1)}{F(n)}
    = \lim_{n → ∞} \left(1 + \frac{F(n-1)}{F(n)} + \frac{F(n-2)}{F(n)} + ⋯ + \frac{F(n-d)}{F(n)}\right).
  \]
  Each ratio in the sum can be rewritten as a product of ratios with
  consecutive Fibonacci numbers, since
  \[
    \frac{F(n - i)}{F(n)}
    = \frac{F(n - i)}{F(n - i + 1)} · \frac{F(n - i + 1)}{F(n - i + 2)} ⋯ \frac{F(n - 1)}{F(n)}.
  \]
  By assumption, each ratio approaches $r^{-1}$ as $n$ increases towards infinity.
  Therefore,
  \[
    r = 1 + \frac{1}{r} + \frac{1}{r^2} + ⋯ + \frac{1}{r^d},
  \]
  which is equivalent to $r^{d+1} - r^d - ⋯ - r - 1 = 0$.
  Furthermore, the ratios are always positive.
  Hence, $r = φ_d$.
\end{proof}

\begin{corollary}
  \label{cor:dfib-conv}
  The ratios $F(n + i)/F(n)$ converge towards $φ_d^i$ for any $i ≥ 0$.
\end{corollary}

\begin{theorem}
  The generalized Euclidean algorithm requires $O(\log p_d)$ iterations,
  when selecting the minimum fractional value in each step.
\end{theorem}

\begin{proof}
  Suppose the algorithm requires $n$ iterations for an input $x = (p₁/q, …, p_d/q)$.
  By Lemma~\ref{thm:dbonacci}, we have $p_d ≥ F(n+1)$
  and by Corollary~\ref{cor:dfib-conv}, we must have $F(n+1) ≥ φ_d^{n-1}$ at some point.
  Then, $\log p_d ≥ (n - 1) \log φ_d$.
  As $d$ increases, $ψ_d$ approaches $2$, so $ψ_d = Θ(1)$.
  Hence, the algorithm requires $n = O(\log p_d)$ iterations.
\end{proof}

Apart from the runtime, we can additionally find an algebraic solution vector,
which is periodic under the algorithm.
We replace the ratios in the solution vector $x$ with the actual golden ratio $φ_d$.
We now have an algebraic vector $x = (x₁, …, x_d) ∈ ℝ^d$ with
\[
  x_i
  = \lim_{n → ∞} \sum_{k=0}^i \frac{F(n - i)}{F(n)}
  = \sum_{k=0}^i φ_d^k.
\]
Importantly, the algorithm is periodic for this input vector.
We have already seen for the approximate solution with Fibonacci numbers
that $d$ steps shifts each number backwards by $d$.
For the algebraic solution, this means that it stays the same after $d$ steps.

\begin{theorem}
  The generalized Euclidean algorithm is periodic for the real vector $x = (x₁, …, x_d)$
  with $x_i = \sum_{k=0}^i φ_d^k$ when taking the minimum fractional value.
\end{theorem}

\begin{proof}
  The fractional part of each $x_i$ is given by
  \[
    \{x_i\} = x_i - \lfloor x_i \rfloor = \sum_{k = 1}^i φ_d^k.
  \]
  Since all $x_i$ have the same integer part $1$, the smallest fractional value
  is $\{x_1\} = φ_d$, so we pivot with $ℓ = 1$.

  Applying the pivot rule, the next vector $x'$ has entries:
  \begin{align*}
    x_1' &= \frac{1}{\{x_1\}} = \frac{1}{φ_d} = \sum_{k = 0}^d φ_d^k = x_d, \\
    x_i' &= \frac{\{x_i\}}{\{x_1\}} = \frac{\sum_{k = 1}^i φ_d^k}{φ_d}
          = \sum_{k = 0}^{i - 1} φ_d^k = x_{i - 1} \quad \text{for } i > 1.
  \end{align*}
  So the vector is cyclically shifted:
  \[
    x' = (x_d, x_1, x_2, \dots, x_{d-1}).
  \]
  After $d$ iterations, the vector returns to its original state.
  Hence, the algorithm is periodic with period $d$.
\end{proof}

% ==============================================================================
\section{Improving the Minimum Strategy over Two Iterations}
% ==============================================================================

We can achieve a significantly faster decrease if we minimize over two
iterations instead of just one.
To this end, we assume that the values are ordered in increasing order by their
fractional value, i.e. $\{x₁\} ≤ \{x₂\} ≤ ⋯ ≤ \{x_d\}$.
In the first iteration, we select
\[
  ℓ = \argmin_{i ∈ \{1,…,d\}} \left\{ \frac{\{x_{i+1}\}}{\{x_i\}} \right\},
\]
where we define the boundary cases $x_0 = 1$ and $x_{d+1} = 1$,
so that $ℓ = 1$ and $ℓ = d$ remain valid choices.
In the second iteration, we pivot with $ℓ + 1$ if $ℓ < d$
and otherwise, we only pivot once.

The reasoning behind this strategy is that it minimizes the relative gap
between the two closest fractional parts,
measured by the ratio of their values.
This often leads to a more substantial reduction in size
compared to always choosing the single minimum fractional value.
While the minimum strategy can only be bounded by its golden ratio over the
entire execution, this improved strategy already leads to a new golden ratio
after just two iterations.

\begin{theorem}
  The determinant decreases by at least $ψ_d$ over two iterations,
  where $ψ_d$ is the root of the polynomial $p(x) = x(x+1)^d - 1$.
\end{theorem}

\begin{proof}
  We assume without loss of generality that every value in $x$
  lies between $0$ and $1$, such that $\{x_i\} = x_i$.
  Furthermore, we assume that $\floor{x_ℓ/x_{ℓ-1}} = 1$ with the exception of $ℓ = 1$,
  since any larger value would lead to a faster decrease.
  The decrease in the second iteration is the largest
  when all ratios $\{x_ℓ/x_{ℓ-1}\}$ for $ℓ ∈ \{1, …, d\}$ are equal, i.e.
  \[
    x₁ = \frac{x₂}{x₁} - 1 = ⋯ = \frac{x_d}{x_{d-1}} - 1 = \frac{1}{x_d} - 1.
  \]

  We show via induction that $x_i = x₁ (x₁ + 1)^{i-1}$.
  The case $i = 1$ is trivial and
  the first equation gives us $x₂ = x₁(x₁ + 1)$.
  Thus, we proceed with the induction step.
  Suppose that $x_{i+1} = x₁ (x₁ + 1)^{i-1}$.
  The other equations (except the last) result in
  \[
    \frac{x_{i+1}}{x_i} - 1 = \frac{x_i}{x_{i-1}} - 1 \iff
    x_{i+1}
    = \frac{x_i^2}{x_{i-1}}
    = \frac{(x_1 (x₁ + 1)^{i-1})^2}{x₁ (x₁ + 1)^{i-2}}
    = x₁ (x₁ + 1)^{i+1}
    .
  \]
  Thus, we have $x_i = x₁(x₁ + 1)^{i-1}$.
  Finally, the last equation leads to the polynomial
  \[
    \frac{x_d^2}{x_{d-1}} = 1
    \iff
    x₁ (x₁ + 1)^d - 1 = 0,
  \]
  which is the exact definition of $ψ_d$.
\end{proof}

As before, the proof leads to an algebraic solution vector $x = (x₁, …, x_d)$ defined by $x_i = ψ_d (ψ_d + 1)^{i-1}$
and we can consider, what happens with this vector under the specified strategy.
Although the previous golden ratio $φ_d$ was periodic under the minimum strategy,
the vector $x$ is not periodic under this strategy.
% TODO: Explain why it's not periodic
Nevertheless, we can use a different strategy under which the input becomes periodic.

\begin{theorem}
  The generalized Euclidean algorithm is periodic for the real vector $x = (x₁, …, x_d)$
  with $x₁ = ψ_d + 1$ and $x_i = ψ_d (ψ_d + 1)^i + 1$ when taking the maximum fractional value.
\end{theorem}

\begin{proof}
  Because $ψ_d$ is positive and $ψ_d (ψ_d + 1)^d = 1$,
  the values $x_i$ strictly lie between $0$ and $1$
  and $x_1$ lies between $1$ and $2$.
  The largest fractional value is $x_d$,
  since it has the highest power of $ψ_d + 1$.
  Pivoting with $x_d$ results in the vector $x' = (x₁', …, x_d')$ with
  \[
    x_d' = \frac{1}{ψ_d (ψ_d + 1)^{d-1}} = ψ_d + 1
  \]
  as well as
  \[
    x_1' = \frac{ψ_d + 1 - 1}{ψ_d (ψ_d + 1)^{d-1}} = ψ_d (ψ_d + 1)
    \quad
    \text{ and }
    \quad
    x_i' = \frac{ψ_d (ψ_d + 1)^{i-1}}{ψ_d (ψ_d + 1)^{d-1}} = ψ_d (ψ_d + 1)^i.
  \]
  This follows from the definition of $ψ_d$, because
  \[
    ψ_d(ψ_d + 1)^d = 1 \iff \frac{1}{ψ(ψ_d + 1)^d} = 1 \iff \frac{ψ_d (ψ_d + 1)^i}{ψ_d (ψ_d + 1)^d} = ψ_d (ψ_d + 1)^{i-1}.
  \]
  Note that the fractional values have not changed in $x'$
  except that now $x_{d-1}$ has the largest fraction value.
  Thus, $x$ is periodic under the maximum strategy
  if we repeat this step $d$ times.
\end{proof}

To find the Fibonacci numbers, we have to go backwards.
We already have a definition of a golden ratio $ψ_d$,
so we have to find a linear recurrence which converges to $ψ_d$.
In its current form, we cannot easily find such a linear recurrence.
Instead, we will use the root $θ_d = ψ_d + 1$.
Then,
\[
  ψ_d (ψ_d + 1)^d - 1 = 0
  \iff
  θ_d^{d+1} - θ^d - 1 = 0.
\]
Using this equation, we define the new Fibonacci numbers as
\[
  F(n + d + 1) = F(n + d) + F(n + 1).
\]
We can then show that the ratios of these Fibonacci numbers converge towards the root $θ_d$.

\begin{theorem}
  If the ratios $\frac{F(n+1)}{F(n)}$ converge, then they approach $θ_d$.
\end{theorem}

\begin{proof}
  By assumption, the ratios are approach some limit $r$.
  Then,
  \[
    r
    = \lim_{n → ∞} \frac{F(n+1)}{F(n)}
    = \lim_{n → ∞} \left(1 + \frac{F(n-d)}{F(n)}\right).
  \]
  The ratio $F(n-d)/F(n)$ can be rewritten as a product of ratios with
  consecutive Fibonacci numbers, since
  \[
    \frac{F(n - d)}{F(n)}
    = \frac{F(n - d)}{F(n - d + 1)} · \frac{F(n - d + 1)}{F(n - d + 2)} ⋯ \frac{F(n - 1)}{F(n)}.
  \]
  By assumption, each ratio approaches $r^{-1}$ as $n$ increases towards infinity.
  Therefore,
  \[
    r = 1 + \frac{1}{r^d},
  \]
  which is equivalent to the equation $r^{d+1} - r^d - 1 = 0$.
  Furthermore, the ratios are always positive.
  Hence, $r = ψ_d$.
\end{proof}

\begin{corollary}
  If the ratios $\frac{F(n+1)}{F(n)} - 1$ converge, then they approach $ψ_d$.
\end{corollary}

We can now construct a vector $x$ using these Fibonacci numbers.
Specifically, we choose the vector
\begin{align*}
  x = \left(
    \frac{F(n-1)}{F(n)},
    \frac{F(n-2)}{F(n)},
    …,
    \frac{F(n-(d-1))}{F(n)},
    \frac{F(n-d) + F(n)}{F(n)} \right).
\end{align*}
We pivot with the largest fractional value, which is $x_1$.
The next vector is $x' = (x₁', …, x_d')$,
where the first element is
\begin{align*}
  x_1'
  = \frac{1}{\left\{\frac{F(n-1)}{F(n)}\right\}}
  = \frac{F(n)}{F(n-1)}
  = \frac{F(n-1) + F(n-1-d)}{F(n)},
\end{align*}
the elements with $i ∈ \{1, …, d - 1\}$ are
\begin{align*}
  x_i'
  = \frac{\frac{F(n-i)}{F(n)}}{\frac{F(n-1)}{F(n)}}
  = \frac{F(n-i)}{F(n-1)}
\end{align*}
and the last element is
\begin{align*}
  x_d'
  = \frac{\frac{F(n-d) + F(n)}{F(n)} - 1}{\frac{F(n-1)}{F(n)}}
  = \frac{F(n-d)}{F(n-1)}.
\end{align*}

However, the maximum Fibonacci numbers no longer represent the worst-case, even
under the new strategy.
The issue is that in the equation
\[
  p_ℓ = a_ℓ p_ℓ' + q'
\]
from Lemma~\ref{lem:divmod} we cannot bound $a_ℓ$ from below by $1$,
which would lead to a worst-case bound by these Fibonacci numbers.
This is because only one element of $x$ can be greater than $1$,
while all others must be strictly less than $1$.
But that element does not have to be the pivot element,
it could be another element.
Therefore, we cannot use them as a bound for the worst-case
in the same way as we did before.
Nevertheless, their ratios still lead to the root $θ_d$,
which is periodic when taking the maximum fractional value.

% ==============================================================================
\section{Convergence of General Linear Recurrences}
\label{sec:fib-conv}
% ==============================================================================


We have seen that the ratios of consecutive Fibonacci numbers for both
definition converges towards their own respective golden ratios.
However, this was under the assumption that the ratios converge.
In this section, we will show in this section that this is actually the case.
To show this for both sequences at once, we use a general linear recurrence $F(n)$
with the initial terms $F(0) = ⋯ = F(d) = 1$ and with the remaining terms
defined according to
\[
  F(n + 1) = c_d F(n) + c_{d-1} F(n-1) + ⋯ + c₁ F(n - d + 1) + c₀ F(n - d)
\]
for some nonnegative integer coefficients $c₀, c₁, …, c_d$.
We assume that $c₀$ and $c_d$ are not zero.
For example, the $d$-bonacci sequence has $c_0 = ⋯ = c_d = 1$,
whereas the other sequence has only $c₀ = c_d = 1$ and the remaining terms are zero.
The goal is to show that the ratios
\[
  r_n
  := \frac{F(n+1)}{F(n)}
  = \frac{c_0 F(n - d)}{F(n)} + \frac{c₁ F(n - d + 1)}{F(n)} + ⋯ + \frac{c_{d-1} F(n-1)}{F(n)} + c_d
\]
converge.

The first step is to rewrite the terms of the recurrence such that each ratio
between non-consecutive Fibonacci numbers can be rewritten as a product of
ratios $r_{n-i}$:
\begin{align*}
  \frac{F(n - d + i)}{F(n)}
  & = \frac{F(n - d + i + 1)}{F(n - d + i)} \frac{F(n - d + i + 2)}{F(n - d + i + 1)} \dots \frac{F(n-1)}{F(n)} \\
  & = \frac{1}{r_{n - d + i}} · \frac{1}{r_{n - d + i + 1}} · \dots · \frac{1}{r_{n-1}}.
\end{align*}
Then we can calculate the ratio $r_n$ using the previous ratios $r_{n-1}, r_{n-2}, …, r_{n-d}$ as follows:
\[
  r_n = c_0 + \frac{c_1}{r_{n-1}} + \frac{c_2}{r_{n-1} r_{n-2}} + ⋯ + \frac{c_d}{r_{n-1} r_{n-2} \dots r_{n-d}}.
\]
Using this equation, we can show that the ratios are bounded.

\begin{lemma}
  \label{lem:fib-bounded}
  The ratios $r_n$ are bounded between $1$ and $C := ∑_{k=0}^d c_d$.
\end{lemma}

\begin{proof}
  The first $d - 1$ ratios are all $1$.
  The ratio $r_d$ is equal to
  \[
    \frac{F(d+1)}{F(d)} = \frac{F(0) + F(1) + ⋯ + F(d)}{F(d)} = \frac{c_d + ⋯ + c_0}{1} = C,
  \]
  which clearly satisfies the bounds of this lemma.
  By induction, suppose that the previous ratios $r_{n-1}, r_{n-2}, …, r_{n-d}$
  all satisfy the bound between $1$ and $d+1$.
  From previous consideration, we can reformulate the ratio $r_n$ as follows:
  \[
    r_n = c₀ + \frac{c₁}{r_{n-1}} + \frac{c₂}{r_{n-1} r_{n-2}} + \dots + \frac{c_d}{r_{n-1} r_{n-2} \dots r_{n-d}}.
  \]
  Since $r_{n-i} ≤ C$, we can bound $r_n$ from below by
  \[
    r_n ≥ c₀ + \frac{c₁}{C} + \frac{c₂}{C^2} + \dots + \frac{c_d}{C^d} ≥ 1
  \]
  and since $r_{n-i} ≥ 1$, we can bound $r_n$ from above by
  \[
    r_n ≤ c₀ + \frac{c₁}{1} + \frac{c₂}{1} + \dots + \frac{c_d}{1} ≤ C.
  \]
  Hence, $1 ≤ r_n ≤ C$ for every $n ≥ 0$.
\end{proof}

\begin{figure}[tbp]
  \centering
  \includestandalone{figures/fibonacci-convergence}
  \caption{
    Illustration of the convergence proof for ratios of Fibonacci numbers.
    The points represent the ratios $r_n$.
    The sequences $s_n$ and $t_n$ are the minimum and maximum from the previous
    $d$ ratios, respectively.
    Both sequences converge towards the same limit $φ$, so the ratios converge
    towards this limit, too.
  }
  \label{fig:fibonacci-convergence}
\end{figure}

Showing that the sequence is monotone would be enough to show the convergence.
However, the sequence is clearly not monotone.
Even the ratios of the original Fibonacci sequence alternate between increasing
and decreasing.
So we cannot possibly prove that the higher-order sequences are monotone.
Instead, we bound the sequence between two other sequences $s_n$ and $t_n$
and show that the two sequences converge to the same limit.
From the squeeze theorem, it follows that $r_n$ converges to the same limit.
The main idea is illustrated in Figure~\ref{fig:fibonacci-convergence}.
The sequences are
\[
  s_n = \min\{r_n, r_{n-1}, …, r_{n-d} \}, \qquad t_n = \max\{r_n, r_{n-1}, …, r_{n-d}\}
\]
Because $r_n$ clearly lies between the two sequences, it must converge to the same limit as $s_n$ and $t_n$.
For these sequences, we already know from the previous lemma that they are bounded,
so it only remains to be shown that they are monotone,
i.e. $s_n$ is increasing and $t_n$ is decreasing.
After which we can show that they converge to the same limit.

\begin{lemma}
  The sequences $s_n$ and $t_n$ are monotone.
\end{lemma}

\begin{proof}
  Each ratio can be represented as a convex combination of the previous ratios, i.e.
  \[
    r_{n+1} = λ₀ r_n + λ₁ r_{n-1} + \dots + λ_d r_{n-d}
  \]
  using $λ_i = F(n - i) / F(n + 1)$.
  To show that this is indeed a convex combination, all coefficients $λ_i$
  need to be nonnegative and $λ₀ + λ₁ + \dots + \lambda_d = 1$.
  The former follows from the fact that Fibonacci numbers are always increasing,
  while the latter follows simply from the definition of the Fibonacci numbers.
  Because $s_n$ is the minimum and $t_n$ the maximum of $r_n, r_{n-1}, …, r_{n-d}$,
  we can bound the next maximum by
  \[
    t_{n+1} ≤ r_{n+1} = λ₀ r_n + λ₁ r_{n-1} + \dots + λ_d r_{n-d} ≤ λ₀ t_n + λ₁ t_n + ⋯ + λ_d t_n = t_n.
  \]
  and the next minimum by
  \[
    s_{n+1} ≥ r_{n+1} = λ₀ r_n + λ₁ r_{n-1} + \dots + λ_d r_{n-d} ≥ λ₀ s_n + λ₁ s_n + ⋯ + λ_d s_n = s_n.
  \]
  Therefore, $s_n ≤ s_{n+1} ≤ t_{n+1} ≤ t_n$.
\end{proof}

\begin{lemma}
  The sequences $s_n$ and $t_n$ converge to the same limit.
\end{lemma}

\begin{proof}
  For a contradiction, suppose there exists some $δ > 0$ such that $t_n - s_n > δ$ for every $n ≥ 0$.
  Out of the previous ratios, there must be one ratio $r_k$ exactly equal to $t_n$.
  Therefore,
  \begin{align*}
    s_{n+1} ≥ r_n & = λ₀ r_{n-d-1} + λ₁ r_{n-d} + ⋯ + λ_d r_{n-1} \\
                  & ≥ λ_k t_n + \sum_{i ≠ k} λ_i s_n \\
                  & = λ_k t_n + (1 - λ_k) s_n = s_n + λ_k (t_n - s_n) ≥ s_n + λ_k δ.
  \end{align*}
  We have
  \[
    λ_k = \frac{F(n+k)}{F(n+d+1)} = \frac{1}{r_{n+k} r_{n+k+1} \dots r_{n+d+1}} ≥ \frac{1}{(d+1)^{d+1-k}}.
  \]
  Hence, $λ_k$ is always greater than some constant $c > 0$ for every $k ≥ 0$.
  But then
  \[
    s_{n+i} ≥ s_{n+i-1} + c δ ≥ s_{n+i-2} + 2c δ ≥ \dots ≥ s_n + i c δ
  \]
  and $s_{n+i}$ would always increase as $i$ approaches infinity,
  which contradicts Lemma~\ref{lem:fib-bounded}.
  Therefore, $δ = 0$ and it follows that $s_n$ and $t_n$ are approaching the same limit.
\end{proof}

\begin{theorem}
  Let $F$ be a linear recurrence with coefficients $c_0, \dots, c_d ≥ 0$
  and let $φ$ be the real positive root of its characteristic polynomial.
  Then,
  \[
    \lim_{n \to \infty} \frac{F(n + 1)}{F(n)} = φ.
  \]
\end{theorem}

\begin{proof}
  By the previous lemma, the ratios $r_n$ approach some limit $r ∈ ℝ$. It follows:
  \[
    r
    = \lim_{n → ∞} r_n
    = \lim_{n → ∞} 1 + \frac{c_d}{r_{n-1}} + \frac{c_{d-1}}{r_{n-1} r_{n-2}} + ⋯ + \frac{a₀}{r_{n-1} r_{n-2} \dots r_{n-d}}.
  \]
  Hence, each denominator in the sum approaches $r^i$,
  which results in the following polynomial:
  \[
    r = 1 + \sum_{i = 1}^d \frac{c_{d - i}}{r^i}
    \iff
    r^{d+1} = c₀ + c₁ r + \dots + c_d r^d,
  \]
  which directly corresponds to a root of its characteristic polynomial.
  Furthermore, the ratios are always positive, so $r = φ$.
\end{proof}

\begin{corollary}
  The ratios $F(n + i) / F(n)$ converge to $φ^i$ for $i > 1$.
\end{corollary}

This concludes the connection between Fibonacci numbers and their golden ratios.
As previously shown, the ratios of $d$-bonacci numbers approach the root of the polynomial $x^{d+1} - x^d - ⋯ - x - 1$
and the ratios of the other Fibonacci numbers approach the root of the polynomial $x^{d+1} - x^d - 1$.
For both of these roots, the generalized Euclidean algorithm becomes periodic after $d$ iterations.
So they represent the first examples of algebraic numbers, which are periodic
under the generalized Euclidean algorithm.
