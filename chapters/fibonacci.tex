\chapter{Higher-Order Fibonacci Sequences and their Golden Ratios}
\label{ch:fibonacci}

It is well known that the worst-case for the classical Euclidean algorithm
are consecutive Fibonacci numbers.
Naturally, we can ask, whether an equivalent concept exists for the generalized
Euclidean algorithm.
Furthermore, the Euclidean algorithm is used to construct continued fractions.
Constructing a continued fraction using the Fibonacci numbers
produces one of the simplest continued fractions, the golden ratio.
For the higher-order Fibonacci numbers,
we can equally construct a multidimensional continued fraction,
which can be considered a generalization of the golden ratio.
Some of these are well known, like the supergolden ratio or the plastic ratio,
which are both roots of cubic polynomials.
They bring us one step closer to Hermite's question,
since they are some of the simplest examples of periodic multidimensional
continued fractions.

Whereas the classical algorithm has one clear worst-case input,
the generalized algorithm has multiple worst-case inputs
depending on the pivot strategy.
In this chapter, we consider two different strategies.
We begin with the locally optimal strategy,
which chooses the minimum fractional value in each iteration.
In the next section, we improve upon the strategy by minimizing the ratios between the values.
Both of these strategies have a respective golden ratio for which the algorithm
becomes periodic and they each have a corresponding linear recurrence, which
can be seen as a generalization of the Fibonacci numbers.
In the end, we will derive a general definition of higher-order Fibonacci
sequences and show that the generalized Euclidean algorithm is periodic for
their golden ratios.

% ==============================================================================
\section{The Classical Case: The Fibonacci Sequence and the Golden Ratio}
% ==============================================================================

We revisit the classical case of Fibonacci numbers and the golden ratio
to understand their relationship with the Euclidean algorithm.
Fibonacci numbers are a recursive sequence.
Starting with the initial conditions $F(0) = 1$ and $F(1) = 1$,
the remaining terms are calculated using the formula
\[
  F(n) = F(n-1) + F(n-2).
\]
The first terms of this sequence are $1, 1, 2, 3, 5, 8, 13, 21, …$

Their relationship with the Euclidean algorithm becomes apparent when analyzing
the runtime of the algorithm.
In fact, the analysis was actually one of the first practical application of
the Fibonacci numbers.
Lamé \cite{Lame44} used the Fibonacci numbers to show that the number of
iterations in the Euclidean algorithm is bounded by the number of digits in the
input.
The proof is considered by some as the beginning of computational
complexity theory.

\begin{theorem}
  \label{thm:lame}
  If the Euclidean algorithm requires at least $n$ steps for $(a, b) ∈ ℤ_{> 0}^2$ with $a < b$,
  then $a ≥ F(n+2)$ and $b ≥ F(n+1)$.
\end{theorem}

\begin{proof}
  For $n = 0$, we have $F(2) = 2$ and $F(1) = 1$
  and they are the smallest pair of positive integers which require one step of
  the Euclidean algorithm.
  In fact, they are the smallest pair which satisfy $a < b$.
  So any pair, which takes only one step, must satisfy $a ≥ F(2)$ and $b ≥ F(1)$.
  Next, suppose that the theorem holds for some $n ≥ 0$.
  Let $(a, b)$ be an input pair which requires $n+1$ steps.
  Then, $a = qb + r$ for some integers $q, r ≥ 1$.
  For the pair $(b, r)$, the algorithm requires $n$ steps, so $b ≥ F(n+2)$ and
  $r ≥ F(n+1)$ by induction.
  But then
  \[
    a = qb + r ≥ F(n+2) + F(n+1) = F(n+3).
  \]
  Therefore, the theorem also holds for $n+1$.
\end{proof}

The theorem also follows easily using the continued fraction of the ratios
between Fibonacci numbers.
Given a finite continued fraction $[a₀; a₁, …, a_n]$, the Euclidean algorithm
requires $n$ steps for any input $(a, b)$ with the same ratio $a/b$ as the
continued fraction.
Since $[1; 1, …, 1] = F(n+1)/F(n)$ for some $n ≥ 0$, the Euclidean algorithm
takes at least $n$ steps for two consecutive Fibonacci numbers.
Furthermore, Fibonacci numbers are the smallest numbers which require $n$ steps
since we can only increase the coefficients of the continued fraction.

The continued fractions also show the connection between the Fibonacci numbers
and the golden ratio.
The convergents of the continued fraction $[1; \overline{1}]$ are ratios
between consecutive Fibonacci numbers $F(n+1)/F(n)$ and as $n$ approaches
infinity the convergents tend towards the golden ratio $φ$.

\iffalse
\begin{example}
  Consider $a = 13$ and $b = 8$.
  The algorithm proceeds as follows:
  \[
    \begin{array}{rclcrcl}
      13/8 & = & 1 & · & 5/8 & + & 3/8 \\
       5/8 & = & 1 & · & 3/8 & + & 2/8 \\
       3/8 & = & 1 & · & 2/8 & + & 1/8 \\
       2/8 & = & 2 & · & 1/8 & + & 0.
    \end{array}
  \]
\end{example}
\fi

\begin{theorem}
  The ratios $\frac{F(n+1)}{F(n)}$ converge towards the golden ratio $φ$ as $n$ increases.
\end{theorem}

\begin{proof}
  First, we show that the ratios are converging by mapping it to the continued fraction $[1; \overline{1}]$.
  By Lemma~\ref{lem:cf-wallis}, we can calculate the convergents of the continued fraction as
  \[
    p_n = a_n p_{n-1} + p_{n-2}, q_n = a_n q_{n-1} + q_{n-2}.
  \]
  The continued fraction consists solely of ones,
  so the Fibonacci sequence and the sequence $p_n$ are identical.
  The same is true for $q_n$, but $q_{-1} = 0$ and $q_{-2} = 1$, so the sequence lags one step behind $p_n$.
  Therefore, the convergents are actually $F(n+1)/F(n)$.
  The convergence is already given by Lemma~\vref{lem:cf-approx},
  so all that remains to be shown is that they actually converge to the golden ratio $φ$.
  Suppose they converge to some different limit $φ'$
  \[
    φ' = \lim_{n → ∞} \frac{F(n+1)}{F(n)} = \lim_{n → ∞} 1 + \frac{F(n-1)}{F(n)} = 1 + \frac{1}{φ'}.
  \]
  Multiplying this equation by $φ'$ results in the defining polynomial of the golden ratio $φ$.
  Because the ratios are always positive, we must have $φ' = φ$.
\end{proof}

% TODO: Metallic means

In summary,
the defining property of Fibonacci numbers with respect to the Euclidean
algorithm is that they represent the worst-case for the algorithm.
At each step, they have a quotient of $1$, so they decrease slower than any other pair of integers.
Additionally, their ratios approximate the golden ratio,
which follows from the continued fraction.

% ==============================================================================
\section{A Higher-Order Fibonacci Sequence for the Minimum Strategy}
% ==============================================================================

% TODO: There should be a discussion on what it even means to have a Fibonacci number in higher dimensions
% TODO: We should also discuss the thing about idempotence under the choice of the pivot.
% TODO: Should we have a discussion about why we choose this particular linear system.
% TODO: The discussion about the linear system should go in the generalized Euclidean algorithm chapter
% TODO: We should explain that very large values give the least decrease, but also give the most decrease in the following iteration

For the generalisation of Fibonacci numbers,
we must find a sequence which similarly has a quotient of $1$ at each iteration.
However, the problem is that no single definitive sequence can exist.
With the generalized Euclidean algorithm we have an additional choice with the
element we choose to pivot with.
Therefore, multiple different notions of Fibonacci numbers exist for this
algorithm.

% I would like a better explanation of what the goal of this section is
We begin by deriving the Fibonacci numbers for one particular strategy.
Recall that the determinant decreases by $\{x_ℓ\}$ in each iteration.
So the first strategy one may think of is to choose the index $ℓ$ with the
smallest fractional value.
After all, this gives us the largest decrease in the determinant over one iteration.
Locally, this would be the highest decrease we can achieve.
So we begin with this strategy and see how the Fibonacci numbers are defined.

% TODO: Actually explain how we derive the Fibonacci numbers for this strategy.
% TODO: Should we have a table for the Fibonacci numbers?
% TODO: What about adding references to their OEIS number?

% TODO: Maybe it's better to show the numbers in the ratios, since we explain
% that they approach some golden ratio anyways
\begin{table}[tbp]
  \caption{The first 10 $d$-bonacci numbers for $d = 1, …, 5$ and their golden ratios.}
  \label{tbl:min-fibonacci}
  \centering
  \begin{tabular}{cc|cccccccccc}
\uzlhline
\uzlemph{$d$} & \uzlemph{$φ$} & \uzlemph{$F(0)$} & \uzlemph{$F(1)$} & \uzlemph{$F(2)$} & \uzlemph{$F(3)$} & \uzlemph{$F(4)$} & \uzlemph{$F(5)$} & \uzlemph{$F(6)$} & \uzlemph{$F(7)$} & \uzlemph{$F(8)$} & \uzlemph{$F(9)$} \\
\hline
$1$ & $1.61803$ & 1 & 2 & 3 & 5 & 8 & 13 & 21 & 34 & 55 & 89 \\
$2$ & $1.83929$ & 1 & 1 & 3 & 5 & 9 & 17 & 31 & 57 & 105 & 193 \\
$3$ & $1.92756$ & 1 & 1 & 1 & 4 & 7 & 13 & 25 & 49 & 94 & 181 \\
$4$ & $1.96595$ & 1 & 1 & 1 & 1 & 5 & 9 & 17 & 33 & 65 & 129 \\
$5$ & $1.98358$ & 1 & 1 & 1 & 1 & 1 & 6 & 11 & 21 & 41 & 81 \\
$6$ & $1.99198$ & 1 & 1 & 1 & 1 & 1 & 1 & 7 & 13 & 25 & 49 \\
\uzlhline
\end{tabular}

\end{table}

\begin{definition}
  The Fibonacci numbers for the minimum strategy are defined as
  \begin{enumerate}
    \item $F(0) = F(1) = ⋯ = F(d) = 1$.
    \item $F(n + 1) = F(n) + F(n - 1) + ⋯ + F(n - d)$.
  \end{enumerate}
  For $d = 2, 3$ and $4$, they are also referred to as the Tribonacci, Tetranacci
  and Pentanacci numbers.
  In general, they are known as the $d$-bonacci numbers.
\end{definition}

For these numbers, we use the solution vector $x = (x₁, …, x_d)$ with
\[
  x_i = \frac{\sum_{k=0}^i F(n - i)}{F(n)}.
\]
Because $F(n) = F(n - 1) + ⋯ + F(n - d - 1) > F(n - 1) + ⋯ + F(n - i)$,
each $x_i$ is bounded between $1$ and $2$.
Thus, each $x_i$ has a fractional value of $1$.
The smallest one is $x₁$, so we pivot with $ℓ = 1$ first.
Let $x' = \mathrm{pivot}_ℓ(x)$, then the first value is
\begin{align*}
  x₁' &
  = \frac{1}{\{x₁\}}
  = \frac{1}{x₁ - 1}
  = \frac{F(n)}{F(n - 1)}
  = \frac{F(n - 1) + ⋯ + F(n - d)}{F(n)}.
\end{align*}
This equals the value of $x_d$, if we would have started with $F(n-1)$ instead of $F(n)$.
The other values in our input vector $x$ are calculated as follows:
\begin{align*}
  xᵢ'
  & = \frac{\{xᵢ\}}{\{x₁\}} = \frac{xᵢ - 1}{x₁ - 1} \\
  & = \frac{F(n - 1) + ⋯ + F(n - i)}{F(n + 1)} · \frac{F(n)}{F(n - 1)} \\
  & = \frac{F(n - 1) + ⋯ + F(n - i)}{F(n - 1)}.
\end{align*}
So in the next iteration the value $x_i'$ has the same value as $x_{i-1}$, if we
would have started with $F(n - 1)$.
Therefore, in the next iteration the smallest fractional value must be $x_2$.
If we repeat this a total of $d$ times, then we end up where we started, but
each term is shifted backwards by $d$ steps.
To reach the end, we need a total $n$ steps.

% TODO: There should be a note, that we're taking the actual minimum, so we
% stop once we see any zero. Because at that point, we're cycling.

For the worst-case analysis, we first need the following lemma, which shows
that the $\mathrm{pivot}$ operation from Equation~\ref{eq:modified-update-rule}
can be viewed as a multi-dimensional division with remainder operation, where
we have a set of integers $p₁, …, p_d, q$ and divide them by one particular
element $p_ℓ$.

\begin{lemma}
  \label{lem:divmod}
  % TODO: This should basically prove that pivot is just division with
  % remainder
  Let $x = \left(\frac{p₁}{q}, \frac{p₂}{q}, …, \frac{p_d}{q}\right)$
  and $x' = \left(\frac{p₁'}{q'}, \frac{p₂'}{q'}, …, \frac{p_d'}{q'}\right)$
  with $x' = \mathrm{pivot}_ℓ(x)$ for some index $ℓ$.
  % TODO: Do we need gcd = 1?
  %Suppose $\gcd(q, p₁, …, p_d) = 1$ and $\gcd(q', p₁', …, p_d') = 1$, then
  Then,
  \[
    q = p_ℓ',
    \qquad
    p_ℓ = a_ℓ p_ℓ' + q',
    \qquad
    p_i = a_i p_ℓ' + p_i',
  \]
  for every $i ∈ \{1, …, d\}$ and $a = \floor{x}$.
\end{lemma}

\begin{proof}
  % TODO: Reference the rule
  By definition of $\mathrm{pivot}$, we have
  \[
    x_ℓ' = \frac{p_ℓ'}{q'} = \frac{1}{\frac{p_ℓ}{q} - a_ℓ} = \frac{q}{p_ℓ - a_ℓ q}.
  \]
  Therefore, $q = p_ℓ'$ and $p_ℓ = a_ℓ q + q' = a_ℓ p_ℓ' + q'$.
  For the remaining elements, we have
  \[
    \frac{p_i'}{q'} = \frac{\frac{p_i}{q} - a_i}{\frac{p_ℓ}{q} - a_ℓ} = \frac{p_i - a_i q}{p_ℓ - a_ℓ q}.
  \]
  Therefore, $p_i = p_i' + a_i q = p_i' + a_i p_ℓ'$.
\end{proof}

To show that the $d$-bonacci numbers are the worst-case for the minimum
strategy, we first need a few requirements.
The first is that the vector $x$ is strictly positive,
which can be easily solved by ignoring the first iteration.
In the second iteration, $x$ must always be positive.
The second requirement is that the values in $x$ are all different.
The reasoning behind this requirement is that one could easily check if there
are two elements $x_i, x_j$ with $x_i = x_j$.
Pivoting with either $ℓ = i$ or $ℓ = j$ removes the other element from the
vector, since the element will be zero in the next iteration.
Hence, this step will always be more efficient than pivoting a different element.
With these requirements, we can show that the $d$-bonacci numbers are the worst-case.

\begin{theorem}
  Let $x = (p₁/q, …, p_d/q) ∈ ℚ^d$.
  If the algorithm requires at least $n ≥ 0$ steps for the vector $x$ when
  taking the minimum fractional value at each step, then there exists a permutation $σ$ such
  that
  \[
    q ≥ F(n+d),
    \qquad
    p_{σ(i)} ≥ \sum_{k = 0}^i F(n+d - k),
    \text{ for every } i ≤ d.
  \]
\end{theorem}

\begin{proof}
  Suppose we require $n = 0$ steps.
  Since each value of $p₁, …, p_d, q$ must be distinct,
  there has to be one out of the $d+1$ integers,
  which is greater than $k$ for each $k ∈ \{1, …, d+1\}$.
  However, the first $d+1$ Fibonacci numbers have the values:
  \begin{align*}
    F(d) = 1,
    \qquad
    \sum_{k=0}^i F(d - k) = i + 1 \text{ for every } i ≤ d.
  \end{align*}
  Furthermore, the smallest value the denominator $q$ can assume is $1$.
  By ordering the elements accordingly,
  we can find a permutation $σ$ such that
  the bounds for this theorem are satisfied.

  % TODO: Fix indices
  Suppose the algorithm requires $n+1$ steps for some input $x = (p₁/q, …, p_d/q)$.
  Hence, $x' = \mathrm{pivot}_ℓ(x)$ requires $n$ steps and by induction,
  there exists a permutation $σ$ such that $q' ≥ F(n)$ and
  \begin{align*}
    p_{σ(i)}' & ≥ \sum_{k=0}^i F(n - k) = F(n + 1).
  \end{align*}
  We assume without loss of generality that $σ(d)$ is the largest element.
  Hence, we have to pivot with $ℓ = σ(d)$.
  Let $a = \floor{x}$, then $a_i ≥ 1$ by construction.
  From Lemma~\ref{lem:divmod}, it follows that
  \begin{align*}
    q        & = p_{σ(d)}' ≥ F(n+1),                                                   \\
    p_{σ(d)} & = a_{σ(d)} p_{σ(d)}' + q' ≥ F(n + 1) + F(n) = \sum_{k=0}^1 F(n + 1 - k) \\
    p_{σ(i)} & = a_{σ(d)} p_{σ(d)}' + p_{σ(i)}' ≥ F(n + 1) + ∑_{k=0}^i F(n - k) = ∑_{k=0}^{i+1} F(n + 1 - k).
  \end{align*}
  We construct a new permutation $σ'$ which orders the bounds correctly
  using $σ'(1) = σ(d)$ and $σ'(i) = σ(i+1)$ for the other indices.
\end{proof}

% TODO: Explain how this can be used to bound the time
Using the contrapositive of the theorem, it follows that if a vector $x$
has only one value smaller than the listed bounds, then it must take less than
$n$ steps.
Furthermore, we have seen that the Fibonacci numbers require exactly $n$ steps.
Therefore, these Fibonacci numbers represent the worst-case input for the
minimum strategy.
This mirrors how classical Fibonacci numbers represent the worst-case for the
classical Euclidean algorithm.

In the same spirit,
we can ask whether this generalization of Fibonacci also lead to a generalization of the golden ratio.
The classical Fibonacci numbers approach the golden ratio, when dividing consecutive numbers.
Therefore, we look at what values the $d$-bonacci numbers approach by dividing consecutive numbers.
This limit can be considered a multi-dimensional generalization of the golden ratio.
For now, we will assume that these ratios are converging.
A general proof of convergence will be shown in Section~\ref{sec:fib-conv}, at the end of the chapter.

\begin{theorem}
  If the ratios $\frac{F(n+1)}{F(n)}$ are converging, then they approach $φ_d$,
  where $φ_d$ is the positive real root of the polynomial $p(x) = x^{d+1} - x^d - ⋯ - 1$.
\end{theorem}
% TODO: Add note about convergence towards some sort of golden ratio

\begin{proof}
  By assumption, the ratios are approach some limit $r$.
  Then,
  \[
    r
    = \lim_{n → ∞} \frac{F(n+1)}{F(n)}
    = \lim_{n → ∞} \left(1 + \frac{F(n-1)}{F(n)} + \frac{F(n-2)}{F(n)} + ⋯ + \frac{F(n-d)}{F(n)}\right).
  \]
  Each ratio in the sum can be rewritten as a product of ratios with
  consecutive Fibonacci numbers, since
  \[
    \frac{F(n - i)}{F(n)}
    = \frac{F(n - i)}{F(n - i + 1)} · \frac{F(n - i + 1)}{F(n - i + 2)} ⋯ \frac{F(n - 1)}{F(n)}.
  \]
  By assumption, each ratio approaches $r^{-1}$ as $n$ increases towards infinity.
  Therefore,
  \[
    r = 1 + \frac{1}{r} + \frac{1}{r^2} + ⋯ + \frac{1}{r^d},
  \]
  which is equivalent to $r^{d+1} - r^d - ⋯ - r - 1 = 0$.
  Furthermore, the ratios are always positive.
  Hence, $r = φ_d$.
\end{proof}

\begin{corollary}
  The ratios $F(n + i)/F(n)$ converge towards $φ_d^i$ for any $i ≥ 0$.
\end{corollary}

We can then replace the ratios in the solution vector $x$ with the actual golden ratio $φ_d$.
We now have an algebraic vector $x = (x₁, …, x_d) ∈ ℝ^d$ with
\[
  x_i
  = \lim_{n → ∞} \sum_{k=0}^i \frac{F(n - i)}{F(n)}
  = \sum_{k=0}^i φ_d^k.
\]
Importantly, the algorithm is periodic for this input vector.
We have already seen for the approximate solution with Fibonacci numbers
that $d$ steps shifts each number backwards by $d$.
For the algebraic solution, this means that it stays the same after $d$ steps.

\begin{theorem}
  The generalized Euclidean algorithm is periodic for the real vector $x = (x₁, …, x_d)$
  with $x_i = \sum_{k=0}^i φ_d^k$ when taking the minimum fractional value.
\end{theorem}

\begin{proof}
  %The integer part of each $x_i$ is still $1$.
  The first value $x₁$ is the smallest again, so $ℓ = 1$.
  In the following iteration, we have
  \[
    x_i' = \frac{\{x_i\}}{\{x_1\}} = \frac{φ_d + φ_d^2 + ⋯ + φ_d^i}{φ_d} = 1 + φ_d + ⋯ + φ_d^{i-1} = x_{i-1}.
  \]
  and
  \[
    x_1' = \frac{1}{\{x_1\}} = \frac{1}{φ_d} = 1 + φ_d + ⋯ + φ_d^d = x_d.
  \]
  Therefore, we reach the original input vector after $d$ iterations.
\end{proof}

% TODO: What property do the ratios have? Are they similar to the golden ratio?

% ==============================================================================
\section{Improving the Minimum Strategy over Two Iterations}
% ==============================================================================

We can achieve a much better decrease if we minimize over two iterations instead of just one.
We assume that the values are ordered in increasing order by their fractional
value, i.e. $\{x₁\} ≤ \{x₂\} ≤ ⋯ ≤ \{x_d\}$.
In the first iteration, we choose the index $ℓ ∈ \{1, …, d\}$,
which minimizes the fractional value $\{\{x_ℓ\}/\{x_{ℓ-1}\}\}$
Then, we choose $ℓ + 1$ in the second iteration.
We additionally define $x₀ = 1$ and $x_{d+1} = 1$ such that we can choose $ℓ = 1$ in the first iteration and $ℓ = d$ in the second.
The reasoning behind this strategy is that it minimizes the distance between the two closest values,
if we measure the distance by their ratios.

Whereas the Fibonacci numbers for the minimum strategy only occur,
when analyzing the whole execution of the algorithm,
the Fibonacci numbers for this strategy occur already after two iterations.
In fact, we will go straight to the golden ratio for this strategy and use it
to bound the decrease of the determinant over two iterations.

\begin{theorem}
  The determinant decreases by at least $ψ_d$ over two iterations,
  where $ψ_d$ is the root of the polynomial $p(x) = x(x+1)^d - 1$.
\end{theorem}

\begin{proof}
  We assume without loss of generality that every value in $x$
  lies between $0$ and $1$, such that $\{x_i\} = x_i$.
  Furthermore, we assume that $\floor{x_ℓ/x_{ℓ-1}} = 1$ with the exception of $ℓ = 1$,
  since any larger value would lead to a faster decrease.
  The decrease in the second iteration is the largest
  when all ratios $\{x_ℓ/x_{ℓ-1}\}$ for $ℓ ∈ \{1, …, d\}$ are equal, i.e.
  \[
    x₁ = \frac{x₂}{x₁} - 1 = ⋯ = \frac{x_d}{x_{d-1}} - 1 = \frac{1}{x_d} - 1.
  \]
  The first equation gives us $x₂ = x₁(x₁ + 1)$.

  Next, we show via induction that $x_i = x₁ (x₁ + 1)^{i-1}$.
  The case $i = 1$ is trivial and we shown the case $i = 2$ just now.
  Thus, we proceed with the induction step.
  Suppose that $x_{i+1} = x₁ (x₁ + 1)^{i-1}$.
  The other equations (except the last) result in
  \[
    \frac{x_{i+1}}{x_i} - 1 = \frac{x_i}{x_{i-1}} - 1 \iff
    x_{i+1}
    = \frac{x_i^2}{x_{i-1}}
    = \frac{(x_1 (x₁ + 1)^{i-1})^2}{x₁ (x₁ + 1)^{i-2}}
    = x₁ (x₁ + 1)^{i+1}
    .
  \]
  Thus, we have $x_i = x₁(x₁ + 1)^{i-1}$.
  Finally, the last equation leads to the polynomial
  \[
    \frac{x_d^2}{x_{d-1}} = 1
    \iff
    x₁ (x₁ + 1)^d - 1 = 0,
  \]
  which is the exact definition of $ψ_d$.
\end{proof}

As before, the proof leads to an algebraic solution vector $x = (x₁, …, x_d)$ defined by $x_i = ψ_d (ψ_d + 1)^{i-1}$
and we consider, what happens with this vector under the specified strategy.
Although the previous golden ratio $φ_d$ was periodic under the minimum strategy,
the vector $x$ is not periodic under this strategy.
Nevertheless, we can use a different strategy under which the input becomes periodic.

\begin{theorem}
  The generalized Euclidean algorithm is periodic for the real vector $x = (x₁, …, x_d)$
  with $x₁ = ψ_d + 1$ and $x_i = ψ_d (ψ_d + 1)^i + 1$ when taking the maximum fractional value.
\end{theorem}

\begin{proof}
  Because $ψ_d$ is positive and $ψ_d (ψ_d + 1)^d = 1$,
  the values $x_i$ strictly lie between $0$ and $1$
  and $x_1$ lies between $1$ and $2$.
  The largest fractional value is $x_d$,
  since it has the highest power of $ψ_d + 1$.
  Pivoting with $x_d$ results in the vector $x' = (x₁', …, x_d')$ with
  \[
    x_d' = \frac{1}{ψ_d (ψ_d + 1)^{d-1}} = ψ_d + 1
  \]
  as well as
  \[
    x_1' = \frac{ψ_d + 1 - 1}{ψ_d (ψ_d + 1)^{d-1}} = ψ_d (ψ_d + 1)
    \quad
    \text{ and }
    \quad
    x_i' = \frac{ψ_d (ψ_d + 1)^{i-1}}{ψ_d (ψ_d + 1)^{d-1}} = ψ_d (ψ_d + 1)^i.
  \]
  This follows from the definition of $ψ_d$, because
  \[
    ψ_d(ψ_d + 1)^d = 1 \iff \frac{1}{ψ(ψ_d + 1)^d} = 1 \iff \frac{ψ_d (ψ_d + 1)^i}{ψ_d (ψ_d + 1)^d} = ψ_d (ψ_d + 1)^{i-1}.
  \]
  Note that the fractional values have not changed in $x'$
  except that now $x_{d-1}$ has the largest fraction value.
  Thus, $x$ is periodic if we repeat this step $d$ times.
\end{proof}

To find the Fibonacci numbers, we have to go backwards.
We already have a definition of a golden ratio $ψ_d$,
so we have to find a linear recurrence which converges to $ψ_d$.
In its current form, we cannot easily find such a linear recurrence.
Instead, we will use the root $θ_d = ψ_d + 1$.
Then,
\[
  ψ_d (ψ_d + 1)^d - 1 = 0
  \iff
  θ_d^{d+1} - θ^d - 1 = 0.
\]
Using this equation, we define the new Fibonacci numbers as
\[
  F(n + d + 1) = F(n + d) + F(n + 1).
\]
We can then show that their ratios are converging towards the root $θ_d$.

\begin{theorem}
  If the ratios $\frac{F(n+1)}{F(n)}$ are converging, then they approach $θ_d$.
\end{theorem}

\begin{proof}
  By assumption, the ratios are approach some limit $r$.
  Then,
  \[
    r
    = \lim_{n → ∞} \frac{F(n+1)}{F(n)}
    = \lim_{n → ∞} \left(1 + \frac{F(n-d)}{F(n)}\right).
  \]
  The ratio $F(n-d)/F(n)$ in the sum can be rewritten as a product of ratios with
  consecutive Fibonacci numbers, since
  \[
    \frac{F(n - d)}{F(n)}
    = \frac{F(n - d)}{F(n - d + 1)} · \frac{F(n - d + 1)}{F(n - d + 2)} ⋯ \frac{F(n - 1)}{F(n)}.
  \]
  By assumption, each ratio approaches $r^{-1}$ as $n$ increases towards infinity.
  Therefore,
  \[
    r = 1 + \frac{1}{r^d},
  \]
  which is equivalent to the equation $r^{d+1} - r^d - 1 = 0$.
  Furthermore, the ratios are always positive.
  Hence, $r = ψ_d$.
\end{proof}

\begin{corollary}
  If the ratios $\frac{F(n+1)}{F(n)} - 1$ are converging, then they approach $ψ_d$.
\end{corollary}

We can now construct a vector $x$ using these Fibonacci numbers.
Specifically, we choose the vector
\begin{align*}
  x = \left(
    \frac{F(n-1)}{F(n)},
    \frac{F(n-2)}{F(n)},
    …,
    \frac{F(n-(d-1)}{F(n)},
    \frac{F(n-d) + F(n)}{F(n)} \right).
\end{align*}
We pivot with the largest fractional value, which is $x_1$.
The next vector is $x' = (x₁', …, x_d')$ where
the first element is
\begin{align*}
  x_1'
  = \frac{1}{\left\{\frac{F(n-1)}{F(n)}\right\}}
  = \frac{F(n)}{F(n-1)}
  = \frac{F(n-1) + F(n-1-d)}{F(n)},
\end{align*}
the elements with $i ≤ d - 1$ are
\begin{align*}
  x_i'
  = \frac{\frac{F(n-i)}{F(n)}}{\frac{F(n-1)}{F(n)}}
  = \frac{F(n-i)}{F(n-1)}
\end{align*}
and the last element is
\begin{align*}
  x_d'
  = \frac{\frac{F(n-d) + F(n)}{F(n)} - 1}{\frac{F(n-1)}{F(n)}}
  = \frac{F(n-d)}{F(n-1)}.
\end{align*}

However, this time the maximum Fibonacci numbers do not represent the worst-case,
even for the new strategy.
The issue is that in the equation
\[
  p_ℓ = a_ℓ p_ℓ' + q'
\]
from Lemma~\ref{lem:divmod} we cannot bound $a_ℓ$ from below by $1$.
The reason is that only element of $x$ can be greater than $1$
and all other elements must be smaller than $1$.
But that element does not have to be the pivot element,
it could be another element.
Therefore, we cannot use them as a bound for the worst-case
in the same way as we did before.
Nevertheless, their ratios still lead to the root $θ_d$,
which is periodic under this strategy.

% ==============================================================================
\section{Convergence of General Linear Recurrences}
\label{sec:fib-conv}
% ==============================================================================


We have seen that the ratios of consecutive Fibonacci numbers for both
definition converges towards their own respective golden ratios.
However, this was under the assumption that the ratios are converging.
In this section, we will show in this section that this is actually the case.
To show this for both sequences at once, we use a general linear recurrence $F(n)$
with the initial terms $F(0) = ⋯ = F(d) = 1$ and with the remaining terms
defined according to
\[
  F(n + 1) = c_d F(n) + c_{d-1} F(n-1) + ⋯ + c₁ F(n - d + 1) + c₀ F(n - d)
\]
for some nonnegative integer coefficients $c₀, c₁, …, c_d$.
We assume that $c₀$ and $c_d$ are not zero.
For example, the $d$-bonacci sequence has $c_0 = ⋯ = c_d = 1$,
whereas the other sequence has only $c₀ = c_d = 1$ and the remaining terms are zero.
The goal is to show that the ratios
\[
  r_n
  := \frac{F(n+1)}{F(n)}
  = \frac{c_0 F(n - d)}{F(n)} + \frac{c₁ F(n - d + 1)}{F(n)} + ⋯ + \frac{c_{d-1} F(n-1)}{F(n)} + c_d
\]
are converging.

The first step is to rewrite the terms of the recurrence such that each ratio
between non-consecutive Fibonacci numbers can be rewritten as a product of
ratios $r_{n-i}$:
\begin{align*}
  \frac{F(n - d + i)}{F(n)}
  & = \frac{F(n - d + i + 1)}{F(n - d + i)} \frac{F(n - d + i + 2)}{F(n - d + i + 1)} \dots \frac{F(n-1)}{F(n)} \\
  & = \frac{1}{r_{n - d + i}} · \frac{1}{r_{n - d + i + 1}} · \dots · \frac{1}{r_{n-1}}.
\end{align*}
Then we can calculate the ratio $r_n$ using the previous ratios $r_{n-1}, r_{n-2}, …, r_{n-d}$ as follows:
\[
  r_n = c_0 + \frac{c_1}{r_{n-1}} + \frac{c_2}{r_{n-1} r_{n-2}} + ⋯ + \frac{c_d}{r_{n-1} r_{n-2} \dots r_{n-d}}.
\]
Using this equation, we can show that the ratios are bounded.

\begin{lemma}
  \label{lem:fib-bounded}
  The ratios $r_n$ are bounded between $1$ and $C := ∑_{k=0}^d c_d$.
\end{lemma}

\begin{proof}
  The first $d - 1$ ratios are all $1$.
  The ratio $r_d$ is equal to
  \[
    \frac{F(d+1)}{F(d)} = \frac{F(0) + F(1) + ⋯ + F(d)}{F(d)} = \frac{c_d + ⋯ + c_0}{1} = C,
  \]
  which clearly satisfies the bounds of this lemma.
  By induction, suppose that the previous ratios $r_{n-1}, r_{n-2}, …, r_{n-d}$
  all satisfy the bound between $1$ and $d+1$.
  From previous consideration, we can reformulate the ratio $r_n$ as follows:
  \[
    r_n = c₀ + \frac{c₁}{r_{n-1}} + \frac{c₂}{r_{n-1} r_{n-2}} + \dots + \frac{c_d}{r_{n-1} r_{n-2} \dots r_{n-d}}.
  \]
  Since $r_{n-i} ≤ C$, we can bound $r_n$ from below by
  \[
    r_n ≥ c₀ + \frac{c₁}{C} + \frac{c₂}{C^2} + \dots + \frac{c_d}{C^d} ≥ 1
  \]
  and since $r_{n-i} ≥ 1$, we can bound $r_n$ from above by
  \[
    r_n ≤ c₀ + \frac{c₁}{1} + \frac{c₂}{1} + \dots + \frac{c_d}{1} ≤ C.
  \]
  Hence, $1 ≤ r_n ≤ C$ for every $n ≥ 0$.
\end{proof}

\begin{figure}[tbp]
  \centering
  \includestandalone{figures/fibonacci-convergence}
  \caption{
    Illustration of the convergence proof for ratios of Fibonacci numbers.
    The points represent the ratios $r_n$.
    The sequences $s_n$ and $t_n$ are the minimum and maximum from the previous
    $d$ ratios, respectively.
    Both sequences converge towards the same limit $φ$, so the ratios converge
    towards this limit, too.
  }
  \label{fig:fibonacci-convergence}
\end{figure}

Showing that the sequence is monotone would be enough to show the convergence.
However, the sequence is clearly not monotone.
Even the ratios of the original Fibonacci sequence alternate between increasing
and decreasing.
So we cannot possibly prove that the higher-order sequences are monotone.
Instead, we bound the sequence between two other sequences $s_n$ and $t_n$
and show that the two sequences are converging to the same limit.
From the squeeze theorem, it follows that $r_n$ converges to the same limit.
The main idea is illustrated in Figure~\ref{fig:fibonacci-convergence}.
The sequences are
\[
  s_n = \min\{r_n, r_{n-1}, …, r_{n-d} \}, \qquad t_n = \max\{r_n, r_{n-1}, …, r_{n-d}\}
\]
Because $r_n$ clearly lies between the two sequences, it must converge to the same limit as $s_n$ and $t_n$.
For these sequences, we already know from the previous lemma that they are bounded,
so it only remains to be shown that they are monotone,
i.e. $s_n$ is increasing and $t_n$ is decreasing.
After which we can show that they are converging to the same limit.

\begin{lemma}
  The sequences $s_n$ and $t_n$ are monotone.
\end{lemma}

\begin{proof}
  Each ratio can be represented as a convex combination of the previous ratios, i.e.
  \[
    r_{n+1} = λ₀ r_n + λ₁ r_{n-1} + \dots + λ_d r_{n-d}
  \]
  using $λ_i = F(n - i) / F(n + 1)$.
  To show that this is indeed a convex combination, all coefficients $λ_i$
  need to be nonnegative and $λ₀ + λ₁ + \dots + \lambda_d = 1$.
  The former follows from the fact that Fibonacci numbers are always increasing,
  while the latter follows simply from the definition of the Fibonacci numbers.
  Because $s_n$ is the minimum and $t_n$ the maximum of $r_n, r_{n-1}, …, r_{n-d}$,
  we can bound the next maximum by
  \[
    t_{n+1} ≤ r_{n+1} = λ₀ r_n + λ₁ r_{n-1} + \dots + λ_d r_{n-d} ≤ λ₀ t_n + λ₁ t_n + ⋯ + λ_d t_n = t_n.
  \]
  and the next minimum by
  \[
    s_{n+1} ≥ r_{n+1} = λ₀ r_n + λ₁ r_{n-1} + \dots + λ_d r_{n-d} ≥ λ₀ s_n + λ₁ s_n + ⋯ + λ_d s_n = s_n.
  \]
  Therefore, $s_n ≤ s_{n+1} ≤ t_{n+1} ≤ t_n$.
\end{proof}

\begin{lemma}
  The sequences $s_n$ and $t_n$ are converging to the same limit.
\end{lemma}

% TODO: Finish this proof
\begin{proof}
  For a contradiction, suppose there exists some $δ > 0$ such that $t_n - s_n > δ$ for every $n ≥ 0$.
  Out of the previous ratios, there must be one ratio $r_k$ exactly equal to $t_n$.
  Therefore,
  \begin{align*}
    s_{n+1} ≥ r_n & = λ₀ r_{n-d-1} + λ₁ r_{n-d} + ⋯ + λ_d r_{n-1} \\
                  & ≥ λ_k t_n + \sum_{i ≠ k} λ_i s_n \\
                  & = λ_k t_n + (1 - λ_k) s_n = s_n + λ_k (t_n - s_n) ≥ s_n + λ_k δ.
  \end{align*}
  We have
  \[
    λ_k = \frac{F(n+k)}{F(n+d+1)} = \frac{1}{r_{n+k} r_{n+k+1} \dots r_{n+d+1}} ≥ \frac{1}{(d+1)^{d+1-k}}.
  \]
  Hence, $λ_k$ is always greater than some constant $c > 0$ for every $k ≥ 0$.
  But then
  \[
    s_{n+i} ≥ s_{n+i-1} + c δ ≥ s_{n+i-2} + 2c δ ≥ \dots ≥ s_n + i c δ
  \]
  and $s_{n+i}$ would always increase as $i$ approaches infinity,
  which contradicts Lemma~\ref{lem:fib-bounded}.
  Therefore, $δ = 0$ and it follows that $s_n$ and $t_n$ are approaching the same limit.
\end{proof}

% TODO: Show that this converges! We're still missing the convergence criteria
\begin{theorem}
  Let $F$ be a linear recurrence with coefficients $c_0, \dots, c_d ≥ 0$
  and let $φ$ be the real positive root of its characteristic polynomial.
  Then,
  \[
    \lim_{n \to \infty} \frac{F(n + 1)}{F(n)} = φ.
  \]
\end{theorem}

\begin{proof}
  By the previous lemma, the ratios $r_n$ approach some limit $r ∈ ℝ$. It follows:
  \[
    r
    = \lim_{n → ∞} r_n
    = \lim_{n → ∞} 1 + \frac{c_d}{r_{n-1}} + \frac{c_{d-1}}{r_{n-1} r_{n-2}} + ⋯ + \frac{a₀}{r_{n-1} r_{n-2} \dots r_{n-d}}.
  \]
  Hence, each denominator in the sum approaches $r^i$,
  which results in the following polynomial:
  \[
    r = 1 + \sum_{i = 1}^d \frac{c_{d - i}}{r^i}
    \iff
    r^{d+1} = c₀ + c₁ r + \dots + c_d r^d,
  \]
  which directly corresponds to a root of its characteristic polynomial.
  Furthermore, the ratios are always positive, so $r = φ$.
\end{proof}

\begin{corollary}
  The ratios $F(n + i) / F(n)$ converge to $φ^i$ for $i > 1$.
\end{corollary}

This concludes the connection between Fibonacci numbers and their golden ratios.
As previously shown, the ratios of $d$-bonacci numbers approach the root of the polynomial $x^{d+1} - x^d - ⋯ - x - 1$
and the ratios of the other Fibonacci numbers approach the root of the polynomial $x^{d+1} - x^d - 1$.
For both of these roots, the generalized Euclidean algorithm becomes periodic after $d$ iterations.
So they represent the first examples of algebraic numbers, which are periodic
under the generalized Euclidean algorithm.
