\chapter{Higher-Order Fibonacci Sequences and their Golden Ratios}

% TODO: Add reference to Fibonacci numbers in the continued fraction chapter
The simplest periodic continued fraction is $[1; \overline{1}]$.
This fraction evaluates to the golden ratio and its convergents are ratios of
consecutive Fibonacci numbers.
Naturally, we can consider the simplest vectors $x ∈ ℝ^d$ for the generalized
Euclidean algorithm, where the integer part in each iteration is $1$.
Such vectors can be seen as a multi-dimensional generalization of the golden
ratio and Fibonacci numbers.
However, there does not exist a single definitive golden ratio,
Since the algorithm has an additional freedom in the choice of the pivot.
So there are actually multiple different possible definitions.
Some of these are well known, like the supergolden ratio or the plastic ratio,
which are both roots of cubic polynomials.

Since the definition of the golden ratios depends entirely on the choice of our
pivot element $x_ℓ$, we consider only two strategies in this chapter: First
choosing the smallest fractional value $\{x_ℓ\}$ and second choosing the largest
fractional value $\{x_ℓ\}$.
From these two strategies we will then derive a general definition of
higher-order Fibonacci sequences and their golden ratios.

% ==============================================================================
\section{Higher-Order Fibonacci Sequence for the Minimum Strategy}
% ==============================================================================

We have already seen that the worst-case input for the one-dimensional
Euclidean algorithm are the Fibonacci numbers.
Plugging in two Fibonacci numbers always requires at least n steps for the
algorithm.
This leads to the question whether such numbers also exist in higher dimensions
and what their golden ratios look like.

% TODO: There should be a discussion on what it even means to have a Fibonacci number in higher dimensions

First, we have to understand what Fibonacci numbers in higher dimensions would
look like.
Fibonacci numbers have two properties which make them special for the Euclidean algorithm.
The first is that applying one iteration of the Euclidean algorithm on two
consecutive Fibonacci numbers~$(F(n + 1), F(n))$ yields the previous Fibonacci numbers~$(F(n), F(n - 1))$.
So the Euclidean algorithm requires exactly $n$ steps on this input.

The second property is that they are minimal with respect to this.
There are other sequences, which fulfill the first property, but the Fibonacci
numbers are special, in that they fit exactly once inside the next number.
This means that in the Euclidean algorithm there is a multiplicative factor of $1$ at each step (except the last):
\[
  \begin{array}{rcl}
    F(n + 1) & = & 1 · F(n) + F(n - 1),     \\
    F(n)     & = & 1 · F(n - 1) + F(n - 2), \\
             & ⋮ &                          \\
    F(2)     & = & 2 · F(1).
  \end{array}
\]
Other numbers may have higher factors -- for example, the silver ratio has a
factor of $2$ at each step -- but the Fibonacci numbers have the smallest factor.

This property is also reflected in the fractions $a/b$ for a given input $(a, b)$.
At each step (except the last) the fraction always has an integer part of $1$:
\[
  \begin{array}{rcl}
    \frac{F(n + 1)}{F(n)} & = & 1 + \frac{F(n - 1)}{F(n)},     \\
    \frac{F(n)}{F(n - 1)} & = & 1 + \frac{F(n - 2)}{F(n - 1)}, \\
                          & ⋮ &                                \\
    \frac{F(2)}{F(1)}     & = & 2.                              \\
  \end{array}
\]
So, a higher-dimensional Fibonacci number would also need to have a fraction
with integer part $1$ at each step.

% TODO: We should also discuss the thing about idempotence under the choice of the pivot.
% TODO: Should we have a discussion about why we choose this particular linear system.

First, we are indeed talking about just numbers and not vectors.
The Euclidean algorithm takes one-dimensional integers as input, so one would
expect that the generalized algorithm would take Fibonacci vectors as input.
But the numbers of steps that the algorithm requires only depends on the
initial solution vector x.
In fact, the entire execution of the algorithm only depends on this vector.
There are infinitely many possible linear systems which have the same solution
vector x, and hence there would be infinitely many "Fibonacci vectors".
Therefore, we look instead only for the numbers inside the rational vector.

There are many different possibilities for choosing a linear system which has the solution $x$.
We use a simple construction, which
given a rational vector $x = \left(\frac{p₁}{q}, \frac{p₂}{q}, …, \frac{p_d}{q}\right)$,
constructs the linear system $Bx = c$ with
\begin{equation}
  \label{eq:linear-system-construction}
  B =
  \begin{pmatrix}
    q & 0 & ⋯ & 0 \\
    0 & q & ⋯ & 0 \\
    ⋮ & ⋮ & ⋱ & ⋮ \\
    0 & 0 & ⋯ & q \\
  \end{pmatrix},
  \quad
  c =
  \begin{pmatrix}
    p₁ \\
    p₂ \\
    ⋮ \\
    p_d
  \end{pmatrix}.
\end{equation}

\begin{table}[tbp]
  \caption{The first 10 Fibonacci numbers for $d = 1, …, 5$ and their respective golden ratio.}
  \label{tbl:min-fibonacci}
  \centering
  \begin{tabular}{cc|cccccccccc}
\uzlhline
\uzlemph{$d$} & \uzlemph{$φ$} & \uzlemph{$F(0)$} & \uzlemph{$F(1)$} & \uzlemph{$F(2)$} & \uzlemph{$F(3)$} & \uzlemph{$F(4)$} & \uzlemph{$F(5)$} & \uzlemph{$F(6)$} & \uzlemph{$F(7)$} & \uzlemph{$F(8)$} & \uzlemph{$F(9)$} \\
\hline
$1$ & $1.61803$ & 1 & 2 & 3 & 5 & 8 & 13 & 21 & 34 & 55 & 89 \\
$2$ & $1.83929$ & 1 & 1 & 3 & 5 & 9 & 17 & 31 & 57 & 105 & 193 \\
$3$ & $1.92756$ & 1 & 1 & 1 & 4 & 7 & 13 & 25 & 49 & 94 & 181 \\
$4$ & $1.96595$ & 1 & 1 & 1 & 1 & 5 & 9 & 17 & 33 & 65 & 129 \\
$5$ & $1.98358$ & 1 & 1 & 1 & 1 & 1 & 6 & 11 & 21 & 41 & 81 \\
$6$ & $1.99198$ & 1 & 1 & 1 & 1 & 1 & 1 & 7 & 13 & 25 & 49 \\
\uzlhline
\end{tabular}

\end{table}

% I would like a better explanation of what the goal of this section is

Although we have reduced the problem just to the vector $x$, there is still
some ambiguity left, when extending Fibonacci numbers to higher dimensions.
The generalized Euclidean algorithm allows for an additional degree of freedom
by choosing which column vector $B_ℓ$ we swap our vector $c$ with.
Therefore, there cannot be just one definitive Fibonacci number, there are
different ones depending on the choice of our index $ℓ$.

The first strategy one may think of is to choose the index $ℓ$ with the
smallest fractional value.
After all, this gives us the largest decrease in the determinant over one
iteration.
Locally, this would be the highest decrease we can achieve.
So we begin with this strategy and try to find a generalization of Fibonacci
numbers under this strategy.

% TODO: Actually explain how we derive the Fibonacci numbers for this strategy.
% TODO: Should we have a table for the Fibonacci numbers?
% TODO: What about adding references to their OEIS number?

The Fibonacci numbers for this strategy are defined as
\begin{align*}
  F(0) = F(1) = ⋯ = F(d) = 1, \qquad F(n + 1) = F(n) + F(n - 1) + ⋯ + F(n - d).
\end{align*}
For $d = 2, 3, 4$, they are also called the Tribonacci, Tetranacci and Pentanacci numbers.
In general, they are known as the $d$-bonacci numbers.
For these numbers,
We use the solution vector
\[
  x =
  \left(
    \frac{F(n)}{F(n + 1)},\;
    \frac{F(n) + F(n - 1)}{F(n + 1)},\;
    ⋯,\;
    \frac{F(n) + F(n - 1) + ⋯ + F(n + 1 - d)}{F(n + 1)}\;
  \right)
\]
and we can use the construction from Equation~\ref{eq:linear-system-construction}
for the actual input to the generalized Euclidean algorithm.

The smallest fractional value in this vector is $x₁$, so we pivot with $ℓ = 1$ first.
\begin{align*}
  \{x₁'\}
  & = \left\{\frac{1}{\{x₁\}}\right\}
  = \left\{\frac{F(n + 1)}{F(n)}\right\}
  = \frac{F(n - 1) + ⋯ + F(n - d)}{F(n)}.
\end{align*}
This gives us the value for $x_d$, if we would have started with $F(n)$ instead of $F(n+1)$.
The other values in our input vector $x$ are calculated as follows:
\begin{align*}
  \{xᵢ'\}
  & = \left\{\frac{\{xᵢ\}}{\{x₁\}}\right\} \\
  & = \left\{\frac{F(n) + F(n - 1) + ⋯ + F(n + 1 - i)}{F(n + 1)} · \frac{F(n + 1)}{F(n)}\right\} \\
  & = \frac{F(n - 1) + ⋯ + F(n - d)}{F(n)}.
\end{align*}
So in the next iteration the value $x_i$ has the same value as $x_{i+1}$ if we
would have started with $F(n)$.
Therefore, in the next iteration the smallest fractional value must be $x_2$.
We continue this until we have cycled through all variables.

Another example would be the opposite strategy: Choosing the maximum in each iteration.
In this case, the Fibonacci numbers are defined as
\begin{align*}
  F(0) = \dots = F(d) = 1, \qquad F(n + 1) = F(n) + F(n - d).
\end{align*}

The golden ratios for each definition is the limit of consecutive Fibonacci numbers
\[
  φ = \lim_{n → ∞} \frac{F(n + 1)}{F(n)},
\]
where $F(n)$ can be any of the previously defined Fibonacci numbers.

% ==============================================================================
\section{Higher-Order Fibonacci Sequence for the Maximum Strategy}
% ==============================================================================

\begin{table}[tbp]
  \caption{The first 10 Fibonacci numbers for $d = 1, …, 5$ and their respective golden ratio.}
  \label{tbl:max-fibonacci}
  \centering
  \begin{tabular}{cc|cccccccccc}
\uzlhline
\uzlemph{$d$} & \uzlemph{$φ$} & \uzlemph{$F(0)$} & \uzlemph{$F(1)$} & \uzlemph{$F(2)$} & \uzlemph{$F(3)$} & \uzlemph{$F(4)$} & \uzlemph{$F(5)$} & \uzlemph{$F(6)$} & \uzlemph{$F(7)$} & \uzlemph{$F(8)$} & \uzlemph{$F(9)$} \\
\hline
$1$ & $1.61803$ & 1 & 2 & 3 & 5 & 8 & 13 & 21 & 34 & 55 & 89 \\
$2$ & $1.46557$ & 1 & 1 & 2 & 3 & 4 & 6 & 9 & 13 & 19 & 28 \\
$3$ & $1.37971$ & 1 & 1 & 1 & 2 & 3 & 4 & 5 & 7 & 10 & 14 \\
$4$ & $1.32143$ & 1 & 1 & 1 & 1 & 2 & 3 & 4 & 5 & 6 & 8 \\
$5$ & $1.29577$ & 1 & 1 & 1 & 1 & 1 & 2 & 3 & 4 & 5 & 6 \\
$6$ & $1.23256$ & 1 & 1 & 1 & 1 & 1 & 1 & 2 & 3 & 4 & 5 \\
\uzlhline
\end{tabular}

\end{table}

Next, we look at the opposite strategy:
Always taking the index with the maximum fractional value at each step.

The Fibonacci numbers are defined by the following recurrence relation:
\begin{align*}
  F(0) = F(1) = \dots = F(d) = 1, \qquad F(n + d + 1) = F(n + d) + F(n).
\end{align*}
The respective golden ratio $φ$ for this sequence is the positive root of the polynomial
\begin{align*}
  p(x) = x^{d+1} - x^d - 1.
\end{align*}

% ==============================================================================
\section{Generalized Fibonacci Sequences}
% ==============================================================================

Dividing two consecutive Fibonacci numbers approaches the golden ratio as $n$ increases.
The golden ratio is a solution to the equation $x^2 - x - 1 = 0$.
For higher dimensions, we will encounter similar polynomial equations with higher degree.
The goal of this section is to generalize the relationship between linear
recurrences like the Fibonacci sequence with their respective golden ratio.

Since the algorithm expects $x_i$ between $0$ and $1$, we have to use $1/\phi$ instead of $\phi$.
The value $1/\phi$ is the root of another polynomial $x^2 + x - 1$, which is the reciprocal
of the polynomial $x^2 - x - 1$.

\begin{definition}
  Given a polynomial $p = \sum_{i=0}^n a_i x^n$, its \emph{reciprocal polynomial},
  denoted as $p^*$, is defined as
  \[
    p^*(x) = x^n p(x^{-1}) = a_n + a_{n-1} x + \dots + a_0 x^n.
  \]
\end{definition}

\begin{example}
  The reciprocal of the golden ratio polynomial $x^2 - x - 1$ is
  \begin{align*}
    x^2 + x - 1,
  \end{align*}
  which has a root at $\psi = 1/\phi$.
\end{example}

\begin{lemma}
  Let $p$ be a polynomial. Then, $a$ is a root of $p$ if and only if $a^{-1}$ is a root of $p^*$.
\end{lemma}

\begin{proof}
  Let $a$ be a root of $p$. It follows $p^*(a^{-1}) = a^n p(a) = a^n \cdot 0 = 0$.
  Now let $a^{-1}$ be a root of $p^*$. Then $p^*(a^{-1}) = a^n p(a) = 0$.
  By construction, $a$ cannot be zero and therefore we must have $p(a) = 0$.
\end{proof}

\begin{definition}
  A linear recurrence with coefficients~$a_0, \dots, a_d$ is an equation of the
  following form:
  \[
    F(n + 1) = a_0 F(n - d) + a_1 F(n - d + 1) + \dots + a_{d-1} F(n - 1) + a_d F(n).
  \]
\end{definition}

\begin{definition}
  Given a linear recurrence~$F$ with coefficients~$a_0, \dots, a_d$, its
  \emph{characteristic polynomial} $p_F$ is defined as
  \[
    p_F(x) = x^{d+1} - a_d x^d - a_{d-1} x^{d-1} - \dots - a_1 x - a_0.
  \]
\end{definition}

\begin{example}
  The linear recurrence which has $p_d^*(x)$ as its characteristic polynomial is:
  \begin{align*}
    F(n) = F(n - 1) + (-1)^{d+1} F(n - d) - \sum_{i=2}^{d - 1} (-1)^{i+1} 2 F(n - i).
  \end{align*}
\end{example}

% TODO: Show that this converges! We're still missing the convergence criteria
\begin{lemma}
  Let $F$ be a linear recurrence with coefficients $a_0, \dots, a_d$.
  If its characteristic polynomial has a real positive root $\phi$
  and the sequence $F(n+1)/F(n)$ is converging, then
  \[
    \lim_{n \to \infty} \frac{F(n + 1)}{F(n)} = \phi.
  \]
\end{lemma}

\begin{proof}
  By assumption, the ratios approach some limit $L$. It follows:
  \[
    L
    = \lim_{n \to \infty} \frac{F(n + 1)}{F(n)}
    = \lim_{n \to \infty} a_0 + \sum_{i = 1}^d \frac{a_i F(n - d + i)}{F(n)}
  \]
  Each term in the sum can be rewritten as a product of consecutive terms:
  \[
    \frac{F(n - 1 - d + i)}{F(n - 1)}
    = \frac{F(n - 1 - d + i)}{F(n - d + i)} \frac{F(n - d + i)}{F(n - d + i + 1)} \dots \frac{F(n - 2)}{F(n-1)}.
  \]
  Hence, each term in the sum approaches $L^{-i}$,
  which results in the following polynomial:
  \[
    L = a_0 + \sum_{i = 1}^d a_{d - i} L^{-i},
    \quad \text{or equivalently} \quad
    L^{d+1} = a_0 + a_1 L + \dots + a_d L^d,
  \]
  which directly corresponds to a root of its characteristic polynomial.
\end{proof}

\begin{corollary}
  Under the same conditions, $\lim_{n \to \infty} F(n + i) / F(n) = \phi^i$ for $i > 1$.
\end{corollary}

Given, the linear recurrence $F(n + d + 1) = F(n) + a_1 F(n - 1) + \dots + a_d F(n + d)$,
the value for $x_i^{(n)}$ in the solution $x^{(n)}$ is
\begin{equation}
  \label{eq:general-solution}
  x_i^{(n)} = \frac{F(n - i) + \sum_{k=1}^{i-1} a_{d-k} F(n - k)}{F(n)}.
\end{equation}

\begin{lemma}
  Pivoting with the first element of $x^{(n)}$ using the modified update rule yields the vector
  \[
    x' = (x^{(n-1)}_2, x^{(n-1)}_3, \dots, x^{(n-1)}_d, x^{(n-1)}_1).
  \]
\end{lemma}

\begin{proof}
  For $x_1$, we get
  \[
    \begin{aligned}
      \frac{1}{x_1}
      & = \frac{F(n)}{F(n - 1)} \\
      & = a_d + \frac{F(n - d - 1) + a_1 F(n - d - 2) + \dots + a_{d-1} F(n - 2)}{F(n - 1)} \\
      & = a_d + x^{(n-1)}_d.
    \end{aligned}
  \]
  For the other values, we get
  \begin{align*}
    \frac{x_i}{x_1}
    & = \frac{F(n - i) + \sum_{k=1}^{i-1} a_{d-k} F(n - k)}{F(n)} \frac{F(n)}{F(n - 1)} \\
    & = a_{d-1} + \frac{F(n - i) + \sum_{k=2}^{i-1} a_{d-k} F(n - k)}{F(n-1)} \\
    & = x^{(n-1)}_{i+1} \qedhere
  \end{align*}
\end{proof}

\begin{corollary}
  Running the generalized Euclidean algorithm with $x^{(n)}$ as the solution to
  the linear system $B x = c$ requires $n$ steps, if $x^{(n)}_1$ is the
  smallest element in $x^{(n)}$.
\end{corollary}

Of course, the Euclidean algorithm receives a matrix $B \in \Z^{d \times d}$
and vector $c \in \Z^d$ as its input and not the solution $x$ itself.
However, we can construct a very simple linear system $B^{(n)} x = c^{(n)}$,
where $x^{(n)}$ is the solution, in the following way:
\[
  B^{(n)} =
  \begin{pmatrix}
    F(n) & 0 & \dots & 0 & 0 \\
    0 & F(n) & \dots & 0 & 0 \\
    \vdots & \vdots & \ddots & \vdots & \vdots \\
    0 & 0 & \dots & F(n) & 0 \\
    0 & 0 & \dots & 0 & F(n) \\
  \end{pmatrix},
\]
\[
  c^{(n)} =
  \begin{pmatrix}
    F(n - 1) \\
    F(n - 2) + a_{d-1} F(n - 1) \\
    \vdots \\
    F(n - d + 1) + \sum_{k=1}^{d-2} a_{d-k} F(n - k) \\
    F(n - d) + \sum_{k=1}^{d-1} a_{d-k} F(n - k) \\
  \end{pmatrix}.
\]

\begin{lemma}
  The vector $x^{(n)}$ from Equation~\ref{eq:general-solution} is a solution to the
  linear system $B^{(n)} x = c^{(n)}$.
\end{lemma}

\begin{proof}
  Follows directly from the definition of $x^{(n)}$.
\end{proof}

% ==============================================================================
\section{An Idempotent Solution for Two Dimensions}
% ==============================================================================

Let $\vec x = (\psi, 1 - \psi)$, where $\psi$ is the only real positive root of $x^2 + x - 1$.
For this solution, it does not matter which element we choose to pivot with, it will always
yield the same result.

Pivoting with $x_1$:
\begin{align*}
  \frac{1}{x_1} - 1 & = \frac{1}{\psi} - 1 = \psi, \\
  1 - \frac{x_2}{x_1} & = 1 - \frac{1 - \psi}{\psi} = 1 - \psi.
\end{align*}

Pivoting with $x_2$:
\begin{align*}
  \frac{1}{x_2} - 2 & = \frac{1}{1 - \psi} - 2 = \psi, \\
  1 - \frac{x_1}{x_2} & = 1 - \frac{\psi}{1 - \psi} = 1 - \psi.
\end{align*}

\begin{align*}
  \left(
  \begin{array}{cc|c}
    F(n) & 0    & F(n-1) \\
    0    & F(n) & F(n) - F(n-1) \\
  \end{array}
  \right)
\end{align*}
