\documentclass[aspectratio=169]{beamer}

\title{On Elementary Properties of a Multidimensional Generalization of the Euclidean Algorithm}
\author{Daniel Knaack}
\date{}

\usepackage{tikz}
\usepackage{fontspec}
\usepackage{unicode-math}
\usepackage{standalone}

\setmathfont{TeX Gyre Pagella Math}
\usetikzlibrary{decorations.pathreplacing, calligraphy, intersections, backgrounds, graphdrawing, graphs, calc, 3d}

\usetheme{metropolis}
\usefonttheme[onlymath]{serif}
\setbeamertemplate{caption}{\raggedright\insertcaption\par}

\begin{document}

\begin{frame}
  \maketitle
\end{frame}

\begin{frame}
  \begin{itemize}
    \item A number is rational if and only if its decimal expansion is eventually periodic
    \item A number is a quadratic irrational if and only if its continued fraction expansion is eventually periodic
    \item A number is a cubic irrational if and only if ?
  \end{itemize}
\end{frame}

\begin{frame}
  \begin{enumerate}
    \item The Euclidean Algorithm
    \item Fibonacci Numbers
    \item Continued Fractions
    \item Klein Polygons
    \item Periodicity
  \end{enumerate}
\end{frame}

\begin{frame}
  \frametitle{The Euclidean Algorithm}
  % TODO: Figure for Euclidean algorithm
\end{frame}

\begin{frame}
  \frametitle{The Euclidean Algorithm}
  % TODO: Transitional figure for continued fractions
\end{frame}

\begin{frame}
  \frametitle{Continued Fractions}
  \[
    [a₀; a₁, …] = a₀ + \cfrac{1}{a₁ + \cfrac{1}{a₂ + \cfrac{1}{⋱}}}.
  \]
\end{frame}

\begin{frame}
  \begin{center}
    \includestandalone{figures/klein-polygon}
  \end{center}
\end{frame}

\begin{frame}
  % TODO: Figure for pivot rule
  \[
    x_i' =
    \begin{cases}
      \frac{1}{x_ℓ - a_ℓ}, & \text{ if } i = ℓ, \\
      \frac{x_i - a_i}{x_ℓ - a_ℓ}, & \text{ otherwise},
    \end{cases}
    \iff
    x_i =
    \begin{cases}
      a_ℓ + \frac{1}{x_ℓ'}, & \text{ if } i = ℓ, \\
      a_i + \frac{x_i'}{x_ℓ'}, & \text{ otherwise}.
    \end{cases}
  \]
\end{frame}

\begin{frame}
  \begin{definition}
    A \emph{multidimensional continued fraction} over a sequence
    $(a^{(n)})_{n≥0}$ is defined inductively as
    \[
      [a^{(0)}] = a^{(0)}, \quad
      [a^{(0)}; a^{(1)}, …, a^{(n)}] = \mathrm{pivot}^{-1}(a^{(0)}, [a^{(1)}; a^{(2)}, …, a^{(n)}]).
    \]
  \end{definition}
\end{frame}

\begin{frame}
  \begin{theorem}
    The convergents of a multidimensional continued fraction converge towards $x$, if
    \begin{enumerate}
      \item Every index is used infinitely often
      \item The distance between two equal indices is constant.
    \end{enumerate}
  \end{theorem}
\end{frame}

\begin{frame}
  \begin{center}
    \includestandalone{figures/projective-space}
  \end{center}
\end{frame}

\begin{frame}
  \begin{theorem}
    If a multidimensional continued fraction of a vector $x ∈ ℝ^d$ is periodic, then \[
      [ℚ(x₁, …, x_d) : ℚ] ≤ d+1.
    \]
  \end{theorem}
\end{frame}

\begin{frame}

\end{frame}

\begin{frame}
  % TODO: References
\end{frame}

\end{document}
