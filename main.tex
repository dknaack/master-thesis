\documentclass[english,version-2020-11]{uzl-thesis}

\UzLThesisSetup{
  Masterarbeit,
  Logo-Dateiname        = {uzl-thesis-logo-itcs.pdf},
  Verfasst              = {am}{Institut für Theoretische Informatik},
  Titel auf Deutsch     = {Über elementare Eigenschaften einer mehrdimensionalen Verallgemeinerung des euklidischen Algorithmus},
  Titel auf Englisch    = {On Elementary Properties of a Multi-Dimensional Generalization of the Euclidean Algorithm},
  Autor                 = {Daniel Knaack},
  Betreuerin            = {Prof. Dr. Kim-Manuel Klein},
  Studiengang           = {Informatik},
  Datum                 = {18. Juni 2025},
  Abstract              = {
    In 1839, Charles Hermite posed the question of whether every real number admits
    a representation as a sequence of integers, periodic if and only if the number
    is a cubic irrational.
    While continued fractions solve this problem for
    quadratic irrationals, no analogous representation is known for higher-degree
    algebraic numbers.
    This thesis explores a new approach to Hermite’s question through the
    construction of multidimensional continued fractions (MDCFs), which extend ordinary
    continued fractions to higher dimensions using a generalization of the
    Euclidean algorithm.

    It is shown that most multidimensional continued fractions converge, and that
    any periodic fraction represents an algebraic number of bounded degree.
    These results establish one direction of Hermite’s question in higher
    dimensions.
    An experimental study complements the theoretical findings, providing
    numerical evidence that a wide class of algebraic numbers -- including many
    cube roots -- give rise to periodic MDCFs.
  },
  Zusammenfassung       = {
    Im Jahr 1839 stellte Charles Hermite die Frage, ob jede reelle Zahl durch eine
    Folge ganzer Zahlen dargestellt werden kann, die genau dann periodisch ist,
    wenn die dargestellte Zahl eine kubisch irrationale Zahl ist.
    Während Kettenbrüche dieses Problem für quadratische Irrationale lösen,
    ist bis heute keine analoge Darstellung für algebraische Zahlen höheren Grades bekannt.
    Diese Arbeit untersucht einen neuen Zugang zu Hermites Frage durch die Konstruktion
    multidimensionaler Kettenbrüche (MDCF), die gewöhnliche Kettenbrüche mithilfe
    einer Verallgemeinerung des euklidischen Algorithmus auf höhere Dimensionen erweitern.

    Es wird gezeigt, dass die meisten multidimensionalen Kettenbrüche konvergieren
    und dass jede periodische Darstellung eine algebraische Zahl beschreiben muss,
    deren Grad nach oben beschränkt ist.
    Diese Ergebnisse beantworten eine Richtung von Hermites Frage in höheren Dimensionen.
    Eine experimentelle Untersuchung ergänzt die theoretischen Resultate und liefert
    numerische Hinweise darauf, dass eine breite Klasse algebraischer Zahlen --
    einschließlich vieler kubischer Wurzeln -- zu periodischen MDCFs führt.
  },
  Numerische Bibliographie,
}

\UzLStyle{pagella contrast design}
\addbibresource{refs.bib}

\usepackage{standalone}
\usepackage{todonotes}
\usepackage{pgf-pie}
\usepackage{pgfplots}
\pgfplotsset{compat=1.18}

\usetikzlibrary{decorations.pathreplacing, calligraphy, intersections, backgrounds, graphdrawing, graphs, calc, 3d}
\usepgflibrary{shadings}
\usegdlibrary{trees}

\newcommand\N{{\mathbb N}}
\newcommand\Z{{\mathbb Z}}
\newcommand\Q{{\mathbb Q}}
\DeclareMathOperator*{\argmin}{arg\,min}
\DeclareMathOperator*{\argmax}{arg\,max}
\DeclareMathOperator*{\lcm}{lcm}
\newcommand\floor[1]{\left\lfloor#1\right\rfloor}
\newcommand\ceil[1]{\left\lceil#1\right\rceil}

\begin{document}

\chapter{Introduction}
\label{ch:intro}

In 1839, Charles Hermite wrote a letter to Jacobi~\cite{Hermite50} about the
representation of real numbers.
He asked whether there exists a representation of the real numbers as a
sequence of integers that is periodic if and only if the represented number
is a cubic irrational, i.e. the root of a cubic polynomial.
Although he posed the question nearly two centuries ago,
it remains unanswered to this day.

The standard way to represent numbers is through decimal notation.
A number is represented as a sequence of digits, which begins with the integer
part and is followed by a (potentially infinite) sequence of digits for the
fractional part.
If the decimal expansion of a number is finite, then the number is clearly rational.
Furthermore, if the decimal expansion is periodic, then the number is also rational.
The same behavior occurs for continued fractions and quadratic irrationals.
Continued fractions are fractions of the form
\[
  a₀ + \cfrac{1}{a₁ + \cfrac{1}{a₂ + \cfrac{1}{⋱}}},
\]
where $a₀, a₁, a₂, …$ are integers.
Every real number has a continued fraction expansion,
which can be constructed using the Euclidean algorithm.
If that continued fraction is periodic, then the number must be a quadratic irrational.
More importantly, the converse is also true:
If a number is a quadratic irrational,
then its continued fraction must be eventually periodic.
Naturally, we may ask whether we can extend this and
find a periodic representation for cubic irrationals.
However, no such representation exists yet.

The interest in a generalization to cubic irrationals
comes from the effectiveness of continued fractions in related fields.
The primary example is Diophantine approximation, where the goal is to
approximate real numbers using rational numbers as closely as possible.
It turns out that the best rational approximations are precisely those provided
by continued fractions.
A generalization of continued fractions could serve a similar role in the field
of simultaneous Diophantine approximation, where the goal is to approximate
multiple real numbers with a single rational vector.

Hermite's original question only applies to cubic irrationals,
but it can be easily generalized to any algebraic number:
Does there exist a representation of the real numbers as a sequence of integers
that is periodic if and only if the represented number is an algebraic number of
degree $d$?
There are two parts to this question.
The first is the representation of real numbers as a sequence of integers.
Each finite subsequence of the representation should give us a rational number,
which approaches the represented number as the sequence grows larger.
The second part is about the periodicity of the integer sequence.
In the original question, the sequence should repeat after some point if and
only if the represented number is a cubic irrational.
The general question asks whether a periodic sequence exists for any algebraic
number with a certain degree.

% ==============================================================================
\section{Background}
\label{sec:jacobi-perron}
% ==============================================================================

Since Hermite originally posed his question to Jacobi, it was Jacobi who first
attempted to answer it.
He developed an algorithm \cite{Jacobi68} inspired by the Euclidean algorithm,
which calculates the greatest common divisor of three numbers instead of two.
At each step,
the algorithm chooses the smallest number and uses it to divide the other two.
In the next triple, the other two numbers are replaced by their remainders.
This process is continued until the greatest common divisor is found.
Later, Oskar Perron generalized Jacobi's method to arbitrary dimensions \cite{Perron07},
resulting in what is now called the Jacobi–Perron Algorithm (JPA).
His algorithm is essentially a generalization of the Euclidean algorithm to $n$ numbers.
At each step, he still chooses the smallest element at each iteration.

Continued fractions are typically constructed using the Gauss transformation,
which is defined as
\begin{align*}
  T(x) = \frac{1}{x - \floor{x}}.
\end{align*}
This transformation is applied repeatedly until $T^m(x) = T^n(x)$ for $m ≠ n$,
in which case the continued fraction is periodic.
For example, $T(\sqrt{2}) = 1/(\sqrt{2} - 1) = \sqrt{2} + 1$
and $T^2(\sqrt{2}) = T(\sqrt{2} + 1) = \sqrt{2} + 1$.
Thus, the continued fraction of $\sqrt{2}$ is periodic after two iterations.

The Jacobi-Perron algorithm uses a similar transform,
which takes a vector $x = (x₁, …, x_d)$ as input and calculates
\begin{align*}
  T(x₁, …, x_d) =
  \left(
  \frac{x_2 - \floor{x_2}}{x_1 - \floor{x_1}},
  \frac{x_3 - \floor{x_3}}{x_1 - \floor{x_1}},
  …,
  \frac{x_d - \floor{x_d}}{x_1 - \floor{x_1}},
  \frac{1}{x_1 - \floor{x_1}}
  \right)
\end{align*}
Again, this transformation is applied repeatedly until $T^m(x) = T^n(x)$ for $m ≠ n$,
at which point it becomes periodic.
Perron showed that if the algorithm becomes periodic,
then each $x_i$ is an algebraic number with degree $d+1$.
However, he was not able to show the other direction.

In an attempt to find an algorithm which solves both directions,
previous work often replaces this transformation with a different one,
which can be considered JPA-type algorithms.
For example, a different transformation could use the second smallest number.
They usually consider only one particular transformation and iterate it until a period has been found.
Despite numerous different transformations being proposed,
Hermite's question remains open.

% ==============================================================================
\section{Contributions of this Thesis}
\label{sec:contributions}
% ==============================================================================

% The other algorithms focus on a single path, whereby the only use their own
% transformation function to find a periodic representation.
Whereas previous algorithms typically consider only a single transformation,
I present a framework that is able to choose any one of these transformations.
This framework, which I call \emph{multidimensional continued fractions} (MCF),
is based on a generalization of the Euclidean algorithm by Klein and
Reuter~\cite{Klein24}.

The main contribution of this thesis is a theoretical analysis of the
multidimensional continued fractions.
This includes two important theorems.
The first theorem shows that many multidimensional continued fractions converge.
This was neglected by Jacobi at first, but it was eventually shown by Perron.
Since then, the main focus has returned to the second part
with the convergence often being implicitly assumed.
However, the convergence is crucial for a correct representation of the real numbers.
For this thesis, I give an explicit proof that many multidimensional continued
fractions converge.
This covers the first part of Hermite's question.
The second theorem partially solves the second part of Hermite's question.
It shows that every periodic multidimensional continued fraction leads to an
algebraic number, thus solving the first direction.
The converse remains open.

Nevertheless, I have performed an experimental analysis on the MCFs,
which strongly suggests that every cubic irrational has a periodic MCF.
The aim of this analysis was to evaluate different strategies
for constructing periodic MCFs for algebraic numbers.
One of these strategies shows particular promise, having produced a periodic
MCF for every cubic and quartic irrational I tested..
In addition, I have evaluated the performance of MCFs in the field of
simultaneous Diophantine approximation.
Since ordinary continued fractions yield the best rational approximations
of a single real number, the idea for MCFs would be that they provide the best
rational approximations for real vectors.
With my analysis, I show that not all convergents lead to good rational approximations.
I provide a specific example where no MCF consistently produces good rational
approximations of a vector.

The basis of these MCFs is a generalization of the Euclidean algorithm from
Klein and Reuter \cite{Klein24}.
The initial aim was to analyze the algorithm's worst-case performance
and to determine whether there exists some generalization of Fibonacci numbers,
as they represent the worst-case for the classical Euclidean algorithm.
As such, a secondary result of this thesis is a proof that such Fibonacci numbers exist, at least for one strategy.
More importantly, I show that they represent the worst case for this strategy.
Using these numbers, I derive the multidimensional analogue of the golden
ratio, which can be seen as one of the simplest examples of a periodic MCF.

In summary, there are three main contributions:
\begin{enumerate}
  \item A new class of multidimensional continued fractions, including a proof
    of convergence and how they lead to algebraic numbers.
  \item An experimental analysis of the multidimensional continued fractions
    on cubic irrationals and their application in simultaneous
    Diophantine approximation.
  \item Worst-case bounds for the generalized Euclidean algorithm
    using higher-order Fibonacci numbers and multidimensional golden ratios.
\end{enumerate}

% ==============================================================================
\section{Related Work}
\label{sec:related-work}
% ==============================================================================

As previously mentioned,
one of the first algorithms studied for this problem is the JPA.
Initially developed by Jacobi and Perron,
the algorithm was analyzed again by Bernstein~\cite{Bernstein71},
who identified explicit classes of cubic irrationals which are periodic under
the JPA \cite{Bernstein64A, Bernstein65, Bernstein64B}.
However, numerical computations by Elsner and Hasse \cite{Elsner67} have shown
that the JPA seems to fail for certain cube roots.
This has led some to conjecture that the JPA does not provide an answer to Hermite's question.

Since then, numerous alternative algorithms have been
proposed~\cite{Assaf05, Hendy81, Schweiger13, Schweiger00}.
However, they usually build upon the Jacobi-Perron algorithm.
This includes the subtractive algorithms by
Brun~\cite{Brun19} and Selmer~\cite{Selmer67}
as well as the fully subtractive algorithms introduced by Schweiger~\cite{Schweiger95}.
Each algorithm is still fundamentally based on the idea of the JPA, though they
do not use the full remainder and instead subtract only the chosen element.
While the main focus has been on periodicity, Selmer's algorithm is one of the
few algorithms, for which convergence has been proven~\cite{Bruin15}.
However, none of them give a full answer to Hermite's question.

There are also generalizations of continued fractions to two dimensions,
called bifurcating~\cite{Gupta00} or ternary continued fractions~\cite{Daus22},
and they are constructed by reversing the JPA in two dimensions.
Traditionally, they only admitted integers as coefficients,
but Murru~\cite{Murru15} used rational coefficients to construct periodic
expansions for all cubic irrationals.
While this addresses one half of Hermite’s question,
it does not provide a representation for arbitrary real numbers.
It only provides a representation of cubic irrationals
and thus, it does not provide a full answer to Hermite's question.

More recently, Karpenkov has proposed two new algorithms \cite{Karpenkov24, Karpenkov21}.
The first, known as the $\sin^2$-algorithm, has been shown to be periodic for
every totally real cubic irrational, i.e. any root of a cubic polynomial
with three real roots.
The second is called the HAPD algorithm, and he conjectures that it is periodic
for all cubic irrationals, though this remains unproven.
Nevertheless, both algorithms are built upon transformations closely related to
those used in the JPA.

Apart from the algorithmic approach,
there has also been significant effort in a geometric generalization of
continued fractions.
Felix Klein famously interpreted the convergents of a continued fraction as
points on integer lattices \cite{Klein95}.
His interpretation has led to a geometric proof of Lagrange’s theorem, which
will also be presented in this thesis, following the work of Korkina~\cite{Korkina96}.
Arnol'd has suggested a generalization of Klein's interpretation to higher
dimensions \cite{Arnold98} and it was conjectured that they satisfy a
multidimensional equivalent of the Lagrange's famous theorem,
which shows that every quadratic irrational has an eventually periodic
continued fraction.
This conjecture was eventually proven by German \cite{German08},
which indicates that there could be a connection between the multidimensional
continued fractions and algebraic numbers.

Beyond multidimensional continued fractions, there have been alternative
generalizations of classical number-theoretic functions.
One notable example is the Minkowski question-mark function $?(x)$,
which maps a quadratic irrational $x$ to a rational number.
Efforts to define higher-dimensional analogues of this function aim to mirror
the relationship between algebraic numbers and their representations.
There has been an extension of this function to two dimensions~\cite{Beaver04},
but this has not yet solved Hermite's question.

% ==============================================================================
\section{Structure of this Thesis}
\label{sec:structure}
% ==============================================================================

Chapter~\ref{ch:preliminaries} introduces the necessary background for this thesis,
which primarily includes algebraic number theory and lattice theory.
Chapter~\ref{ch:quadratic} examines the case of quadratic irrationals and continued fractions.
It presents a geometric interpretation of continued fractions based on Klein polygons,
which is used in the subsequent proof of Lagrange's theorem.
Chapter~\ref{ch:generalized-euclidean} introduces the generalized Euclidean algorithm.
In Chapter~\ref{ch:fibonacci}, the generalized Euclidean algorithm is analyzed
for worst-case performance under two different strategies.
Each strategy results in a different generalization of the Fibonacci numbers
and its own definition of a golden ratio.
This golden ratio is the first case of a periodic representation of an algebraic number.
Building on this result, Chapter~\ref{ch:mdcf} generalizes continued fractions to higher dimensions
and presents the two main theoretical results of the thesis: convergence and periodicity.
Chapter~\ref{ch:implementation} analyzes the second part of Hermite's problem,
whether multidimensional continued fractions of algebraic numbers are always periodic.
This chapter presents examples of such continued fractions for cubic irrationals,
and compares different strategies for constructing these continued fractions.

\chapter{Preliminaries}
\label{ch:preliminaries}

This chapter introduces key concepts needed throughout this thesis.
We begin with the most important topic first,
which is the Euclidean algorithm.
It serves as the basis for continued fractions and its generalization.
We proceed with a brief introduction to lattice theory.
Finally, the chapter ends with the basics of algebraic number theory,
which is needed for the second part of Hermite's problem.

% ==============================================================================
\section{The Euclidean Algorithm}
% ==============================================================================

The input to the algorithm is a pair of integers $(a, b)$ with $a > b$ and the
goal is to find the greatest common divisor between the two inputs.
The algorithm works in two steps.
The first step is to calculate the remainder when $a$ is divided by $b$.
We find an integer $q$ for the quotient and a remainder $r < b$ such that
\[
  a = q b + r.
\]
In the next step, $a$ is exchanged with $b$ and $b$ with $r$.
The algorithm continues with these two steps until the remainder $r$ equals zero.

\begin{example}
  Let $a = 252$ and $b = 105$.
  The Euclidean algorithm proceeds as follows:
  \begin{alignat*}{2}
    252 & = 2 · 105 + 42 \\
    105 & = 2 · 42 + 21 \\
    42 & = 2 · 21 + 0.
  \end{alignat*}
  Thus, the greatest common divisor of $252$ and $105$ is $21$.
\end{example}

\begin{lemma}
  If $a = qb + r$, then $\gcd(a, b) = \gcd(b, r)$.
\end{lemma}

\begin{proof}
  Suppose $d = \gcd(b, r)$, then $b = b'd$ and $r = r'd$ for some integers $b'$ and $r'$.
  But then $d$ must also divide $a$, since $a = qb + r = d(qb' + r')$.
  Thus, $d ≤ \gcd(a, b)$.
  Next, suppose that $d = \gcd(a, b)$.
  Because $d$ divides $a = qb + r$ and $b$,
  it must also divide $r$.
  Hence, $d ≤ \gcd(b, r)$.
  Combining both inequalities, we get $\gcd(a, b) = \gcd(b, r)$.
\end{proof}

Since $r < b$, the input decreases by at least $1$ after each iteration
and the Euclidean algorithm terminates at some point.
Therefore, it correctly calculates the greatest common divisor between the two
inputs $a$ and $b$.
In summary, the algorithm uses the following steps:
\begin{itemize}
  \item \textbf{Modulo}: $b' ← a \bmod b$.
  \item \textbf{Exchange}: $(a, b) ← (b', a)$.
  \item \textbf{Termination}: $b = 0$
\end{itemize}

The Euclidean algorithm not only works on integers,
but can be extended to any ring with a division with remainder operation.
Recall that a ring is a set $R$ equipped with two operations -- addition and
multiplication -- that behaves like the integers:
it forms an abelian group under addition and is closed under multiplication,
which is associative and distributes over addition.
A ring need not have multiplicative inverses or a multiplicative identity.
In this thesis, all rings will contain $1$ and be commutative unless stated
otherwise.

\begin{definition}
  An \emph{integral domain} is a ring $R$ in which $1 ≠ 0$ and for every $a, b ∈ R$,
  \[
    ab = 0 \implies a = 0 \text{ or } b = 0.
  \]
\end{definition}

An integral domain is a ring without zero divisors.
The Euclidean algorithm can be extended to any integral domain,
which defines an additional division with remainder operation.
Such a ring is called an Euclidean domain.

\begin{definition}
  A \emph{Euclidean domain} $(R, f)$ is an integral domain $R$ with a function $f$,
  which maps any nonzero element of $R$ to a nonnegative integer,
  such that for every two elements $a$ and $b$ of $R$, there exist elements $q$ and $r$ of $R$ with
  \[
    a = qb + r \quad \text{ and } \quad r = 0 \text{ or } f(r) < f(b).
  \]
  The element $q$ is called the \emph{quotient} and $r$ is called the \emph{remainder}.
  The function $f$ is called a \emph{Euclidean function}.
\end{definition}

\begin{example}
  \label{ex:real-divmod}
  Consider the real numbers with the Euclidean function $f(r) = r$.
  One possible division just uses the quotient $q = ab^{-1}$.
  However,
  another possible division operation would be
  \[
    a = \underbrace{\floor{\frac{a}{b}}}_q b + \underbrace{\left\{ \frac{a}{b} \right\} b}_r
  \]
\end{example}

\begin{example}
  Consider the ring $ℚ[X]$ of polynomials with rational coefficients.
  For any two polynomials $f(X), g(X) ∈ ℚ[X]$,
  we can find polynomials $q(X), r(X) ∈ ℚ[X]$
  using polynomial division such that
  \[
    f(X) = q(X) g(X) + r(X)
    \quad
    \text{ and }
    \quad
    \deg(r) ≤ \deg(f).
  \]
  In this case, $\deg(f)$ is the Euclidean function and $q(X), r(X)$ are
  the quotient and remainder, respectively.
\end{example}

% ==============================================================================
\section{Lattices}
% ==============================================================================

This section is heavily based on the lecture notes by Rothvoss
\cite{Rothvoss23} and the book by Micciancio and Goldwasser~\cite{Micciancio02}.
We begin with the introduction of lattices.

One of the fundamental structures in linear algebra is the vector space.
Given a set of vectors $A = \{A_1, …, A_n\} ∈ ℝ^d$, the span of $A$ is the set
of all linear combinations of the basis vectors $A_i$.
A lattice is similarly defined over a set of vectors $A$, but instead of linear
combinations of vectors, a lattice consists of only integral linear
combinations.

% TODO: Should we allow real vectors?
\begin{definition}
  Let $A$ be an integer matrix with column vectors $A_1, \dots, A_n ∈ ℤ^d$.
  The \emph{lattice} generated by $A$ is defined as
  \[
    \mathcal{L}(A) = A ℤ^d
    = \left\{\, z_1 A_1 + \dots + z_n A_n \mid z_1, \dots, z_n \in ℤ \,\right\}.
  \]
  The \emph{rank} of $\mathcal{L}(A)$ is defined as the rank of the matrix $A$.
  The \emph{dimension} of the lattice is $d$.
  If $\mathrm{rank}(A) = d$,
  then $\mathcal{L}(A)$ is called a \emph{full-rank lattice} in $ℝ^d$,
  and the columns of $A$ form a \emph{basis} of the lattice.
\end{definition}

Different sets of column vectors can generate the same lattice.
That is, a lattice $L$ can have multiple different bases.
If two matrices $B, B' \in \mathbb{Z}^{d \times d}$ generate the same lattice,
then they must be related by a matrix $U \in \mathbb{Z}^{d \times d}$
such that
\[
  B' = BU.
\]
To ensure that $B$ and $B'$ generate exactly the same set of integer linear combinations,
the matrix $U$ must preserve volume and invertibility over $ℤ$.
This leads to the following definition.

\begin{definition}
  A matrix $U ∈ ℤ^{n×n}$ is \emph{unimodular} if $\det(U) = ±1$.
\end{definition}

\begin{lemma}
  If $U$ is unimodular, then $U^{-1}$ is unimodular.
\end{lemma}

\begin{proof}
  We begin by showing that $U^{-1}$ consists of integer entries.
  Let $U^{(ij)}$ denote the matrix $U$ where the $i$-th column is replaced by the $j$-th unit vector $e_j$.
  Then, $\det(U^{(ij)}) ∈ ℤ$, and by Cramer's rule,
  \[
    U_{ij}^{-1} = \frac{\det(U^{(ij)})}{\det(U)}
  \]
  For the determinant, we have $\det(U^{-1}) = (\det(U))^{-1} = ±1$.
  Thus, the inverse $U^{-1}$ is unimodular.
\end{proof}

\begin{lemma}
  \label{lem:unimodular}
  Let $B$ and $B'$ be two bases.
  The lattices $\mathcal L(B)$ and $\mathcal L(B')$ are the same if and only if
  there exists a unimodular matrix $U ∈ ℤ^{n×n}$ such that $B' = BU$.
\end{lemma}

\begin{proof}
  Suppose $B' = BU$ for some unimodular matrix $U ∈ ℤ^{n×n}$.
  The function $f \colon ℤ^n → ℤ^n, f(x) = Ux$ is a bijection.
  Thus, any vector $y ∈ ℤ^n$ has a corresponding vector $x ∈ ℤ^n$ with $y = Ux$.
  Hence,
  \[
    \mathcal L(B)  = \{\, B λ \mid λ ∈ ℤ^n \,\} = \{\, B' U λ \mid λ ∈ ℤ^n \,\} = \mathcal L(B').
  \]

  Next, suppose $\mathcal L(B) = \mathcal L(B')$.
  Then, any vector in $B$ can be represented in a integral combination of the
  vectors of $B'$ and vice-versa.
  These coefficients can be combined into matrices $U, V ∈ ℤ^{n×n}$ such that $B' = BU$ and $B = B'V$.
  Thus, the determinant of $B$ is
  \[
    \det(B) = \det(BUV) = \det(B) \det(U) \det(V).
  \]
  Since $U$ and $V$ are both integer matrices,
  they must have determinant $±1$.
\end{proof}

\begin{figure}[tbp]
  \centering
  \includegraphics{build/lattice.pdf}
  \caption{
    A two-dimensional lattice $\mathcal L(B)$ with the basis vectors $B_1 = (2, 1)$
    and $B_2 = (1, 3)$.
    The fundamental parallelepiped $Π(B)$ is colored in {\color{cyan}cyan}.
  }
\end{figure}

While different bases may generate the same lattice, the geometric region the
basis spans still captures the same volume and structure.
To study this structure more concretely, we consider the region formed by all
linear combinations of the basis vectors with coefficients in the unit
interval.
This region is a geometric representation of the “unit cell” of the lattice.

\begin{definition}
  The \emph{fundamental parallelepiped} of a lattice $\mathcal{L}(B)$ with a basis $B ∈ ℤ^{d×d}$ is defined as
  \[
    Π(B) = \left\{\, B₁ x₁ + \dots + B_d x_d \mid x_1, \dots, x_d ∈ [0, 1) \,\right\}.
  \]
  The \emph{volume} of this parallelepiped is given by
  \[
    \mathrm{vol}(Π(B)) = |\det(B)|.
  \]
\end{definition}

An important property of the fundamental parallelepiped is that
for $B$ any square integer matrix, both the volume and the number of integer points
inside the parallelepiped $Π(B)$ is entirely determined by $\mathrm{det}(B)$.
Specifically,
\[
  \mathrm{vol}(Π(B)) = |Π(B) ∩ ℤ^n| = |\det(B)|.
\]

% ==============================================================================
\section{Algebraic Numbers}
% ==============================================================================

This section draws on standard results from algebraic number theory, as presented in the book by Dummit \cite{Dummit04}.
For a ring $R$,
the set $R[t]$ denotes all polynomial expressions in $t$,
i.e. expressions of the form $a₀ + a₁ t + ⋯ + aₙ t^n$ with $a₀, …, aₙ ∈ R$.
For a given field $K ⊆ L$ and an subset $S ⊆ L$,
the set $K(S)$ denotes the smallest field containing $K$ and $S$.
If $S = \{α₁, …, α_n\} ⊆ L$, then we simplify the
expression $ℚ(\{α₁, …, αₙ\})$ to $ℚ(α₁, …, αₙ)$.
In this thesis, we consider fields $ℚ(α)$ where the element $α$ is an algebraic number.

\begin{definition}
  A complex number $α$ is said to be \emph{algebraic} if its the root of some polynomial
  with integer coefficients.
  If $α$ is not algebraic, then it is said to be \emph{transcendental}.
\end{definition}

The set of algebraic numbers includes many familiar constants, such as $\sqrt{2}$ or $\frac{1}{2}$,
and plays a central role in number theory.
As it stands, the definition currently only states that it must be a root of some polynomial.
However, we can identify each algebraic number by one unique monic polynomial,
i.e. a polynomial where its leading coefficient is $1$.

\begin{lemma}
  If $α$ is algebraic, then there exists a unique monic polynomial $m_α(X) ∈ ℚ[X]$
  of smallest degree with $α$ as a root.
\end{lemma}

\begin{proof}
  For a contradiction, suppose there exists another monic polynomial $g(X) ≠
  m_α(X)$ with $\deg(g) = \deg(m_α)$.
  Then we can find polynomials $q(X), r(X) ∈ ℚ[X]$ such that
  \[
    g(X) = q(X) m_α(X) + r(X), \quad \text{ with } r(X) = 0 \text{ or } \deg(r) < \deg(m_α).
  \]
  If $r(X) = 0$, then $q(X)$ has to have degree $0$.
  However, then either $q(X) = 1$, which means $m_α(X) = g(X)$ or one of
  the two polynomials is not monic.
  Both options would contradict our assumptions.

  Therefore, we must have $r(X) ≠ 0$.
  Because $m_α(α) = 0$ and $g(α) = 0$, we must also have $r(α) = 0$.
  However, then $r(X)$ would be a polynomial with smaller degree than $f(X)$
  and it would have $α$ as a root, which is a contradiction.
  Thus, $m_α$ is the unique monic polynomial for $α$.
\end{proof}

Every algebraic number is associated with a unique monic polynomial over the rational numbers
of minimal degree, known as its \emph{minimal polynomial}.
Since $m_α(α) = 0$,
we can think of $ℚ(α)$ as the set of rational polynomials in $α$ modulo $m_α$.
Therefore, any polynomial expression in $α$ with degree higher than $\deg(m_α)$
can be replaced by an expression with lower degree.
Hence, every element in $ℚ(α)$ can be expressed as
\[
  a₀ + a₁ α + ⋯ + a_{n-1} α^{n-1},
\]
where $n$ is the degree of the minimal polynomial $m_α$
and the coefficients $a₀, a₁, …, a_{n-1}$ are rational numbers.
Thus,
we can consider the field $ℚ(α)$ as a vector space over $ℚ$
with the set $\{1, α, …, α^{n-1}\}$ as a basis.
Then,
any element in $ℚ(α)$ can be expressed as a linear combination of this basis.
The basis is also known as the \emph{power basis}.

\begin{definition}
  The \emph{degree} $[K : ℚ]$ of a field extension $K/ℚ$ is defined as the
  dimension of $K$ as a vector space over $ℚ$.
  If the dimension is finite, then $K$ is called a \emph{number field}.
\end{definition}

\begin{example}
  \hfill
  \begin{enumerate}
    \item The rational numbers form a number field and they have degree $1$.
    \item The quadratic field $ℚ(\sqrt{2})$ is a number field.
      Any element in $ ℚ(\sqrt{2})$ can be written as $a + b \sqrt{2}$ with $a,b ∈ ℚ$
      and it is therefore a number field with degree $[ℚ(\sqrt{2}) : ℚ] = 2$.
    \item The field $ℚ(\sqrt[3]{2})$ is a number field.
      It contains all elements of the form $a + b \sqrt[3]{2} + c \sqrt[3]{2^2}$ with $a,b,c ∈ ℚ$.
      Therefore, $1, \sqrt[3]{{2}}, \sqrt[3]{4}$ forms a basis for the vector space over $ℚ$
      and the field $ℚ(\sqrt[3]{2})$ has degree $[ℚ(\sqrt[3]{2}) : ℚ] = 3$.
  \end{enumerate}
\end{example}

Multiplication by a fixed element $β$ of a number field $K$ can be viewed as a
linear transformation on its corresponding vector space.
If $\{1, α, …, α^{n-1}\}$ is the basis of the vector space, then we begin by
evaluating the multiplication of $β$ with one element $α^k$ of the basis.
The resulting element can be represented as a linear combination again,
where one column can be calculated as follows:
\[
  β · α^k = \sum_{j=0}^{n-1} m_{kj} α^j \Rightarrow
  \begin{pmatrix}
    m_{k,0} \\
    m_{k,1} \\
    ⋮ \\
    m_{k,n-1} \\
  \end{pmatrix}
\]
for some coefficients $m_{jk} ∈ ℚ$.
Combining each column into one matrix gives us the final multiplication matrix.

\begin{example}
  \label{ex:mult-matrix}
  Consider a quadratic field $K = ℚ(\sqrt{d})$ for some non-perfect square $d$.
  Each element $α ∈ ℚ(\sqrt{d})$ can be represented as $a + b \sqrt{d}$ for some $a, b ∈ ℚ$.
  The multiplication matrix of $α$ is
  \[
    M_α =
    \begin{pmatrix}
      a & bd \\
      b & a \\
    \end{pmatrix},
  \]
  since
  $(a + b \sqrt{d}) · 1 = a + b \sqrt{d}$ and
  $(a + b \sqrt{d}) · \sqrt{d} = bd + a \sqrt{d}$,
  each of which represents one column in the multiplication matrix.
\end{example}

\begin{definition}
  Let $K$ be a number field.
  The \emph{field norm} $N_{K/ℚ}(α)$ of an element $α$ of $K$ is defined as the
  determinant of its multiplication matrix $M_α$.
\end{definition}

\begin{example}
  By Example~\ref{ex:mult-matrix},
  the norm of an element $a + b \sqrt{d} ∈ ℚ(\sqrt{d})$ is
  \[
    N_{K/ℚ}(a + b \sqrt{d}) = a^2 - d b^2.
  \]
\end{example}

While the number field $K = ℚ(α)$ forms an $n$-dimensional vector space using
the power basis, the ring $ℤ[α]$ forms an $n$-dimensional lattice under the same basis.
If $ε ∈ ℤ[α]$ with $N_{K/ℚ}(ε) = 1$, then $ε^{-1} ∈ ℤ[α]$,
since its multiplication matrix is unimodular.
Thus, $ε$ is a unit in the ring $ℤ[α]$.
If we can find a nontrivial unit $ε ≠ 1$,
then we have a nontrivial transformation of the lattice $ℤ[α]$.
Finding such a nontrivial unit requires Dirichlet's unit theorem,
for which we first need the notion of integer rings and orders.

The \emph{integer ring} $\mathcal O_K$ of a number field $K$
contains all elements $x ∈ \mathcal O_K$, which are algebraic integers,
i.e. their minimal polynomial is a monic polynomial with integer coefficients.
In general, algebraic numbers which are the root of a monic polynomial with integer coefficients
are known as \emph{algebraic integers}.
The ring of integers is not necessarily equal to $ℤ[α]$.
For example, in quadratic fields $ℚ(\sqrt{d})$
the ring of integers is equal to $ℤ\left[\frac{1+\sqrt{d}}{2}\right]$ if $d ≡ 1\ (\mathrm{mod}\ 4)$.

\begin{definition}
  Let $K$ be a number field and $\mathcal O_K$ its ring of integers.
  An \emph{order} $\mathcal O$ is a subring of $\mathcal O_K$,
  which is a finitely generated $ℤ$-module,
  i.e. every element $x ∈ \mathcal O$ can be represented as an integral
  combination of a fixed, finite set of elements $x₁, x₂, …, xₙ ∈ \mathcal O$.
\end{definition}

\begin{theorem}[Dirichlet's Unit Theorem]
  \label{thm:dirichlet-unit}
  Let $K$ be a number field with $r₁$ real embeddings and $r₂$ complex embeddings.
  The unit group of an order of $K$ is finitely generated by $r₁ + r₂ - 1$
  independent generators of infinite order.
\end{theorem}

\begin{corollary}
  \label{cor:nontrivial-unit}
  There exists a nontrivial unit in $ℤ[α]$,
  unless $α$ is an imaginary quadratic irrational number.
\end{corollary}

\begin{proof}
  Theorem~\ref{thm:dirichlet-unit} only applies to orders,
  but the ring $ℤ[α]$ is not an order in general.
  If the minimal polynomial of $α$ contains rational coefficients,
  then $ℤ[α]$ cannot be an order, since $α ∈ ℤ[α]$.
  However, we can find a subring of $ℤ[α]$ which is an order.
  Suppose $m_α(x) = x^n + a_{n-1} x^{n-1} + ⋯ + a_1 x + a_0$ is the minimal polynomial of $α$.
  Let $q$ be the common denominator of its coefficients.
  Then, $q α$ is a root of the monic polynomial
  \begin{align*}
    m_{qα}(x) & = x^n + a_{n-1} q x^{n-1} + ⋯ + a_1 q^{n-1} + a_0 q^n,
  \intertext{since}
    m_{qα}(qα)
    & = (qα)^n + a_{n-1} q (qα)^{n-1} + ⋯ + a_1 q^{n-1} (q α) + a_0 q^n \\
    & = q^n \underbrace{(α^n + a_{n-1} α^{n-1} + ⋯ + a_1 α + a_0)}_{= 0}
  \end{align*}
  Thus, $ℤ[qα]$ is an order and Dirichlet's unit theorem applies.
  If $α$ is not an imaginary quadratic irrational, then there exist at least $r₁ + r₂ - 1 > 0$
  generators for the unit group of $ℤ[qα]$.
  Therefore, there exists a nontrivial unit $ε ∈ ℤ[qα] ⊆ ℤ[α]$.
\end{proof}

The corollary tells us that there always exists a unit $ε ∈ ℚ(α)$ with integral
coefficients in the power basis $\{1, α, …, α^{n-1}\}$.
The multiplication matrix of $ε$ is therefore unimodular.
This multiplication matrix actually has the power basis as one of its eigenvectors.
The other eigenvectors are similarly the power basis of the conjugates of $α$,
that is any root $β$ of $m_α(x)$ defines an eigenvector $(1, β, …, β^{n-1})$.

\begin{lemma}
  \label{lem:mult-eig}
  Let $α$ be algebraic and let $α₁, …, α_n ∈ ℂ$ be its conjugates.
  Then, the multiplication matrix of an element $β ∈ ℚ(α)$ has the row vector
  $(1, α, …, α^{n-1})$ as one of its left eigenvectors.
\end{lemma}

\begin{proof}
  The row vector $a_i = (1, α_i, …, α_i^{n-1})$ is a linear form $a_i(x) = a_i x$,
  which maps any vector $x ∈ ℚ^d$ of a corresponding element in $ℚ(α)$ to its
  embedding in the complex plane.
  Thus, it maps each column of the multiplication matrix $M_β$ to its embedding.
  By definition,
  each column is $β · α^k$.
  Thus, $a_i M_β = (β, β α_i, …, β α_i^{n-1}) = β a_i$ and $a_i$ is a left
  eigenvector.
\end{proof}

\begin{theorem}
  \label{thm:unimodular-algebraic}
  Let $α$ be an algebraic number and let $α_1, …, α_n$ be its conjugates.
  There exists a unimodular matrix $U ≠ I$,
  which has $(1, α_i, …, α_i^{n-1})$ as one of its \emph{right} eigenvectors.
\end{theorem}

\begin{proof}
  This follows from Corollary~\ref{cor:nontrivial-unit} and Lemma~\ref{lem:mult-eig}.
  Although the eigenvector of Lemma~\ref{lem:mult-eig} is one the right,
  transposing the multiplication matrix produces a right eigenvector.
\end{proof}

\begin{example}
  \label{ex:sqrt2-unit}
  Consider the quadratic irrational $\sqrt{2}$
  and its conjugate $-\sqrt{2}$.
  One nontrivial unit in $ℚ(\sqrt{2})$ is $ε = 3 + 2\sqrt{2}$
  since it has norm $3^2 - 2 · 2^2 = 1$.
  It has the unimodular multiplication matrix
  \[
    U = \begin{pmatrix}
      3 & 4 \\
      2 & 3 \\
    \end{pmatrix}.
  \]
  Furthermore, it has $(1, \sqrt{2})$ and $(1, -\sqrt{2})$ as its left eigenvectors,
  since
  \begin{align*}
    (1, \sqrt{2})
    \begin{pmatrix}
      3 & 4 \\
      2 & 3 \\
    \end{pmatrix}
    & = (3 + 2\sqrt{2}, 4 + 3\sqrt{2})
    = (3 + 2\sqrt{2}) · (1, \sqrt{2}), \\
    (1, -\sqrt{2})
    \begin{pmatrix}
      3 & 4 \\
      2 & 3 \\
    \end{pmatrix}
    & = (3 - 2\sqrt{2}, 4 - 3\sqrt{2})
    = (3 - 2\sqrt{2}) · (1, -\sqrt{2}).
  \end{align*}
\end{example}

\chapter{Continued Fractions}
\label{ch:quadratic}

% TODO: Find a source for this chapter
We begin with the case of quadratic irrationals.
A \emph{quadratic irrational} is any real number
that is a root of a polynomial with degree $2$.
The main goal of this chapter is to show that the continued fraction expansion
of a number is periodic if and only if the number is a quadratic irrational.
Traditionally, this result is proven using the algebraic properties of the
continued fraction and the minimal polynomial of the quadratic irrational.
In contrast, this chapter takes a geometric approach based on Klein polygons.
These polygons visualize the continued fractions as points in a two-dimensional
integer lattice and reduce the proof to a simple geometric argument based on unimodular matrices.

% ==============================================================================
\section{Definition and Basic Properties}
\label{sec:cf-def}
% ==============================================================================

Although continued fractions are typically defined over a sequence of
positive integers, we extend the definition to allow for real numbers.
This will be useful for the theorems which follow.
A continued fraction will be denoted more compactly as $[r₀; r₁, …]$.

\begin{definition}
  \label{def:cont-frac}
  Given a sequence $(rₙ)_{n≥0}$ of real numbers, the finite continued
  fractions over this sequence are defined inductively as
  \[
    [r₀] = r₀, \qquad
    [r₀; r₁, …, rₙ] = r₀ + \frac{1}{[r₁; r₂, …, rₙ]}.
  \]
  The infinite continued fraction $[r₀; r₁, r₂, …]$ is then defined if the limit
  \[
    r = \lim_{n → ∞} [r₀; r₁, …, rₙ]
  \]
  exists.
\end{definition}

The definition gives us one way to calculate the value of a continued fraction.
The following lemma gives us the other direction,
starting from the last coefficient going right to left:

\begin{lemma}
  \label{lem:cf-nesting}
  Let $r₀, r₁, …, r_n, x ∈ ℝ$, then
  \[
    [r₀; r₁, …, r_n, x] = [r₀; r₁, …, r_n + 1/x]
  \]
\end{lemma}

\begin{proof}
  If $n = 0$, then
  \[
    [r₀; x] = r₀ + \frac{1}{[x]} = r₀ + \frac{1}{x} = [r₀ + 1/x].
  \]
  Suppose the lemma is true for some $n ≥ 0$, then
  \begin{align*}
    [r₀; r₁, …, rₙ, x]
    & = r₀ + \frac{1}{[r₁; r₂, …, rₙ, x]} \\
    & = r₀ + \frac{1}{[r₁; r₂, …, rₙ + 1/x]} \\
    & = [r₀; a₁, …, rₙ, x]. \qedhere
  \end{align*}
\end{proof}

The \emph{$n$-th convergent} of $[r₀; r₁, …]$ is the finite continued fraction $[r₀; r₁, …, rₙ]$.
For a continued fraction $[a₀; a₁, …]$ with integer coefficients,
we denote each convergent $[a₀; a₁, …, aₙ]$ by a unique rational value $p/q$.
This is representation is called the \emph{canonical representation} of the
convergent.
We can define this representation inductively as follows:

The canonical representation of the first convergent is simply
\[
  [a₀] = a₀.
\]
Suppose that the $(n-1)$-th convergent of $[a₁; a₂, …]$ is $p'/q'$,
then the $n$-th convergent of the continued fractions $[a₀; a₁, …]$ is
\[
  [a₀; a₁, …, aₙ]
  = a₀ + \frac{1}{[a₁; a₂, …, aₙ]}
  = a₀ + \frac{q'}{p'}
  = \frac{p_{n-1}' a₀ + q'}{p'}.
\]
Thus, the canonical representation $p/q$ of $[a₀; a₁, …, aₙ]$ is defined as:
\begin{align*}
  p = p' a₀ + q', \quad q = p'.
\end{align*}

Of course, this requires the convergent of a different continued fraction.
Instead, we use a more direct way to calculate the canonical representation $pₙ/qₙ$
from the previous convergents $p_{n-1}/q_{n-1}$ and $p_{n-2}/q_{n-2}$
of the same continued fraction.
We use the following recurrence to calculate $p_n/q_n$:
\begin{align*}
  p_n & = p_{n-1} r_n + p_{n-2}, & p_{-1} & = 1, & p_{-2} & = 0, \\
  q_n & = q_{n-1} r_n + q_{n-2}, & q_{-1} & = 1, & q_{-2} & = 0.
\end{align*}

\begin{lemma}
  \label{lem:cf-wallis}
  Let $x ∈ ℝ$, then
  \[
    [a₀; a₁, …, a_{n-1}, x] = \frac{pₙ}{qₙ} = \frac{p_{n-1} x + p_{n-2}}{q_{n-1} x + q_{n-2}}.
  \]
\end{lemma}

\begin{proof}
  If $n = 0$, then
  \[
    [x] = x = \frac{1x + 0}{0x + 1} = \frac{p_{-1} x + p_{-2}}{q_{-1} x + q_{-2}}.
  \]
  Suppose, the lemma is true for $n ≥ 0$.
  By Lemma~\ref{lem:cf-nesting}, we have
  \begin{align*}
    [a₀; a₁, …, aₙ, x]
    & = [a₀; a₁, …, aₙ + 1/x].
  \end{align*}
  From the induction hypothesis, it follows that
  \begin{align*}
    [a₀; a₁, …, aₙ + 1/x]
    & = \frac{p_{n - 1} (aₙ + 1/x) + p_{n-2}}{q_{n-1} (aₙ + 1/x) + q_{n-2}} \\
    & = \frac{x (p_{n-1} aₙ + p_{n-2}) + p_{n-1}}{x (q_{n-1} aₙ + q_{n-2}) + q_{n-1}} \\
    & = \frac{pₙ x + p_{n-1}}{qₙ x + q_{n-1}}. \qedhere
  \end{align*}
\end{proof}

The canonical representation of a convergent is unique
if we cannot reduce the fraction $pₙ/qₙ$ any further.
In other words, $pₙ$ and $qₙ$ have a greatest common divisor of $1$.
By Bezout's identity, $pₙ$ and $qₙ$ have a greatest common divisor of $1$
if and only if there are integers $a, b$ such that $apₙ + bqₙ = 1$.

\begin{lemma}
  \label{lem:cf-det}
  For every $n ≥ 0$, we have $p_{n-1} q_{n-2} - q_{n-1} p_{n-2} = (-1)^n$.
\end{lemma}

\begin{proof}
  For $n = 0$, we have
  \[
    p_{-1} q_{-2} - q_{-1} p_{-2} = 1 - 0 = 1.
  \]
  Suppose that the lemma holds for $n ≥ -1$, then
  \begin{align*}
    p_n q_{n-1} - q_n p_{n-1}
    & = (p_{n-1} a_n + p_{n-2}) q_{n-1} - (q_{n-1} a_n + q_{n-2}) p_{n-1} \\
    & = p_{n-2} q_{n-1} - q_{n-2} p_{n-1} \\
    & = (-1) (p_{n-1} q_{n-2} - q_{n-1} p_{n-2}) \\
    & = (-1)^{n+1}. \qedhere
  \end{align*}
\end{proof}

% ==============================================================================
\section{Construction Using the Euclidean Algorithm}
\label{sec:cf-construction}
% ==============================================================================

% TODO: Should we jump out of the gate with this problem?
The first part of Hermite's question requires a representation of the real numbers.
This means we have to find a unique continued fraction $[r₀; r₁, …]$ for every real number $x$.
So far, we have allowed continued fractions with real coefficients
but for the construction we use only integer coefficients.
More specifically, we will only look at continued fractions $[a₀; a₁, a₂, …]$
where the first coefficient $a₀$ can be any integer but the remaining ones have to be positive.
These constraints do not guarantee a unique representation alone.
Consider $x = 3/2$, with the current requirements there are two possible representations:
\[
  x = [1; 1, 1] = 1 + \cfrac{1}{1 + \cfrac{1}{1}} \qquad \text{ or } \qquad x = [1; 2] = 1 + \cfrac{1}{2}.
\]
The issue is that we can always split the last coefficient in the continued fraction.
In general, if $x = [a₀; a₁, …, aₙ]$, then also $x = [a₀; a₁, …, aₙ - 1, 1]$.
Therefore, we additionally require that in a finite continued fraction the last value is never $1$.

We begin with the representation of rational numbers.
We use the Euclidean algorithm to construct the continued fraction for a number $x ∈ ℚ$.
More specifically, if $x = p/q$, then we run the algorithm with the input pair $(p, q)$.
At each step, we keep track of the quotient,
which will be used as a coefficient in the continued fraction.
So if we calculate $p = a₀q + r$ in the first division step, then the quotient $a₀$ will be
the first coefficient in the continued fraction $[a₀; a₁, …, aₙ]$ for $p/q$.
This is also the reason why we allow $a₀$ to be negative.
If the fraction $p/q$ is negative, then $a₀$ is negative, too.

\begin{example}
  \label{ex:euclidean-cf}
  Consider $x = 13/5$.
  The Euclidean algorithm computes
  \begin{align*}
    13 & = 2 · 5 + 3 \\
     5 & = 1 · 3 + 2 \\
     3 & = 1 · 2 + 1 \\
     2 & = 2 · 1 + 0.
  \end{align*}
  The quotient in each line correspond directly to the continued fraction of $13/5$:
  \[
    \frac{13}{5}
    = [2; 1, 1, 2]
    = 2 + \cfrac{1}{1 + \cfrac{1}{1 + \cfrac{1}{2}}}
    = 2 + \cfrac{1}{1 + \cfrac{2}{3}}
    = 2 + \cfrac{3}{5}
    = \frac{13}{5}.
  \]
\end{example}

\begin{lemma}
  \label{lem:cf-rat}
  Every rational number has a finite continued fraction.
\end{lemma}

\begin{proof}
  Let $p/q$ be the reduced fraction of a rational number, i.e. $\gcd(p, q) = 1$.
  We proceed by induction over the number of steps when running the
  Euclidean algorithm on $(p, q)$.
  Suppose only one step is required, then $p = a₀ q + 0$.
  Because $\gcd(p, q) = 1$, the fraction $p/q$ must be an integer.
  Hence, the continued fraction $[a₀]$ represents $p/q$ correctly.

  Next, suppose that there exists a valid continued fraction for any rational
  number which requires $n$ steps in the Euclidean algorithm.
  Let $p = a₀ q + r$ be the first step of the Euclidean algorithm
  and suppose $(p, q)$ requires $n+1$ steps.
  Then, $(q, r)$ requires $n$ steps and by induction there exists a continued
  fraction $[a₁; a₂, …, aₙ]$ for $q/r$.
  Using this fact, we can construct the continued fraction for $p/q$, since
  \[
    \frac{p}{q}
    = a₀ + \frac{r}{q}
    = a₀ + \frac{1}{\frac{r}{q}}
    = a₀ + \frac{1}{[a₁; a₂, …, aₙ]}
    = [a₀; a₁, …, aₙ].
  \]
  Therefore, $[a₀; a₁, …, aₙ]$ is a valid continued fraction for $p/q$.
\end{proof}

So for every rational number, we have a finite continued fraction.
It remains to be shown that there exists a continued fraction for every irrational number.
However, we cannot use the Euclidean algorithm again.
In order to run the Euclidean algorithm for an irrational number $x$,
we would have to find two integers $p, q$ such that $p/q = x$
and that is by definition not possible.
Therefore, we cannot use the Euclidean algorithm directly to construct a
continued fraction.

Instead, we take a different approach.
For a given number $x₀ ∈ ℝ$,
we begin the algorithm with the input $(x₀, 1)$
and we use the division from Example~\vref{ex:real-divmod}, i.e.
we split $x$ into its integer and fractional part:
\[
  x = \floor{x} · 1 + \{x\}
\]
In the next iteration, we multiply the pair $(1, \{x\})$ by $1/\{x\}$
such that we have $(1/\{x\}, 1)$.
Then, the next input pair is $(x₁, 1) = (1/\{x₀\}, 1)$
and  we repeat the division step:
\[
  \frac{1}{\{x\}} = \floor{\frac{1}{\{x\}}} + \left\{ \frac{1}{\{x\}} \right\}.
\]
We continue this process until $xₙ = 0$.
At each step, we take the integer part $aₙ = \floor{xₙ}$ as one coefficient in
the continued fraction $[a₀; a₁, …]$.
The values $xₙ$ are also called \emph{complete quotients} of $x$.

\begin{example}
  \label{ex:cf-rat}
  Consider $x = 13/5$, again.
  The new algorithm computes
  \begin{align*}
    \frac{13}{5} = 2 + \frac{3}{5}
    \quad \Rightarrow \quad
    \frac{5}{3} = 1 + \frac{2}{3}
    \quad \Rightarrow \quad
    \frac{3}{2} = 1 + \frac{1}{2}
    \quad \Rightarrow \quad
    \frac{2}{1} = 2 + 0.
  \end{align*}
  Thus, we get the same continued fraction $[2; 1, 1, 2]$ for $13/5$
  as in Example~\ref{ex:euclidean-cf}.
\end{example}

\begin{example}
  \label{ex:cf-irrat}
  Consider $x = \sqrt{2}$, which has an integer part of $1$.
  The first iteration of the algorithm results in
  \[
    \sqrt{2} = 1 + (\sqrt{2} - 1).
  \]
  The inverse of $\sqrt{2} - 1$ is $\sqrt{2} + 1$,
  since $(\sqrt{2} - 1)(\sqrt{2} + 1) = 2 - 1 = 1$.
  Thus, the next input pair is $(\sqrt{2} + 1, 1)$ and the algorithm computes
  \[
    \sqrt{2} = 2 + (\sqrt{2} - 1).
  \]
  After this step,
  the algorithm cycles.
  Therefore, the continued fraction for $\sqrt{2}$ is $[1; 2, 2, …]$.
\end{example}

% TODO: Explain that we need to show that it converges
The next step is to show that the constructed continued fraction actually
produces a correct representation for the number.
By definition, a number $x ∈ ℝ$ is represented by a continued fraction $[a₀;
a₁, …]$ if and only if the limit
\[
  x = \lim_{n → ∞} [a₀; a₁, …, aₙ]
\]
exists.
In other words, the convergents must approach the number $x$ as $n$ increases.
Thus, the next step is to prove that convergents actually approach the number $x$.

Lemma~\ref{lem:cf-det} already gives us a hint towards the convergence of
continued fractions, because dividing the equation by $q_n q_{n-1}$ results in
\[
  \frac{p_n}{q_n} - \frac{p_{n-1}}{q_{n-1}} = \frac{(-1)^{n+1}}{q_n q_{n-1}}.
\]
Since $aₙ > 0$ for $n ≥ 1$, the denominators $q_n$ and $q_{n-1}$ are always
increasing after some point.
Consequently, the distance between consecutive convergents is decreasing.
But we need to show that the convergents actually approach the represented number.
For the proof, we use the fact that the algorithm always guarantees $xₙ - aₙ < 1$.

\begin{lemma}
  \label{lem:cf-approx}
  Let $x ∈ ℝ$ and let $[a₀; a₁, …]$ be its continued fraction expansion.
  If $pₙ/qₙ$ is the $n$-th convergent, then
  \[
    \left| x - \frac{pₙ}{qₙ} \right| < \frac{1}{qₙ^2}.
  \]
\end{lemma}

% TODO: Have we proven that (p_{n-1} q_n - p_n q_{n-1}) = (-1)^n yet?
\begin{proof}
  Let $x_n = [a_n; a_{n+1}, …]$.
  Then, $x = [a₀; a₁, …, a_{n-1}, x_n]$ and $a_n = \floor{x_n}$.
  Using Lemma~\ref{lem:cf-wallis},
  we can represent $x$ as well as $p_n/q_n$ using the previous convergents
  $p_{n-1}/q_{n-1}$ and $p_{n-2}/q_{n-2}$ together with the coefficients $x_n$
  and $a_n$, respectively.
  Hence,
  \begin{align*}
    \left| x - \frac{pₙ}{qₙ} \right|
    & = \left| \frac{x_n p_{n-1} + p_{n-2}}{x_n q_{n-1} + q_{n-2}} - \frac{a_n p_{n-1} + p_{n-2}}{a_n q_{n-1} + q_{n-2}} \right| \\
    & = \left| \frac{(x_n p_{n-1} + p_{n-2})(a_n q_{n-1} + q_{n-2}) - (x_n q_{n-1} + q_{n-2})(a_n p_{n-1} + p_{n-2})}{(x_n q_{n-1} + q_{n-2})(a_n q_{n-1} + q_{n-2})} \right| \\
    & = \left| \frac{(p_{n-1} q_{n-2} - q_{n-1} p_{n-2})(x_n - a_n)}{((x_n - a_n) q_{n-1} + q_n) q_n} \right|.
  \end{align*}
  From Lemma~\ref{lem:cf-det} it follows that
  \begin{align*}
    \left| x - \frac{pₙ}{qₙ} \right|
    & = \Biggl| \frac{(-1)^{n+1} \overbrace{(x_n - a_n)}^{≤ 1}}{q_n^2 + \underbrace{(x_n - a_n)}_{≥ 0} q_{n-1} q_n} \Biggr| < \frac{1}{q_n^2}. \qedhere
  \end{align*}
\end{proof}

% TODO: Mention that we can also prove the other direction
We now have a continued fraction for both rational number and irrational numbers.
What remains to be shown is that the continued fractions are unique.
This gives us a unique representation for every real number $x ∈ ℝ$ by a
continued fraction $[a₀; a₁, …]$.

\begin{theorem}
  \label{thm:irrat-cf}
  Every real number $x$ has a unique continued fraction.
\end{theorem}

\begin{proof}
  From the previous considerations, it follows that for every number $x ∈ ℝ$,
  there exists a continued fraction $[a₀; a₁, …]$ such that $[a₀; a₁, …] = x$.
  Because the continued fraction $[a₀; a₁, …]$ is made up of integers and the
  ones after $a₀$ are all positive, its integer part is entirely determined by
  the first coefficient $a₀$ and its fractional part is determined by $[0; a₁, a₂, …]$.
  Although, there is one special case of $[0; 1] = 1$,
  this cannot be constructed using the algorithm.
  The algorithm would have to begin with a fractional part of $1$, which is impossible.

  Now, suppose there is a different continued fraction $[b₀; b₁, …]$ with $[b₀; b₁, …] = x$.
  Because $[0; a₁, a₂, …]$ and $[0; b₁, b₂, …]$ both lie between $0$ and $1$,
  the continued fractions must share the same first coefficient $a₀ = b₀$.
  Otherwise, they would not have the same integer part and represent different numbers.
  By induction, suppose that the first $n ≥ 0$ terms are equal.
  Then, the continued fractions $[a_{n+1}; a_{n+2}, …]$ and $[b_{n+1}; b_{n+2}, …]$ must be equal.
  But by the same argument, we have $a_{n+1} = b_{n+1}$.
  Therefore, the continued fraction $[a₀; a₁, …]$ is unique.
\end{proof}

This concludes the first part for Hermite's question.
We now have a representation of the real numbers using continued fractions.
The second part is about periodic continued fractions and quadratic irrationals.
But before we look into that, we will look into the geometry behind the continued fractions
to understand them better.
The geometrical view will also be useful for the second part of Hermite's question.

% ==============================================================================
\section{Geometrical Interpretation Based on Klein Polygons}
\label{sec:cf-geometry}
% ==============================================================================

The geometry of continued fraction was first analyzed by Klein \cite{Klein95}.
He viewed the convergents $p_n/q_n$ of a continued fraction $[a₀; a₁, …]$ not
as rational numbers on a one-dimensional number line,
but as a convergent \emph{vector} $bₙ = (pₙ, qₙ)$ in the two-dimensional integer lattice $ℤ^2$.
Klein has shown many geometrical analogues of important theorems for continued fraction
using these vectors.
However, we will focus on a small subset, which will be useful for the
continued fractions of quadratic irrationals.

\begin{figure}[tb]
  \centering
  \includestandalone{figures/klein-polygon}
  \caption{
    The convergents $pₙ/qₙ$ of $\sqrt{2}$ as vectors $bₙ = (pₙ, qₙ)^⊤$.
    The even and odd convergents form two different polygonal chains which
    approach the irrational line given by $x/y = \sqrt{2}$.
  }
  \label{fig:klein-polygon}
\end{figure}

Figure~\ref{fig:klein-polygon} shows an example with $\sqrt{2}$ and its convergent vectors $bₙ$.
The figure clearly shows that there are two different polygonal chains.
One formed by the even convergents and one formed by the odd convergents.
These chains both seemingly approach a line where $x/y = \sqrt{2}$.
The reason for this follows from the convergence of continued fractions proven in Lemma~\ref{lem:cf-approx}.
If we scale down each convergent $(pₙ, qₙ)$ by its second coordinate,
then we get the vector $(pₙ/qₙ, 1)$.
This vector cannot be further away than $1/qₙ^2$ from the vector $(\sqrt{2}, 1)$.
Therefore, scaling things back up results in a distance $1/qₙ$, which decreases as $n$ increases.
So convergence in this geometrical interpretation means that the vectors are
approaching a line spanned by the vector $(x, 1)$.

Convergence is one property which can be interpreted geometrically.
There is also a geometrical interpretation for the definition of continued fractions themselves.
Suppose we have the vector $\tilde b_{n-1} = (\tilde p_{n-1}, \tilde q_{n-1})^⊤$
for the continued fraction $[a₁; a₂, …, aₙ]$.
To calculate the convergent of $[a₀; a₁, …, aₙ]$,
we would calculate the reciprocal of $\tilde p_{n-1} / \tilde q_{n-1}$ first
and then add the integer part $a₀$ to it.
The equivalent of calculating the reciprocal in the vector space is swapping the two coordinates
since $\tilde b_{n-1} = (\tilde p_{n-1}, \tilde q_{n-1})^⊤$ is the vector for $\tilde p_{n-1} / \tilde q_{n-1}$
and therefore $(\tilde q_{n-1}, \tilde p_{n-1})^⊤$ must be the vector for $\tilde q_{n-1} / \tilde p_{n-1}$.
The second part is adding some integer part $a₀$ to the reciprocal.
This is equivalent to adding $a₀ q$ to the first coordinates
or skewing the vector using linear transformation $S$.
In fact, the reciprocal can also be done by a matrix transformation $R$.
These matrices are defined as follows:
\[
  S^a =
  \begin{pmatrix}
    1 & 1 \\
    0 & 1 \\
  \end{pmatrix}^a
  =
  \begin{pmatrix}
    1 & a \\
    0 & 1 \\
  \end{pmatrix},
  \quad
  R =
  \begin{pmatrix}
    0 & 1 \\
    1 & 0 \\
  \end{pmatrix}.
\]
The vector $b_n$ can then be calculated using
\[
  b_n = S^{a_0} R \tilde b_{n-1}.
\]

For the continued fractions, we quickly moved from the definition to Lemma~\ref{lem:cf-wallis},
which tells us how to calculate the convergent $pₙ/qₙ$ using $p_{n-1}/q_{n-1}$ and $p_{n-2}/q_{n-2}$.
Naturally, we can ask whether this has a nice geometrical interpretation, too.
First, we can easily derive a formula to calculate the vector $bₙ$, since
\begin{align*}
  b_n =
  \begin{pmatrix}
    p_n \\ q_n
  \end{pmatrix}
  =
  \begin{pmatrix}
    p_{n-1} a_n + p_{n-2} \\ q_{n-1} a_n + q_{n-2}
  \end{pmatrix}
  =
  \begin{pmatrix}
    p_{n-1} \\ q_{n-1}
  \end{pmatrix}
  a_n
  +
  \begin{pmatrix}
    p_{n-2} \\ q_{n-2}
  \end{pmatrix}
  = b_{n-1} a_n + b_{n-2}.
\end{align*}
However, there is also an explanation for the choice of the coefficients $aₙ$.
If we view $aₙ$ as a variable between $0$ and infinity, then this forms a ray
starting at $b_{n-2}$ and going in the direction of $b_{n-1}$.
This line must intersect the line spanned by $(x, 1)$ at some point
and we choose $aₙ$ such that $bₙ$ comes just before the intersection.
Importantly, we can use the complete quotient $xₙ$ to calculate the intersection point:
\[
  λ
  \begin{pmatrix}
    x \\
    1 \\
  \end{pmatrix}
  =
  b_{n-1} x_n + b_{n-2}
  \iff
  \frac{λ x}{λ} = \frac{p_{n-1} x_n + p_{n-2}}{q_{n-1} x_n + q_{n-2}}.
\]

For the first new lemma, we return to what we saw in
Figure~\ref{fig:klein-polygon} and show that even and odd convergent lie on
different sides of the line spanned by $(x, 1)$.
In terms of rational numbers,
this means that the convergents alternate between being less than $x$ and being
greater than $x$.

\begin{lemma}
  \label{lem:klein-conv}
  Even and odd convergent lie on different sides of the line spanned by $(x, 1)$.
\end{lemma}

\begin{proof}
  We proceed via induction on $n$.
  For the first two vectors $b_{-2}$ and $b_{-1}$, this is obviously true.
  Now suppose, that $b_{n-1}$ and $b_{n-2}$ lie in different cones.
  The next vector $b_n$ can be calculated according to
  \[
    b_n = b_{n-1} a_n + b_{n-2}.
  \]
  Furthermore, we can find an intersection with the line spanned $(x, 1)$ using
  the complete quotient $xₙ$ of $x$:
  \[
    λ
    \begin{pmatrix}
      x \\
      1 \\
    \end{pmatrix}
    = b_{n-1} xₙ + b_{n-2}.
  \]
  By the induction hypothesis, the vector $b_{n-2}$ is on a different side than $b_{n-1}$.
  However, $a_n$ is smaller than $x_n$, so $b_n = b_{n-1} a_n + b_{n-2}$ never crosses the line.
  Therefore, $b_n$ is on the same side as $b_{n-2}$.
\end{proof}

If we combine the previous two vectors in a matrix $B_n = \begin{pmatrix}
  b_{n-1} & b_{n-2} \\
\end{pmatrix}$,
then we can calculate the next matrix by multiplying with the matrices $S$ and $R$, since
\[
  B_n S^{a_n} R =
  \begin{pmatrix}
    p_{n-1} & p_{n-2} \\
    q_{n-1} & q_{n-2} \\
  \end{pmatrix}
  \begin{pmatrix}
    a_n & 1 \\
    1   & 0 \\
  \end{pmatrix}
  =
  \begin{pmatrix}
    p_{n-1} a_n + p_{n-2} & p_{n-1} \\
    q_{n-1} a_n + q_{n-2} & q_{n-1} \\
  \end{pmatrix}
  =
  B_n.
\]
Because $\det(S^a) = \det(S)^a = 1$ and $\det(R) = -1$, it is straightforward to
see that $B_n$ always has determinant $±1$.
Geometrically, this means that there is no integer point inside the
parallelogram spanned by $b_{n-1}$ and $b_{n-2}$.
This is the geometrical analogue of Lemma~\ref{lem:cf-det},
because $\det(B_n) = p_{n-1} q_{n-2} - p_{n-2} q_{n-1}$.

\begin{lemma}
  \label{lem:klein-empty}
  Between the polygonal chains of the even and odd convergent vectors,
  there exists no nonzero integer point.
\end{lemma}

\begin{proof}
  We can cover the area between the chains
  using parallelograms made up of the points $b_n$, $b_{n-1}, b_{n-2}$, and a suitable fourth point.
  Each parallelogram contains exactly
  \begin{align*}
    \left|\det\begin{pmatrix}
      b_n - b_{n-2} & b_{n-1} - b_{n-2}
    \end{pmatrix}\right|
    & = \left|\det\begin{pmatrix}
      a_n b_{n-1} & b_{n-1} - b_{n-2}
    \end{pmatrix}\right| \\
    & = \left|\det\begin{pmatrix}
      a_n b_{n-1} & -b_{n-2}
    \end{pmatrix}\right| \\
    & = a_n
  \end{align*}
  integer points.
  However, these points all lie on the line between $b_n$ and $b_{n-2}$.
  Therefore, the area between the chains cannot contain any integer point.
\end{proof}

\begin{definition}
  \label{def:klein-polygon}
  Let $v₁, v₂ ∈ ℝ^2$ be two linearly independent vectors.
  Consider the cone $C$ in $ℝ^2$ bounded by the two rays of $v₁, v₂$. That is
  \[
    C = \{\, λ₁ v₁ + λ₂ v₂ \mid λ₁, λ₂ ≥ 0 \,\}.
  \]
  The \emph{Klein polygon} $K$ spanned by $v₁, v₂$ is defined as the convex hull
  of all nonzero integer points inside the cone, i.e.
  \[
    K = \mathrm{conv}(C ∩ ℤ^2 \setminus \{(0, 0)\}).
  \]
  The \emph{boundary} of a Klein polygon is denoted as $Π(K)$.
\end{definition}

The even and odd convergent vectors are part of two such Klein polygons.
The even convergents are part of the Klein polygon $K_0$ spanned by $(0, 1)$ and $(x, 1)$,
and the odd convergents are part of the Klein polygon $K_1$ spanned by $(x, 1)$ and $(1, 0)$.
This is shown in the following theorem.

\begin{theorem}
  The Klein polygons $K_0$ and $K_1$ are equal to the
  convex hull of the even and odd convergents, respectively.
\end{theorem}

\begin{proof}
  Suppose there is a point in the Klein polygons,
  which is not in the convex hull of either the even or odd convergents.
  This means that the point must lie between the two polygonal chains,
  which is impossible by Lemma~\ref{lem:klein-empty}.
  Now suppose there is a point in the convex hull of the convergents,
  but not in one of the Klein polygons.
  Then, there must be one convergent outside the Klein polygon,
  which is closer to the line $(x, 1)$.
  However, this integer point is still inside the cone,
  so it is also contained in the Klein polygon.
\end{proof}

\begin{corollary}
  The vertices of $K_0$ and $K_1$ are equal to the
  even and odd convergents, respectively.
\end{corollary}

% ==============================================================================
\section{Quadratic Irrationals and Periodicity}
\label{sec:cf-quadratic}
% ==============================================================================

% TODO: Write introduction for this section
This section is about the second part of Hermite's question for quadratic
irrationals, i.e. whether the continued fraction of a number is periodic if and only
if the number is a quadratic irrational.
Formally, we call a continued fraction $[a₀; a₁, …]$ \emph{periodic}
if there exists a starting index $k₀ ≥ 0$ and a period $ℓ ≥ 1$ such that $aₖ = a_{k+ℓ}$ for every $k ≥ k₀$.
A continued fraction is called \emph{purely periodic} if $k₀ = 0$,
i.e. the period starts immediately.
For a periodic continued fraction starting at $K$ with length $ℓ$,
we will denote it as $[a₀; a₁, …, a_{k₀-1}, \overline{a_{k₀}, …, a_{k₀+ℓ}}]$.
This is similar to how in decimal notation, we denote a period with a bar over the digits,
e.g. $1/3 = 0.\overline{3}$.
In a continued fraction, we similarly denote the period with a line over the
coefficients that are infinitely repeated.
For example, $\sqrt{2} = [1; \overline{2}]$.

For the proof, we have to show two directions.
The first is that every periodic continued fraction is a quadratic irrational
and the second is that every quadratic irrational has a periodic continued fraction.
We begin with the former and we will use the geometry from the previous section
in the proof.

\begin{theorem}
  If the continued fraction of $x ∈ ℝ$ is periodic, then $x$ is a quadratic irrational.
\end{theorem}

\begin{proof}
  Let $x$ be a purely periodic continued fraction $[a₀; a₁, …]$.
  Then, there exists a complete quotient such that $x = x_ℓ = [a_ℓ; a_{ℓ+1}, …]$ for some $ℓ ≥ 1$.
  For each pair of convergent vectors $b_{n-1}, b_{n-2}$, we can use the
  complete quotient $x_ℓ$ to find an intersection point with the line $(x, 1)$
  and since $x_ℓ = x$, we can also use $x$ itself.
  Stated differently, there exists a scalar $λ ∈ ℝ$ such that
  \[
    B_ℓ
    \begin{pmatrix}
      x \\
      1 \\
    \end{pmatrix}
    =
    b_{ℓ-1} x + b_{ℓ-2}
    = λ
    \begin{pmatrix}
      x \\
      1 \\
    \end{pmatrix}
  \]
  Therefore, $(x, 1)^⊤$ is an eigenvector of the matrix $B_ℓ$
  and $λ$ is the eigenvalue.
  Because $B$ is a 2-by-2 matrix,
  the eigenvalue can only be a quadratic irrational.
  The eigenvector is a solution to the linear system $(B_ℓ - λ I) (x, 1)^⊤ = 0$,
  where the coefficients come from the field $ℚ(λ)$.
  Therefore, $x$ must be a quadratic irrational.

  We proceed with the eventually-periodic case.
  Suppose $x$ is periodic only after some index $k ≥ 0$,
  then $xₖ = [a_k; a_{k+1}, …]$ is a quadratic irrational.
  By Lemma~\ref{lem:cf-wallis},
  \[
    x = \frac{p_{k-1} x_k + p_{k-2}}{q_{k-1} x_k + q_{k-2}}.
  \]
  Because $x$ is a rational expression of $x_k$ with integer coefficients,
  it lives in the same field as $x_k$.
  Therefore, $x$ is a quadratic irrational, too.
\end{proof}

% TODO: Figure for shift of Klein polygon
The converse was originally proven by Lagrange \cite{Lagrange70}.
Here, a proof by Korkina \cite{Korkina96} is presented,
which uses the geometrical interpretation of Klein to show periodicity.
The idea is that for quadratic irrationals,
there always exists a unimodular matrix $U$ which shifts the corner points of a
Klein polygon along its boundary and preserves volume.
Because the volume between $b_n$ and $b_{n-2}$ directly corresponds to
the coefficients $a_0, a_1, a_2, …$ of the continued fraction,
they must repeat at some point.

\begin{figure}[tb]
  \centering
  \includestandalone{figures/full-klein-polygon}
  \caption{
    The four Klein polygons spanned by $±(\sqrt{2}, 1)$ and $±(-\sqrt{2}, 1)$.
  }
  \label{fig:full-klein-polygon}
\end{figure}

For the proof, we not only consider the Klein polygon in the positive quadrant,
but in all four quadrants.
Instead of the vectors $(1, 0)$ and $(0, 1)$, we use the conjugate
$\overline{x}$ and the vector $(\overline{x}, 1)$.
There are now four different Klein polygons each spanned by $±(x, 1)$
and $±(-\overline{x}, 1)$.
An example is shown in Figure~\ref{fig:full-klein-polygon}.
The idea behind the proof is that the matrix $U$ does not change any of the four
Klein polygons, but it is not the identity and therefore it must move the
points along the boundary.

\begin{theorem}
  If $x$ is a quadratic irrational,
  then its continued fraction is periodic.
\end{theorem}

\begin{proof}
  Let $\overline{x}$ denote the conjugate of $x$.
  By Theorem~\ref{thm:unimodular-algebraic}, we can find a unimodular matrix $U$ for $x$,
  which has $(x, 1)^⊤$ and $(\overline{x}, 1)^⊤$ as eigenvectors.
  Applying $U$ on the Klein polygon, we have
  \begin{align*}
    UK
    = U \mathrm{conv}(C ∩ ℤ^2 \setminus \{(0, 0)\})
    = \mathrm{conv}(UC ∩ Uℤ^2 \setminus \{(0, 0)\}).
  \end{align*}
  The matrix $U$ is unimodular, so by Lemma~\ref{lem:unimodular} we have $Uℤ^2 = ℤ^2$.
  Let $v ∈ C$.
  By definition of $C$,
  there exist coefficients $c₁, c₂ ≥ 0$ such that $v = c₁ (x, 1)^⊤ + c₂ (\overline{x}, 1)^⊤$.
  Because the vectors of $C$ are eigenvectors of $U$, we have
  \begin{align*}
    U v
    = c₁
    U\begin{pmatrix}
      x \\
      1 \\
    \end{pmatrix}
    +
    c₂ U\begin{pmatrix}
      x \\
      1 \\
    \end{pmatrix}
    = c₁ λ₁ \begin{pmatrix}
      x \\
      1 \\
    \end{pmatrix}
    + c₂ λ₂ \begin{pmatrix}
      x \\
      1 \\
    \end{pmatrix}
  \end{align*}
  Furthermore, the eigenvalues must be positive.
  If not, then we simply apply $U$ twice.
  Thus, $UC = C$ and $K$ must be invariant under $U$.

  Similarly, the boundary $Π(K)$ must also be invariant under this transformation.
  Because $U$ is a linear transformation which preserves the orientation of
  vectors and is not the identity, it must shift the integer points along the
  boundary.
  We assume that it shifts it in the positive direction,
  i.e. $B_{n+k} = U B_n$ for some $k ≥ 1$.
  If not, then we can choose $U^{-1}$ to shift them in the positive direction.
  Because $\det(U) = 1$, the matrix also preserves volume and
  \[
    a_{n+k}
    = \det\begin{pmatrix}
      b_{n+k} & b_{n+k-2}
    \end{pmatrix}
    = \det(U) \det\begin{pmatrix}
      b_n & b_{n-2}
    \end{pmatrix}
    = a_n.
  \]
  Hence, the continued fraction is periodic after some point.
\end{proof}

% TODO: Fix this example
\begin{example}
  Consider the continued fraction $[1; \overline{2}]$ of $\sqrt{2}$.
  Even if we only know the first few terms of the continued fraction, i.e. that
  $\sqrt{2} ≈ [1; 2]$, we can still deduce that it must be periodic.
  We use the matrix representation to calculate the second convergent
  \[
    B_2 = B_1 S^2 R = B_0 S^1 R (S^2 R)^{n-1} = SRS · S^1 R (S^2 R)^{n-2}.
  \]
  The first three matrices in this product are
  \[
    SRS =
    \begin{pmatrix}
      1 & 1 \\
      0 & 1 \\
    \end{pmatrix}
    \begin{pmatrix}
      1 & 0 \\
      0 & 1 \\
    \end{pmatrix}
    \begin{pmatrix}
      1 & 1 \\
      0 & 1 \\
    \end{pmatrix}
    =
    \begin{pmatrix}
      1 & 2 \\
      1 & 1 \\
    \end{pmatrix}.
  \]
  This matrix is the mutliplication matrix for $1 + \sqrt{2}$.
  In Example~\vref{ex:sqrt2-unit},
  we have already seen that this matrix has $(±\sqrt{2}, 1)$ as eigenvectors.
  Therefore, if $B_2$ is a convergent, then so is $U^n B_2$.
  Thus, the continued fraction of $\sqrt{2}$ is periodic.
\end{example}

\chapter{The Generalized Euclidean Algorithm}

In this chapter, we look at the generalized version of the Euclidean algorithm \cite{Klein24}.
While the original algorithm works on numbers,
the generalized version works with vectors.
More specifically, the generalized version works on lattices.
For this chapter, we proceed analogously to the Euclidean algorithm.
First, we look at how the generalized algorithm works and then use it to find a
higher-dimensional analogue to the Fibonacci numbers and the golden ratio.
Using the golden ratio, we can naturally extend this generalized algorithm to
the real numbers, just like the original single-dimensional algorithm.

\section{Basics of Lattice Theory}

\begin{figure}[b]
  \centering
  \includestandalone{figures/lattice}
  \caption{A two-dimensional lattice with vectors $B_1 = (2, 1)$ and $B_2 = (1, 3)$.}
\end{figure}

\begin{itemize}
  \item Vector space as the linear combination over a basis
  \item Lattices as an integral combination over a basis
\end{itemize}

\begin{definition}
  Given a basis $B ∈ ℤ^{d × n}$, the \emph{lattice} over the basis $B$ is defined as
  \[
    \mathcal{L}(B) = \left\{\, B₁z₁ + \dots + B_n z_n \mid z_1, \dots, z_n ∈ ℤ^d \,\right\}.
  \]
  The \emph{rank} of $\mathcal{L}(B)$ is $n$ and its \emph{dimension} is $d$.
  If $n = d$, then $\mathcal{L}(B)$ is a \emph{full rank} lattice.
\end{definition}

\begin{problem}[Lattice Basis Reduction]~
  \begin{itemize}
    \item \textbf{Input}: A matrix $A ∈ ℤ^{d × n}$ with $\text{rank}(A) = d$.
    \item \textbf{Output}: A matrix $B ∈ ℤ^{d × d}$ with $\mathcal{L}(B) = \mathcal{L}(A)$.
  \end{itemize}
\end{problem}

In this thesis, I only consider the case for one additional vector, i.e. $n = d + 1$.

% TODO: Example for an over-defined basis and what the reduced basis is.
\begin{example}
  Consider $A = \begin{pmatrix}
    2 & 1 & 3 \\
    1 & 3 & 4 \\
  \end{pmatrix}$.
  The matrix $B = \begin{pmatrix}
    2 & 1 \\
    1 & 3 \\
  \end{pmatrix}$
  spans the same lattice,
  since $A_3 = A_1 + A_2$.
  Therefore, $B$ would is the reduced basis of $\mathcal L(A)$.
\end{example}

% TODO: Another example which shows that you can't just take a submatrix of the
% original matrix.

\begin{definition}
  The \emph{fundamental parallelepiped} of a lattice $\mathcal{L}(B)$ with $B ∈ ℤ^{d × n}$ is defined as
  \[
    Π(B) = \left\{\, B₁ x₁ + \dots + B_n x_n \mid x_1, \dots, x_n ∈ [0, 1) \,\right\}
  \]
\end{definition}

A useful fact about the fundamental parallelepiped of a lattice $\mathcal L(B)$ is that
if $B$ is a square integer matrix,
then the volume of the parallelepiped $Π(B)$ and
the number of integer points $ℤ^n$ contained in $Π(B)$ is determined by $\mathrm{det}(B)$,
i.e.
\[
  \mathrm{vol}(Π(B)) = |Π(B) ∩ ℤ^n| = |\det(B)|.
\]

\section{Description of the Algorithm}

\begin{Pseudocode}[float=tb,caption={The Generalized Euclidean Algorithm \cite{Klein24}.}]
solve $Bx = c$
while $x$ is not integral do
  find $x_ℓ$ which is not integral
  $c ← B_ℓ$
  $B_ℓ ← B\{x\}$
  solve $Bx = c$
end
\end{Pseudocode}

In the previous example,
we saw that we could represent the last column vector as an integral
combination of the previous two,
which allows us to reduce the basis for the lattice to only those two column vectors.
However, in general it is not as easy as this.
Consider the matrix $A = ?$.
In this case, $A_3 = ? + ?$, which is clearly not an integral combination.
So $A' = ?$ does not span the same lattice as $A$.

Each point $a ∈ ℝ^d$ can be represented as a combination of a lattice point $z
∈ \mathcal{L}(B)$ and a point in the fundamental parallelepiped $r ∈ Π(B)$.
Specifically,
\[
  a = z + r = B\floor{x} + B\{x\}.
\]
This is essentially a division with remainder inside a lattice.
It allows us to define a modulo operation on the lattice:
\[
  a \pmod{Π(B)} := a - B\floor{B^{-1} x}.
\]

The algorithm requires solving a linear system in each iteration.
However, we do not have to do this in every iteration.
We only have to do this in the first iteration and in the following iterations
we simply update this solution from the old solution.
If $x = (x₁, …, x_d)$ is the solution in the previous iteration,
then $x' = (x₁', …, x_d')$ with
\begin{align*}
  x_i' =
  \begin{cases}
    \frac{1}{\{x_ℓ\}},  & \text{ if } i = ℓ, \\
    -\frac{\{x_i\}}{\{x_ℓ\}} & \text{ otherwise,}
  \end{cases}
\end{align*}
is the solution in the next iteration.
This update rule follows from
\[
  B_ℓ \{x_ℓ\} + \sum_{i ≠ ℓ} B_i \{x_i\} = r
  \iff
  r - \sum_{i ≠ ℓ} B_i \{x_i\} = B_ℓ \{x_ℓ\}
  \iff
  r \frac{1}{\{x_ℓ\}} - \sum_{i ≠ ℓ} B_i \frac{\{x_i\}}{\{x_ℓ\}} = B_ℓ.
\]

% TODO: Should we add a citation for Northshield and explain that continued
% fractions map positive to positive values which seems to be a fundamental
% requirement for the continued fractions to be periodic?

% I think a better wording would be, that the update rule makes the negation
% visible, which is not optimal. The update rule itself doesn't negate the
% variables, even without the update rule we would still have negated
% variables, since the update rule is just an improvement of the original
% algorithm.

Although the update rule speeds up the algorithm considerably, it is not
optimal for the analysis in the following sections.
The rule flips the sign of all elements inside the solution vector in each
iteration.
Instead, I propose a slight modification to the generalized algorithm which
maps each $xᵢ$ to another positive value.
After we replace $B_ℓ$ with $c$, we flip the signs of all vectors $B_i$ with $i ≠ ℓ$.
This leads to the modified update rule, where the values $x_i$ for $i ≠ ℓ$ are
no longer negated:
\begin{align*}
  x_i' =
  \begin{cases}
    \frac{1}{\{x_ℓ\}},  & \text{ if } i = ℓ, \\
    \frac{\{x_i\}}{\{x_ℓ\}} & \text{ otherwise.}
  \end{cases}
\end{align*}
By $\mathrm{pivot}_ℓ(x) = x'$, we denote this modified update rule.
The modified algorithm can be seen in Listing~\ref{lst:modified-generalized-euclidean}.
In the algorithm, first $B_ℓ$ is flipped and then the whole matrix $B$ is flipped,
This is the same as only flipping the vectors $B_i$ for $i ≠ ℓ$.

\begin{Pseudocode}[float=tb, caption={The Modified Algorithm.}, label={lst:modified-generalized-euclidean}]
solve $Bx = c$
while $x$ is not integral do
  find index $ℓ$ for which $x_ℓ$ is not integral
  $c ← B_ℓ$
  $B_ℓ ← -B\{x\}$
  $B ← -B$
  $x ← \mathrm{pivot}_ℓ(x)$
end
\end{Pseudocode}

\begin{lemma}
  The algorithm terminates in at most $\det(B)$ steps.
\end{lemma}

\begin{proof}

\end{proof}

\begin{lemma}
  In each iteration, $\mathcal L(B ∪ c) = \mathcal L(B' ∪ c')$.
\end{lemma}

\begin{proof}

\end{proof}

\begin{theorem}
  The generalized Euclidean algorithm solves the lattice basis reduction problem.
\end{theorem}

\begin{figure}[t]
  \centering
  \includestandalone{figures/pivot-choice}
  \caption{
    Different choices for the remainder of vector $c$. The original algorithm
    always uses $r$ as the remainder, but the modified update rule would also consider $r'$.}
\end{figure}

\section{Extension to Real Numbers}

% ==============================================================================
\section{Comparison to the Jacobi-Perron Algorithm}
% ==============================================================================

Many generalizations to the Euclidean algorithm have been considered.
One of them was proposed by Jacobi to answer Hermite's question.
In his version, he computes the GCD of three numbers by successively dividing
the smallest number from the larger numbers.
This algorithm was later extended by Oskar Perron to arbitrarily many numbers.
The algorithm works as follows:

Given a list of positive integers $a₀, a₁, …, aₙ$, take the smallest number $a_ℓ$
and compute the remainder $a_i'$ resulting from the division of $a_i$ with $a_ℓ$.
The value $a_ℓ$ is kept until the next iteration, i.e. $a_ℓ' = a_ℓ$.
Continue this process until all but one value remains.
\begin{align*}
  a₀' = a₀ \bmod a_ℓ, a₁ = a₁ \bmod a_ℓ, …, a_ℓ' = a_ℓ, …, aₙ' = aₙ \bmod a_ℓ; \\
\end{align*}

Perron modified this algorithm for the purposes of his analysis.
The integers $a₁, …, aₙ$ are kept in a list.
We remove the first element from the list, calculate the remainders for each
remaining element and append the element to the end of the list.
One iteration in this modified version produces the values:
\begin{align*}
  a₀' = a₁ \bmod a₀, a₁' = a₂ \bmod a₀, …, a_{n-1}' = a_n \bmod a₀, aₙ = a₀. \\
\end{align*}
This process is repeated until the first element is zero.

Of course, the termination condition is not sufficient.
When this algorithm terminates, the remaining elements might not all be zero.
Therefore, we remove the first element from the list and continue with the
remaining list.

By allowing real numbers as inputs, this algorithm proceeds infinitely but
always converges to zero.

The Jacobi-Perron algorithm is actually a subset of the generalized Euclidean algorithm.
The generalized Euclidean algorithm is periodic for all real numbers where the Jacobi-Perron algorithm is periodic.
This comes from the fact, that the Jacobi-Perron algorithm is really the
generalized Euclidean algorithm with the specific sequence of pivots $L = \overline{12…d}$.
So in the $i$th iteration, we are choosing index $(i \bmod d) + 1$ as our pivot.

This has the convenient property that if the all inputs which are periodic for
the Jacobi-Perron algorithm must also be periodic for the brute-force algorithm.

\begin{theorem}
  The brute-force algorithm is periodic on input $(1, r, r^2)$ if
  \[
    r = \sqrt[n]{D + d}, \text{ where } d | D.
  \]
\end{theorem}

\chapter{Higher-Order Fibonacci Numbers and their Golden Ratios}
\label{ch:fibonacci}

% TODO: Add reference to Fibonacci numbers in the continued fraction chapter
One of the simplest periodic continued fraction is $[1; \overline{1}]$.
This fraction evaluates to the golden ratio and its convergents are ratios of
consecutive Fibonacci numbers.
Naturally, we can consider the simplest vectors $x ∈ ℝ^d$ for the generalized
Euclidean algorithm, where the integer part in each iteration is $1$.
Such vectors can be seen as a generalization of the golden
ratio to higher dimensions and their approximation can be seen as a
generalization of Fibonacci numbers.
However, there does not exist a single definitive golden ratio,
Since the generalized algorithm allows an additional freedom in the choice of the pivot.
So there are actually multiple different possible definitions.
Some of these are well known, like the supergolden ratio or the plastic ratio,
which are both roots of cubic polynomials.

Since the definition of the golden ratios depends entirely on the choice of our
pivot element $x_ℓ$, we consider two strategies for choosing $x_ℓ$ in this
chapter: First choosing the smallest fractional value $\{x_ℓ\}$ and second
choosing the largest fractional value $\{x_ℓ\}$.
Both of these strategies have a respective golden ratio for which the algorithm
becomes periodic and they each have a corresponding linear recurrence, which
can be seen as a generalization of the Fibonacci numbers.
In the end, we will derive a general definition of higher-order Fibonacci
sequences and show that the generalized Euclidean algorithm is periodic for
their golden ratios.

% ==============================================================================
\section{The Classical Case: Fibonacci Numbers and the Golden Ratio}
% ==============================================================================

We revisit the classical case of Fibonacci numbers and the golden ratio
to understand their relationship with the Euclidean algorithm.
Fibonacci numbers are a recursive sequence.
Starting with the initial conditions $F(0) = 1$ and $F(1) = 1$,
the remaining terms are calculated using the formula
\[
  F(n) = F(n-1) + F(n-2).
\]
The first terms of this sequence are $1, 1, 2, 3, 5, 8, 13, 21, …$

Their relationship with the Euclidean algorithm becomes apparent when analyzing
the runtime of the algorithm.
In fact, the analysis was actually one of the first practical application of
the Fibonacci numbers.
Lamé \cite{Lame44} used the Fibonacci numbers to show that the number of
iterations in the Euclidean algorithm is bounded by the number of digits in the
input.
The proof is considered by some as the beginning of computational
complexity theory.

\begin{theorem}
  \label{thm:lame}
  If the Euclidean algorithm requires at least $n$ steps for $(a, b) ∈ ℤ_{> 0}^2$ with $a < b$,
  then $a ≥ F(n+2)$ and $b ≥ F(n+1)$.
\end{theorem}

\begin{proof}
  For $n = 0$, we have $F(2) = 2$ and $F(1) = 1$
  and they are the smallest pair of positive integers which require one step of
  the Euclidean algorithm.
  In fact, they are the smallest pair which satisfy $a < b$.
  So any pair, which takes only one step, must satisfy $a ≥ F(2)$ and $b ≥ F(1)$.
  Next, suppose that the theorem holds for some $n ≥ 0$.
  Let $(a, b)$ be an input pair which requires $n+1$ steps.
  Then, $a = qb + r$ for some integers $q, r ≥ 1$.
  For the pair $(b, r)$, the algorithm requires $n$ steps, so $b ≥ F(n+2)$ and
  $r ≥ F(n+1)$ by induction.
  But then
  \[
    a = qb + r ≥ F(n+2) + F(n+1) = F(n+3).
  \]
  Therefore, the theorem also holds for $n+1$.
\end{proof}

The theorem also follows easily using the continued fraction of the ratios
between Fibonacci numbers.
Given a finite continued fraction $[a₀; a₁, …, a_n]$, the Euclidean algorithm
requires $n$ steps for any input $(a, b)$ with the same ratio $a/b$ as the
continued fraction.
Since $[1; 1, …, 1] = F(n+1)/F(n)$ for some $n ≥ 0$, the Euclidean algorithm
takes at least $n$ steps for two consecutive Fibonacci numbers.
Furthermore, Fibonacci numbers are the smallest numbers which require $n$ steps
since we can only increase the coefficients of the continued fraction.

The continued fractions also show the connection between the Fibonacci numbers
and the golden ratio.
The convergents of the continued fraction $[1; \overline{1}]$ are ratios
between consecutive Fibonacci numbers $F(n+1)/F(n)$ and as $n$ approaches
infinity the convergents tend towards the golden ratio $φ$.

\iffalse
\begin{example}
  Consider $a = 13$ and $b = 8$.
  The algorithm proceeds as follows:
  \[
    \begin{array}{rclcrcl}
      13/8 & = & 1 & · & 5/8 & + & 3/8 \\
       5/8 & = & 1 & · & 3/8 & + & 2/8 \\
       3/8 & = & 1 & · & 2/8 & + & 1/8 \\
       2/8 & = & 2 & · & 1/8 & + & 0.
    \end{array}
  \]
\end{example}
\fi

\begin{theorem}
  The ratios $\frac{F(n+1)}{F(n)}$ converge towards the golden ratio $φ$ as $n$ increases.
\end{theorem}

\begin{proof}
  First, we show that the ratios are converging by mapping it to the continued fraction $[1; \overline{1}]$.
  By Lemma~\ref{lem:cf-wallis}, we can calculate the convergents of the continued fraction as
  \[
    p_n = a_n p_{n-1} + p_{n-2}, q_n = a_n q_{n-1} + q_{n-2}.
  \]
  The continued fraction consists solely of ones,
  so the Fibonacci sequence and the sequence $p_n$ are identical.
  The same is true for $q_n$, but $q_{-1} = 0$ and $q_{-2} = 1$, so the sequence lags one step behind $p_n$.
  Therefore, the convergents are actually $F(n+1)/F(n)$.
  The convergence is already given by Lemma~\vref{lem:cf-approx},
  so all that remains to be shown is that they actually converge to the golden ratio $φ$.
  Suppose they converge to some different limit $φ'$
  \[
    φ' = \lim_{n → ∞} \frac{F(n+1)}{F(n)} = \lim_{n → ∞} 1 + \frac{F(n-1)}{F(n)} = 1 + \frac{1}{φ'}.
  \]
  Multiplying this equation by $φ'$ results in the defining polynomial of the golden ratio $φ$.
  Because the ratios are always positive, we must have $φ' = φ$.
\end{proof}

% TODO: Metallic means

In summary,
the defining property of Fibonacci numbers with respect to the Euclidean
algorithm is that they represent the worst-case for the algorithm.
At each step, they have a quotient of $1$, so they decrease slower than any other pair of integers.
Additionally, their ratios approximate the golden ratio,
which follows from the continued fraction.

% ==============================================================================
\section{Higher-Order Fibonacci Numbers for the Minimum Strategy}
% ==============================================================================

% TODO: There should be a discussion on what it even means to have a Fibonacci number in higher dimensions
% TODO: We should also discuss the thing about idempotence under the choice of the pivot.
% TODO: Should we have a discussion about why we choose this particular linear system.
% TODO: The discussion about the linear system should go in the generalized Euclidean algorithm chapter
% TODO: We should explain that very large values give the least decrease, but also give the most decrease in the following iteration

For the generalisation of Fibonacci numbers,
we must find a sequence which similarly has a quotient of $1$ at each iteration.
However, the problem is that no single definitive sequence can exist.
With the generalized Euclidean algorithm we have an additional choice with the
element we choose to pivot with.
Therefore, multiple different notions of Fibonacci numbers exist for this
algorithm.

% I would like a better explanation of what the goal of this section is
We begin by deriving the Fibonacci numbers for one particular strategy.
Recall that the determinant decreases by $\{x_ℓ\}$ in each iteration.
So the first strategy one may think of is to choose the index $ℓ$ with the
smallest fractional value.
After all, this gives us the largest decrease in the determinant over one iteration.
Locally, this would be the highest decrease we can achieve.
So we begin with this strategy and see how the Fibonacci numbers are defined.

% TODO: Actually explain how we derive the Fibonacci numbers for this strategy.
% TODO: Should we have a table for the Fibonacci numbers?
% TODO: What about adding references to their OEIS number?

% TODO: Maybe it's better to show the numbers in the ratios, since we explain
% that they approach some golden ratio anyways
\begin{table}[tbp]
  \caption{The first 10 $d$-bonacci numbers for $d = 1, …, 5$ and their golden ratios.}
  \label{tbl:min-fibonacci}
  \centering
  \begin{tabular}{cc|cccccccccc}
\uzlhline
\uzlemph{$d$} & \uzlemph{$φ$} & \uzlemph{$F(0)$} & \uzlemph{$F(1)$} & \uzlemph{$F(2)$} & \uzlemph{$F(3)$} & \uzlemph{$F(4)$} & \uzlemph{$F(5)$} & \uzlemph{$F(6)$} & \uzlemph{$F(7)$} & \uzlemph{$F(8)$} & \uzlemph{$F(9)$} \\
\hline
$1$ & $1.61803$ & 1 & 2 & 3 & 5 & 8 & 13 & 21 & 34 & 55 & 89 \\
$2$ & $1.83929$ & 1 & 1 & 3 & 5 & 9 & 17 & 31 & 57 & 105 & 193 \\
$3$ & $1.92756$ & 1 & 1 & 1 & 4 & 7 & 13 & 25 & 49 & 94 & 181 \\
$4$ & $1.96595$ & 1 & 1 & 1 & 1 & 5 & 9 & 17 & 33 & 65 & 129 \\
$5$ & $1.98358$ & 1 & 1 & 1 & 1 & 1 & 6 & 11 & 21 & 41 & 81 \\
$6$ & $1.99198$ & 1 & 1 & 1 & 1 & 1 & 1 & 7 & 13 & 25 & 49 \\
\uzlhline
\end{tabular}

\end{table}

\begin{definition}
  The Fibonacci numbers for the minimum strategy are defined as
  \begin{enumerate}
    \item $F(0) = F(1) = ⋯ = F(d) = 1$.
    \item $F(n + 1) = F(n) + F(n - 1) + ⋯ + F(n - d)$.
  \end{enumerate}
  For $d = 2, 3$ and $4$, they are also referred to as the Tribonacci, Tetranacci
  and Pentanacci numbers.
  In general, they are known as the $d$-bonacci numbers.
\end{definition}

For these numbers, we use the solution vector $x = (x₁, …, x_d)$ with
\[
  x_i = \frac{\sum_{k=0}^i F(n - i)}{F(n)}.
\]
Because $F(n) = F(n - 1) + ⋯ + F(n - d - 1) > F(n - 1) + ⋯ + F(n - i)$,
each $x_i$ is bounded between $1$ and $2$.
Thus, each $x_i$ has a fractional value of $1$.
The smallest one is $x₁$, so we pivot with $ℓ = 1$ first.
Let $x' = \mathrm{pivot}_ℓ(x)$, then the first value is
\begin{align*}
  x₁' &
  = \frac{1}{\{x₁\}}
  = \frac{1}{x₁ - 1}
  = \frac{F(n)}{F(n - 1)}
  = \frac{F(n - 1) + ⋯ + F(n - d)}{F(n)}.
\end{align*}
This equals the value of $x_d$, if we would have started with $F(n-1)$ instead of $F(n)$.
The other values in our input vector $x$ are calculated as follows:
\begin{align*}
  xᵢ'
  & = \frac{\{xᵢ\}}{\{x₁\}} = \frac{xᵢ - 1}{x₁ - 1} \\
  & = \frac{F(n - 1) + ⋯ + F(n - i)}{F(n + 1)} · \frac{F(n)}{F(n - 1)} \\
  & = \frac{F(n - 1) + ⋯ + F(n - i)}{F(n - 1)}.
\end{align*}
So in the next iteration the value $x_i'$ has the same value as $x_{i-1}$, if we
would have started with $F(n - 1)$.
Therefore, in the next iteration the smallest fractional value must be $x_2$.
If we repeat this a total of $d$ times, then we end up where we started, but
each term is shifted backwards by $d$ steps.
To reach the end, we need a total $n$ steps.

% TODO: There should be a note, that we're taking the actual minimum, so we
% stop once we see any zero. Because at that point, we're cycling.

For the worst-case analysis, we first need the following lemma, which shows
that the $\mathrm{pivot}$ operation from Equation~\ref{eq:modified-update-rule}
can be viewed as a multi-dimensional division with remainder operation, where
we have a set of integers $p₁, …, p_d, q$ and divide them by one particular
element $p_ℓ$.

\begin{lemma}
  \label{lem:divmod}
  % TODO: This should basically prove that pivot is just division with
  % remainder
  Let $x = \left(\frac{p₁}{q}, \frac{p₂}{q}, …, \frac{p_d}{q}\right)$
  and $x' = \left(\frac{p₁'}{q'}, \frac{p₂'}{q'}, …, \frac{p_d'}{q'}\right)$
  with $x' = \mathrm{pivot}_ℓ(x)$ for some index $ℓ$.
  % TODO: Do we need gcd = 1?
  %Suppose $\gcd(q, p₁, …, p_d) = 1$ and $\gcd(q', p₁', …, p_d') = 1$, then
  Then,
  \[
    q = p_ℓ',
    \qquad
    p_ℓ = a_ℓ p_ℓ' + q',
    \qquad
    p_i = a_i p_ℓ' + p_i',
  \]
  for every $i ∈ \{1, …, d\}$ and $a = \floor{x}$.
\end{lemma}

\begin{proof}
  % TODO: Reference the rule
  By definition of $\mathrm{pivot}$, we have
  \[
    x_ℓ' = \frac{p_ℓ'}{q'} = \frac{1}{\frac{p_ℓ}{q} - a_ℓ} = \frac{q}{p_ℓ - a_ℓ q}.
  \]
  Therefore, $q = p_ℓ'$ and $p_ℓ = a_ℓ q + q' = a_ℓ p_ℓ' + q'$.
  For the remaining elements, we have
  \[
    \frac{p_i'}{q'} = \frac{\frac{p_i}{q} - a_i}{\frac{p_ℓ}{q} - a_ℓ} = \frac{p_i - a_i q}{p_ℓ - a_ℓ q}.
  \]
  Therefore, $p_i = p_i' + a_i q = p_i' + a_i p_ℓ'$.
\end{proof}

To show that the $d$-bonacci numbers are the worst-case for the minimum
strategy, we first need a few requirements.
The first is that the vector $x$ is strictly positive,
which can be easily solved by ignoring the first iteration.
In the second iteration, $x$ must always be positive.
The second requirement is that the values in $x$ are all different.
The reasoning behind this requirement is that one could easily check if there
are two elements $x_i, x_j$ with $x_i = x_j$.
Pivoting with either $ℓ = i$ or $ℓ = j$ removes the other element from the
vector, since the element will be zero in the next iteration.
Hence, this step will always be more efficient than pivoting a different element.
With these requirements, we can show that the $d$-bonacci numbers are the worst-case.

\begin{theorem}
  Let $x = (p₁/q, …, p_d/q) ∈ ℚ^d$.
  If the algorithm requires at least $n ≥ 0$ steps for the vector $x$ when
  taking the minimum fractional value at each step, then there exists a permutation $σ$ such
  that
  \[
    q ≥ F(n+d),
    \qquad
    p_{σ(i)} ≥ \sum_{k = 0}^i F(n+d - k),
    \text{ for every } i ≤ d.
  \]
\end{theorem}

\begin{proof}
  Suppose we require $n = 0$ steps.
  Since each value of $p₁, …, p_d, q$ must be distinct,
  there has to be one out of the $d+1$ integers,
  which is greater than $k$ for each $k ∈ \{1, …, d+1\}$.
  However, the first $d+1$ Fibonacci numbers have the values:
  \begin{align*}
    F(d) = 1,
    \qquad
    \sum_{k=0}^i F(d - k) = i + 1 \text{ for every } i ≤ d.
  \end{align*}
  Furthermore, the smallest value the denominator $q$ can assume is $1$.
  By ordering the elements accordingly,
  we can find a permutation $σ$ such that
  the bounds for this theorem are satisfied.

  % TODO: Fix indices
  Suppose the algorithm requires $n+1$ steps for some input $x = (p₁/q, …, p_d/q)$.
  Hence, $x' = \mathrm{pivot}_ℓ(x)$ requires $n$ steps and by induction,
  there exists a permutation $σ$ such that $q' ≥ F(n)$ and
  \begin{align*}
    p_{σ(i)}' & ≥ \sum_{k=0}^i F(n - k) = F(n + 1).
  \end{align*}
  We assume without loss of generality that $σ(d)$ is the largest element.
  Hence, we have to pivot with $ℓ = σ(d)$.
  Let $a = \floor{x}$, then $a_i ≥ 1$ by construction.
  From Lemma~\ref{lem:divmod}, it follows that
  \begin{align*}
    q        & = p_{σ(d)}' ≥ F(n+1),                                                   \\
    p_{σ(d)} & = a_{σ(d)} p_{σ(d)}' + q' ≥ F(n + 1) + F(n) = \sum_{k=0}^1 F(n + 1 - k) \\
    p_{σ(i)} & = a_{σ(d)} p_{σ(d)}' + p_{σ(i)}' ≥ F(n + 1) + ∑_{k=0}^i F(n - k) = ∑_{k=0}^{i+1} F(n + 1 - k).
  \end{align*}
  We construct a new permutation $σ'$ which orders the bounds correctly
  using $σ'(1) = σ(d)$ and $σ'(i) = σ(i+1)$ for the other indices.
\end{proof}

% TODO: Explain how this can be used to bound the time
Using the contrapositive of the theorem, it follows that if a vector $x$
has only one value smaller than the listed bounds, then it must take less than
$n$ steps.
Furthermore, we have seen that the Fibonacci numbers require exactly $n$ steps.
Therefore, these Fibonacci numbers represent the worst-case input for the
minimum strategy.
This mirrors how classical Fibonacci numbers represent the worst-case for the
classical Euclidean algorithm.

In the same spirit,
we can ask whether this generalization of Fibonacci also lead to a generalization of the golden ratio.
The classical Fibonacci numbers approach the golden ratio, when dividing consecutive numbers.
Therefore, we look at what values the $d$-bonacci numbers approach by dividing consecutive numbers.
This limit can be considered a multi-dimensional generalization of the golden ratio.
For now, we will assume that these ratios are converging.
A general proof of convergence will be shown in Section~\ref{sec:fib-conv}, at the end of the chapter.

\begin{theorem}
  If the ratios $\frac{F(n+1)}{F(n)}$ are converging, then they approach $φ_d$,
  where $φ_d$ is the positive real root of the polynomial $p(x) = x^{d+1} - x^d - ⋯ - 1$.
\end{theorem}
% TODO: Add note about convergence towards some sort of golden ratio

\begin{proof}
  By assumption, the ratios are approach some limit $r$.
  Then,
  \[
    r
    = \lim_{n → ∞} \frac{F(n+1)}{F(n)}
    = \lim_{n → ∞} \left(1 + \frac{F(n-1)}{F(n)} + \frac{F(n-2)}{F(n)} + ⋯ + \frac{F(n-d)}{F(n)}\right).
  \]
  Each ratio in the sum can be rewritten as a product of ratios with
  consecutive Fibonacci numbers, since
  \[
    \frac{F(n - i)}{F(n)}
    = \frac{F(n - i)}{F(n - i + 1)} · \frac{F(n - i + 1)}{F(n - i + 2)} ⋯ \frac{F(n - 1)}{F(n)}.
  \]
  By assumption, each ratio approaches $r^{-1}$ as $n$ increases towards infinity.
  Therefore,
  \[
    r = 1 + \frac{1}{r} + \frac{1}{r^2} + ⋯ + \frac{1}{r^d},
  \]
  which is equivalent to $r^{d+1} - r^d - ⋯ - r - 1 = 0$.
  Furthermore, the ratios are always positive.
  Hence, $r = φ_d$.
\end{proof}

\begin{corollary}
  The ratios $F(n + i)/F(n)$ converge towards $φ_d^i$ for any $i ≥ 0$.
\end{corollary}

We can then replace the ratios in the solution vector $x$ with the actual golden ratio $φ_d$.
We now have an algebraic vector $x = (x₁, …, x_d) ∈ ℝ^d$ with
\[
  x_i
  = \lim_{n → ∞} \sum_{k=0}^i \frac{F(n - i)}{F(n)}
  = \sum_{k=0}^i φ_d^k.
\]
Importantly, the algorithm is periodic for this input vector.
We have already seen for the approximate solution with Fibonacci numbers
that $d$ steps shifts each number backwards by $d$.
For the algebraic solution, this means that it stays the same after $d$ steps.

\begin{theorem}
  The generalized Euclidean algorithm is periodic for the real vector $x = (x₁, …, x_d)$
  with $x_i = \sum_{k=0}^i φ_d^k$ when taking the minimum fractional value.
\end{theorem}

\begin{proof}
  %The integer part of each $x_i$ is still $1$.
  The first value $x₁$ is the smallest again, so $ℓ = 1$.
  In the following iteration, we have
  \[
    x_i' = \frac{\{x_i\}}{\{x_1\}} = \frac{φ_d + φ_d^2 + ⋯ + φ_d^i}{φ_d} = 1 + φ_d + ⋯ + φ_d^{i-1} = x_{i-1}.
  \]
  and
  \[
    x_1' = \frac{1}{\{x_1\}} = \frac{1}{φ_d} = 1 + φ_d + ⋯ + φ_d^d = x_d.
  \]
  Therefore, we reach the original input vector after $d$ iterations.
\end{proof}

% TODO: What property do the ratios have? Are they similar to the golden ratio?

% ==============================================================================
\section{Higher-Order Fibonacci Numbers for the Maximum Strategy}
% ==============================================================================

The next strategy we will look at is the maximum strategy,
where we choose the element with the largest fractional value.
Although this leads to the worst possible decrease in terms of the determinant,
it does provide another definition of Fibonacci sequences and a new golden ratio.
The sequence for this strategy will be called maximum Fibonacci numbers and they
are defined by the following recurrence relation:
\begin{align*}
  F(n) =
  \begin{cases}
    1, & \text{ if } n ≤ 0, \\
    F(n - 1) + F(n - 1 - d), & \text{ if } n > 0.
  \end{cases}
\end{align*}
The first few terms of this sequences are listed in Table~\ref{tbl:max-fibonacci}.

\begin{table}[tbp]
  \caption{The first 10 Fibonacci numbers for $d = 1, …, 5$ and their respective golden ratio.}
  \label{tbl:max-fibonacci}
  \centering
  \begin{tabular}{cc|cccccccccc}
\uzlhline
\uzlemph{$d$} & \uzlemph{$φ$} & \uzlemph{$F(0)$} & \uzlemph{$F(1)$} & \uzlemph{$F(2)$} & \uzlemph{$F(3)$} & \uzlemph{$F(4)$} & \uzlemph{$F(5)$} & \uzlemph{$F(6)$} & \uzlemph{$F(7)$} & \uzlemph{$F(8)$} & \uzlemph{$F(9)$} \\
\hline
$1$ & $1.61803$ & 1 & 2 & 3 & 5 & 8 & 13 & 21 & 34 & 55 & 89 \\
$2$ & $1.46557$ & 1 & 1 & 2 & 3 & 4 & 6 & 9 & 13 & 19 & 28 \\
$3$ & $1.37971$ & 1 & 1 & 1 & 2 & 3 & 4 & 5 & 7 & 10 & 14 \\
$4$ & $1.32143$ & 1 & 1 & 1 & 1 & 2 & 3 & 4 & 5 & 6 & 8 \\
$5$ & $1.29577$ & 1 & 1 & 1 & 1 & 1 & 2 & 3 & 4 & 5 & 6 \\
$6$ & $1.23256$ & 1 & 1 & 1 & 1 & 1 & 1 & 2 & 3 & 4 & 5 \\
\uzlhline
\end{tabular}

\end{table}

To construct the solution vector $x$ for these numbers, we choose a consecutive
ratio of two Fibonacci numbers for each $x_i$, where the index $i$ determines
the distance between the two.
More specifically, we choose the vector
\begin{align*}
  x = \left(
    \frac{F(n-1)}{F(n)},
    \frac{F(n-2)}{F(n)},
    …,
    \frac{F(n-(d-1)}{F(n)},
    \frac{F(n-d) + F(n)}{F(n)} \right).
\end{align*}

This time, we pivot with the largest fractional value, which is $x_1$.
Let $a = \floor{x}$, then the next vector is $x' = (x₁', …, x_d')$ where
the first element is
\begin{align*}
  x_1'
  = \frac{1}{\left\{\frac{F(n-1)}{F(n)}\right\}}
  = \frac{F(n)}{F(n-1)}
  = \frac{F(n-1) + F(n-1-d)}{F(n)},
\end{align*}
the elements with $i ≤ d - 1$ are
\begin{align*}
  x_i'
  = \frac{\frac{F(n-i)}{F(n)}}{\frac{F(n-1)}{F(n)}}
  = \frac{F(n-i)}{F(n-1)}
\end{align*}
and the last element is
\begin{align*}
  x_d'
  = \frac{\frac{F(n-d) + F(n)}{F(n)} - 1}{\frac{F(n-1)}{F(n)}}
  = \frac{F(n-d)}{F(n-1)}.
\end{align*}

However, this time the maximum Fibonacci numbers do not represent the worst-case.
At least, we cannot show any worst-case bounds using these numbers in the same way as before.
The issue is that in the equation
\[
  p_ℓ = a_ℓ p_ℓ' + q'
\]
from Lemma~\ref{lem:divmod} we cannot bound $a_ℓ$ from below by $1$.
The reason is that only element of $x$ can be greater than $1$
and all other elements must be smaller than $1$.
But that element does not have to be the pivot element,
it could be another element.

\begin{theorem}
  There exists a vector $x = (p₁/q, …, p_d/q) ∈ ℚ^d$ which requires at least $n$ steps
  and there is at least one element $p_i$ with $p_i < F(n-d)$
\end{theorem}

\begin{proof}
  % TODO
\end{proof}

Although they do not represent the worst-case,
they do lead to another golden ratio.
Looking at the ratios $F(n+1)/F(n)$ again,
they converge towards a limit.
This limit represents the golden ratio $ψ_d$ for the sequence and
the following theorem shows that they converge towards this ratio.

\begin{theorem}
  If the ratios $\frac{F(n+1)}{F(n)}$ are converging, then they approach $ψ_d$,
  where $ψ_d$ is the positive real root of the polynomial $p_d(x) = x^{d+1} - x^d - 1$.
\end{theorem}

\begin{proof}
  By assumption, the ratios are approach some limit $r$.
  Then,
  \[
    r
    = \lim_{n → ∞} \frac{F(n+1)}{F(n)}
    = \lim_{n → ∞} \left(1 + \frac{F(n-d)}{F(n)}\right).
  \]
  The ratio $F(n-d)/F(n)$ in the sum can be rewritten as a product of ratios with
  consecutive Fibonacci numbers, since
  \[
    \frac{F(n - d)}{F(n)}
    = \frac{F(n - d)}{F(n - d + 1)} · \frac{F(n - d + 1)}{F(n - d + 2)} ⋯ \frac{F(n - 1)}{F(n)}.
  \]
  By assumption, each ratio approaches $r^{-1}$ as $n$ increases towards infinity.
  Therefore,
  \[
    r = 1 + \frac{1}{r^d},
  \]
  which is equivalent to the equation $r^{d+1} - r^d - 1 = 0$.
  Furthermore, the ratios are always positive.
  Hence, $r = ψ_d$.
\end{proof}

\begin{corollary}
  The ratios $F(n + i)/F(n)$ converge towards $ψ_d^i$ for any $i ≥ 0$.
\end{corollary}

We can then replace the ratios in the solution vector $x$ with the actual
golden ratio $ψ_d$, again.
The algebraic vector $x = (x₁, …, x_d) ∈ ℝ^d$ is defined as
\[
  x_i
  = \lim_{n → ∞} \sum_{k=0}^i \frac{F(n - i)}{F(n)}
  = \sum_{k=0}^i ψ_d^k.
\]
Importantly, the algorithm is periodic for this input vector.
We have already seen for the approximate solution with Fibonacci numbers
that $d$ steps shifts each number backwards by $d$.
For the algebraic solution, this means that it stays the same after $d$ steps.
This gives us the second example of a periodic input for the generalized Euclidean algorithm.

\begin{theorem}
  The generalized Euclidean algorithm is periodic for the real vector $x = (x₁, …, x_d)$
  with $x_i = ψ_d^i$ when taking the minimum fractional value.
\end{theorem}

\begin{proof}
  %The integer part of each $x_i$ is still $1$.
  The first value $x₁$ is the smallest again, so $ℓ = 1$.
  In the following iteration, we have
  \[
    x_i' = \frac{\{x_i\}}{\{x_1\}} = \frac{ψ_d + ψ_d^2 + ⋯ + ψ_d^i}{ψ_d} = 1 + ψ_d + ⋯ + ψ_d^{i-1} = x_{i-1}.
  \]
  and
  \[
    x_1' = \frac{1}{\{x_1\}} = \frac{1}{ψ_d} = 1 + ψ_d + ⋯ + ψ_d^d = x_d.
  \]
  Therefore, we reach the original input vector after $d$ iterations.
\end{proof}

% Worst-case analysis, which does not work!
\iffalse

\begin{theorem}
  Suppose the Euclidean algorithm requires at least $n$ steps for $x = (p₁/q, …, p_d/q)$
  with $0 < p₁ < ⋯ < p_{d-1} < q < p_d$ when taking the maximum fractional value at each step.
  Then, there exists a permutation $σ$ such that
  \[
    q ≥ F(n+d),
    \qquad
    p_{σ(d)} ≥ F(n + d) + F(n),
    \qquad
    p_{σ(i)} ≥ F(n + i)
    \quad
    \text{ for every } i ≤ d - 1.
  \]
\end{theorem}

\begin{proof}
  For $n = 0$, we have
  \[
    \left(\frac{F(2)}{F(1)}, \frac{F(3)}{F(1)}, …, \frac{F(d+1)}{F(1)} \right)
    = \left(\frac{2}{1}, \frac{3}{1}, …, \frac{d+1}{1} \right).
  \]
  This is the first positive input, which satisfies the ordering and it requires only one step.
  Therefore, any other input that requires one step must be larger.
  Now suppose that the theorem holds for any $n ≥ 0$.
  Let $x ∈ ℚ^d$ and $a = \floor{x}$.
  Suppose that $x' = \mathrm{pivot}_ℓ(x) = (p₁'/q', …, p_d'/q')$ requires at
  least $n$ steps, where $ℓ$ is chosen according to the strategy.
  By induction, there exists a permutation $σ$ such that
  \begin{align*}
    q'        & ≥ F(n),                     \\
    p_{σ(d)}' & ≥ F(n) + F(n - d) = F(n+1), \\
    p_{σ(i)}' & ≥ F(n - i)
    \text{ for every } i ≤ d - 1.
  \end{align*}
  We assume without loss of generality that $σ(i) = i$ and that it orders the elements in increasing order.
  Because $p_{σ(d)}$ is the largest element in $x'$, we have $ℓ = σ(d) = d$.
  However, $x_ℓ$ cannot be the largest element in $x$ because of the strategy.
  Therefore, $a_ℓ = 0$ and there exists another element $x_k$ with $k ≠ ℓ$ and $a_k ≥ 1$.
  Futhermore, that element must come before $x_ℓ$ since $ℓ = d$.
  We can now bound $x$ using $x'$ and Lemma~\ref{lem:divmod}:
  \begin{align*}
    q   & = p_ℓ' ≥ F(n+1), \\
    p_ℓ & = a_ℓ p_ℓ' + q' = q' ≥ F(n) ≥ F((n+1) - 1), \\
    p_k & = a_k p_ℓ' + p_k' ≥ p_ℓ' + p_k' ≥ F(n + 1) + F(n + 1 - (k + 1)), \\
    p_i & = a_k p_ℓ' + p_i' ≥ p_i' ≥ F(n - i) = F((n+1) - (i+1)).
  \end{align*}
  where $i < ℓ = d$ and $i ≠ k$.
  We're missing the bound for $F(n+1) + F(n+1 - d)$ and $F(n + 1 - k)$.
  The former follows from the bound of $p_k$.
  Because $k < ℓ$, we have
  \begin{align*}
    p_k & ≥ F(n + 1) + F(n + 1 - (k + 1)) ≥ F(n+1) + F(n+1 - d).
  \end{align*}
\end{proof}

% ==============================================================================
\section{Combining both Strategies into the Minimax Strategy}
% ==============================================================================

We choose two indices $k, ℓ$ at once such that they minimize $\{\{x_k\}/\{x_ℓ\}\}$.
We pivot with $k$ first to get $x'$, then with $ℓ$ to get $x''$.
It follows that:

\begin{itemize}
  \item $1/\{\{x_ℓ\}/\{x_k\}\}$ is the largest element in $x''$
  \item $\{\{x_ℓ\}/\{x_k\}\}$ is the smallest fractional value in $x'$
  \item $1/\{x_k\}$ is the largest element in $x'$
  \item $\{x_k\}$ has the largest fractional value in $x$
  \item $\{x_i\}/\{x_k\} < 1$ for every $i$.
\end{itemize}

For this strategy, we interleave the previous two sequences.
We have an even sequence $F_0$ and an odd sequence $F_1$,
which are defined as follows:
\begin{align*}
  F_1(n) & = F_0(n-1) + F_0(n-2) + ⋯ + F_0(n-d-1), \\
  F_0(n) & = F_1(n-1) + F_1(n-d-1).
\end{align*}

\begin{align*}
  \frac{F_1(n) + ∑_{k=0}^i F(n - k)}{F_1(n)}
  \frac{F_1(n)}{F}
\end{align*}

\begin{theorem}
  Let $x = (p₁/q, …, p_d/q) ∈ ℚ^d$ with distinct elements $p_i/q > 1$.
  If the algorithm requires $n ≥ 0$ steps for $x$ with the minimax strategy,
  then there exists a permutation $σ$ such that
  \[
    q ≥ F_1(n),
    \qquad
    p_{σ(i)} ≥ F_1(n) + \sum_{k=1}^i F_0(n - k).
  \]
\end{theorem}

\begin{proof}
  Suppose the algorithm requires $n$ steps for $x''$.
  By induction,
  \[
    q'' ≥ F_1(n),
    \qquad
    p_{σ(i)}'' ≥ ∑_{k=0}^i F_0(n - k).
  \]
  In the first iteration,
  $x_k$ has the largest fractional value.
  Therefore, $x_i' = \{x_i\}/\{x_k\}$ must be smaller than $1$ and $a_i' = 0$ for every $i ≠ k$.
  Furthermore, $x_k' = 1/\{x_k\}$ must be greater than $1$ and $a_k' ≥ 1$.
  By Lemma~\ref{lem:divmod},
  \begin{align*}
    q' & = p_ℓ'' ≥ F_0(n), \\
    p_ℓ' & = a_ℓ' p_ℓ'' + q'' ≥ q'' ≥ F_0(n), \\
    p_i' & = a_i' p_ℓ'' + p_i'' ≥ p_i'' ≥ \sum_{k=0}^i F_0(n - k), \\
    p_k' & = a_k' p_ℓ'' + p_k'' ≥ p_ℓ'' + p_k'' ≥ \sum_{k=0}^d F_0(n - k) = F_1(n+1).
  \end{align*}
  By assumption, every element $x_i$ in the initial input is greater than $1$ and
  so $a_i ≥ 1$ for every $i ∈ \{1, …, d\}$.
  Again by Lemma~\ref{lem:divmod},
  \begin{align*}
    q & = p_k' ≥ F_1(n + 1), \\
    p_k & = a_k' p_k' + q' ≥ p_k' + q' ≥ F_1(n+1) + F_0(n + 1 - 1), \\
    p_i & = a_i' p_k' + p_i' ≥ p_k' + p_i' ≥ F_1(n+1) + \sum_{k=1}^{i+1} F_0(n+1 - (i+1)).
  \end{align*}
\end{proof}

% TODO: Is the bound tight?

% ==============================================================================
\section{Higher-Order Fibonacci Sequences}
% ==============================================================================


%
% Periodicity
%

% TODO: THIS DOES NOT WORK AT ALL, THE INTEGER PART IS NOT CORRECT!!!!!


We proceed with the general case.
Now we are given a \emph{linear recurrence} with coefficients~$c_0, c_1, \dots, c_{d-1} ∈ \{0, 1\}$ such that
the initial terms are $F(n) = 1$ for $n < 0$ and
the remaining terms are calculated using the recurrence
\[
  F(n + 1) = F(n - d) + c_{d-1} F(n - d + 1) + \dots + c_1 F(n - 1) + c_0 F(n).
\]
Importantly, the coefficients satisfy
\[
  c
\]
Given such a linear recurrence $F$, we choose the solution vector $x^{(n)} = (x_1^{(n)}, …, x_d^{(n)})$ with
\begin{align*}
  x_i^{(n)} = \frac{\sum_{k=0}^i c_k F(n - k)}{F(n)}.
\end{align*}

\begin{lemma}
  $1 ≤ x_i^{(n)} < 2$.
\end{lemma}

\begin{proof}
  For $n = 0$, the first terms are all $1$, so
  \[
    x_i^{(0)}
    = \frac{\sum_{k=0}^i c_k F(n - k)}{F(n)}
    = c_0 + \frac{\sum_{k=1}^i c_k F(n - k)}{F(n)}
    = c_0 + \frac{\sum_{k=1}^i c_k}{\sum_{k=0}^d c_k}
    ≤ c₀ + 1.
  \]
  Suppose that the bounds hold for any $n ≥ 0$.
  By induction,
  \[
    c₀ + 1
    ≥ \frac{\sum_{k=0}^i c_k F(n - k)}{F(n)}
    ≥ \frac{F(n)}{F(n+1)} \frac{\sum_{k=0}^i c_k F(n - k)}{F(n)}
    ≥ \frac{F(n)}{F(n+1)}
    = c_0 + \frac{\sum_{k=1}^i c_k}{\sum_{k=0}^d c_k}
    ≤ c₀ + 1.
  \]
\end{proof}

\begin{lemma}
  Pivoting with the first element of $x^{(n)}$ using the modified update rule yields the vector
  \[
    x' = (x^{(n-1)}_2, x^{(n-1)}_3, \dots, x^{(n-1)}_d, x^{(n-1)}_1).
  \]
\end{lemma}

\begin{proof}
  For the first value, we have
  \[
    \begin{aligned}
      \frac{1}{\{x_1\}}
      & = \frac{F(n+1)}{F(n)} \\
      & = 1 + \frac{\sum_{k=0}^d c_k F(n - k)}{F(n)} \\
      & = 1 + x^{(n-1)}_d.
    \end{aligned}
  \]
  For the other values, we have
  \begin{align*}
    \frac{x_i}{x_1}
    & = \frac{F(n - i) + \sum_{k=1}^{i-1} a_{d-k} F(n - k)}{F(n)} \frac{F(n)}{F(n - 1)} \\
    & = a_{d-1} + \frac{F(n - i) + \sum_{k=2}^{i-1} a_{d-k} F(n - k)}{F(n-1)} \\
    & = x^{(n-1)}_{i+1} \qedhere
  \end{align*}
\end{proof}

\begin{corollary}
  Running the generalized Euclidean algorithm with $x^{(n)}$ as the solution to
  the linear system $B x = c$ requires $n$ steps, if $x^{(n)}_1$ is the
  smallest element in $x^{(n)}$.
\end{corollary}

Of course, the Euclidean algorithm receives a matrix $B \in \Z^{d \times d}$
and vector $c \in \Z^d$ as its input and not the solution $x$ itself.
However, we can construct a very simple linear system $B^{(n)} x = c^{(n)}$,
where $x^{(n)}$ is the solution, in the following way:
\[
  B^{(n)} = F(n) I_d, \qquad c^{(n)}_k = \sum_{i=0}^k a_{d-i} F(n - i) \text{ for } k ≤ d.
\]
% end periodicity
\fi

% ==============================================================================
\section{Proof of Convergence for General Linear Recurrences}
\label{sec:fib-conv}
% ==============================================================================


%
% Convergence
%


We have seen that the ratios of consecutive Fibonacci numbers for both
definition converges towards their own respective golden ratios.
However, this was under the assumption that the ratios are converging in the first place.
Therefore, we will show in this section that this is actually the case.
To show this for both sequences at once, we use a general linear recurrence $F(n)$
with the initial terms $F(0) = ⋯ = F(d) = 1$ and the remaining terms are
defined according to the recurrence
\[
  F(n + 1) = c_d F(n) + c_{d-1} F(n-1) + ⋯ + c₁ F(n - d + 1) + c₀ F(n - d)
\]
with nonnegative integer coefficients $c₀, c₁, …, c_d$.
We assume that the first and last coefficient, $c₀$ and $c_d$, are not zero.
As an example, the $d$-bonacci sequence would have all coefficients set to one,
whereas the other sequence would have only $c₀$ and $c_d$ set to one.
The goal is to show that the ratios
\[
  r_n
  := \frac{F(n+1)}{F(n)}
  = \frac{c_0 F(n - d)}{F(n)} + \frac{c₁ F(n - d + 1)}{F(n)} + ⋯ + \frac{c_{d-1} F(n-1)}{F(n)} + c_d
\]
are converging.

The first step is to rewrite the terms of the recurrence such that each ratio
between non-consecutive Fibonacci numbers can be rewritten as a product of
ratios $r_{n-i}$:
\begin{align*}
  \frac{F(n - d + i)}{F(n)}
  & = \frac{F(n - d + i + 1)}{F(n - d + i)} \frac{F(n - d + i + 2)}{F(n - d + i + 1)} \dots \frac{F(n-1)}{F(n)} \\
  & = \frac{1}{r_{n - d + i}} · \frac{1}{r_{n - d + i + 1}} · \dots · \frac{1}{r_{n-1}}.
\end{align*}
Then we can calculate the ratio $r_n$ using the previous ratios $r_{n-1}, r_{n-2}, …, r_{n-d}$ as follows:
\[
  r_n = c_0 + \frac{c_1}{r_{n-1}} + \frac{c_2}{r_{n-1} r_{n-2}} + ⋯ + \frac{c_d}{r_{n-1} r_{n-2} \dots r_{n-d}}.
\]
Using this equation, we can show that the ratios are bounded.

\begin{lemma}
  \label{lem:fib-bounded}
  The ratios $r_n$ are bounded between $1$ and $C := ∑_{k=0}^d c_d$.
\end{lemma}

\begin{proof}
  The first $d - 1$ ratios are all $1$.
  The ratio $r_d$ is equal to
  \[
    \frac{F(d+1)}{F(d)} = \frac{F(0) + F(1) + ⋯ + F(d)}{F(d)} = \frac{c_d + ⋯ + c_0}{1} = C,
  \]
  which clearly satisfies the bounds of this lemma.
  By induction, suppose that the previous ratios $r_{n-1}, r_{n-2}, …, r_{n-d}$
  all satisfy the bound between $1$ and $d+1$.
  From previous consideration, we can reformulate the ratio $r_n$ as follows:
  \[
    r_n = c₀ + \frac{c₁}{r_{n-1}} + \frac{c₂}{r_{n-1} r_{n-2}} + \dots + \frac{c_d}{r_{n-1} r_{n-2} \dots r_{n-d}}.
  \]
  Since $r_{n-i} ≤ C$, we can bound $r_n$ from below by
  \[
    r_n ≥ c₀ + \frac{c₁}{C} + \frac{c₂}{C^2} + \dots + \frac{c_d}{C^d} ≥ 1
  \]
  and since $r_{n-i} ≥ 1$, we can bound $r_n$ from above by
  \[
    r_n ≤ c₀ + \frac{c₁}{1} + \frac{c₂}{1} + \dots + \frac{c_d}{1} ≤ C.
  \]
  Hence, $1 ≤ r_n ≤ C$ for every $n ≥ 0$.
\end{proof}

\begin{figure}[tbp]
  \centering
  \includestandalone{figures/fibonacci-convergence}
  \caption{
    Illustration of the convergence proof for ratios of Fibonacci numbers.
    The points represent the ratios $r_n$.
    The sequences $s_n$ and $t_n$ are the minimum and maximum from the previous
    $d$ ratios, respectively.
    Both sequences converge towards the same limit $φ$, so the ratios converge
    towards this limit, too.
  }
  \label{fig:fibonacci-convergence}
\end{figure}

Showing that the sequence is monotone would be enough to show the convergence.
However, the sequence is clearly not monotone.
Even the ratios of the original Fibonacci sequence alternate between increasing
and decreasing.
So we cannot possibly prove that the higher-order sequences are monotone.
Instead, we bound the sequence between two other sequences $s_n$ and $t_n$
and show that the two sequences are converging to the same limit.
From the squeeze theorem, it follows that $r_n$ converges to the same limit.
The main idea is illustrated in Figure~\ref{fig:fibonacci-convergence}.
The sequences are
\[
  s_n = \min\{r_n, r_{n-1}, …, r_{n-d} \}, \qquad t_n = \max\{r_n, r_{n-1}, …, r_{n-d}\}
\]
Because $r_n$ clearly lies between the two sequences, it must converge to the same limit as $s_n$ and $t_n$.
For these sequences, we already know from the previous lemma that they are bounded,
so it only remains to be shown that they are monotone,
i.e. $s_n$ is increasing and $t_n$ is decreasing.
After which we can show that they are converging to the same limit.

\begin{lemma}
  The sequences $s_n$ and $t_n$ are monotone.
\end{lemma}

\begin{proof}
  Each ratio can be represented as a convex combination of the previous ratios, i.e.
  \[
    r_{n+1} = λ₀ r_n + λ₁ r_{n-1} + \dots + λ_d r_{n-d}
  \]
  using $λ_i = F(n - i) / F(n + 1)$.
  To show that this is indeed a convex combination, all coefficients $λ_i$
  need to be nonnegative and $λ₀ + λ₁ + \dots + \lambda_d = 1$.
  The former follows from the fact that Fibonacci numbers are always increasing,
  while the latter follows simply from the definition of the Fibonacci numbers.
  Because $s_n$ is the minimum and $t_n$ the maximum of $r_n, r_{n-1}, …, r_{n-d}$,
  we can bound the next maximum by
  \[
    t_{n+1} ≤ r_{n+1} = λ₀ r_n + λ₁ r_{n-1} + \dots + λ_d r_{n-d} ≤ λ₀ t_n + λ₁ t_n + ⋯ + λ_d t_n = t_n.
  \]
  and the next minimum by
  \[
    s_{n+1} ≥ r_{n+1} = λ₀ r_n + λ₁ r_{n-1} + \dots + λ_d r_{n-d} ≥ λ₀ s_n + λ₁ s_n + ⋯ + λ_d s_n = s_n.
  \]
  Therefore, $s_n ≤ s_{n+1} ≤ t_{n+1} ≤ t_n$.
\end{proof}

\begin{lemma}
  The sequences $s_n$ and $t_n$ are converging to the same limit.
\end{lemma}

% TODO: Finish this proof
\begin{proof}
  For a contradiction, suppose there exists some $δ > 0$ such that $t_n - s_n > δ$ for every $n ≥ 0$.
  Out of the previous ratios, there must be one ratio $r_k$ exactly equal to $t_n$.
  Therefore,
  \begin{align*}
    s_{n+1} ≥ r_n & = λ₀ r_{n-d-1} + λ₁ r_{n-d} + ⋯ + λ_d r_{n-1} \\
                  & ≥ λ_k t_n + \sum_{i ≠ k} λ_i s_n \\
                  & = λ_k t_n + (1 - λ_k) s_n = s_n + λ_k (t_n - s_n) ≥ s_n + λ_k δ.
  \end{align*}
  We have
  \[
    λ_k = \frac{F(n+k)}{F(n+d+1)} = \frac{1}{r_{n+k} r_{n+k+1} \dots r_{n+d+1}} ≥ \frac{1}{(d+1)^{d+1-k}}.
  \]
  Hence, $λ_k$ is always greater than some constant $c > 0$ for every $k ≥ 0$.
  But then
  \[
    s_{n+i} ≥ s_{n+i-1} + c δ ≥ s_{n+i-2} + 2c δ ≥ \dots ≥ s_n + i c δ
  \]
  and $s_{n+i}$ would always increase as $i$ approaches infinity,
  which contradicts Lemma~\ref{lem:fib-bounded}.
  Therefore, $δ = 0$ and it follows that $s_n$ and $t_n$ are approaching the same limit.
\end{proof}

% TODO: Show that this converges! We're still missing the convergence criteria
\begin{theorem}
  Let $F$ be a linear recurrence with coefficients $c_0, \dots, c_d ≥ 0$
  and let $φ$ be the real positive root of its characteristic polynomial.
  Then,
  \[
    \lim_{n \to \infty} \frac{F(n + 1)}{F(n)} = φ.
  \]
\end{theorem}

\begin{proof}
  By the previous lemma, the ratios $r_n$ approach some limit $r ∈ ℝ$. It follows:
  \[
    r
    = \lim_{n → ∞} r_n
    = \lim_{n → ∞} 1 + \frac{c_d}{r_{n-1}} + \frac{c_{d-1}}{r_{n-1} r_{n-2}} + ⋯ + \frac{a₀}{r_{n-1} r_{n-2} \dots r_{n-d}}.
  \]
  Hence, each denominator in the sum approaches $r^i$,
  which results in the following polynomial:
  \[
    r = 1 + \sum_{i = 1}^d \frac{c_{d - i}}{r^i}
    \iff
    r^{d+1} = c₀ + c₁ r + \dots + c_d r^d,
  \]
  which directly corresponds to a root of its characteristic polynomial.
  Furthermore, the ratios are always positive, so $r = φ$.
\end{proof}

\begin{corollary}
  The ratios $F(n + i) / F(n)$ converge to $φ^i$ for $i > 1$.
\end{corollary}

This concludes the connection between Fibonacci numbers and their golden ratios.
As previously shown, the ratios of $d$-bonacci numbers approach the root of the polynomial $x^{d+1} - x^d - ⋯ - x - 1$
and the ratios of the other Fibonacci numbers approach the root of the polynomial $x^{d+1} - x^d - 1$.
For both of these roots, the generalized Euclidean algorithm becomes periodic after $d$ iterations.
So they represent some of the first algebraic numbers, for which we know that the algorithm becomes periodic.

\chapter{Multidimensional Continued Fractions}
\label{ch:mdcf}

% TODO: This needs to be rewritten since we have already introduced a
% representation in the previous chapter on the generalized Euclidean algorithm
For quadratic irrationals, we used continued fractions to represent them as a
periodic sequence of integers.
The continued fractions were constructed using the Euclidean algorithm.
Naturally, we can generalize continued fractions to higher dimensions using the
generalized Euclidean algorithm.
This leads to a concept of Multidimensional Continued Fractions (MCFs), which could
potentially be an answer of Hermite's question.
Since they are based on the generalized Euclidean algorithm,
they extend previous generalizations of continued fractions based on the
Jacobi-Perron algorithm, like bifurcating or ternary continued fractions \cite{Gupta00}.

This chapter introduces the concept of MCFs and discusses their potential as
an answer to Hermite's question.
We begin by defining what they are and deriving many properties similar to
continued fractions.
The chapter contains two main results for MCFs.
The first is that they converge under certain conditions
and the second is that periodic MCFs always consist of algebraic numbers with degree $≤ d+1$.
For the latter, we will first analyze the geometry behind MCFs in the same
style as Klein did for continued fractions.
What is missing from this chapter is the other direction,
that MCFs containing certain algebraic numbers are always periodic.
This will be discussed in more detail in the next chapter.

% ==============================================================================
\section{Construction Using the Generalized Euclidean Algorithm}
% ==============================================================================

% TODO: Why do we need a bottom-up definition? Because the construction only
% associates the integer sequence with a real number, we don't have any way to
% calculate which number it corresponds to.
% Also, the top-down construction gives us no notion of convergents.
The construction of MCFs using the generalized Euclidean algorithm gives us a
top-down definition, where we start with a vector $x$ and derive its
representation using the algorithm.
The continued fraction defined on page~\pageref{def:cont-frac}
use a bottom-up definition:
We begin with the continued fraction $[a₀]$ of a single coefficent
and then we inductively define the continued fraction $[a₀; a₁, …, aₙ]$
based on the smaller continued fraction $[a₁; a₂, …, aₙ]$.
For MCFs, we can similarly derive a bottom-up definition based on the inverse
pivot operation.

Let $x ∈ ℝ^d$ and $a = \floor{x}$.
Suppose that $x' = \mathrm{pivot}_ℓ(x)$ for a given index $ℓ ∈ \{1, …, d\}$.
We can derive $x$ from the vectors $x'$ and $a$ as follows:
By definition,
\[
  x_ℓ' = \frac{1}{x_ℓ - a_ℓ}
  \; \text{ and } \;
  x_i' = \frac{x_i - a_i}{x_ℓ - a_ℓ}
  \; \text{ for all } i ≠ ℓ.
\]
By rearranging these equations, we can calculate $x$ from $x'$ and $a$ using
\[
  x_ℓ = a_ℓ + \frac{1}{x_ℓ'}
  \; \text{ and } \;
  x_i = a_i + \frac{x_i'}{x_ℓ'}
  \; \text{ for all } i ≠ ℓ.
\]
Thus, we have an inverse function of the pivot operation.

The inverse function does not require the index $ℓ$ used in $x' = \mathrm{pivot}_ℓ(x)$.
The reason is that $x_ℓ' = 1/\{x_ℓ\}$ is always the largest element, since $\{x_i\} < 1$ for every element.
So we can derive $ℓ$ from the vector $x'$ by finding the index of the largest element.
Therefore, the inverse function $\mathrm{pivot}^{-1}$ only takes
a vector $x' ∈ ℝ^d$ and a vector $a ∈ ℤ^d$ as input.
It calculates the vector $x$ according to the equations above.
In summary, we have
\[
  \mathrm{pivot}_ℓ(x) = x' \iff \mathrm{pivot}^{-1}(a, x') = x.
\]

Using this inverse operation we can directly derive the bottom-up definition for MCFs.
So far we have only used integer vectors $a ∈ ℤ^d$,
however, in the definition we will allow any real vector, just like with the continued fractions.
Again this is for subsequent lemmas, where we will need rational or even real vectors coefficients.

\begin{definition}
  Given a sequence of $d$-dimensional real vectors $(r^{(n)})_{n ≥ 0}$,
  the \emph{multidimensional continued fraction} $[r^{(0)}; r^{(1)}, …]$ is defined as
  \[
    [r^{(0)}; r^{(1)}, …] = \lim_{n → ∞} [r^{(0)}; r^{(1)}, …, r^{(n)}],
  \]
  where the finite continued fractions $[r^{(0)}; r^{(1)}, …, r^{(n)}]$
  are defined inductively as
  \[
    [r^{(0)}] = r^{(0)},
    \qquad
    [r^{(0)}, r^{(1)}, …, r^{(n)}]
    = \mathrm{pivot}^{-1}\big(r^{(0)}, [r^{(1)}, r^{(2)}, …, r^{(n)}]\big).
  \]
\end{definition}

For the representation to be correct, we require $\max_i r_i^{(n)} ≠ 0$ for $n ≥ 1$.
This is similar to the continued fractions, where only the first value could be zero,
while all subsequent values had to be positive.
For the multidimensional counterpart we only require that the pivot element is not zero.
The other values in the coefficients $r^{(n)}$ can assume any non-negative value, including zero.

\begin{example}
  Consider the MCF $x = [(1,\, 2); (3,\, 2),\, (4,\, 5)]$.
  By definition,
  \begin{align*}
    x & = \mathrm{pivot}^{-1}\Bigl((1,\, 2),\, [(3,\, 2); (3,\, 5)]\Bigr) \\
      & = \mathrm{pivot}^{-1}\Bigl((1,\, 2),\, \mathrm{pivot}^{-1}\bigl((3,\, 2),\, [(3,\, 5)]\bigr)\Bigr) \\
      & = \mathrm{pivot}^{-1}\Bigl((1,\, 2),\, \mathrm{pivot}^{-1}\bigl((3,\, 2),\, (3,\, 5)\bigr)\Bigr).
  \end{align*}
  We start with the inner most pivot operation,
  where the largest element in $(4,\, 5)$ is $5$.
  Therefore, we invert the second element and divide the first element by $5$
  and we add the vector $(3,\, 2)$ to the result, which leads to
  \begin{align*}
    x & = \mathrm{pivot}^{-1}\bigl((1,\, 2),\, (3,\, 2) + (3/5,\, 1/5)\bigr)
        = \mathrm{pivot}^{-1}\bigl((1,\, 2),\, (18/5,\, 11/5)\bigr).
  \end{align*}
  In the next iteration, $18/5$ is the largest element.
  Therefore, we invert of the first element and divide the second element by $18/5$,
  resulting in
  \begin{align*}
    x & = (1,\, 2) + (5/18,\, 11/5 · 5/18) = (23/18,\, 41/18).
  \end{align*}
  Thus, $[(1,\, 2); (3,\, 2),\, (3,\, 5)] = (23/18,\, 41/18)$.
\end{example}

The terminology from one-dimensional continued fractions naturally carry over to its
multidimensional counterpart.
Given an infinite MCF representation~$[r^{(0)}; r^{(1)}, …]$ of a vector $x ∈ ℝ^d$, we define the following:

\begin{itemize}
  \item The \emph{$k$-th convergent} of $x$ is the finite MCF $[r^{(0)}; r^{(1)}, …, r^{(n)}]$.
  \item The \emph{$k$-th complete quotient} is the MCF $[r^{(n)}; r^{(n+1)}, …]$.
  \item The MCF is \emph{eventually periodic} if there exists an index~$n₀ ≥ 0$
    and a period~$k ≥ 1$ such that $aₙ = a_{n+k}$ and $ℓₙ = ℓ_{n+k}$
    for every $n ≥ n₀$.
    The MCF is \emph{purely periodic} if $n₀ = 0$.
\end{itemize}

For the analysis of MCFs,
we begin with two lemmas which generalize Lemma~\ref{lem:cf-nesting} and
Lemma~\vref{lem:cf-wallis}.
The first allows us to merge the last two coefficients of an MCF.
The second gives us a formula for calculating the convergents.

\begin{lemma}
  \label{lem:mdcf-nesting}
  Let $x ∈ ℝ^d$, then
  \[
    [r^{(0)}; r^{(1)}, …, r^{(n)}, x]
    = [r^{(0)}; r^{(1)}, …, r^{(n-1)}, \mathrm{pivot}^{-1}(r^{(n)}, x)]
  \]
\end{lemma}

\begin{proof}
  If $n = 0$, then by definition,
  \[
    [r^{(0)}; x] = \mathrm{pivot}^{-1}(r^{(0)}, [x]) = \mathrm{pivot}^{-1}(r^{(0)}, x) = [\mathrm{pivot}^{-1}(r^{(0)}, x)].
  \]
  Suppose the lemma holds for any $n ≥ 0$, then
  \begin{align*}
    [r^{(0)}; r^{(1)}, …, r^{(n+1)}, x]
    & = \mathrm{pivot}^{-1}(r^{(0)}, [r^{(1)}; r^{(2)}, …, r^{(n+1)}, x]) \\
    & = \mathrm{pivot}^{-1}(r^{(0)}, [r^{(1)}; r^{(2)}, …, r^{(n)}, \mathrm{pivot}^{-1}(r^{(n)}, x)] \\
    & = [r^{(0)}; r^{(1)}, …, r^{(n)}, \mathrm{pivot}^{-1}(r^{(n+1)}, x)]. \qedhere
  \end{align*}
\end{proof}

% TODO: Explain how to derive the sequences
In the lemma for continued fractions,
we defined the convergents of a continued fraction~$pₙ/qₙ$
using a linear recurrence based on the previous two terms~$p_{n-1}/q_{n-1}$ and $p_{n-2}/q_{n-2}$.
For MCFs, we can similarly derive a recursive formula to derive the values of
the convergent vector $(p₁/q, \dots, p_d/q)$ using the previous convergents.
Deriving the sequence is more involved than the one-dimensional case,
since we have an additional pivot index $ℓ$ at each step.

There are two types of sequences:
A sequence of vectors $P_0^{(n)}, P_1^{(n)}, …, P_d^{(n)} ∈ ℤ^{d+1}$ and a sequence
of scalars $Q_0^{(n)}\!,\, Q_1^{(n)}\!,\, …, Q_d^{(n)} ∈ ℤ$.
The initial terms are
\[
  \begin{aligned}
    P_0^{(-1)} & = 0, & P_i^{(-1)} & = e_i, \\
    Q_0^{(-1)} & = 1, & Q_i^{(-1)} & = 0,
  \end{aligned}
\]
where $e_i$ is the $i$-th unit vector.
For the remaining terms,
Let $ℓ$ be the index of the maximum element in the complete quotient $x^{(n)}$.
Then, the next terms in the sequences are:
\[
  \begin{aligned}
    P_ℓ^{(n)} & = P_0^{(n-1)} + P_1^{(n-1)} r_1^{(n)} + ⋯ + P_d^{(n-1)} r_d^{(n)}, &
    P_0^{(n)} & = P_{ℓₙ}^{(n-1)}, \\
    Q_ℓ^{(n)} & = Q_0^{(n-1)} + Q_1^{(n-1)} r_1^{(n)} + ⋯ + Q_d^{(n-1)} r_d^{(n)}, &
    Q_0^{(n)} & = Q_{ℓₙ}^{(n-1)}.
  \end{aligned}
\]
The terms $P_i^{(n)}$ and $Q_i^{(n)}$ where $i ≠ ℓ$ are carried over from the previous iteration,
i.e.
\[
  P_i^{(n)} = P_i^{(n-1)}, Q_i^{(n)} = Q_i^{(n-1)}.
\]

The idea behind these sequences is that they behave like the generalized
Euclidean algorithm, but in reverse.
We can consider the vectors $P_0^{(n)}, P_1^{(n)}, …, P_d^{(n)}$ as the basis,
which is reduced by the algorithm.
The Euclidean algorithm terminates when one vector is an integral combination
of the other vectors, because then the remainder is zero.
The sequence $P_0^{(n)}$ starts with the zero vector,
i.e. when the basis has been fully reduced.
Therefore, $P_0^{(n)}$ represents the remainder $c$ and the vectors $P_1^{(n)},
…, P_d^{(n)}$ represent the basis $B$.
Then, the formula calculates a new vector using an integral combination of the
old vectors and stores it in $P_ℓ^{(n)}$.
This corresponds directly to the modulo and exchange operation of the
generalized Euclidean algorithm.
As the vectors $P_0^{(n)}, P_1^{(n)}, …, P_d^{(n)}$ would grow infinitely, we
divide them by the scalars $Q_0^{(n)}, Q_1^{(n)}, …, Q_d^{(n)}$ to ensure that
they converge to a limit as $n$ increases.

% TODO: Ensure that this is correct for the first index!
\begin{lemma}
  \label{lem:mdcf-wallis}
  Let $r^{(0)}, r^{(1)}, …, r^{(n-1)}, x ∈ ℝ^d$, then
  \[
    [r^{(0)}; r^{(1)}, …, r^{(n-1)}, x]
    = \frac{P_0^{(n-1)} + P_1^{(n-1)} x_1 + ⋯ + P_d^{(n-1)} x_d}{Q_0^{(n-1)} + Q_1^{(n-1)} x_1 + ⋯ + Q_d^{(n-1)} x_d}.
  \]
\end{lemma}

\begin{proof}
  If $n = 0$, then
  \[
    [x]
    = x
    = \frac{x}{1}
    = \frac{0 + e₁ x₁ + ⋯ + e_d x_d}{1 + 0 x₁ + ⋯ + 0 x_d}
    = \frac{P₀^{(0)} + P₁^{(0)} x₁ + ⋯ + P_d^{(0)} x_d}{Q_0^{(0)} + Q_1^{(0)} x₁ + ⋯ + Q_d^{(0)} x_d}.
  \]
  Suppose the lemma holds for $n ≥ 0$,
  we show that it also holds for $n+1$.
  From the previous lemma, it follows that
  \begin{align*}
    [r^{(0)}; r^{(1)}; …, r^{(n)}, x] & = [r^{(0)}; r^{(1)}, …, r^{(n-1)}, \mathrm{pivot}^{-1}(r^{(n)}, x)].
  \end{align*}
  Let $y = \mathrm{pivot}^{-1}(r^{(n)}, x)$ and $ℓ = \argmax_i x_i$.
  By the induction hypothesis,
  \begin{align*}
    [r^{(0)}; r^{(1)}; …, r^{(n)}, x] & = \frac{x_ℓ}{x_ℓ} · \frac{P_0^{(n-1)} + P_1^{(n-1)} y_1 + ⋯ + P_d^{(n-1)} y_d}{Q_0^{(n-1)} + Q_1^{(n-1)} y_1 + ⋯ + Q_d^{(n-1)} y_d},
  \end{align*}
  where the fraction is expanded with $x_ℓ/x_ℓ$ such that the numerator and denominator are each multiplied by $x_ℓ$.
  The numerator can then be simplified as follows:
  \begin{align*}
    & \hphantom{{} = {}} x_ℓ \left( P_0^{(n-1)} + P_ℓ^{(n-1)} y_ℓ + \sum_{i ∉ \{0,ℓ\}} P_i^{(n-1)} y_i \right) \\
    & = x_ℓ \left( P_0^{(n-1)} + P_ℓ^{(n-1)} \left( r_ℓ^{(n)} + \frac{1}{x_ℓ} \right) + \sum_{i ∉ \{0,ℓ\}} P_i^{(n-1)} \left(r_i^{(n)} + \frac{x_i}{x_ℓ} \right) \right) \\
    & = P_0^{(n-1)} x_ℓ + P_ℓ^{(n-1)} r_ℓ^{(n)} x_ℓ + P_ℓ^{(n-1)} + \sum_{i ∉ \{0,ℓ\}} P_i^{(n-1)} r_i^{(n)} x_ℓ + P_i^{(n-1)} x_i \\
    & = \underbrace{\left( P_0^{(n-1)} + P_ℓ^{(n-1)} r_ℓ^{(n)} + \sum_{i ∉ \{0,ℓ\}} P_i^{(n-1)} r_i^{(n)} \right)}_{P_ℓ^{(n)}} x_ℓ
      + \underbrace{P_ℓ^{(n-1)}}_{P_0^{(n)}}
      + \sum_{i ∉ \{0,ℓ\}} \underbrace{P_i^{(n)}}_{P_i^{(n)}} x_i \\
    & = P_0^{(n)} + P_1^{(n)} x_1 + ⋯ + P_d^{(n)} x_d.
  \end{align*}
  The simplification for the denominator is identical.
  Therefore,
  \[
    [r^{(0)}; r^{(1)}, …, r^{(n)}, x]
    = \frac{P_0^{(n)} + P_1^{(n)} x_1 + ⋯ + P_d^{(n)} x_d}{Q_0^{(n)} + Q_1^{(n)} x_1 + ⋯ + Q_d^{(n)} x_d}.
    \qedhere
  \]
\end{proof}

% ==============================================================================
\section{Infinite Multidimensional Continued Fractions and their Convergence}
% ==============================================================================

% TODO: I don't like how we say that the requirement is from Perron, when it's
% not really from him, but just based on his version.
% TODO: The indices here are wrong...
So far in our analysis, we have implicitly assumed that the MCF $[a^{(0)}; a^{(1)}, a^{(2)} …]$
constructed using the generalized Euclidean algorithm always converges to the
original input vector $x ∈ ℝ^d$.
The aim of this section is to show that the convergents actually live up to
their name and converge towards the vector $x$ as $n$ increases.

Let $x ∈ ℝ$ and let $[a^{(0)}; a^{(1)}, …]$ be its MCF expansion,
constructed using the generalized Euclidean algorithm.
The vector $x^{(n)}$ denotes the $n$-th complete quotient of $x$
and $r^{(n)}$ denotes the $n$-th convergent.
We say that $ℓ ∈ \{1, …, d\}$ is the \emph{pivot index} of $x^{(n)}$,
if $x^{(n)} = \mathrm{pivot}_ℓ(x^{(n-1)})$.
With this index $ℓ$, the $n$-th convergent is defined as
\[
  r^{(n)}
  = (r_1^{(n)}, …, r_d^{(n)})
  = \left( \frac{P_{1ℓ}^{(n)}}{Q_ℓ^{(n)}}, \dots, \frac{P_{dℓ}^{(n)}}{Q_ℓ^{(n)}} \right)
  = \frac{P_ℓ^{(n)}}{Q_ℓ^{(n)}}.
\]
We say that $r^{(n)}$ converges towards $x$ if
\[
  x_i = \lim_{n → ∞} r_i^{(n)} \text{ for every } i ∈ \{1, …, d\}.
\]
The proof for this is based on Perron's original proof \cite{Perron07} for the
convergence of his algorithm.
For the proof we need three additional requirements:
\begin{enumerate}
  \item
    For every $n ≥ 0$,
    \[
      0 < \frac{1}{a_ℓ^{(n)}} ≤ A
      \quad \text{ and } \quad
      0 ≤ \frac{a_i^{(n)}}{a_ℓ^{(n)}} ≤ A \quad \text{ for every } i ≠ ℓ,
    \]
    where $ℓ$ is the index of the largest element in the complete quotient $x^{(n)}$.
  \item
    During the construction of the MCF,
    every index $ℓ ∈ \{1, …, d\}$ is used infinitely often.
    Formally, for every $ℓ ∈ \{1, …, d\}$ and $N ≥ 0$,
    we can find an index $n ≥ N$ such that $ℓ$ is the pivot index of $x^{(n)}$.
  \item
    The distance between the same index is bounded by some constant $L$,
    i.e. for every $n ≥ 0$ we can find a index $m ∈ \{n+1, …, n+L\}$ such that
    the pivot indices of $x^{(m)}$ and $x^{(n)}$ are the same.
\end{enumerate}

The first requirement is already satisfied by the generalized Euclidean algorithm for $A = 1$.
Because $x_ℓ^{(n)}$ is the largest element
and the integer part of every element in the complete quotient is determined by $a_i^{(n)}$,
the element $a_ℓ^{(n)}$ must also be the largest element in $a^{(n)}$.
Furthermore $a_ℓ^{(n)}$ cannot be zero due to the requirements.
Hence, this condition is satisfied for $A = 1$.
Nevertheless,
this condition can be used when replacing the floor function with a different function.
If one can guarantee that the new function satisfies the first requirement,
then the convergence will still hold.

The last two requirements are specifically for MCFs.
The vector $r^{(n)}$ is only one possible convergent out of several.
If $n$ is large enough, then we can consider the other vectors
$P_i^{(n)}/Q_i^{(n)}$ with $i ≠ ℓ$ as convergent vectors,
since $x_i^{(m)}$ must have been the largest element at some index $m < n$.
We will call $r^{(n)}$ the \emph{primary convergent} and the other ones
\emph{secondary convergents}.
When updating the secondary convergents,
we exchange $P_0^{(n)} / Q_0^{(n)}$ with $P_ℓ^{(n)}/Q_ℓ^{(n)}$
and replace  $P_ℓ^{(n)}/Q_ℓ^{(n)}$ with a new vector.
Thus, we only move one of the secondary convergents.
The second requirement guarantees that we move every secondary convergent
and the last requirement guarantees that we have moved every convergent after at most $L$ iterations.

\begin{example}
  Consider the MCF $x = [(2, 1); \overline{(2, 1)}]$,
  where the vector $(2, 1)$ repeats periodically.
  This MCF does not meet the requirements for this section.
  In this case, the first complete quotient is $x^{(1)}$, and the pivot index is $\ell = 1$,
  since the coefficient $a^{(1)} = (2, 1)$ determines the integer part
  of each coordinate in $x^{(1)}$, and $2$ is the largest entry.
  The following complete quotient $x^{(2)}$ only affects the fractional part
  of each coordinate, so the first coordinate continues to be the pivot.

  However, this violates the second requirement of the proof,
  which states that every index $\ell \in \{1, \dots, d\}$ must occur
  as a pivot index infinitely often.
  In this example, the pivot index is always $\ell = 1$,
  so the other index (in this case, $2$) is never selected.
  Therefore, the convergence proof presented in this section does not apply to this example.
\end{example}

Since we use the Euclidean algorithm to construct an MCF,
we can assume that all vectors $a^{(n)}$ are positive integers,
including the initial coefficient $a^{(0)}$.
Although the first coefficient could be negative,
the second coefficient is always positive.
If we can prove the convergence for the second complete quotient $x^{(1)} = [a^{(1)}; a^{(2)}, …]$,
then this implies the convergence for the whole MCF.
Thus, it suffices to show the convergence for the second complete quotient,
which is positive.
Finally, we assume that the target vector $x = (x₁, …, x_d)$ is irrational,
which means that the MCF is infinite.

\begin{lemma}
  \label{lem:conv-conv}
  There exists an integer $N ≥ 0$ such that for all $n ≥ N$,
  there are nonnegative coefficients $λ₀^{(n)}, λ₁^{(n)}, …, λ_d^{(n)}$
  satisfying $λ₀^{(n)} + λ₁^{(n)} + ⋯ + λ_d^{(n)} = 1$ and
  \[
    r_i^{(n)} = λ₀^{(n)} \frac{P_{i0}^{(n-1)}}{Q_0^{(n-1)}} + λ₁^{(n)} \frac{P_{i1}^{(n-1)}}{Q_1^{(n-1)}} + ⋯ + λ_d^{(n)} \frac{P_{id}^{(n-1)}}{Q_d^{(n-1)}}.
  \]
\end{lemma}

\begin{proof}
  We choose $N$ such that every index in the construction has been used at least once.
  In particular, we choose $N$ such that all denominators $Q_i^{(n-1)}$ are not zero.
  Let $ℓ$ be the index of the largest element in $x^{(n)}$.
  By Lemma~\ref{lem:mdcf-wallis}, we can calculate $P_{iℓ}^{(n)}$ using the previous values as follows:
  \begin{align*}
    P_{iℓ}^{(n)} = \sum_{k = 1}^d P_{ik}^{(n-1)} a_k^{(n)} + P_{i0}^{(n-1)}.
  \end{align*}
  Dividing by $Q_ℓ^{(n)}$ gives us
  \begin{align*}
    \frac{P_{iℓ}^{(n)}}{Q_ℓ^{(n)}} = \sum_{k = 1}^d \frac{P_{ik}^{(n-1)}}{Q_ℓ^{(n)}} a_k^{(n)} + \frac{P_{i0}^{(n-1)}}{Q_ℓ^{(n)}}.
  \end{align*}
  Using the coefficients $λ_k^{(n)} = a_k^{(n)} \frac{Q_k^{(n-1)}}{Q_ℓ^{(n)}}$ for $k ∈ \{1, …, d\}$
  and $λ₀^{(n)} = \frac{Q_0^{(n-1)}}{Q_ℓ^{(n)}}$,
  we can reformulate the convergent $r^{(n)}$ as
  \[
    λ₀^{(n)} \frac{P_{i0}^{(n-1)}}{Q_0^{(n-1)}} + λ₁^{(n)} \frac{P_{i1}^{(n-1)}}{Q_1^{(n-1)}} + ⋯ + λ_d^{(n)} \frac{P_{id}^{(n-1)}}{Q_d^{(n-1)}}.
  \]
  Because $N$ was chosen large enough such that every index has occurred at least once,
  the denominators $Q_i^{(n-1)}$ cannot be zero.
  Therefore, this is a valid representation of the convergent $r_i^{(n)}$.
  For the coefficients themselves, we require $λ₀ + λ₁ + ⋯ + λ_d = 1$ and $0 ≤ λᵢ ≤ 1$ for every index $i$.
  The first property follows from the definition of $Q_ℓ^{(n)}$:
  \[
    Q_ℓ^{(n)} = Q_0^{(n-1)} + Q_1^{(n-1)} a_1^{(n)} + ⋯ + Q_d^{(n-1)} a_d^{(n)}
  \]
  which is equivalent to
  \[
    1 = \frac{Q_0^{(n-1)}}{Q_ℓ^{(n)}} + \frac{Q_1^{(n-1)}}{Q_ℓ^{(n)}} a_1^{(n)} + ⋯ + \frac{Q_d^{(n-1)}}{Q_ℓ^{(n)}} a_d^{(n)} = λ₀^{(n)} + λ₁^{(n)} + ⋯ + λ_d^{(n)}.
  \]
  The second property follows from the fact that every vector $a^{(n)}$ is nonnegative.
  Therefore, the denominator $Q_ℓ^{(n)}$ is always less than or equal to $a_i^{(n)} Q_i^{(n-1)}$ and
  $λ_i$ is always bounded between $0$ and $1$.
\end{proof}

The lemma shows that the next convergent lies inside the convex hull of the secondary convergents.
We enclose the entire hull inside an axis-aligned bounding box
and we show that this box converges to a single point,
which implies the convergence of the original convergent sequence $r^{(n)}$.
At each iteration, the bounding box is characterized by its minimum and maximum corner,
which are defined as $s^{(n)} = (s_1^{(n)}, …, s_d^{(n)})$ and $t^{(n)} = (t_1^{(n)}, …, t_d^{(n)})$ with
\[
  s_i^{(n)} = \min\left\{\frac{P_{i0}^{(n)}}{Q_0^{(n)}}, \frac{P_{i1}^{(n)}}{Q_1^{(n)}}, …, \frac{P_{id}^{(n)}}{Q_d^{(n)}}\right\}
\]
and
\[
  t_i^{(n)} = \max\left\{\frac{P_{i0}^{(n)}}{Q_0^{(n)}}, \frac{P_{i1}^{(n)}}{Q_1^{(n)}}, …, \frac{P_{id}^{(n)}}{Q_d^{(n)}}\right\}.
\]
The primary convergent $r^{(n)}$ lies inside the box,
since it is one of the secondary convergent vectors
used in the definition of $s^{(n)}$ and $t^{(n)}$.
Therefore, if $s^{(n)}$ and $t^{(n)}$ converge,
then the sequence $r^{(n)}$ must converge, too.

\begin{lemma}
  The sequences $s_i^{(n)}$ and $t_i^{(n)}$ converge.
\end{lemma}

\begin{proof}
  We show that the sequences are monotone and bounded, which implies convergence.
  We begin with $s^{(n)}$ and show that this sequence is non-decreasing.
  The sequence is non-decreasing if every secondary convergent at iteration $n$
  is greater than the previous minimum $s^{(n-1)}$.
  Let $ℓ$ be the pivot index in iteration $n$.
  From the previous iteration, we already have for every $j ≠ ℓ$:
  \[
    \frac{P_{ij}^{(n)}}{Q_j^{(n)}} = \frac{P_{ij}^{(n-1)}}{Q_j^{(n-1)}} ≥ s_i^{(n-1)}.
  \]
  We can bound the primary convergent $r^{(n)}$ using Lemma~\ref{lem:conv-conv}, since
  \begin{align*}
    \frac{P_{iℓ}^{(n)}}{Q_ℓ^{(n)}}
    & = λ₀^{(n)} \frac{P_{i0}^{(n-1)}}{Q_0^{(n-1)}} + λ₁^{(n)} \frac{P_{i1}^{(n-1)}}{Q_1^{(n-1)}} + ⋯ + λ_d^{(n)} \frac{P_{id}^{(n-1)}}{Q_d^{(n-1)}} \\
    & ≥ λ₀^{(n)} s_i^{(n-1)} + λ₁^{(n)} s_i^{(n-1)} + ⋯ + λ_d^{(n)} s_i^{(n-1)}
      = s_i^{(n-1)}.
  \end{align*}
  Thus, $s_i^{(n)} ≥ s_i^{(n-1)}$ and $s^{(n)}$ is non-decreasing.
  We can show $t_i^{(n)} ≤ t_i^{(n-1)}$ by bounding the convergents from above using a similar argument.
  Futhermore, $s_i^{(n)} ≤ t_i^{(n)}$ for every index $n ≥ 0$.
  Thus, both sequences must converge.
\end{proof}

In order to show that the sequences $s^{(n)}$ and $t^{(n)}$ are actually
converging to the same limit, we first need the following crucial lemma about the
coefficients $λ₀^{(n)}, λ₁^{(n)}, …, λ_d^{(n)}$ from Lemma~\ref{lem:conv-conv}.
% TODO: Actually add the figure here. Or do we wanna use the combined one, which is already here?
%The main idea behind the lemma is illustrated in Figure~\ref{fig:lambda-pos}.
It states that one of the coefficients is always greater than some constant
independently of $n$ and importantly that coefficient is for the previous convergent $r^{(n-1)}$.
From the recurrence for the convergents,
it follows that we always carry over the primary convergent $r^{(n-1)}$ as one
of the secondary convergents in the next iteration.
Therefore, we cannot change its position.
However, we can change one of the other convergents.
But the area available for the next convergent $r^{(n)}$ is restricted by this lemma
such that the convex hull always shrinks by a constant amount.

% TODO: We could maybe save this by only stating that this occurs infinitely often.
% I think so, because if this cannot happen less than $d$ times. Otherwise, we must have a duplicate index.
% -> This should also work if we're doing this for any index.
\begin{lemma}
  \label{lem:lambda-pos}
  Let $ℓ$ be the pivot index in $x^{(n)}$,
  then $λ_ℓ^{(n)} > 1/C$ for some integer $C > 0$.
\end{lemma}

% TODO: Show that it is not equal to 1!
\begin{proof}
  The ratio between the coefficients $λ_i^{(n)}$ and $λ_ℓ^{(n)}$ is
  \begin{equation}
    \label{eq:lambda-ratio}
    \frac{λ_i^{(n)}}{λ_ℓ^{(n)}}
    = \frac{a_i^{(n)}}{a_ℓ^{(n)}} · \frac{Q_i^{(n-1)}}{Q_ℓ^{(n-1)}}
    ≤ A · \frac{Q_i^{(n-1)}}{Q_ℓ^{(n-1)}}.
  \end{equation}
  If $Q_i^{(n-1)} < Q_ℓ^{(n-1)}$,
  then we can guarantee that $λ_ℓ^{(n)} ≥ 1/C$.
  For $Q_0^{(n-1)} = Q_ℓ^{(n-2)}$, this is straightforward.
  But for the other values the issue is that $Q_i^{(n-1)}$ might be larger than $Q_ℓ^{(n-1)}$,
  if $i$ is the pivot index, for example.

  We solve this issue using the initial requirements.
  Specifically, the third requirement,
  which states that after at most $L$ iterations,
  we must have used $ℓ$ as the pivot index again.
  Therefore, there must have been some index $m$ between $n$ and $n - L$
  where we used $ℓ = \argmax_i x_i^{(m)}$ such that $Q_ℓ^{(n-1)} = Q_ℓ^{(m)}$.
  If $x^{(m)} = \mathrm{pivot}_ℓ(x^{(m-1)})$,
  then we must have $x_ℓ^{(m+1)} = 1/\{x_ℓ^{(m)}\} > 1$
  and therefore the integer part $a_ℓ^{(m+1)}$ must be at least $1$.
  Furthermore, if $k$ is the pivot index in the next iteration $m+1$,
  then
  \[
    Q_k^{(m+1)}
    = Q_0^{(m)} + Q_1^{(m)} a_1^{(m+1)} + ⋯ + Q_d^{(m)} a_d^{(m+1)}
    ≥ a_ℓ^{(m+1)} Q_ℓ^{(m)}
    ≥ \frac{1}{A} Q_ℓ^{(m)}.
  \]
  We can repeat this step for every pivot index from $m$ to $n$
  tracing the bound back from $Q_ℓ^{(m)}$ to $Q_i^{(n-1)}$.
  This path consists of at most $L$ steps.
  Thus, we can bound $Q_i^{(n-1)}$ from below by $Q_ℓ^{(n-1)}$ according to
  \[
    Q_i^{(n-1)} ≥ \frac{1}{A^L} Q_ℓ^{(m)} = \frac{1}{A^L} Q_ℓ^{(n-1)},
  \]
  which we can finally use to bound the fraction $Q_i^{(n-1)}/Q_ℓ^{(n-1)}$
  from Equation~\ref{eq:lambda-ratio}.
  The equation leads to the bound $λᵢ^{(n)} ≤ A^{L-1} λ_ℓ^{(n)}$,
  from which it follows that
  \[
    1 = λ₀^{(n)} + λ₁^{(n)} + ⋯ + λ_d^{(n)} ≤ λ_ℓ^{(n)} (1 + dA^{L-1}).
  \]
  Finally, if we set $C = 1 + dA^{L-1} > 0$, then $λ_ℓ^{(n)} > 1/C$.
\end{proof}

\begin{figure}[tbp]
  \centering
  \includestandalone{figures/convergence}
  \caption{
    Illustration of the proof for the convergence of MCFs.
    The points $r^{(i)}, r^{(j)}$ and $r^{(k)}$ represent the convergents and $x$ is
    the vector, which is approximated by them.
    If $r^{(k)}$ is the convergent of the current iteration,
    then either $r^{(i)}$ or $r^{(j)}$ has to be moved in the next iteration.
    However, neither can go in the red portion, since $λ_k^{(n)} > 1/C$.
  }
  \label{fig:convergence}
\end{figure}

If the sequences $s^{(n)}$ and $t^{(n)}$ converge to the same limit,
then that means that the bounding box shrinks to a point.
Therefore, the convergent $r^{(n)}$, which is inside this box, converges to the
same point.
However, the box could also turn out to be nonempty,
if the sequences approach different limits, i.e.
\[
  \lim_{n → ∞} s_i^{(n)} < \lim_{n → ∞} t_i^{(n)}.
\]
But the previous lemma contradicts this assumption.
The area of the convex hull always shrinks by a at least a constant proportion.
So if the sequences would approach different limits, then we can always find a
point at which all convergents step over the box and all of them are contained
in the supposed limit.
This is the main idea behind the following proof.

% TODO: Should we add the requirements, like every index occurs infinitely
% often (but the correct version)?
\begin{lemma}
  \label{lem:min-max-conv}
  The minimum $s_i^{(n)}$ and maximum $t_i^{(n)}$ are converging to the same
  limit.
\end{lemma}

\begin{proof}
  Let $s_i = \lim_{n → ∞} s_i^{(n)}$
  and $t_i = \lim_{n → ∞} t_i^{(n)}$ for every $i ≤ d$.
  Suppose $s_i < t_i$.
  For the minimum~$s_i$, there must exist an $ε > 0$ and an index $N ≥ 0$ such
  that for every $n ≥ N$,
  \[
    \frac{P_{ij}^{(n)}}{Q_j^{(n)}} ≥ s_i^{(n)} > s_i - ε.
  \]
  Let $ℓ$ be the index of the largest element in the previous complete quotient $x^{(n-1)}$.
  Since the limit $s_i - ε$ bounds the secondary convergents,
  we can use this to bound the primary convergent
  %$Q_i^{(n-1)} = Q_i^{(n-2)}$ and $Q_k^{(n-1)} = Q_0^{(n-2)} + Q_1^{(n-2)} a_1^{(n-1)} + ⋯ + Q_d^{(n-2)} a_d^{(n-1)}$. % TODO: Why was this here?
  \begin{align*}
    r_i^{(n)}
    & = λ₀^{(n)} \frac{P_{i0}^{(n-1)}}{Q_0^{(n-1)}} + λ₁^{(n)} \frac{P_{i1}^{(n-1)}}{Q_1^{(n-1)}} + ⋯ + λ_d^{(n)} \frac{P_{id}^{(n-1)}}{Q_d^{(n-1)}} \\
    & > λ_ℓ^{(n)} \frac{P_{iℓ}^{(n-1)}}{Q_ℓ^{(n-1)}} + \sum_{i ≠ ℓ} λ_i^{(n)} (s_i - ε) \\
    & = λ_ℓ^{(n)} r_i^{(n-1)} + (1 - λ_ℓ^{(n)}) (s_i - ε),
  \end{align*}
  or equivalently
  \begin{align*}
    r_i^{(n)} - s_i > λ_ℓ^{(n)} \left( r_i^{(n-1)} - s_i \right) - ε.
  \end{align*}
  Lemma~\ref{lem:lambda-pos} tells us that there is a positive constant $C$
  which bounds $λ_ℓ^{(n)}$ from below.
  It follows that
  \begin{align*}
    r_i^{(n)} - s_i > \frac{1}{C} \left( r_i^{(n-1)} - s_i \right) - ε
  \end{align*}
  for some constant $C > 0$.
  This inequality holds for all $n ≥ N$.
  Therefore, we can advance $n$ to some point $n + k$
  such that
  \begin{align*}
    r_i^{(n+k)} - s_i
    & > \frac{1}{C} \left( r_i^{(n+k-1)} - s_i \right) - ε \\
    & > \frac{1}{C} \left(\frac{1}{C} \left( r_i^{(n+k-2)} - s_i \right) - ε\right) - ε \\
    & = \frac{1}{C^2} \left(r_i^{(n+k-2)} - s_i \right) - ε\left(1 + \frac{1}{C}\right) \\
    & \, ⋮ \\
    & > \frac{1}{C^k} \left( r_i^{(n-1)} - s_i \right) - ε\left( \frac{1}{C} + \frac{1}{C^2} + ⋯ + \frac{1}{C^{k-1}} \right).
  \end{align*}
  Since every index occurs infinitely often,
  we can find one index $n$ where $r_i^{(n-1)} ≤ t_i$
  and another index $n+k$ where $r_i^{(n+k)} ≥ s_i$.
  Thus,
  \begin{align*}
    0 > \frac{1}{C^k} \left( t_i - s_i \right) - ε\left( \frac{1}{C} + \frac{1}{C^2} + ⋯ + \frac{1}{C^{k-1}} \right).
  \end{align*}
  But $ε$ can be chosen arbitrarily small,
  which is a contradiction to the fact that $t_i - s_i > 0$.
  Therefore, $t_i$ and $s_i$ converge to the same limit.
\end{proof}

We have shown that $s^{(n)}$ and $t^{(n)}$ converge to the same limit
and that $r^{(n)}$ therefore also converges to that limit.
What remains to be shown is what they converge to,
which is the original vector $x$ used for construction of the MCF.

% TODO: Fix statement, i.e. ℓₙ no longer exists...
\begin{theorem}
  \label{thm:mdcf-conv}
  The sequence $r^{(n)}$ converges to $x$.
\end{theorem}

% TODO: We should improve the last section of this proof,
% because I don't think this is correctly showing that it converges to the same
% limit. Or rather, we're already assuming that we are approaching x_i in the
% limit.
\begin{proof}
  Let $x^{(n)}$ denote the $n$-th complete quotient of $x$.
  By Lemma~\ref{lem:mdcf-wallis}, we can represent each element in $x$ as
  \[
    x_i = \frac{P_{i0}^{(n-1)} + P_{i1}^{(n-1)} x_1^{(n)} + ⋯ + P_{id}^{(n-1)} x_d^{(n)}}{Q_{i0}^{(n-1)} + Q_{i1}^{(n-1)} x_1^{(n)} + ⋯ + Q_{id}^{(n-1)} x_d^{(n)}}.
  \]
  Using a similar argument as in Lemma~\ref{lem:conv-conv}, we can represent this
  as a convex combination
  \begin{align*}
    x_i = μ₀^{(n)} \frac{P_{i0}^{(n-1)}}{Q_0^{(n-1)}}  + μ₁^{(n)} \frac{P_{i1}^{(n-1)}}{Q_1^{(n-1)}} + μ_d^{(n)} \frac{P_{id}^{(n-1)}}{Q_d^{(n-1)}}
  \intertext{with}
    μ_i^{(n)} = x_i^{(n)} \frac{Q_i^{(n-1)}}{Q_0^{(n-1)} + Q_d^{(n)} x_1^{(n)} + ⋯ + Q_d^{(n-1)} x_d^{(n)}}.
  \end{align*}
  From the previous lemma, we know that $r^{(n)}$ converges to some limit $x' ∈ ℝ^d$.
  Since every index occurs infinitely often,
  we can find an index $m ≤ n - 1$ such that the primary convergent $r_i^{(m)}$ is
  \[
    \frac{P_{ij}^{(n-1)}}{Q_j^{(n-1)}} = \frac{P_{ij}^{(m)}}{Q_j^{(m)}}.
  \]
  As $n$ increases there are infinitely many such convergents, so each term converges to $x_i'$.
  Thus, for each secondary convergent there are sufficiently small values $ε₀, ε₁, …, ε_d$ such that
  \begin{align*}
    x_i
    & = μ₀^{(n)} \frac{P_{i0}^{(n-1)}}{Q_0^{(n-1)}}  + μ₁^{(n)} \frac{P_{i1}^{(n-1)}}{Q_1^{(n-1)}} + μ_d^{(n)} \frac{P_{id}^{(n-1)}}{Q_d^{(n-1)}} \\
    & = μ₀^{(n)} (x_i' - ε₀) + μ₁^{(n)} (x_i' - ε₁) + ⋯ + μ_d^{(n)} (x_d' - ε_d) \\
    & = x_i' - (μ₀^{(n)} ε₀ + μ₁^{(n)} ε₁ + ⋯ + μ_d^{(n)} ε_d).
  \end{align*}
  However, as $n$ increases the values $εᵢ$ must become arbitrarily small
  and because the coefficients $μ₀^{(n)}, μ₁^{(n)}, …, μ_d^{(n)}$ all lie between $0$ and $1$,
  we can conclude that $x_i = x_i'$.
\end{proof}

One important example where this theorem applies is for periodic MCFs,
where the period contains every index as a pivot index.
Since the period is constant,
the distance between the same pivot indices is also constant.
Thus, they satisfy the requirements of this theorem.
Besides periodic MCFs, the theorem also includes other JPA-like algorithms.
For example, an algorithm may construct an MCF based on a fixed list of indices $[ℓ₁, ℓ₂, …, ℓₘ]$,
which is repeated throughout the construction.
If this list contains every possible index,
then this theorem would guarantee the convergence.

% ==============================================================================
\section{Geometrical Interpretation Based on Projective Spaces}
\label{sec:mdcf-geometry}
% ==============================================================================

% TODO: I'm not sold on the square brackets. With the list notation for the
% continued fraction, they're kind of ambiguous. Maybe we should just switch to
% regular parentheses instead.
In the geometrical interpretation of continued fractions,
each convergent $pₙ/qₙ$ is represented as a two-dimensional vector $(pₙ, qₙ)$.
These vectors approach an irrational line spanned by the vector $(1, α)$
where $α$ is some irrational number.
For the generalization to multidimensional continued fractions,
each convergent, which is a $d$-dimensional rational vector,
as a $(d+1)$-dimensional integer vector.
Specifically, given a convergent $r^{(n)} = (p₁/q₁, …, p_d/q_d) ∈ ℚ^d$,
we first find a common denominator $(p₁'/q, …, p_d'/q)$ and
then we map it to the vector $\hat r = (q, p₁', …, p_d') ∈ ℤ^{d+1}$.
Similarly, if we have a vector $(x₀, x₁, …, x_d) ∈ ℤ^{d+1}$,
then we map it back to $(x₁/x₀, …, x_d/x₀) ∈ ℚ^d$.

In this space, the representation for a particular convergent is not unique,
there can be multiple integer vectors representing the same convergent.
For example, if we have a vector $r ∈ ℤ^{d+1}$ for a convergent,
then we can multiply with some scalar $λ ∈ ℤ$ and get a new vector $r' = λ r$
which represents the same convergent.
This is because the scalar is eliminated when mapping it back to the rational vector:
\[
  λ (c₀, c₁, …, c_d)
  ↦ \left(\frac{λ c₁}{λ c₀}, …, \frac{λ c_d}{λ c_0} \right)
  = \left(\frac{c₁}{c₀}, …, \frac{c_d}{c_0} \right).
\]

We can extend this to any real vector in $ℝ^{d+1}$:
Two nonzero vectors $a, b ∈ ℝ^{d+1}$ are equivalent,
denoted as $a \sim b$, if $a = λ b$ for some scalar $λ ∈ ℝ$.
This relation then defines the equivalence class of an element $a ∈ ℝ^{d+1}$ as
\[
  [a] = \mathrm{span}(a) = \{\, λ a \mid λ ∈ ℝ, λ ≠ 0 \,\}.
\]
Formally, this is known as a real \emph{projective space}, denoted as $\mathbb{RP}^d$.
It is the set of equivalence classes in $ℝ^{d+1} \setminus \{0\}$ defined by the
equivalence relation $\sim$.
An element $x$ of this space is denoted as $[x₀, x₁, …, x_d]$,
where the square brackets indicate that this element is an equivalence
class.
In summary, we have the following mappings from $ℝ^d$ to $\mathbb{RP}^d$
and vice-versa:

\begin{center}
  \begin{tikzpicture}
    \matrix[
      column sep=2cm,
      nodes={text width=3cm, align=center},
    ] {
      \node (L0) {$\mathbb{R}^d$}; &
      \node (R0) {$\mathbb{RP}^d$}; \\
      \node (L1) {$(x₁, …, x_d)$}; &
      \node (R1) {$[1, x₁, …, x_d]$}; \\
      \node (L2) {$(x₁/x₀, …, x_d/x₀)$}; &
      \node (R2) {$[x₀, x₁, …, x_d]$}; \\
    };

    \draw[->] (L1) -- node[above] {} (R1);
    \draw[<-] (L2) -- node[above] {} (R2);
  \end{tikzpicture}
\end{center}

\begin{figure}[tbp]
  \centering
  \includestandalone{figures/projective-space}
  \caption{
    The convergents as vectors in a $d$-dimensional projective space.
    The ordinary convergents are projections at the $x₀ = 1$ plane.
  }
  \label{fig:projective-space}
\end{figure}

Next, we consider the pivot operation in a projective space.
Before we had to differentiate to cases to define the pivot operation,
now the pivot operation is just a linear operation on the projective coordinates.
For example, consider the vector $[1, x₁, x₂]$ and suppose that $0 ≤ x₁, x₂ < 1$.
A pivot operation with $ℓ = 1$ would result in the vector $[1, 1/x₁, x₂/x₁]$.
This vector is equivalent to $[x₁, 1, x₂]$.
Therefore, we can reformulate this operation as a coordinate swap of $x_ℓ$ with
the new coordinate $x₀$:
\[
  \begin{bmatrix}
    0 & 1 & 0 \\
    1 & 0 & 0 \\
    0 & 0 & 1 \\
  \end{bmatrix}
  ·
  \begin{bmatrix} 1 \\ x₁ \\ x₂ \\ \end{bmatrix}
  =
  \begin{bmatrix} x₁ \\ 1 \\ x₂ \\ \end{bmatrix}
  =
  \begin{bmatrix} 1 \\ 1/x₁ \\ x₂/x₂ \\ \end{bmatrix}.
\]
If $x₁$ and $x₂$ have a nonzero integer part,
then we first have to subtract this part from the vector $x$.
In the projective space, this is equivalent to a series of skew operations:
\[
  \begin{bmatrix}
    1 & 0 & 0 \\
    -\floor{x₁} & 1 & 0 \\
    0 & 0 & 1 \\
  \end{bmatrix}
  ·
  \begin{bmatrix}
    1 & 0 & 0 \\
    0 & 1 & 0 \\
    -\floor{x₂} & 0 & 1 \\
  \end{bmatrix}
  ·
  \begin{bmatrix} 1 \\ x₁ \\ x₂ \\ \end{bmatrix}
  =
  \begin{bmatrix} 1 \\ x₁ - \floor{x₁} \\ x₂ - \floor{x₂} \\ \end{bmatrix}.
\]
In general,
let $S(a)$ denote the skew matrix by a vector $a ∈ ℝ^d$,
i.e. the identity matrix with zeros in the first column swapped with $a$,
and let $R(ℓ)$ denote the permutation matrix,
which swaps $x_ℓ$ with $x_0$.
Then, we can define the pivot operation as
\[
  \mathrm{pivot}_ℓ(x) = R(ℓ) S(-\floor{x}) \hat x,
\]
where $\hat x = [1, x₁, …, x_d]$.

Importantly, we can invert each matrix.
For the matrix $S(a)$, we simply skew in the opposite direction
and the matrix $R(ℓ)$ is its own inverse, since swapping the same coordinate
twice yields the same vector.
Therefore, the whole operation can be easily reversed by inverting the matrix.
This is the equivalent of the inverse pivot operation in the projective space $\mathbb{RP}^d$.
Thus, we can reformulate MCFs as a series of matrix multiplications.
The projective definition of an MCF can be written as
\[
  [a^{(0)}] = \hat a^{(0)}, \qquad
  [a^{(0)}; a^{(1)}, …, a^{(n)}] = S(a₀) · R(ℓ) · [a^{(1)}; a^{(2)}, …, a^{(n)}],
\]
where $ℓ$ is the pivot index of $[a^{(1)}; a^{(2)}, …, a^{(n)}]$
and $\hat a^{(0)} = [1, a_1^{(0)}, …, a_d^{(0)}]$.

We can also use the projective space to dramatically simplify Lemma~\ref{lem:mdcf-wallis}.
Instead of two different sequences $P_i^{(n)}$ and $Q_i^{(n)}$, we simplify it to a single matrix sequence $(B^{(n)})_{n ≥ 0}$.
Each matrix $B^{(n)}$ consists of the column vectors $B₀^{(n)}, B₁^{(n)}, …, B_d^{(n)}$.
The sequence begins with $B^{(0)} = I_d$
and the remaining terms are calculated according to the recurrence
\begin{align*}
  B_ℓ^{(n)} = B^{(n-1)} \hat a^{(n)},
  \qquad B_i^{(n)} = B_i^{(n-1)},
  \qquad B_0^{(n)} = B_ℓ^{(n-1)},
\end{align*}
where $ℓ$ is the pivot index and $\hat a^{(n)} = [1, a_1^{(n)}, …, a_d^{(n)}]$.
By construction, $B^{(n)}$ is the combined matrix of the original sequences
$P_i^{(n)}$ and $Q_i^{(n)}$ defined for Lemma~\vref{lem:mdcf-wallis}:
\[
  B^{(n)} = \begin{bmatrix}
    Q_0^{(n)} & Q_1^{(n)} & ⋯ & Q_d^{(n)} \\
    P_0^{(n)} & P_1^{(n)} & ⋯ & P_d^{(n)} \\
  \end{bmatrix}.
\]
In fact, we can prove an equivalent statement from this lemma using the new sequence.

\begin{lemma}
  \label{lem:mdcf-wallis'}
  Let $x ∈ ℝ^d$ and $\hat x = (1, x₁, …, x_d)$, then
  \[
    [r^{(0)}; r^{(1)}, …, r^{(n-1)}, x] \sim B^{(n-1)} \hat x.
  \]
\end{lemma}

% TODO
\begin{proof}
  By construction of $B^{(n-1)}$, we have
  \begin{align*}
    B^{(n-1)} \hat x
    & = B_0^{(n-1)} \hat x_0 + B_1^{(n-1)} \hat x_1 + ⋯ + B_d^{(n-1)} x_d \\
    & =
    \begin{bmatrix}
      P_0^{(n-1)} \\
      Q_0^{(n-1)} \\
    \end{bmatrix} \hat x_0
    + \begin{bmatrix}
      P_1^{(n-1)} \\
      Q_1^{(n-1)} \\
    \end{bmatrix} \hat x_1
    + ⋯ + \begin{bmatrix}
      P_d^{(n-1)} \\
      Q_d^{(n-1)} \\
    \end{bmatrix} \hat x_d \\
    & = \begin{bmatrix}
      P_0^{(n-1)} \hat x_0 + P_1^{(n-1)} \hat x_1 + ⋯ + P_d^{(n-1)} \hat x_d \\
      Q_0^{(n-1)} \hat x_0 + Q_1^{(n-1)} \hat x_1 + ⋯ + Q_d^{(n-1)} \hat x_d \\
    \end{bmatrix}.
  \end{align*}
  Projecting this vector back to $ℚ^d$ results exactly in the vector
  \[
    r^{(n)} = \frac{P_0^{(n-1)} + P_1^{(n-1)} x_1 + ⋯ + P_d^{(n-1)} x_d}{Q_0^{(n-1)} + Q_1^{(n-1)} x_1 + ⋯ + Q_d^{(n-1)} x_d}
  \]
  from
  Lemma~\ref{lem:mdcf-wallis}.
\end{proof}

\iffalse
% TODO: Klein polyhedra?
Last but not least,
there also exists a generalization of Klein polygons to higher dimensions.
For three dimensions, they are known as Klein polyhedra
and in general they are known as Klein polytopes.

\begin{definition}
  Let $B = \{b₁, …, b_d\} ⊆ ℝ^d$ be a basis and let $C = \{ λ₁ b₁ + ⋯ + λ_d b_d \mid λ_i ≥ 0 \}$.
  The \emph{Klein polytope} $K$ generated by $B$ is defined as
  \[
    K = \mathrm{conv}(C ∩ ℤ^d \setminus \{\symbf 0\}).
  \]
\end{definition}

The connection between Klein polytopes and the convergents is not clear to me, however.
As before, we can show that the area between the convergents is empty.
The idea for a Klein polyhedra is visualized in Figure~\ref{fig:klein-polytope}.
This time, we consider the parallelepiped between the secondary convergents of the
current iteration and the previous iteration.
The number of integer points inside this parallelepiped can be calculated using
the determinant between the convergents.
Although the parallelepiped is not empty,
all of its integer points must be its the boundary.
Therefore, the volume between the convergents is empty.

Similarly,
Equation~\ref{eq:??} already shows that the line lies inside the convex hull of
the secondary convergents
and the convergence shows that the area of the convergents decreases towards zero.
\fi

% ==============================================================================
\section{Algebraic Numbers and Periodicity}
\label{sec:mcf-periodic}
% ==============================================================================

The proof for periodic MCFs is based on the same theorem for the Jacobi-Perron
algorithm, originally proven by Perron \cite{Perron07}.
We begin with the purely periodic case.
The idea behind this proof is that in a purely periodic MCF of a vector $x ∈ ℝ^d$,
the vector itself is an eigenvector for one of the matrices $B^{(k)}$.
Furthermore, the elements of this eigenvector can only be algebraic numbers with degree $≤ d+1$.

\begin{lemma}
  \label{lem:mdcf-purely-periodic}
  If there exists a purely periodic MCF for $x ∈ ℝ^d$,
  then $[ℚ(x₁, …, x_d) : ℚ] ≤ d+1$.
\end{lemma}

% TODO: Should we use x ≡ y or [x] = [y]?
\begin{proof}
  If the MCF is purely periodic, then there is some index $n ≥ 1$ such that $x = x^{(n)}$.
  Let $\hat x = [1, x₁, …, x_d]$ and $\hat x^{(n)} = [1, x_1^{(n)}, …, x_d^{(n)}]$.
  By Lemma~\ref{lem:mdcf-wallis'},
  \[
    \hat x \sim B^{(n)} \hat x^{(n)} \sim B^{(n)} \hat x \iff λ \hat x = B^{(n)} \hat x,
  \]
  for some nonzero $λ ∈ ℝ$.
  Therefore, we are looking for an eigenvector $\hat x$ and an eigenvalue $λ$ of $B^{(n)}$.
  The characteristic polynomial $\det(B^{(n)} - λ I)$ can have a degree of at most $d+1$,
  therefore the eigenvalue $λ$ is an algebraic number of degree $d+1$.
  For the eigenvector $\hat x$, we have to find a nontrivial solution to the
  homogeneous linear system
  \[
    (B^{(n)} - λ I) \hat x = 0.
  \]
  Each coefficient in this linear system is either an integer or $λ$ and is
  therefore contained in the field $ℚ(λ)$.
  Hence, we have $[ℚ(\hat x_0, \hat x_1, …, \hat x_d) : ℚ] ≤ d+1$.

  Finally, the eigenvector $\hat x$ has to be projected back from homogeneous coordinates $x$.
  Since $xᵢ = \hat xᵢ / \hat x₀$ is a rational expression and the values $\hat xᵢ$ and $\hat x₀$ are members of the same field $ℚ(λ)$,
  the projected value $xᵢ$ must also be contained in the same field.
  Therefore, each element in $x$ is an algebraic number
  and we have $[ℚ(x₁, …, x_d) : ℚ] ≤ d+1$.
\end{proof}

\begin{theorem}
  \label{thm:mdcf-periodic}
  If there exists a periodic MCF for $x ∈ ℝ^d$,
  then $[ℚ(x₁, …, x_d) : ℚ] ≤ d + 1$.
\end{theorem}

\begin{proof}
  Given such a MCF for $x$, let $x^{(k)}$ denote the $k$-th complete quotient
  of this fraction.
  Suppose that the MCF is periodic after $K ≥ 0$ with period $ℓ ≥ 0$, i.e.
  $x^{(k)} = x^{(k+ℓ)}$ for every $k ≥ K$.
  By Lemma~\ref{lem:mdcf-wallis'}, $\hat x \sim B^{(k)} \hat x^{(k)}$,
  which means that very element in the projection $x$ can be represented
  as a rational expression of $x^{(k)}$:
  \[
    x_i = \frac{∑_{j=1}^d B_{ij}^{(k)} x_j^{(k)} + B_{i0}^{(k)}}{\sum_{j=1}^d B_{0j}^{(k)} x_j^{(k)} + B_{00}^{(k)}}.
  \]
  % TODO: I think we should be more precise here since the elements could be
  % inside different fields. They are not, but this sentence does not indicate
  % that.
  From Lemma~\ref{lem:mdcf-purely-periodic},
  it follows that the elements of $\hat x^{(k)}$
  are contained in a field $ℚ(λ)$ with degree $[ℚ(λ) : ℚ] ≤ d+1$.
  Since $B^{(k)}$ consists solely of integers, every element in $x$ is contained in the same field $ℚ(λ)$.
  Therefore, they must also be algebraic numbers
  and we have $[ℚ(x₁, …, x_d) : ℚ] ≤ d+1$.
\end{proof}

This section has shown that a purely periodic multidimensional continued fraction must represent a vector of algebraic numbers.
More precisely, the components of the vector lie in a number field of degree at
most $d + 1$, where d is the dimension of the continued fraction.
This result establishes one direction of Hermite's question in higher dimensions.
The converse direction, however, remains open.

Although Theorem~\ref{thm:unimodular-algebraic} shows that for any algebraic
number $α$ there exists a unimodular matrix $U$ with eigenvector $(1, α, …, α^d)$,
it remains unclear whether the existence of this matrix implies a periodic MCF.
For two dimensions, we were able to show that the convergents and the vertices of a Klein polygon are equivalent.
Since $U$ preserves the Klein polygon, it implies that the continued fraction is periodic.
In higher dimensions the Klein polygon generalises to a Klein polytope.
While there exists a multidimensional analogue of Lagrange's theorem \cite{German08},
the connection between the convergents and the vertices of Klein polytopes is not known yet.

\chapter{Experimental Analysis on Multi-Dimensional Continued Fractions}
\label{ch:implementation}

In the previous chapter, we have analyzed
In particular, we saw that most MDCFs converge,
and that any periodic MDCF consists of algebraic numbers.
The former solves the first part of Hermite's question, but the latter solves
only one direction of the second part.
The first part of this chapter will focus on the remaining direction:
whether every algebraic number admits a periodic MDCF.

We have already seen in Chapter~\ref{ch:fibonacci},
that the simplest periodic MDCFs can be considered a generalization of the
golden ratio.
The first part of this chapter extends this result by presenting further
examples of periodic MDCF for algebraic numbers.
These include a wide range of cube roots,
all of which appear to have periodic representations under one particular strategy.
While this does not amount to a proof, the evidence strongly supports
the possibility of a positive answer to the second part of Hermite’s question.

The second part of this chapter focuses on the approximation rate of MDCFs.
For ordinary continued fractions, the convergents are known to give
exceptionally good approximations to irrational numbers.
Here, we test whether MDCFs offer similar approximation behavior in higher dimensions.

\iffalse
% ==============================================================================
\section{Implementation details}
% ==============================================================================

% ==============================================================================
\begin{Python}[
    float=tbp,
    numbers=left,
    label={lst:bfs},
    caption={
      The implementation of the brute-force search for finding a periodic representation.
      The program iterates over all sequences with a maximum length of $N$
      until it finds a duplicate vector.
    }
  ]
def brute_force_search(x, N):
  d = len(x)
  indices = list(range(d))
  for n in range(N):
    for L in product(indices, repeat=n):
      y = x
      seen = {y: 0}
      for i in range(n):
        y = pivot(y, L[i])
        if y in seen:
          j = seen[y]
          start = L[:j]
          period = L[j:i+1]
          return start, period
        seen[y] = i + 1
\end{Python}
% ==============================================================================

% Brute-force search
The goal is to find a periodic MDCF for some algebraic number $α$.
For an MDCF, we require a sequence of indices $ℓ₁, ℓ₂, …$ which determine the
element to pivot with.
To find this sequence, different types of searches were constructed.

The program is an implementation of the generalized Euclidean algorithm
in Python and SageMath.
More specifically, only the pivot operation from the algorithm was implemented,
since this the actual part that is relevant for the construction of an MDCF.

The actual code for the search I implemented is shown in Listing~\ref{lst:bfs}.
The input to the algorithm is a vector $x$ containing algebraic numbers of
degree $≤ d+1$ and a maximum search depth $N$.
If possible, it outputs two index sequences of the start and period for a periodic MDCF,
which represents the original input vector.
To find this sequence, the algorithm uses a simple brute-force search over all
possible sequences $\{1,\dots,d\}^*$ with a maximum length of $N$.
We simply try every sequence of possible pivot indices in a breadth-first manner.
So we begin with all sequences of length $1$ and see if any vector occurs twice.
If not, then we continue with all sequences of length $2$ and check again if
any previous vector has occurred twice.
We continue this process indefinitely until we hopefully find a duplicate vector.
To keep track of duplicates, the dictionary \verb|seen| maps vectors to the index,
where they first occurred.

% ==============================================================================
\begin{Python}[
    float=tbp,
    numbers=left,
    caption={
      The implementation of the nondeterministic search.
      The search begins with the empty sequence and then queries the strategy
      for the next valid sequences.
      At the same time, it checks whether any vector has occurred twice
      and stops once it has found a duplicate.
    },
    label={lst:nondet-search},
  ]
def nondeterministic_search(x, N, strat):
  d = len(x)
  sequences = [[]]
  for n in range(N):
    new_sequences = []
    for L in sequences:
      y = x
      seen = {y: 0}
      for l in L:
        y = pivot(y, l)
        seen[y] = i + 1
      for l in strat(x, L):
        z = pivot(y, l)
        if z in seen:
          j = seen[z]
          start = L[:j]
          period = L[j:i+1]
          return start, period
        new_sequences.append(L + [l])
    sequences = new_sequences
\end{Python}
% ==============================================================================

% Nondeterministic search
The main goal of the brute-force search is to find a periodic representation for a cubic root at all.
The problem is that the search is quite expensive.
For a given maximum search depth, the search will take $O(d^n)$ steps,
so even for the lowest dimension $d = 2$, the search is already expensive.
The hope is that the sequences share something in common such that we can find
a more optimized search strategy, ideally one which can be decided without
trying every possible combination.
Therefore, two more types of searches were studied.
The first type is a deterministic search,
which only looks at one possible path in the tree.
For example, the minimum strategy would only look at one possible path.
The other type is a non-deterministic search,
which looks at multiple paths simultaneously.
For this type of search, the main point of interest was the approximation rate of the convergents
and whether convergents are the best rational approximations of the original input vector $x$,
i.e. whether they fulfill
\[
  \left|x_i - \frac{p_i^{(n)}}{q^{(n)}}\right| < \frac{1}{q^{(n)} \sqrt[d]{q^{(n)}}}, \text{ for every } i ≤ d
\]
at each step.
The idea would be that this could be one possible optimization to the brute-force search.
Instead of trying every possible candidate, we only choose paths which lead to good approximations.
For $d = 1$, this is already known by Lemma~\ref{lem:cf-approx},
but for higher dimensions it is not known whether the convergents are good
approximations, yet.

The strategy is given the initial input $x ∈ ℝ^d$ and a sequence of indices $L
∈ \{1, …, d\}^*$ and must return all valid indices which can be appended to the
current sequence to form a new sequence.
For the brute-force search, the strategy always outputs the entire set; any index is allowed.
The minimum strategy only outputs one index and its the one where $x^{(n)}$ has
the minimum fractional value, if $n$ is the length of the list $L$.
\[
  \texttt{strat} \colon ℝ^d → \mathcal P(\{1, …, d\}), x ↦ \{ℓ_1, …, ℓ_k\}.
\]

% Deterministic search
For the deterministic search, a strategy $s$ is a function $R^d → \{1, …, d\}$
which takes in the complete quotient $x^{(n)} ∈ ℝ^d$ of the input vector $x ∈ R^d$
and the number of iterations $n$.
The strategy outputs only a single index $ℓ$,
which is used to find the next complete quotient by $x^{(n+1)} =
\mathrm{pivot}_ℓ(x^{(n)})$.
For example, the minimum strategy would be defined as
\[
  \texttt{strat}(x, n) = \underset{\substack{ℓ ∈ \{1, …, d\} \\ \{x_ℓ\} ≠ 0}}{\text{arg min}} \{x_ℓ\}.
\]
The additional index $n$ is used for example in the Jacobi-Perron algorithm,
which chooses always the next index in the sequence.
So,
\[
  \texttt{strat}(x, n) = (n \bmod d) + 1.
\]

For the non-deterministic search, only the approximation criterion was tested,
but with different rates, i.e. whether for some constant $c ≥ 1$,
\[
  \left|x_i - \frac{p_i^{(n)}}{q^{(n)}}\right| < \frac{c}{q^{(n)} \sqrt[d]{q^{(n)}}}, \text{ for every } i ≤ d.
\]

In summary,
there are three different types of searches.
The first is the brute-force search, which is used to find a periodic MDCF at all.
The second is the nondeterministic search, which is used to see how well the
convergents approximate the original input vector.
The third is the deterministic search, which is used to compare the different
strategies.
\fi

\iffalse
% ==============================================================================
\section{Periodic MDCFs for Cube Roots}
% ==============================================================================

% TODO: Measure the times for each MDCF and list them!
For the first search, the cube roots $\sqrt[3]{2}$ to $\sqrt[3]{100}$ were tested.
The MDCFs for these roots is listed in Table~\ref{tbl:cubics}.
For cubic irrationals there are $O(2^n)$ possible sequences,
so the search is already quite expensive.
The search for the first 30 cube roots had a maximum search depth of $24$ and
this already took over two hours to complete.
Almost half of the time was spent on the root for $\sqrt[3]{29}$
and it did not find a representation for this root.
Instead, I used different strategies to find an MDCF for this root.

Regarding the representation of the roots,
there is no perceivable patterns between the roots and their length.
However, this is to be expected since continued fractions also follow no simple
pattern between square roots and their length.
What the MDCFs share in common is that their periods always have an even length
and the period always contains both indices;
that is, there is no periodic sequence consisting solely of one repeated index
(e.g., only ones or only twos).

Apart from the even length, there is one specific set of roots which have a
predetermined period length
and they are the roots with the shortest period, i.e. $\sqrt[3]{2},
\sqrt[3]{3}, \sqrt[3]{9}$ and $\sqrt[3]{28}$.
Out of all observed roots, they have the shortest period with only two indices.
The reason comes from a theorem proven by Bernstein \cite{Bernstein71}.
In his analysis of the Jacobi-Perron algorithm,
he has shown that Jacobi-Perron algorithm is periodic for any root of the form
% TODO: Is the period length always 1?
% TODO: Check the correct conditions. Specifically is c < D correct?
\[
  \sqrt[3]{D^3 + c}, \qquad \text{ where } c < D \text{ and } c|D.
\]
and that the period length is exactly $2$ in the case of cubic irrationals.
So other roots with a short periodic sequence are $\sqrt[3]{65}, \sqrt[3]{66}$,
for example.

\begin{table}[tbp]
  \caption{Representation of $ψ = \sqrt[3]{4}$ using the brute-force search.}
  \label{table:cube-root-4}
  \centering
  \footnotesize
  \begin{tabular}{lllllll}
  \uzlhline
  \uzlemph{$\ell$} & \uzlemph{$x_1$} & \uzlemph{$x_2$} & \uzlemph{$x_1$} & \uzlemph{$x_2$} & \uzlemph{$a_1$} & \uzlemph{$a_2$} \\
  \hline
  $0$ & $\psi$ & $\psi^{2}$ & $1.5874$ & $2.51984$ & $0$ & $1$ \\
  \hline
  \hline
  $0$ & $\frac{1}{4} \psi^{2}$ & $\psi - 1$ & $0.62996$ & $0.5874$ & $1$ & $0$ \\
  $0$ & $\psi - 1$ & $\psi^{2} - \psi$ & $0.5874$ & $0.93244$ & $1$ & $1$ \\
  $1$ & $\frac{1}{3} \psi^{2} + \frac{1}{3} \psi - \frac{2}{3}$ & $\psi - 1$ & $0.70241$ & $0.5874$ & $1$ & $1$ \\
  $0$ & $\frac{1}{3} \psi - \frac{1}{3}$ & $\frac{1}{3} \psi^{2} + \frac{1}{3} \psi - \frac{2}{3}$ & $0.1958$ & $0.70241$ & $5$ & $3$ \\
  $1$ & $\psi^{2} + \psi - 4$ & $\psi - 1$ & $0.10724$ & $0.5874$ & $0$ & $1$ \\
  $1$ & $-\frac{2}{3} \psi^{2} + \frac{1}{3} \psi + \frac{4}{3}$ & $\frac{1}{3} \psi^{2} + \frac{1}{3} \psi - \frac{2}{3}$ & $0.18257$ & $0.70241$ & $0$ & $1$ \\
  $1$ & $\frac{1}{2} \psi^{2} - 1$ & $\frac{1}{4} \psi^{2} + \frac{1}{2} \psi - 1$ & $0.25992$ & $0.42366$ & $0$ & $2$ \\
  $0$ & $-\frac{1}{5} \psi^{2} + \frac{1}{5} \psi + \frac{4}{5}$ & $\frac{2}{5} \psi^{2} + \frac{3}{5} \psi - \frac{8}{5}$ & $0.61351$ & $0.36038$ & $1$ & $0$ \\
  \uzlhline
\end{tabular}

\end{table}

Finding periodic MDCFs for higher dimensions is even more difficult than two dimensions.
There are now $O(d^N)$ possible sequences with a maximum depth of $N$.
So each search is exponentially more expensive than the cubic case.
Some of the easier ones to find were again the roots identified by Bernstein.
\fi

% ==============================================================================
\section{Comparison of More Efficient Strategies}
% ==============================================================================

The first part of the analysis is meant to find periodic MDCFs for cubic
irrationals.
For the construction, we use the $\mathrm{pivot}$ operation from generalized
Euclidean algorithm and some sequence of indices, which we pivot with.
The choice of our indices essentially defines a tree
with the initial input vector $x$ as the root
and subsequent nodes as complete quotients of one MDCF.
The edges in this tree are the indices $ℓ$, which we use to get from one
complete quotient $x^{(n)}$ to another $x^{(n+1)}$ by computing $x^{(n+1)} =
\mathrm{pivot}_ℓ(x^{(n)})$.
We have found a periodic representation if there is a path from the root to a
node such that the last node occurs twice on the path.

Initially I tried searching for the MDCF of cube roots
using a brute-force search.
For any irrational vector $x$,
I tried every possible sequence of indices,
which could be used for the construction of an MDCF.
Using this search, I was only able to find the cube roots from $\sqrt[3]{2}$
and $\sqrt[3]{32}$ with the exception of $\sqrt[3]{29}$.
The results are listed in Table~\ref{tbl:cubics}.
The MDCFs I found using this search are the shortest possible representation
for the number, when we measure the combined length of the preperiod and
period.
Since there are a total of $O(2^N)$ sequences,
the search took quite a lot of time.
Therefore, I proceeded to look into better strategies,
which could more easily find MDCFs for cube roots.

\begin{table}[tbp]
  \caption{
    The shortest periodic index sequences for cube roots found using the
    brute-force search algorithm. The maximum search depth was set to $20$ and
    only the sequence for $29$ was not found. The roots for $8$ and $27$ are
    omitted since they are perfect cubes.}
  \label{tbl:cubics}
  \centering
  \begin{minipage}{0.48\textwidth}
\footnotesize
\begin{tabular}{ll}
\uzlhline
\uzlemph{$x$} & \uzlemph{MDCF} \\ \hline
$\sqrt[3]{2}$ & $\left[
\begin{matrix} 1 \\ 1 \\ \end{matrix}\,\,
\overline{
\begin{matrix} 0 \\ 1 \\ \end{matrix}\,\,
\begin{matrix} 2 \\ 1 \\ \end{matrix}\,\,
}\right]$ \\
$\sqrt[3]{3}$ & $\left[
\begin{matrix} 1 \\ 2 \\ \end{matrix}\,\,
\begin{matrix} 2 \\ 0 \\ \end{matrix}\,\,
\overline{
\begin{matrix} 1 \\ 5 \\ \end{matrix}\,\,
\begin{matrix} 2 \\ 1 \\ \end{matrix}\,\,
}\right]$ \\
$\sqrt[3]{4}$ & $\left[
\begin{matrix} 1 \\ 2 \\ \end{matrix}\,\,
\overline{
\begin{matrix} 1 \\ 0 \\ \end{matrix}\,\,
\begin{matrix} 1 \\ 1 \\ \end{matrix}\,\,
\begin{matrix} 1 \\ 3 \\ \end{matrix}\,\,
\begin{matrix} 1 \\ 1 \\ \end{matrix}\,\,
\begin{matrix} 1 \\ 0 \\ \end{matrix}\,\,
\begin{matrix} 1 \\ 1 \\ \end{matrix}\,\,
\begin{matrix} 0 \\ 1 \\ \end{matrix}\,\,
\begin{matrix} 3 \\ 1 \\ \end{matrix}\,\,
}\right]$ \\
$\sqrt[3]{5}$ & $\left[
\begin{matrix} 1 \\ 2 \\ \end{matrix}\,\,
\overline{
\begin{matrix} 1 \\ 1 \\ \end{matrix}\,\,
\begin{matrix} 2 \\ 0 \\ \end{matrix}\,\,
\begin{matrix} 2 \\ 1 \\ \end{matrix}\,\,
\begin{matrix} 0 \\ 1 \\ \end{matrix}\,\,
\begin{matrix} 0 \\ 1 \\ \end{matrix}\,\,
\begin{matrix} 0 \\ 1 \\ \end{matrix}\,\,
\begin{matrix} 1 \\ 0 \\ \end{matrix}\,\,
\begin{matrix} 3 \\ 0 \\ \end{matrix}\,\,
}\right]$ \\
$\sqrt[3]{6}$ & $\left[
\begin{matrix} 1 \\ 3 \\ \end{matrix}\,\,
\overline{
\begin{matrix} 1 \\ 0 \\ \end{matrix}\,\,
\begin{matrix} 4 \\ 1 \\ \end{matrix}\,\,
\begin{matrix} 2 \\ 1 \\ \end{matrix}\,\,
\begin{matrix} 0 \\ 2 \\ \end{matrix}\,\,
\begin{matrix} 0 \\ 1 \\ \end{matrix}\,\,
\begin{matrix} 0 \\ 1 \\ \end{matrix}\,\,
\begin{matrix} 1 \\ 0 \\ \end{matrix}\,\,
\begin{matrix} 3 \\ 1 \\ \end{matrix}\,\,
}\right]$ \\
$\sqrt[3]{7}$ & $\left[
\begin{matrix} 1 \\ 3 \\ \end{matrix}\,\,
\overline{
\begin{matrix} 1 \\ 0 \\ \end{matrix}\,\,
\begin{matrix} 10 \\ 7 \\ \end{matrix}\,\,
\begin{matrix} 0 \\ 1 \\ \end{matrix}\,\,
\begin{matrix} 1 \\ 0 \\ \end{matrix}\,\,
\begin{matrix} 0 \\ 1 \\ \end{matrix}\,\,
\begin{matrix} 4 \\ 0 \\ \end{matrix}\,\,
}\right]$ \\
$\sqrt[3]{9}$ & $\left[
\begin{matrix} 2 \\ 4 \\ \end{matrix}\,\,
\begin{matrix} 12 \\ 4 \\ \end{matrix}\,\,
\overline{
\begin{matrix} 6 \\ 12 \\ \end{matrix}\,\,
\begin{matrix} 12 \\ 6 \\ \end{matrix}\,\,
}\right]$ \\
$\sqrt[3]{10}$ & $\left[
\begin{matrix} 2 \\ 4 \\ \end{matrix}\,\,
\overline{
\begin{matrix} 6 \\ 4 \\ \end{matrix}\,\,
\begin{matrix} 3 \\ 6 \\ \end{matrix}\,\,
\begin{matrix} 0 \\ 2 \\ \end{matrix}\,\,
\begin{matrix} 6 \\ 0 \\ \end{matrix}\,\,
}\right]$ \\
$\sqrt[3]{11}$ & $\left[
\begin{matrix} 2 \\ 4 \\ \end{matrix}\,\,
\overline{
\begin{matrix} 4 \\ 4 \\ \end{matrix}\,\,
\begin{matrix} 2 \\ 4 \\ \end{matrix}\,\,
\begin{matrix} 0 \\ 2 \\ \end{matrix}\,\,
\begin{matrix} 6 \\ 0 \\ \end{matrix}\,\,
}\right]$ \\
$\sqrt[3]{12}$ & $\left[
\begin{matrix} 2 \\ 5 \\ \end{matrix}\,\,
\overline{
\begin{matrix} 1 \\ 4 \\ \end{matrix}\,\,
\begin{matrix} 5 \\ 0 \\ \end{matrix}\,\,
\begin{matrix} 0 \\ 1 \\ \end{matrix}\,\,
\begin{matrix} 0 \\ 2 \\ \end{matrix}\,\,
\begin{matrix} 0 \\ 2 \\ \end{matrix}\,\,
\begin{matrix} 3 \\ 0 \\ \end{matrix}\,\,
\begin{matrix} 0 \\ 1 \\ \end{matrix}\,\,
\begin{matrix} 2 \\ 2 \\ \end{matrix}\,\,
\begin{matrix} 0 \\ 2 \\ \end{matrix}\,\,
\begin{matrix} 6 \\ 1 \\ \end{matrix}\,\,
}\right]$ \\
$\sqrt[3]{13}$ & $\left[
\begin{matrix} 2 \\ 5 \\ \end{matrix}\,\,
\begin{matrix} 2 \\ 1 \\ \end{matrix}\,\,
\overline{
\begin{matrix} 1 \\ 0 \\ \end{matrix}\,\,
\begin{matrix} 5 \\ 3 \\ \end{matrix}\,\,
\begin{matrix} 1 \\ 3 \\ \end{matrix}\,\,
\begin{matrix} 1 \\ 0 \\ \end{matrix}\,\,
\begin{matrix} 3 \\ 2 \\ \end{matrix}\,\,
\begin{matrix} 2 \\ 0 \\ \end{matrix}\,\,
}\right]$ \\
$\sqrt[3]{14}$ & $\left[
\begin{matrix} 2 \\ 5 \\ \end{matrix}\,\,
\begin{matrix} 2 \\ 1 \\ \end{matrix}\,\,
\begin{matrix} 0 \\ 1 \\ \end{matrix}\,\,
\begin{matrix} 2 \\ 0 \\ \end{matrix}\,\,
\overline{
\begin{matrix} 3 \\ 15 \\ \end{matrix}\,\,
\begin{matrix} 2 \\ 1 \\ \end{matrix}\,\,
\begin{matrix} 0 \\ 1 \\ \end{matrix}\,\,
\begin{matrix} 1 \\ 1 \\ \end{matrix}\,\,
}\right]$ \\
$\sqrt[3]{15}$ & $\left[
\begin{matrix} 2 \\ 6 \\ \end{matrix}\,\,
\begin{matrix} 2 \\ 0 \\ \end{matrix}\,\,
\begin{matrix} 6 \\ 1 \\ \end{matrix}\,\,
\overline{
\begin{matrix} 4 \\ 4 \\ \end{matrix}\,\,
\begin{matrix} 6 \\ 4 \\ \end{matrix}\,\,
\begin{matrix} 2 \\ 0 \\ \end{matrix}\,\,
\begin{matrix} 0 \\ 18 \\ \end{matrix}\,\,
\begin{matrix} 2 \\ 0 \\ \end{matrix}\,\,
\begin{matrix} 6 \\ 0 \\ \end{matrix}\,\,
}\right]$ \\
$\sqrt[3]{16}$ & $\left[
\begin{matrix} 2 \\ 6 \\ \end{matrix}\,\,
\overline{
\begin{matrix} 1 \\ 0 \\ \end{matrix}\,\,
\begin{matrix} 1 \\ 0 \\ \end{matrix}\,\,
\begin{matrix} 0 \\ 1 \\ \end{matrix}\,\,
\begin{matrix} 8 \\ 3 \\ \end{matrix}\,\,
\begin{matrix} 1 \\ 0 \\ \end{matrix}\,\,
\begin{matrix} 0 \\ 2 \\ \end{matrix}\,\,
\begin{matrix} 1 \\ 1 \\ \end{matrix}\,\,
\begin{matrix} 1 \\ 0 \\ \end{matrix}\,\,
\begin{matrix} 2 \\ 2 \\ \end{matrix}\,\,
\begin{matrix} 3 \\ 1 \\ \end{matrix}\,\,
\begin{matrix} 0 \\ 1 \\ \end{matrix}\,\,
\begin{matrix} 1 \\ 0 \\ \end{matrix}\,\,
\begin{matrix} 1 \\ 1 \\ \end{matrix}\,\,
\begin{matrix} 5 \\ 4 \\ \end{matrix}\,\,
}\right]$ \\
\uzlhline
\end{tabular}
\end{minipage}
\begin{minipage}{0.48\textwidth}
\footnotesize
\begin{tabular}{ll}
\uzlhline
\uzlemph{$x$} & \uzlemph{MDCF} \\ \hline
$\sqrt[3]{17}$ & $\left[
\begin{matrix} 2 \\ 6 \\ \end{matrix}\,\,
\begin{matrix} 1 \\ 1 \\ \end{matrix}\,\,
\begin{matrix} 1 \\ 0 \\ \end{matrix}\,\,
\overline{
\begin{matrix} 3 \\ 0 \\ \end{matrix}\,\,
\begin{matrix} 0 \\ 3 \\ \end{matrix}\,\,
\begin{matrix} 0 \\ 1 \\ \end{matrix}\,\,
\begin{matrix} 0 \\ 1 \\ \end{matrix}\,\,
\begin{matrix} 0 \\ 4 \\ \end{matrix}\,\,
\begin{matrix} 3 \\ 3 \\ \end{matrix}\,\,
\begin{matrix} 1 \\ 0 \\ \end{matrix}\,\,
\begin{matrix} 1 \\ 1 \\ \end{matrix}\,\,
}\right]$ \\
$\sqrt[3]{18}$ & $\left[
\begin{matrix} 2 \\ 6 \\ \end{matrix}\,\,
\begin{matrix} 1 \\ 1 \\ \end{matrix}\,\,
\begin{matrix} 1 \\ 0 \\ \end{matrix}\,\,
\begin{matrix} 1 \\ 1 \\ \end{matrix}\,\,
\begin{matrix} 1 \\ 0 \\ \end{matrix}\,\,
\begin{matrix} 1 \\ 0 \\ \end{matrix}\,\,
\overline{
\begin{matrix} 5 \\ 17 \\ \end{matrix}\,\,
\begin{matrix} 1 \\ 0 \\ \end{matrix}\,\,
\begin{matrix} 1 \\ 0 \\ \end{matrix}\,\,
\begin{matrix} 1 \\ 0 \\ \end{matrix}\,\,
\begin{matrix} 1 \\ 0 \\ \end{matrix}\,\,
\begin{matrix} 1 \\ 1 \\ \end{matrix}\,\,
}\right]$ \\
$\sqrt[3]{19}$ & $\left[
\begin{matrix} 2 \\ 7 \\ \end{matrix}\,\,
\begin{matrix} 1 \\ 0 \\ \end{matrix}\,\,
\begin{matrix} 2 \\ 0 \\ \end{matrix}\,\,
\overline{
\begin{matrix} 0 \\ 2 \\ \end{matrix}\,\,
\begin{matrix} 0 \\ 1 \\ \end{matrix}\,\,
\begin{matrix} 0 \\ 3 \\ \end{matrix}\,\,
\begin{matrix} 5 \\ 0 \\ \end{matrix}\,\,
\begin{matrix} 1 \\ 0 \\ \end{matrix}\,\,
\begin{matrix} 2 \\ 1 \\ \end{matrix}\,\,
}\right]$ \\
$\sqrt[3]{20}$ & $\left[
\begin{matrix} 2 \\ 7 \\ \end{matrix}\,\,
\overline{
\begin{matrix} 1 \\ 0 \\ \end{matrix}\,\,
\begin{matrix} 2 \\ 1 \\ \end{matrix}\,\,
\begin{matrix} 1 \\ 3 \\ \end{matrix}\,\,
\begin{matrix} 1 \\ 2 \\ \end{matrix}\,\,
\begin{matrix} 1 \\ 0 \\ \end{matrix}\,\,
\begin{matrix} 1 \\ 0 \\ \end{matrix}\,\,
\begin{matrix} 1 \\ 2 \\ \end{matrix}\,\,
\begin{matrix} 6 \\ 3 \\ \end{matrix}\,\,
}\right]$ \\
$\sqrt[3]{21}$ & $\left[
\begin{matrix} 2 \\ 7 \\ \end{matrix}\,\,
\begin{matrix} 1 \\ 0 \\ \end{matrix}\,\,
\begin{matrix} 3 \\ 2 \\ \end{matrix}\,\,
\overline{
\begin{matrix} 6 \\ 3 \\ \end{matrix}\,\,
\begin{matrix} 1 \\ 1 \\ \end{matrix}\,\,
\begin{matrix} 0 \\ 1 \\ \end{matrix}\,\,
\begin{matrix} 2 \\ 2 \\ \end{matrix}\,\,
\begin{matrix} 1 \\ 0 \\ \end{matrix}\,\,
\begin{matrix} 0 \\ 22 \\ \end{matrix}\,\,
\begin{matrix} 1 \\ 0 \\ \end{matrix}\,\,
\begin{matrix} 3 \\ 0 \\ \end{matrix}\,\,
}\right]$ \\
$\sqrt[3]{22}$ & $\left[
\begin{matrix} 2 \\ 7 \\ \end{matrix}\,\,
\begin{matrix} 1 \\ 1 \\ \end{matrix}\,\,
\begin{matrix} 4 \\ 0 \\ \end{matrix}\,\,
\begin{matrix} 0 \\ 4 \\ \end{matrix}\,\,
\begin{matrix} 4 \\ 0 \\ \end{matrix}\,\,
\begin{matrix} 1 \\ 0 \\ \end{matrix}\,\,
\overline{
\begin{matrix} 3 \\ 20 \\ \end{matrix}\,\,
\begin{matrix} 1 \\ 0 \\ \end{matrix}\,\,
\begin{matrix} 4 \\ 2 \\ \end{matrix}\,\,
\begin{matrix} 0 \\ 3 \\ \end{matrix}\,\,
\begin{matrix} 5 \\ 1 \\ \end{matrix}\,\,
\begin{matrix} 1 \\ 1 \\ \end{matrix}\,\,
}\right]$ \\
$\sqrt[3]{23}$ & $\left[
\begin{matrix} 2 \\ 8 \\ \end{matrix}\,\,
\overline{
\begin{matrix} 1 \\ 0 \\ \end{matrix}\,\,
\begin{matrix} 5 \\ 0 \\ \end{matrix}\,\,
\begin{matrix} 2 \\ 1 \\ \end{matrix}\,\,
\begin{matrix} 1 \\ 2 \\ \end{matrix}\,\,
\begin{matrix} 4 \\ 2 \\ \end{matrix}\,\,
\begin{matrix} 1 \\ 1 \\ \end{matrix}\,\,
\begin{matrix} 3 \\ 3 \\ \end{matrix}\,\,
\begin{matrix} 0 \\ 1 \\ \end{matrix}\,\,
\begin{matrix} 10 \\ 1 \\ \end{matrix}\,\,
\begin{matrix} 0 \\ 1 \\ \end{matrix}\,\,
\begin{matrix} 4 \\ 11 \\ \end{matrix}\,\,
\begin{matrix} 1 \\ 1 \\ \end{matrix}\,\,
\begin{matrix} 5 \\ 1 \\ \end{matrix}\,\,
\begin{matrix} 5 \\ 1 \\ \end{matrix}\,\,
\begin{matrix} 2 \\ 5 \\ \end{matrix}\,\,
\begin{matrix} 6 \\ 2 \\ \end{matrix}\,\,
\begin{matrix} 0 \\ 1 \\ \end{matrix}\,\,
\begin{matrix} 1 \\ 0 \\ \end{matrix}\,\,
\begin{matrix} 0 \\ 2 \\ \end{matrix}\,\,
\begin{matrix} 7 \\ 1 \\ \end{matrix}\,\,
}\right]$ \\
$\sqrt[3]{24}$ & $\left[
\begin{matrix} 2 \\ 8 \\ \end{matrix}\,\,
\overline{
\begin{matrix} 1 \\ 0 \\ \end{matrix}\,\,
\begin{matrix} 7 \\ 2 \\ \end{matrix}\,\,
\begin{matrix} 1 \\ 1 \\ \end{matrix}\,\,
\begin{matrix} 2 \\ 5 \\ \end{matrix}\,\,
\begin{matrix} 1 \\ 0 \\ \end{matrix}\,\,
\begin{matrix} 0 \\ 1 \\ \end{matrix}\,\,
\begin{matrix} 0 \\ 2 \\ \end{matrix}\,\,
\begin{matrix} 6 \\ 4 \\ \end{matrix}\,\,
}\right]$ \\
$\sqrt[3]{25}$ & $\left[
\begin{matrix} 2 \\ 8 \\ \end{matrix}\,\,
\overline{
\begin{matrix} 1 \\ 1 \\ \end{matrix}\,\,
\begin{matrix} 1 \\ 1 \\ \end{matrix}\,\,
\begin{matrix} 2 \\ 4 \\ \end{matrix}\,\,
\begin{matrix} 0 \\ 1 \\ \end{matrix}\,\,
\begin{matrix} 2 \\ 0 \\ \end{matrix}\,\,
\begin{matrix} 3 \\ 4 \\ \end{matrix}\,\,
\begin{matrix} 1 \\ 0 \\ \end{matrix}\,\,
\begin{matrix} 12 \\ 1 \\ \end{matrix}\,\,
\begin{matrix} 6 \\ 3 \\ \end{matrix}\,\,
\begin{matrix} 4 \\ 2 \\ \end{matrix}\,\,
\begin{matrix} 3 \\ 4 \\ \end{matrix}\,\,
\begin{matrix} 22 \\ 6 \\ \end{matrix}\,\,
\begin{matrix} 0 \\ 1 \\ \end{matrix}\,\,
\begin{matrix} 3 \\ 11 \\ \end{matrix}\,\,
\begin{matrix} 1 \\ 0 \\ \end{matrix}\,\,
\begin{matrix} 12 \\ 2 \\ \end{matrix}\,\,
\begin{matrix} 0 \\ 1 \\ \end{matrix}\,\,
\begin{matrix} 1 \\ 0 \\ \end{matrix}\,\,
\begin{matrix} 0 \\ 2 \\ \end{matrix}\,\,
\begin{matrix} 7 \\ 1 \\ \end{matrix}\,\,
}\right]$ \\
$\sqrt[3]{26}$ & $\left[
\begin{matrix} 2 \\ 8 \\ \end{matrix}\,\,
\begin{matrix} 1 \\ 0 \\ \end{matrix}\,\,
\begin{matrix} 25 \\ 20 \\ \end{matrix}\,\,
\overline{
\begin{matrix} 1 \\ 1 \\ \end{matrix}\,\,
\begin{matrix} 8 \\ 17 \\ \end{matrix}\,\,
\begin{matrix} 1 \\ 0 \\ \end{matrix}\,\,
\begin{matrix} 25 \\ 18 \\ \end{matrix}\,\,
}\right]$ \\
$\sqrt[3]{28}$ & $\left[
\begin{matrix} 3 \\ 9 \\ \end{matrix}\,\,
\begin{matrix} 27 \\ 6 \\ \end{matrix}\,\,
\overline{
\begin{matrix} 9 \\ 27 \\ \end{matrix}\,\,
\begin{matrix} 27 \\ 9 \\ \end{matrix}\,\,
}\right]$ \\
$\sqrt[3]{30}$ & $\left[
\begin{matrix} 3 \\ 9 \\ \end{matrix}\,\,
\overline{
\begin{matrix} 9 \\ 6 \\ \end{matrix}\,\,
\begin{matrix} 3 \\ 9 \\ \end{matrix}\,\,
\begin{matrix} 0 \\ 3 \\ \end{matrix}\,\,
\begin{matrix} 9 \\ 0 \\ \end{matrix}\,\,
}\right]$ \\
$\sqrt[3]{31}$ & $\left[
\begin{matrix} 3 \\ 9 \\ \end{matrix}\,\,
\begin{matrix} 0 \\ 1 \\ \end{matrix}\,\,
\begin{matrix} 1 \\ 6 \\ \end{matrix}\,\,
\begin{matrix} 13 \\ 8 \\ \end{matrix}\,\,
\overline{
\begin{matrix} 1 \\ 0 \\ \end{matrix}\,\,
\begin{matrix} 5 \\ 9 \\ \end{matrix}\,\,
\begin{matrix} 0 \\ 7 \\ \end{matrix}\,\,
\begin{matrix} 29 \\ 2 \\ \end{matrix}\,\,
\begin{matrix} 2 \\ 7 \\ \end{matrix}\,\,
\begin{matrix} 14 \\ 1 \\ \end{matrix}\,\,
}\right]$ \\
$\sqrt[3]{32}$ & $\left[
\begin{matrix} 3 \\ 10 \\ \end{matrix}\,\,
\overline{
\begin{matrix} 2 \\ 12 \\ \end{matrix}\,\,
\begin{matrix} 4 \\ 2 \\ \end{matrix}\,\,
\begin{matrix} 1 \\ 1 \\ \end{matrix}\,\,
\begin{matrix} 2 \\ 44 \\ \end{matrix}\,\,
\begin{matrix} 1 \\ 0 \\ \end{matrix}\,\,
\begin{matrix} 4 \\ 1 \\ \end{matrix}\,\,
\begin{matrix} 6 \\ 1 \\ \end{matrix}\,\,
\begin{matrix} 6 \\ 4 \\ \end{matrix}\,\,
\begin{matrix} 0 \\ 2 \\ \end{matrix}\,\,
\begin{matrix} 2 \\ 0 \\ \end{matrix}\,\,
\begin{matrix} 2 \\ 1 \\ \end{matrix}\,\,
\begin{matrix} 8 \\ 9 \\ \end{matrix}\,\,
\begin{matrix} 3 \\ 3 \\ \end{matrix}\,\,
\begin{matrix} 9 \\ 1 \\ \end{matrix}\,\,
}\right]$ \\
\uzlhline
\end{tabular}
\end{minipage}

\end{table}

The code used for the comparison is shown in Listing~\ref{lst:det-search}.
It is a search on the tree guided by a specific strategy.
Each strategy is given the current input vector $x^{(n)}$ beginning with $x^{(0)} = x$.
It would then output one index $ℓ$ and the search would continue with
$x^{(1)} = \mathrm{pivot}_ℓ(x^{(0)})$.
The search would continue until we find a duplicate vector $x^{(n)} = x^{(m)}$
with $n < m$ on the path determined by the strategy.
Since they are the same, $x^{(n)}$ would mark the beginning of the period.

% ==============================================================================
\begin{Python}[
    float=tbp,
    numbers=left,
    caption={
      The implementation of the search for periodic MDCFs.
      The strategy \texttt{strat} outputs a single index $ℓ$, which is used
      for pivoting.
      The search stops once a duplicate vector $x$ has been found and the
      program returns the preperiod and period once found.
    },
    label={lst:det-search},
  ]
def search(x, N, strat):
  seen = {x: 0}
  for n in range(N):
    l = strat(x)
    x = pivot(x, l)
    if x in seen:
      j = seen[x]
      start = L[:j]
      period = L[j:i+1]
      return start, period
    else:
      seen[y] = i + 1
\end{Python}
% ==============================================================================

In summary, the following deterministic strategies were tried:
\begin{itemize}
  \item $\textbf{Min}, \textbf{Max}$: Choosing the minimum and maximum fractional value, respectively.
    These are the strategies which have been analyzed in Chapter~\ref{ch:fibonacci}.
  \item $\textbf{JPA}$: The Jacobi-Perron algorithm,
    which chooses indices in a fixed order.
    Specifically, it chooses the indices $1, 2, …, d$, and repeats this sequence indefinitely.
  \item $\textbf{JPA}'$: A modification of the Jacobi-Perron algorithm introduced by Podsypanin \cite{Podsypanin77}.
    Given a vector $x = (x₁, x₂)$ it chooses the index
    \[
      ℓ =
      \begin{cases}
        1, & \text{ if } x₁ > x₂, \\
        2, & \text{ if } x₁ < x₂.
      \end{cases}
    \]
    For higher dimensions, the algorithm chooses the largest element in $x$.
  \item $\textbf{TY}$:
    The algorithm of Tamura and Yasutomi \cite{Tamura09},
    which is based on the idea of the modified JPA.
    Given vector $x = (x₁, x₂)$, the algorithm chooses the index
    \[
      ℓ =
      \begin{cases}
        1, & \text{ if } \frac{x₁}{\sqrt{|N(x₁)|}} > \frac{x₂}{\sqrt{|N(x₂)|}}, \\
        2, & \text{ if } \frac{x₁}{\sqrt{|N(x₁)|}} < \frac{x₂}{\sqrt{|N(x₂)|}}.
      \end{cases}
    \]
    Again, the algorithm chooses the largest element in each iteration.
    However, it scales down each element by the square root of its norm.
    For higher dimensions, the algorithm chooses
    \[
      ℓ = \argmax_i \frac{x_i}{\sqrt[d+1]{|N(x_i)|}}.
    \]
  \item $\textbf{CC}$: Choosing the closest convergent.
    Out of the $d$ possible indices,
    we choose the one which produces the closest convergent $r^{(n)}$,
    which means that it minimizes the distance to the original input vector.
    This is measured either using the Euclidean norm $\|x - r^{(n)}\|_2$ or using the maximum norm $\|x - r^{(n)}\|_{\infty}$.
\end{itemize}
% TODO: Mention that the Tamura and Yasutomi algorithm was tested by the authors themselves for both cubic and quadratic cases.
% TODO: For how many steps did we run the construction?

Table~\ref{tbl:comparison} lists the results for this section.
The clear winner is the algorithm by Tamura and Yatusomi.
It has found a periodic representation for every cubic irrational I tested.

Since Tamura and Yasutomi's algorithm worked so well,
I have also tested it on fifth, sixth and seventh roots
from $\sqrt[d]{2}$ up to $\sqrt[d]{200}$.
After fourth roots, the algorithm no longer returns a periodic MDCF for all roots.
In fact, I have only found periodic MDCFs for the following roots:
\begin{itemize}
  \item Fifth roots:
    \sqrt[5]{2}, \sqrt[5]{7}, \sqrt[5]{11}, \sqrt[5]{13}, \sqrt[5]{19},
    \sqrt[5]{25}, \sqrt[5]{29}, \sqrt[5]{31}, \sqrt[5]{33}, \sqrt[5]{59},
    \sqrt[5]{82}, \sqrt[5]{123}, \sqrt[5]{152}.
  \item Sixth roots: \sqrt[6]{18}, \sqrt[6]{65}, \sqrt[6]{66},\sqrt[6]{198}.
  \item Seventh roots: \sqrt[6]{2}.
\end{itemize}

% ==============================================================================
\section{Usage in Simultaneous Approximation}
% ==============================================================================

The continued fractions play an important role in Diophantine approximation,
where the goal is to approximate real numbers using rational numbers.
In Lemma~\vref{lem:cf-approx}, we have already seen that the convergents
$pₙ/qₙ$ of a continued fraction $x$ approximate the represented number $x$
particularly well.
More specifically, that every convergent satisfies the bound
\[
  \left|α - \frac{pₙ}{qₙ}\right| < \frac{1}{qₙ^2}.
\]
The Lemma also follows from Dirichlet's approximation theorem.

For its multidimensional counterpart,
the question is whether they approximate the vector particularly well.
Approximating a vector instead of a single number is also known as simultaneous
Diophantine approximation.
Given an irrational vector $(α₁, …, α_d)$, the task is to find a good
rational approximation $(p₁/q, …, p_d/q)$ for every number $α_i$ at once.
Using the simultaneous version of Dirichlet's approximation theorem \cite{Schmidt80},
one can show that there are infinitely many rational vectors $(p₁/q, …, p_d/q)$,
which satisfy
\begin{equation}
  \label{eq:sim-approx}
  \left|α_i - \frac{p_i}{q}\right| ≤ \frac{1}{q^{1 + 1/d}}
  \quad
  \text{ for every } i ∈ \{1, …, d\}.
\end{equation}
Such vectors will be called \emph{good rational approximations} of $(α₁, …, α_d)$.
The idea would be that the convergents of MDCFs are such good rational approximations.
However, if that is not the case, there is still a possibility that some path
during the construction has good rational approximations.
From strongest to weakest, I have analyzed the following questions:
\begin{enumerate}
  \item Do all paths have convergents which are good rational approximations?
  \item Is there a path where all convergents are good rational approximations?
  \item Is there a path where infinitely many convergents are good rational approximations?
\end{enumerate}

To answer these question,
I have implemented a breadth-first search over the construction of an MDCF for the vector $x$.
The search begins with the root $x^{(0)} = x$ and it expands a node $x^{(n)} ∈ ℝ^d$
by adding the vector $\mathrm{pivot}_ℓ(x^{(n)})$ for every $ℓ ∈ \{1, …, d\}$ to the queue.
However, it only expands a node $x^{(n)}$
if its convergent vector $r^{(n)}$ is a good approximation,
i.e. satisfies the bound from Equation~\ref{eq:sim-approx}.
Otherwise, the node is considered a leaf.
The search terminates either after a maximum number of steps is reached or if
there are no more nodes in the queue.
The latter of which would show that there are examples, where no path with the
given approximation bound exists.
This covers the first and second questions.
For the third question, I have implemented a brute-force search,
which simply tries all possible sequences to test for the approximation rate.
The goal would be to see how many convergents satisfy the approximation bound.

Figure~\ref{fig:results-approx} shows the results of this analysis.

% TODO: When repeating the period multiple times, does this actually keep the
% approximation bound intact or does it violate it after some point?
For most cubic roots from the previous test,
there are MDCFs which satisfy the approximation bound.
However, they are notably different MDCFs than those found in the brute-force
search.
So some cubic roots can have different MDCFs
and not all of them must necessarily be good simultaneous approximations.

The most surprising result of this analysis is that there are some cubic roots
for which there exists no path which satisfies the approximation bound of Equation~\ref{eq:sim-approx}.
Perron already suggested that his algorithm does not satisfy this bound \cite{Perron07},
so a simple strategy ought to violate the bound at some point.
However, it turns out that no strategy can keep the approximation bound at each step.
One example is the vector $(\sqrt[3]{5}, \sqrt[3]{25})$.
In this case, the number of convergents starts out growing but quickly drops
after only a few number of iterations.

Since not all cubic roots can be approximated well using MDCFs,
I weakened the bound to allow all convergents which satisfy
\[
  \left|x_i - \frac{p_i}{q}\right| < \frac{c}{q^{1 + 1/d}} \qquad \text{ for every } i ≤ d,
\]
where $c$ is some constant independent of $n$.
The specific constant for each root is listed in Table~\ref{tbl:approx-const}.
% TODO: Make the actual table
\begin{table}[tbp]
  \centering
  \begin{tabular}{cc}
    \uzlhline
    Root & Constant \\
    \hline
    TODO & TODO \\
    \uzlhline
  \end{tabular}
  \caption{Approximation constants for the cubic roots.}
  \label{tbl:approx-const}
\end{table}

% TODO: Add plot for growth

% TODO: Add table for numbers which admit good approximations

% TODO: What about transcendental numbers?

\chapter{Conclusion}
\label{ch:conclusion}


\end{document}
