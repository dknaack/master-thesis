\documentclass[english,version-2020-11]{uzl-thesis}

\UzLThesisSetup{
  Masterarbeit,
  Logo-Dateiname        = {uzl-thesis-logo-itcs.pdf},
  Verfasst              = {am}{Institut für Theoretische Informatik},
  Titel auf Deutsch     = {Über elementare Eigenschaften einer mehrdimensionalen Verallgemeinerung des euklidischen Algorithmus},
  Titel auf Englisch    = {On Elementary Properties of a Multi-Dimensional Generalization of the Euclidean Algorithm},
  Autor                 = {Daniel Knaack},
  Betreuerin            = {Prof. Dr. Kim-Manuel Klein},
  Studiengang           = {Informatik},
  Datum                 = {18. Juni 2025},
  Abstract              = {TODO},
  Zusammenfassung       = {TODO},
  Numerische Bibliographie,
}

\UzLStyle{pagella contrast design}

\usepackage{standalone}
\usepackage{todonotes}

\usetikzlibrary{decorations.pathreplacing, calligraphy}

\newcommand\N{{\mathbb N}}
\newcommand\Z{{\mathbb Z}}
\newcommand\Q{{\mathbb Q}}
\DeclareMathOperator*{\argmin}{arg\,min}
\DeclareMathOperator*{\argmax}{arg\,max}
\newcommand\floor[1]{\lfloor#1\rfloor}
\newcommand\ceil[1]{\lceil#1\rceil}

\begin{document}

\chapter{Introduction}
\label{ch:intro}

In 1839, Charles Hermite wrote a letter to Jacobi~\cite{Hermite50} about the
representation of real numbers.
He asked whether there exists a representation of the real numbers as a
sequence of integers that is periodic if and only if the represented number
is a cubic irrational, i.e. the root of a cubic polynomial.
Although he posed the question nearly two centuries ago,
it remains unanswered to this day.

The standard way to represent numbers is through decimal notation.
A number is represented as a sequence of digits, which begins with the integer
part and is followed by a (potentially infinite) sequence of digits for the
fractional part.
If the decimal expansion of a number is finite, then the number is clearly rational.
Furthermore, if the decimal expansion is periodic, then the number is also rational.
The same behavior occurs for continued fractions and quadratic irrationals.
Continued fractions are fractions of the form
\[
  a₀ + \cfrac{1}{a₁ + \cfrac{1}{a₂ + \cfrac{1}{⋱}}},
\]
where $a₀, a₁, a₂, …$ are integers.
Every real number has a continued fraction expansion,
which can be constructed using the Euclidean algorithm.
If that continued fraction is periodic, then the number must be a quadratic irrational.
More importantly, the converse is also true:
If a number is a quadratic irrational,
then its continued fraction must be eventually periodic.
Naturally, we may ask whether we can extend this and
find a periodic representation for cubic irrationals.
However, no such representation exists yet.

The interest in a generalization to cubic irrationals
comes from the effectiveness of continued fractions in related fields.
The primary example is Diophantine approximation, where the goal is to
approximate real numbers using rational numbers as closely as possible.
It turns out that the best rational approximations are precisely those provided
by continued fractions.
A generalization of continued fractions could serve a similar role in the field
of simultaneous Diophantine approximation, where the goal is to approximate
multiple real numbers with a single rational vector.

Hermite's original question only applies to cubic irrationals,
but it can be easily generalized to any algebraic number:
Does there exist a representation of the real numbers as a sequence of integers
that is periodic if and only if the represented number is an algebraic number of
degree $d$?
There are two parts to this question.
The first is the representation of real numbers as a sequence of integers.
Each finite subsequence of the representation should give us a rational number,
which approaches the represented number as the sequence grows larger.
The second part is about the periodicity of the integer sequence.
In the original question, the sequence should repeat after some point if and
only if the represented number is a cubic irrational.
The general question asks whether a periodic sequence exists for any algebraic
number with a certain degree.

% ==============================================================================
\section{Background}
\label{sec:jacobi-perron}
% ==============================================================================

Since Hermite originally posed his question to Jacobi, it was Jacobi who first
attempted to answer it.
He developed an algorithm \cite{Jacobi68} inspired by the Euclidean algorithm,
which calculates the greatest common divisor of three numbers instead of two.
At each step,
the algorithm chooses the smallest number and uses it to divide the other two.
In the next triple, the other two numbers are replaced by their remainders.
This process is continued until the greatest common divisor is found.
Later, Oskar Perron generalized Jacobi's method to arbitrary dimensions \cite{Perron07},
resulting in what is now called the Jacobi–Perron Algorithm (JPA).
His algorithm is essentially a generalization of the Euclidean algorithm to $n$ numbers.
At each step, he still chooses the smallest element at each iteration.

Continued fractions are typically constructed using the Gauss transformation,
which is defined as
\begin{align*}
  T(x) = \frac{1}{x - \floor{x}}.
\end{align*}
This transformation is applied repeatedly until $T^m(x) = T^n(x)$ for $m ≠ n$,
in which case the continued fraction is periodic.
For example, $T(\sqrt{2}) = 1/(\sqrt{2} - 1) = \sqrt{2} + 1$
and $T^2(\sqrt{2}) = T(\sqrt{2} + 1) = \sqrt{2} + 1$.
Thus, the continued fraction of $\sqrt{2}$ is periodic after two iterations.

The Jacobi-Perron algorithm uses a similar transform,
which takes a vector $x = (x₁, …, x_d)$ as input and calculates
\begin{align*}
  T(x₁, …, x_d) =
  \left(
  \frac{x_2 - \floor{x_2}}{x_1 - \floor{x_1}},
  \frac{x_3 - \floor{x_3}}{x_1 - \floor{x_1}},
  …,
  \frac{x_d - \floor{x_d}}{x_1 - \floor{x_1}},
  \frac{1}{x_1 - \floor{x_1}}
  \right)
\end{align*}
Again, this transformation is applied repeatedly until $T^m(x) = T^n(x)$ for $m ≠ n$,
at which point it becomes periodic.
Perron showed that if the algorithm becomes periodic,
then each $x_i$ is an algebraic number with degree $d+1$.
However, he was not able to show the other direction.

In an attempt to find an algorithm which solves both directions,
previous work often replaces this transformation with a different one,
which can be considered JPA-type algorithms.
For example, a different transformation could use the second smallest number.
They usually consider only one particular transformation and iterate it until a period has been found.
Despite numerous different transformations being proposed,
Hermite's question remains open.

% ==============================================================================
\section{Contributions of this Thesis}
\label{sec:contributions}
% ==============================================================================

% The other algorithms focus on a single path, whereby the only use their own
% transformation function to find a periodic representation.
Whereas previous algorithms typically consider only a single transformation,
I present a framework that is able to choose any one of these transformations.
This framework, which I call \emph{multidimensional continued fractions} (MCF),
is based on a generalization of the Euclidean algorithm by Klein and
Reuter~\cite{Klein24}.

The main contribution of this thesis is a theoretical analysis of the
multidimensional continued fractions.
This includes two important theorems.
The first theorem shows that many multidimensional continued fractions converge.
This was neglected by Jacobi at first, but it was eventually shown by Perron.
Since then, the main focus has returned to the second part
with the convergence often being implicitly assumed.
However, the convergence is crucial for a correct representation of the real numbers.
For this thesis, I give an explicit proof that many multidimensional continued
fractions converge.
This covers the first part of Hermite's question.
The second theorem partially solves the second part of Hermite's question.
It shows that every periodic multidimensional continued fraction leads to an
algebraic number, thus solving the first direction.
The converse remains open.

Nevertheless, I have performed an experimental analysis on the MCFs,
which strongly suggests that every cubic irrational has a periodic MCF.
The aim of this analysis was to evaluate different strategies
for constructing periodic MCFs for algebraic numbers.
One of these strategies shows particular promise, having produced a periodic
MCF for every cubic and quartic irrational I tested..
In addition, I have evaluated the performance of MCFs in the field of
simultaneous Diophantine approximation.
Since ordinary continued fractions yield the best rational approximations
of a single real number, the idea for MCFs would be that they provide the best
rational approximations for real vectors.
With my analysis, I show that not all convergents lead to good rational approximations.
I provide a specific example where no MCF consistently produces good rational
approximations of a vector.

The basis of these MCFs is a generalization of the Euclidean algorithm from
Klein and Reuter \cite{Klein24}.
The initial aim was to analyze the algorithm's worst-case performance
and to determine whether there exists some generalization of Fibonacci numbers,
as they represent the worst-case for the classical Euclidean algorithm.
As such, a secondary result of this thesis is a proof that such Fibonacci numbers exist, at least for one strategy.
More importantly, I show that they represent the worst case for this strategy.
Using these numbers, I derive the multidimensional analogue of the golden
ratio, which can be seen as one of the simplest examples of a periodic MCF.

In summary, there are three main contributions:
\begin{enumerate}
  \item A new class of multidimensional continued fractions, including a proof
    of convergence and how they lead to algebraic numbers.
  \item An experimental analysis of the multidimensional continued fractions
    on cubic irrationals and their application in simultaneous
    Diophantine approximation.
  \item Worst-case bounds for the generalized Euclidean algorithm
    using higher-order Fibonacci numbers and multidimensional golden ratios.
\end{enumerate}

% ==============================================================================
\section{Related Work}
\label{sec:related-work}
% ==============================================================================

As previously mentioned,
one of the first algorithms studied for this problem is the JPA.
Initially developed by Jacobi and Perron,
the algorithm was analyzed again by Bernstein~\cite{Bernstein71},
who identified explicit classes of cubic irrationals which are periodic under
the JPA \cite{Bernstein64A, Bernstein65, Bernstein64B}.
However, numerical computations by Elsner and Hasse \cite{Elsner67} have shown
that the JPA seems to fail for certain cube roots.
This has led some to conjecture that the JPA does not provide an answer to Hermite's question.

Since then, numerous alternative algorithms have been
proposed~\cite{Assaf05, Hendy81, Schweiger13, Schweiger00}.
However, they usually build upon the Jacobi-Perron algorithm.
This includes the subtractive algorithms by
Brun~\cite{Brun19} and Selmer~\cite{Selmer67}
as well as the fully subtractive algorithms introduced by Schweiger~\cite{Schweiger95}.
Each algorithm is still fundamentally based on the idea of the JPA, though they
do not use the full remainder and instead subtract only the chosen element.
While the main focus has been on periodicity, Selmer's algorithm is one of the
few algorithms, for which convergence has been proven~\cite{Bruin15}.
However, none of them give a full answer to Hermite's question.

There are also generalizations of continued fractions to two dimensions,
called bifurcating~\cite{Gupta00} or ternary continued fractions~\cite{Daus22},
and they are constructed by reversing the JPA in two dimensions.
Traditionally, they only admitted integers as coefficients,
but Murru~\cite{Murru15} used rational coefficients to construct periodic
expansions for all cubic irrationals.
While this addresses one half of Hermite’s question,
it does not provide a representation for arbitrary real numbers.
It only provides a representation of cubic irrationals
and thus, it does not provide a full answer to Hermite's question.

More recently, Karpenkov has proposed two new algorithms \cite{Karpenkov24, Karpenkov21}.
The first, known as the $\sin^2$-algorithm, has been shown to be periodic for
every totally real cubic irrational, i.e. any root of a cubic polynomial
with three real roots.
The second is called the HAPD algorithm, and he conjectures that it is periodic
for all cubic irrationals, though this remains unproven.
Nevertheless, both algorithms are built upon transformations closely related to
those used in the JPA.

Apart from the algorithmic approach,
there has also been significant effort in a geometric generalization of
continued fractions.
Felix Klein famously interpreted the convergents of a continued fraction as
points on integer lattices \cite{Klein95}.
His interpretation has led to a geometric proof of Lagrange’s theorem, which
will also be presented in this thesis, following the work of Korkina~\cite{Korkina96}.
Arnol'd has suggested a generalization of Klein's interpretation to higher
dimensions \cite{Arnold98} and it was conjectured that they satisfy a
multidimensional equivalent of the Lagrange's famous theorem,
which shows that every quadratic irrational has an eventually periodic
continued fraction.
This conjecture was eventually proven by German \cite{German08},
which indicates that there could be a connection between the multidimensional
continued fractions and algebraic numbers.

Beyond multidimensional continued fractions, there have been alternative
generalizations of classical number-theoretic functions.
One notable example is the Minkowski question-mark function $?(x)$,
which maps a quadratic irrational $x$ to a rational number.
Efforts to define higher-dimensional analogues of this function aim to mirror
the relationship between algebraic numbers and their representations.
There has been an extension of this function to two dimensions~\cite{Beaver04},
but this has not yet solved Hermite's question.

% ==============================================================================
\section{Structure of this Thesis}
\label{sec:structure}
% ==============================================================================

Chapter~\ref{ch:preliminaries} introduces the necessary background for this thesis,
which primarily includes algebraic number theory and lattice theory.
Chapter~\ref{ch:quadratic} examines the case of quadratic irrationals and continued fractions.
It presents a geometric interpretation of continued fractions based on Klein polygons,
which is used in the subsequent proof of Lagrange's theorem.
Chapter~\ref{ch:generalized-euclidean} introduces the generalized Euclidean algorithm.
In Chapter~\ref{ch:fibonacci}, the generalized Euclidean algorithm is analyzed
for worst-case performance under two different strategies.
Each strategy results in a different generalization of the Fibonacci numbers
and its own definition of a golden ratio.
This golden ratio is the first case of a periodic representation of an algebraic number.
Building on this result, Chapter~\ref{ch:mdcf} generalizes continued fractions to higher dimensions
and presents the two main theoretical results of the thesis: convergence and periodicity.
Chapter~\ref{ch:implementation} analyzes the second part of Hermite's problem,
whether multidimensional continued fractions of algebraic numbers are always periodic.
This chapter presents examples of such continued fractions for cubic irrationals,
and compares different strategies for constructing these continued fractions.

\chapter{The Euclidean Algorithm}

\section{Analysis}

\begin{Pseudocode}
algorithm Euclidean(a, b)
  while $b ≠ 0$ do
    $(a, b) ← (b, a \bmod b)$
  end

  return $a$
end
\end{Pseudocode}

\begin{proposition}
  The ratio $b/a$ decreases by at least $1/2$ over two iterations.
\end{proposition}

\begin{proof}
  Suppose $b/a > 1/2$ in the first iteration.
  Over two iterations, we have
  \[
    \frac{b}{a} · \frac{a \bmod b}{b}
    ≤ \frac{b}{a} · \frac{a - b}{b}
    = 1 - \frac{b}{a}
    < 1 - \frac{1}{2}
    = \frac{1}{2}.
  \]
  Therefore, it decreases by at least $1/2$ over two iterations.
\end{proof}

\section{Fibonacci Numbers}

\begin{definition}
  The \emph{Fibonacci numbers} are defined as
  \begin{enumerate}
    \item $F(0) = F(1) = 1$.
    \item $F(n) = F(n - 1) + F(n - 2)$ for $n > 2$.
  \end{enumerate}
\end{definition}

\begin{proposition}[Lamé's Theorem \cite{Lame1844}]
  The Euclidean algorithm requires at most $5h$ steps,
  where $h = \log_{10}(b)$.
\end{proposition}

\begin{proof}
  \todo[inline]{Using Fibonacci numbers.}
\end{proof}

\section{Extension to the Real Numbers using Continued Fractions}

What if the ratio $a/b$ perfectly equals the golden ratio?
In this case, the Euclidean algorithm no longer runs on two integer inputs,
since the golden ratio is irrational.
Therefore, we extend the algorithm to the real numbers.
Our input now consists of a single number $x ∈ ℝ$, which in the original
algorithm represents the ratio between the two inputs.
This allows us to run the algorithm on the golden ratio.

Given $x ∈ ℝ$, the algorithm proceeds just as the original.
We assume $x$ actually represents the fraction $x/1$,
then we find a number $a₀ ∈ ℕ$ times until $x - 1 · a₀ < 1$.
Once this is the case, we swap the inputs, i.e. we calculate the inverse fraction $1/(x - a)$.
Now, we subtract $x - a$ from $1$ until $1 - a₁ (x - a₀) < x - a₀$
and then calculate the inverse fraction again.
We repeat this process until $x = 0$.
Subtracting $a_i$ from the numerator such that it is smaller than the
denominator and then calculating the inverse.

\begin{example}
  Let $x = 13/5$.
  The algorithm proceeds as follows:
  \[
    \begin{array}{rclcrcl}
      13/5 & = & 2 & · & 1   & + & 3/5 \\
         1 & = & 1 & · & 3/5 & + & 2/5 \\
       3/5 & = & 1 & · & 2/5 & + & 1/5 \\
       2/5 & = & 2 & · & 1/5 & + & 0.
    \end{array}
  \]
\end{example}

\begin{example}
  Let $x$ be the golden ratio.
  The algorithm proceeds as follows:
  \[
    \begin{array}{rclcrcl}
      φ & = & 1 & · & 1   & + & φ - 1 \\
         1 & = & 1 & · & (φ - 1) & + & (2 - φ) \\
       φ - 1 & = & 1 & · & (2 - φ) & + & (2φ - 3) \\
       2 - φ & = & 1 & · & (2φ - 3) & + & (5 - 3φ) \\
       & \vdots &
    \end{array}
  \]
\end{example}


The real algorithm computes the coefficients of the continued fractions representation of $x$.
Continued fractions are fractions of the form
\[
  x = a_0 + \cfrac{1}{a_1 + \cfrac{1}{a_2 + \cfrac{1}{a_3 + \cfrac{1}{a_4 + \ddots}}}}.
\]
A continued fraction is finite, when its sequence $a_n$ is finite.
In this case, the continued fraction represents a rational number.

\section{Periodicity and Quadratic Irrationals}

\begin{definition}
  A continued fraction $[a₀; a₁, …]$ is called \emph{eventually periodic}
  if there exists an index $k₀ ≥ 0$ and a period $ℓ ≥ 1$ such that $aₖ = a_{k+ℓ}$ for every $k ≥ k₀$.
  A continued fraction is called \emph{purely periodic} if $k₀ = 0$.
\end{definition}

\begin{definition}
  A number $x ∈ ℝ$ is said to be a \emph{quadratic irrational} if $x$ is the root of some
  polynomial $p(x)$ over the rational numbers with degree $2$.
\end{definition}

\begin{example}
  Let $x = \sqrt{7}$.
  The algorithm proceeds as follows:
  \[
    \begin{array}{rclcrcl}
      \sqrt{7}      & = & 2 · 1               + (\sqrt{7} - 2)    & ≈ & 2.6458 & = & 2 · 1 + 0.6458      \\
      1             & = & 1 · (\sqrt{7} - 2)  + (3 - \sqrt{7})    & ≈ & 1      & = & 1 · 0.6458 + 0.3542 \\
      \sqrt{7} - 2  & = & 1 · (3 - \sqrt{7})  + (2\sqrt{7} - 5)   & ≈ & 0.6458 & = & 1 · 0.3542 + 0.2915 \\
      3 - \sqrt{7}  & = & 1 · (2\sqrt{7} - 5) + (8 - 3\sqrt{7})   & ≈ & 0.3542 & = & 1 · 0.2915 + 0.0627 \\
      2\sqrt{7} - 5 & = & 4 · (8 - 3\sqrt{7}) + (14\sqrt{7} - 37) & ≈ & 0.2915 & = & 4 · 0.0627 + 0.0407 \\
      & \vdots & & \vdots & & \vdots &
    \end{array}
  \]
\end{example}

The interesting connection between the quadratic irrationals and continued
fraction is that a continued fraction of a number $x ∈ ℝ$ is eventually
periodic if and only if $x$ is a quadratic irrational.
We begin with the first direction:

\begin{lemma}
  Let $x ∈ ℝ$, then
  \[
    [a₀; a₁, …, a_n, x] = [a₀; a₁, …, a_n + 1/x]
  \]
\end{lemma}

\begin{proof}
  \label{lem:nesting}
  If $n = 0$, then
  \[
    [a₀; x] = a₀ + \frac{1}{[x]} = a₀ + \frac{1}{x} = [a₀ + 1/x].
  \]
  Suppose the lemma is true for some $n ≥ 0$, then
  \begin{align*}
    [a₀; a₁, …, aₙ, x]
    & = a₀ + \frac{1}{[a₁; a₂, …, aₙ, x]} \\
    & = a₀ + \frac{1}{[a₁; a₂, …, aₙ + 1/x]} \\
    & = [a₀; a₁, …, aₙ, x]. \qedhere
  \end{align*}
\end{proof}

We define the following sequences over the sequence $\{a_n\}$:
\begin{align*}
  pₙ & = p_{n-1} a_n + p_{n - 2}, & p_{-1} & = 1, & p_{-2} & = 0, \\
  qₙ & = q_{n-1} a_n + q_{n - 2}, & q_{-1} & = 0, & q_{-2} & = 1.
\end{align*}

\begin{lemma}
  \label{lem:wallis}
  Let $x ∈ ℝ$, then
  \[
    [a₀; a₁, …, a_{n-1}, x] = \frac{p_{n-1} x + p_{n-2}}{q_{n-1} x + q_{n-2}}.
  \]
\end{lemma}

\begin{proof}
  If $n = 0$, then
  \[
    [x] = x = \frac{1x + 0}{0x + 1} = \frac{p_{-1} x + p_{-2}}{q_{-1} x + q_{-2}}.
  \]
  Suppose, the lemma is true for $n ≥ 0$.
  By Lemma~\ref{lem:nesting}, we have
  \begin{align*}
    [a₀; a₁, …, aₙ, x]
    & = [a₀; a₁, …, aₙ + 1/x].
  \end{align*}
  From the induction hypothesis, it follows that
  \begin{align*}
    [a₀; a₁, …, aₙ + 1/x]
    & = \frac{p_{n - 1} (aₙ + 1/x) + p_{n-2}}{q_{n-1} (aₙ + 1/x) + q_{n-2}} \\
    & = \frac{x (p_{n-1} aₙ + p_{n-2}) + p_{n-1}}{x (q_{n-1} aₙ + q_{n-2}) + q_{n-1}} \\
    & = \frac{x pₙ + p_{n-1}}{x pₙ + p_{n-1}}. \qedhere
  \end{align*}
\end{proof}

\begin{theorem}
  If the continued fraction representation of a number $x ∈ ℝ$ is eventually periodic,
  then $x$ is a quadratic irrational.
\end{theorem}

\begin{proof}
  Let $x$ be a continued fraction $[a₀; a₁, …]$ with a period of length $ℓ ≥ 1$
  starting at an index $k ≥ 0$,
  i.e. $x_k = [a_k; a_{k+1}, …] = [a_{k+ℓ}; a_{k+ℓ+1}, …] = x_{k+ℓ}$.
  By Lemma~\ref{lem:wallis}, we have
  \[
    x
    = \frac{p_k x_k + p_{k-1}}{q_k x_k + q_{k-1}}
    = \frac{p_{k+ℓ} x_{k+ℓ} + p_{k+ℓ-1}}{q_{k+ℓ} x_{k+ℓ} + q_{k+ℓ-1}}
    = \frac{p_{k+ℓ} x_k + p_{k+ℓ-1}}{q_{k+ℓ} x_k + q_{k+ℓ-1}}
  \]
  Multiplying by the denominators on both sides results in the quadratic equation
  \[
    (q_{k+ℓ} x_k + q_{k+ℓ-1})(p_k x_k + p_{k-1}) - (q_k x_k + q_{k-1}) (p_{k+ℓ} x_k + p_{k+ℓ-1}) = 0.
  \]
  Hence, $x_k$ is a quadratic irrational and $x$ must be one too since $x ∈ ℚ(x_k)$.
\end{proof}

The converse was originally shown by Lagrange \cite{Lagrange1770}.
Here I present a proof based on \cite{Northshield11},
which is much shorter and easier to understand.
Let $x$ be a quadratic irrational with polynomial $p$.
Furthermore, let $x = [a₀; a₁, \dots]$ be its continued fraction and $x_k = [a_k; a_{k+1}, \dots]$.
For every $x_i$ we construct a polynomial $p_i$ with $x_i$ as its root.
We begin by expanding $x$ once as $a₀ + x₁^{-1}$.
By definition, $a₀ + x₁^{-1}$ is the root of $p$.
However, this gives us a new polynomial in $x₁$ (if we multiply by $x₁^2$):
\begin{align*}
  p₁(x₁) = x₁^2 p(a_0 + x_1^{-1}) = A x₁^2 (a₀ + x₁^{-1})^2 + B x₁^2 (a₀ + x₁^{-1}) + C x₁^2 = 0. \\
\end{align*}
Reordering the polynomial into the standard quadratic form results in
\begin{align*}
  p₁(x₁) = \underbrace{(A a₀^2 + B a₀ + C)}_{A₁} + \underbrace{(2A a₀ + B)}_{B₁} + \underbrace{A}_{C₁}.
\end{align*}

We can continue this process for every $i$.
In particular, the coefficients of each polynomial $p_{i+1}$ is calculated
using the following recurrence relation:
\begin{align*}
  A_{i+1} = A_i a_i^2 + B_i a_i + C_i, \qquad
  B_{i+1} = 2 A_i a_i + B_i, \qquad
  C_{i+1} = A_i.
\end{align*}

The goal is to show that these coefficients are bounded in some way.
It follows that there must be some triple which repeats and therefore the
representation must be periodic.
The first step towards this proof is bounding the discriminant of each
polynomial $pᵢ$.

\begin{lemma}
  \label{lem:same-disc}
  The polynomials $p₁, p₂, …$ have the same discriminant.
\end{lemma}

\begin{proof}
  Let $Δ$ be the discriminant of $p₁$.
  By induction, suppose $p_i$ has discriminant $Δ$.
  Then the polynomial $p_{i+1}$ has discriminant
  \begin{align*}
    \phantom{= {}} B_{i+1}^2 - 4 A_{i+1} C_{i+1}
    & = (2 A_i a_i + B_i)^2 - 4 (A_i a_i^2 + B_i a_i + C_i) A_i \\
    & = 4 A_i^2 a_i^2 + 4 A_i B_i a_i + B_i^2 - 4 A_i^2 a_i^2 - 4 B_i A_i a_i - 4 C_i A_i \\
    & = B_i^2 - 4 A_i C_i \\
    & = Δ. \qedhere
  \end{align*}
\end{proof}

In order for the continued fraction to be periodic, the coefficients $A_i, B_i, C_i$ must be bounded.
The only way this could not be the case is when $A_i C_i > 0$,
since then there would be infinitely many values for such that $B_i^2 - 4 A_i C_i$.
However, this is not the case.

\begin{lemma}
  \label{lem:infinite-neg}
  There exists infinitely many polynomials $p_i$ with $A_i C_i < 0$.
\end{lemma}

\begin{proof}
  Suppose $A_i C_i > 0$ for every $i$ after some point.
  We assume WLOG that $A_i > 0$ since the negation of $p_i$ has the same roots.
  Except in the first iteration, every $x_i$ is positive and therefore $B_i$ must be negative.
  But $B_i = 2 A_{i-1} a_{i-1} + B_{i-1}$ would be strictly increasing since
  $A_i$ and $a_i$ are both assumed to be positive.
  This is a contradiction and therefore there must be infinitely many $p_i$ with $A_i C_i < 0$.
\end{proof}

\begin{theorem}[Lagrange's Theorem]
  If $x$ is a quadratic irrational, then its continued fraction is eventually periodic.
\end{theorem}

\begin{proof}
  By Lemma~\ref{lem:same-disc}, every polynomial~$p_i$ has the same
  discriminant~$Δ = B_i^2 - 4 A_i C_i > 0$ independent of the index~$i$
  and by Lemma~\ref{lem:infinite-neg} there are infinitely many polynomials
  such that $B_i^2$ and $-4 A_i C_i$ are both positive.
  But there can only be finitely many positive integers which sum up to another positive integer.
  Hence, there must be some triple $(A_i, B_i, C_i)$ which repeats, i.e. there
  exists an integer $i ≠ j$ with $(A_i, B_i, C_i) = (A_j, B_j, C_j)$.
  These polynomials must have the same root $x_i = x_j$ and by construction of
  the continued fraction, we have $a_{i+ℓ} = a_{j+ℓ}$ for every $ℓ ≥ 0$.
\end{proof}

\chapter{The Generalized Euclidean Algorithm}

In this chapter, we look at the generalized version of the Euclidean algorithm \cite{Klein24}.
While the original algorithm works on numbers,
the generalized version works with vectors.
More specifically, the generalized version works on lattices.
For this chapter, we proceed analogously to the Euclidean algorithm.
First, we look at how the generalized algorithm works and then use it to find a
higher-dimensional analogue to the Fibonacci numbers and the golden ratio.
Using the golden ratio, we can naturally extend this generalized algorithm to
the real numbers, just like the original single-dimensional algorithm.

\section{Basics of Lattice Theory}

\begin{figure}[b]
  \centering
  \includestandalone{figures/lattice}
  \caption{A two-dimensional lattice with vectors $B_1 = (2, 1)$ and $B_2 = (1, 3)$.}
\end{figure}

\begin{itemize}
  \item Vector space as the linear combination over a basis
  \item Lattices as an integral combination over a basis
\end{itemize}

\begin{definition}
  Given a basis $B ∈ ℤ^{d × n}$, the \emph{lattice} over the basis $B$ is defined as
  \[
    \mathcal{L}(B) = \left\{\, B₁z₁ + \dots + B_n z_n \mid z_1, \dots, z_n ∈ ℤ^d \,\right\}.
  \]
  The \emph{rank} of $\mathcal{L}(B)$ is $n$ and its \emph{dimension} is $d$.
  If $n = d$, then $\mathcal{L}(B)$ is a \emph{full rank} lattice.
\end{definition}

\begin{problem}[Lattice Basis Reduction]~
  \begin{itemize}
    \item \textbf{Input}: A matrix $A ∈ ℤ^{d × n}$ with $\text{rank}(A) = d$.
    \item \textbf{Output}: A matrix $B ∈ ℤ^{d × d}$ with $\mathcal{L}(B) = \mathcal{L}(A)$.
  \end{itemize}
\end{problem}

In this thesis, I only consider the case for one additional vector, i.e. $n = d + 1$.

% TODO: Example for an over-defined basis and what the reduced basis is.
\begin{example}
  Consider $A = \begin{pmatrix}
    2 & 1 & 3 \\
    1 & 3 & 4 \\
  \end{pmatrix}$.
  The matrix $B = \begin{pmatrix}
    2 & 1 \\
    1 & 3 \\
  \end{pmatrix}$
  spans the same lattice,
  since $A_3 = A_1 + A_2$.
  Therefore, $B$ would is the reduced basis of $\mathcal L(A)$.
\end{example}

% TODO: Another example which shows that you can't just take a submatrix of the
% original matrix.

\begin{definition}
  The \emph{fundamental parallelepiped} of a lattice $\mathcal{L}(B)$ with $B ∈ ℤ^{d × n}$ is defined as
  \[
    Π(B) = \left\{\, B₁ x₁ + \dots + B_n x_n \mid x_1, \dots, x_n ∈ [0, 1) \,\right\}
  \]
\end{definition}

A useful fact about the fundamental parallelepiped of a lattice $\mathcal L(B)$ is that
if $B$ is a square integer matrix,
then the volume of the parallelepiped $Π(B)$ and
the number of integer points $ℤ^n$ contained in $Π(B)$ is determined by $\mathrm{det}(B)$,
i.e.
\[
  \mathrm{vol}(Π(B)) = |Π(B) ∩ ℤ^n| = |\det(B)|.
\]

\section{Description of the Algorithm}

\begin{Pseudocode}[float=tb,caption={The Generalized Euclidean Algorithm \cite{Klein24}.}]
solve $Bx = c$
while $x$ is not integral do
  find $x_ℓ$ which is not integral
  $c ← B_ℓ$
  $B_ℓ ← B\{x\}$
  solve $Bx = c$
end
\end{Pseudocode}

In the previous example,
we saw that we could represent the last column vector as an integral
combination of the previous two,
which allows us to reduce the basis for the lattice to only those two column vectors.
However, in general it is not as easy as this.
Consider the matrix $A = ?$.
In this case, $A_3 = ? + ?$, which is clearly not an integral combination.
So $A' = ?$ does not span the same lattice as $A$.

Each point $a ∈ ℝ^d$ can be represented as a combination of a lattice point $z
∈ \mathcal{L}(B)$ and a point in the fundamental parallelepiped $r ∈ Π(B)$.
Specifically,
\[
  a = z + r = B\floor{x} + B\{x\}.
\]
This is essentially a division with remainder inside a lattice.
It allows us to define a modulo operation on the lattice:
\[
  a \pmod{Π(B)} := a - B\floor{B^{-1} x}.
\]

The algorithm requires solving a linear system in each iteration.
However, we do not have to do this in every iteration.
We only have to do this in the first iteration and in the following iterations
we simply update this solution from the old solution.
If $x = (x₁, …, x_d)$ is the solution in the previous iteration,
then $x' = (x₁', …, x_d')$ with
\begin{align*}
  x_i' =
  \begin{cases}
    \frac{1}{\{x_ℓ\}},  & \text{ if } i = ℓ, \\
    -\frac{\{x_i\}}{\{x_ℓ\}} & \text{ otherwise,}
  \end{cases}
\end{align*}
is the solution in the next iteration.
This update rule follows from
\[
  B_ℓ \{x_ℓ\} + \sum_{i ≠ ℓ} B_i \{x_i\} = r
  \iff
  r - \sum_{i ≠ ℓ} B_i \{x_i\} = B_ℓ \{x_ℓ\}
  \iff
  r \frac{1}{\{x_ℓ\}} - \sum_{i ≠ ℓ} B_i \frac{\{x_i\}}{\{x_ℓ\}} = B_ℓ.
\]

% TODO: Should we add a citation for Northshield and explain that continued
% fractions map positive to positive values which seems to be a fundamental
% requirement for the continued fractions to be periodic?

% I think a better wording would be, that the update rule makes the negation
% visible, which is not optimal. The update rule itself doesn't negate the
% variables, even without the update rule we would still have negated
% variables, since the update rule is just an improvement of the original
% algorithm.

Although the update rule speeds up the algorithm considerably, it is not
optimal for the analysis in the following sections.
The rule flips the sign of all elements inside the solution vector in each
iteration.
Instead, I propose a slight modification to the generalized algorithm which
maps each $xᵢ$ to another positive value.
After we replace $B_ℓ$ with $c$, we flip the signs of all vectors $B_i$ with $i ≠ ℓ$.
This leads to the modified update rule, where the values $x_i$ for $i ≠ ℓ$ are
no longer negated:
\begin{align*}
  x_i' =
  \begin{cases}
    \frac{1}{\{x_ℓ\}},  & \text{ if } i = ℓ, \\
    \frac{\{x_i\}}{\{x_ℓ\}} & \text{ otherwise.}
  \end{cases}
\end{align*}
By $\mathrm{pivot}_ℓ(x) = x'$, we denote this modified update rule.
The modified algorithm can be seen in Listing~\ref{lst:modified-generalized-euclidean}.
In the algorithm, first $B_ℓ$ is flipped and then the whole matrix $B$ is flipped,
This is the same as only flipping the vectors $B_i$ for $i ≠ ℓ$.

\begin{Pseudocode}[float=tb, caption={The Modified Algorithm.}, label={lst:modified-generalized-euclidean}]
solve $Bx = c$
while $x$ is not integral do
  find index $ℓ$ for which $x_ℓ$ is not integral
  $c ← B_ℓ$
  $B_ℓ ← -B\{x\}$
  $B ← -B$
  $x ← \mathrm{pivot}_ℓ(x)$
end
\end{Pseudocode}

\begin{lemma}
  The algorithm terminates in at most $\det(B)$ steps.
\end{lemma}

\begin{proof}

\end{proof}

\begin{lemma}
  In each iteration, $\mathcal L(B ∪ c) = \mathcal L(B' ∪ c')$.
\end{lemma}

\begin{proof}

\end{proof}

\begin{theorem}
  The generalized Euclidean algorithm solves the lattice basis reduction problem.
\end{theorem}

\begin{figure}[t]
  \centering
  \includestandalone{figures/pivot-choice}
  \caption{
    Different choices for the remainder of vector $c$. The original algorithm
    always uses $r$ as the remainder, but the modified update rule would also consider $r'$.}
\end{figure}

\section{Extension to Real Numbers}

% ==============================================================================
\section{Comparison to the Jacobi-Perron Algorithm}
% ==============================================================================

Many generalizations to the Euclidean algorithm have been considered.
One of them was proposed by Jacobi to answer Hermite's question.
In his version, he computes the GCD of three numbers by successively dividing
the smallest number from the larger numbers.
This algorithm was later extended by Oskar Perron to arbitrarily many numbers.
The algorithm works as follows:

Given a list of positive integers $a₀, a₁, …, aₙ$, take the smallest number $a_ℓ$
and compute the remainder $a_i'$ resulting from the division of $a_i$ with $a_ℓ$.
The value $a_ℓ$ is kept until the next iteration, i.e. $a_ℓ' = a_ℓ$.
Continue this process until all but one value remains.
\begin{align*}
  a₀' = a₀ \bmod a_ℓ, a₁ = a₁ \bmod a_ℓ, …, a_ℓ' = a_ℓ, …, aₙ' = aₙ \bmod a_ℓ; \\
\end{align*}

Perron modified this algorithm for the purposes of his analysis.
The integers $a₁, …, aₙ$ are kept in a list.
We remove the first element from the list, calculate the remainders for each
remaining element and append the element to the end of the list.
One iteration in this modified version produces the values:
\begin{align*}
  a₀' = a₁ \bmod a₀, a₁' = a₂ \bmod a₀, …, a_{n-1}' = a_n \bmod a₀, aₙ = a₀. \\
\end{align*}
This process is repeated until the first element is zero.

Of course, the termination condition is not sufficient.
When this algorithm terminates, the remaining elements might not all be zero.
Therefore, we remove the first element from the list and continue with the
remaining list.

By allowing real numbers as inputs, this algorithm proceeds infinitely but
always converges to zero.

The Jacobi-Perron algorithm is actually a subset of the generalized Euclidean algorithm.
The generalized Euclidean algorithm is periodic for all real numbers where the Jacobi-Perron algorithm is periodic.
This comes from the fact, that the Jacobi-Perron algorithm is really the
generalized Euclidean algorithm with the specific sequence of pivots $L = \overline{12…d}$.
So in the $i$th iteration, we are choosing index $(i \bmod d) + 1$ as our pivot.

This has the convenient property that if the all inputs which are periodic for
the Jacobi-Perron algorithm must also be periodic for the brute-force algorithm.

\begin{theorem}
  The brute-force algorithm is periodic on input $(1, r, r^2)$ if
  \[
    r = \sqrt[n]{D + d}, \text{ where } d | D.
  \]
\end{theorem}

\chapter{Hermite's Question}

\begin{problem}[Hermite's Question]
  Is there a representation of the real numbers as a sequence of natural
  numbers such that the sequence is eventually periodic if and only if the real
  number is a cubic irrational?
\end{problem}

More formally: Does there exists a function $f$ which maps any $α ∈ ℝ$ to an
infinite sequence $(a_n)_{n ∈ ℕ}$ of natural numbers such that $a_n$ is
eventually periodic if and only if $α$ is a cubic irrational?
The problem can very easily be generalized to higher degrees of algebraic numbers:
The representation is periodic if and only if $x$ is the root of a polynomial of degree $d + 1$.

\section{Comparison to the Jacobi-Perron Algorithm}

The algorithm in this form, where we always choose the same pivot,
is equivalent to the Jacobi-Perron algorithm, which was first proposed
by Jacobi \cite{Jacobi68} and later analyzed by Perron \cite{Perron07}.
In modern times, the algorithm was picked up again by Bernstein \cite{Bernstein06}.
Each of them has analyzed the algorithm in the context of Hermite's Question.

Charles Hermite initially posed a question to Jacobi, whether there exists a
representation of the real numbers as an infinite sequence of natural numbers
such that the sequence is eventually periodic if and only if the number is a
cubic irrational. The question remains unanswered with the latest attempt
\cite{Murru15} showing that there is an infinite sequence which is eventually
periodic for every cubic irrational.

\section{Finding a Periodic Representation through Brute-Force}

We use the generalized Euclidean algorithm to find a representation for $α ∈ ℝ$.
More specifically, we use the update rule from the algorithm to find the representation.
The update rule takes a vector and not a real number, so we use the vector $x_i = α^i$ for $i ≤ d$.

For our input $x ∈ ℝ^d$, we can build up a $d$-ary tree where $x$ is the root and any
two nodes of this tree $x', x''$ are connected if $\mathrm{pivot}(x, i) = x''$
for some $i$.
The brute-force algorithm performs a breadth-first search over this tree to
find a node which points to a higher node in the tree, i.e. the path forms a loop.
As our pivot sequence we use the path from the root to the this node.

\begin{example}
  The sequences for the prime numbers are:
  \begin{itemize}
    \item $2^{1/3}$: $0\overline{10}$.
    \item $3^{1/3}$: $01\overline{01}$.
    \item $5^{1/3}$: $0\overline{00111000}$.
    \item $7^{1/3}$: $0\overline{010100}$.
    \item $11^{1/3}$: $0\overline{1100}$.
    \item $13^{1/3}$: $00\overline{010000}$.
    \item $17^{1/3}$: $000\overline{11110000}$.
    \item Another choice for $5^{1/3}$ with shorter period: $00110\overline{101010}$.
  \end{itemize}
\end{example}

\begin{remark}
  The first number that has a leading $1$ as the pivot is $12$,
  which has the pivot sequence $1\overline{0111011101}$.
  However, there are other possible choices where it does not have a leading $0$.
\end{remark}

We can likely rule out alternating pivot sequences since this corresponds to
the Jacobi-Perron algorithm and for this algorithm it is conjectured
\cite{Karpenkov2024} that the algorithm is not periodic for $\sqrt[3]{4}$.
However, the brute-force algorithm is periodic for this input (see Table~\ref{table:cube-root-4}).

The choice of the initial input also matters.
We generally choose $x = (α, q(α))$, where $q$ is some polynomial of degree $2$.
But different choices of $q$ produce different sequences.
For example, $(\sqrt[3]{2}, \sqrt[3]{4})$ produces a different sequence of coefficients than $(\sqrt[3]{2}, \sqrt[3]{6})$.
Even though both inputs represent the same number $\sqrt[3]{2}$.
Choosing $q(α) = α^2 - α$ makes the sequence purely periodic for $\sqrt[3]{3}$ and $\sqrt[3]{4}$.

\begin{conjecture}
  The brute-force algorithm is periodic if and only if $α$ is a cubic irrational.
\end{conjecture}

\begin{lemma}
  \label{lem:consecutive-same}
  For every $k ∈ ℕ$, there exists a number $n ∈ ℕ$,
  such that $\sqrt[3]{n}$ and $\sqrt[3]{n + 1}$ have the same first $k$ terms in
  their pivot sequences.
\end{lemma}

\begin{corollary}
  Any strategy which only compares the fractional values does not produce the
  same pivot sequence as the brute-force algorithm.
\end{corollary}

\begin{corollary}
  The minimum strategy does not produce the same pivot sequence as the
  brute-force algorithm.
\end{corollary}

\iffalse
\begin{theorem}
  There exists no algorithm which given $x ∈ ℝ^d$ can determine the same
  strategy as the brute-force search using only $f(d)$ comparison, addition
  and subtraction operations, for some computable function $f \colon ℕ → ℕ$.
\end{theorem}

\begin{proof}
  Idea: A constant-number of additions and subtraction only compare the integer
  parts of each $x_i$ when the values are chosen close enough to an integer.
  Because the trees are the same for the first level, the output for both
  inputs must be the same.

  We choose $x₁ = f(d)$ and $x₂ = \sqrt{f(d) + 1}$ where $n^3 > f(d)$.
  In particular, $x₁$ is an integer and $x₂$ is between $x₁$ and the next integer.
  Suppose the algorithm compares $a₁ x₁$ with $a₂ x₂$ for $|a₁|, |a₂| ≤ f(d)$.
  \[
    a₂ x₂
    = a₂ \sqrt{(f(d))^2 + i + 1}
    ≤ a₂ \left( f(d) + \frac{1}{3 f(d)} \right)
    = a₂ · f(d) + \frac{a₂}{3 f(d)} \right)
  \]
  Furthermore, $a₂ x₂ ≥ a₂ · f(d)$.
  Therefore, the integer part of $a₂ x₂$ is equal to $a₂ x₁$,
  even if $a₂ = f(d)$.
  Hence, any algorithm can only compare the integer parts of $x₁$ and $x₂$.
\end{proof}
\fi

\begin{table}[t]
  \caption{Representation of $ψ = \sqrt[3]{4}$ using the brute-force search.}
  \label{table:cube-root-4}
  \centering
  \begin{tabular}{lllllll}
  \uzlhline
  \uzlemph{$\ell$} & \uzlemph{$x_1$} & \uzlemph{$x_2$} & \uzlemph{$x_1$} & \uzlemph{$x_2$} & \uzlemph{$a_1$} & \uzlemph{$a_2$} \\
  \hline
  $0$ & $\psi$ & $\psi^{2}$ & $1.5874$ & $2.51984$ & $0$ & $1$ \\
  \hline
  \hline
  $0$ & $\frac{1}{4} \psi^{2}$ & $\psi - 1$ & $0.62996$ & $0.5874$ & $1$ & $0$ \\
  $0$ & $\psi - 1$ & $\psi^{2} - \psi$ & $0.5874$ & $0.93244$ & $1$ & $1$ \\
  $1$ & $\frac{1}{3} \psi^{2} + \frac{1}{3} \psi - \frac{2}{3}$ & $\psi - 1$ & $0.70241$ & $0.5874$ & $1$ & $1$ \\
  $0$ & $\frac{1}{3} \psi - \frac{1}{3}$ & $\frac{1}{3} \psi^{2} + \frac{1}{3} \psi - \frac{2}{3}$ & $0.1958$ & $0.70241$ & $5$ & $3$ \\
  $1$ & $\psi^{2} + \psi - 4$ & $\psi - 1$ & $0.10724$ & $0.5874$ & $0$ & $1$ \\
  $1$ & $-\frac{2}{3} \psi^{2} + \frac{1}{3} \psi + \frac{4}{3}$ & $\frac{1}{3} \psi^{2} + \frac{1}{3} \psi - \frac{2}{3}$ & $0.18257$ & $0.70241$ & $0$ & $1$ \\
  $1$ & $\frac{1}{2} \psi^{2} - 1$ & $\frac{1}{4} \psi^{2} + \frac{1}{2} \psi - 1$ & $0.25992$ & $0.42366$ & $0$ & $2$ \\
  $0$ & $-\frac{1}{5} \psi^{2} + \frac{1}{5} \psi + \frac{4}{5}$ & $\frac{2}{5} \psi^{2} + \frac{3}{5} \psi - \frac{8}{5}$ & $0.61351$ & $0.36038$ & $1$ & $0$ \\
  \uzlhline
\end{tabular}

\end{table}

\begin{table}[t]
  \caption{Period Length of the first $28$ numbers.}
  \centering
  \begin{tabular}{ll}
  \uzlhline
  \uzlemph{$n^3$} & \uzlemph{Period Length of $n^3$} \\
  \hline
  2 & 2 \\
  3 & 2 \\
  4 & 8 \\
  5 & 8 \\
  6 & 8 \\
  7 & 6 \\
  9 & 2 \\
  10 & 4 \\
  11 & 4 \\
  12 & 10 \\
  13 & 6 \\
  14 & 4 \\
  15 & 6 \\
  16 & 14 \\
  17 & 8 \\
  18 & 6 \\
  19 & 6 \\
  20 & 8 \\
  21 & 8 \\
  22 & 6 \\
  23 & 20 \\
  24 & 8 \\
  25 & 20 \\
  26 & 4 \\
  28 & 2 \\
  \uzlhline
\end{tabular}

\end{table}

\section{Multi-Dimensional Continued Fractions}

\begin{definition}
  Given a sequence of $d$-dimensional vectors $\{rₙ\}_{n ≥ 0}$ and a sequence of
  indices $\{ℓₙ\}_{n ≥ 0}$, the \emph{multi-dimensional fraction} $[r₀; ℓ₁, r₁; …]$ is defined as
  \[
    [r₀] = r₀, \qquad [r₀; ℓ₁, r₁; …; ℓₙ, rₙ] = r₀ + \mathrm{pivot}_{ℓ₁}[r₁; ℓ₂, r₂; …; ℓₙ, rₙ].
  \]
\end{definition}

\begin{remark}~
  \begin{enumerate}
    \item
      The first item has no index.
      The index sequence $ℓₙ$ is always one shorter than the vector sequence $rₙ$.
    \item
      For the generalized Euclidean algorithm, the sequence $r_n$ consists solely of integer vectors.
      However, the MDCF can also be defined over rational or even real vectors.
      To differentiate the two, a sequence $rₙ$ will denote a sequence of real
      vectors and $aₙ$ will denote a sequence of integer vectors.
  \end{enumerate}
\end{remark}

The concepts from one-dimensional continued fractions naturally carry over to its
multi-dimensional counterpart.

\begin{definition}[Complete Quotient, Convergent, Periodicity]~
  \begin{itemize}
    \item The \emph{$k$-th complete quotient} is defined as $[rₖ; r_{k+1}, …]$.
    \item The \emph{$k$-th convergent} of $x$ is defined as the finite fraction $[r₀; ℓ₁, r₁; …; ℓ_k, r_k]$.
    \item The MDCF is \emph{eventually periodic} if there exists $k₀ ≥ 0$ and $ℓ ≥ 1$ such that $rₖ = r_{k+ℓ}$ for every $k ≥ k₀$.
      The MDCF is \emph{purely periodic} if $k₀ = 0$.
  \end{itemize}
\end{definition}

To show that any periodic MDCF contains elements of a field with degree $d + 1$
over the rationals, we proceed similarly to the one-dimensional continued
fraction.
We first show the multi-dimensional analogue of Lemma~\ref{lem:nesting}.

\begin{lemma}[Nesting]
  \label{lem:mdcf-nesting}
  Let $x ∈ ℝ^d$, then
  \[
    [a₀; ℓ₁, a₁; …; ℓₙ, aₙ; ℓ, x]
    = [a₀; ℓ₁, a₁; \cdots; ℓₙ, aₙ + \mathrm{pivot}_{ℓ}(x)]
  \]
\end{lemma}

\begin{proof}
  If $n = 0$, then by definition,
  \[
    [a₀; ℓ, x] = a₀ + \mathrm{pivot}_{ℓ}[x] = a₀ + \mathrm{pivot}_{ℓ}(x) = [a₀ + \mathrm{pivot}_ℓ(x)].
  \]
  Suppose the lemma holds for any $n ≥ 0$, then
  \begin{align*}
    [a₀; ℓ₁, a₁; …; ℓₙ, aₙ; ℓ, x]
    & = a₀ + \mathrm{pivot}_{ℓ₀}[a₁; …; ℓₙ, aₙ; ℓ, x] \\
    & = a₀ + \mathrm{pivot}_{ℓ₀}[a₁; …; ℓₙ, aₙ + \mathrm{pivot}_ℓ(x)] \\
    & = [a₀; ℓ₁, a₁; …; ℓₙ, aₙ + \mathrm{pivot}_ℓ(x)]. \qedhere
  \end{align*}
\end{proof}

In Lemma~\vref{lem:wallis}, we had to define two sequences $pₙ$ and $qₙ$ over
the sequence of the continued fraction $aₙ$.
Each sequence would only return a single scalar.
This time we define two matrix sequences $P^{(n)}, Q^{(n)}$ over the sequences $aₙ$ and $ℓₙ$ as
\begin{align*}
  P_{ℓₙ}^{(n)} & = P_{n-1} a_n + p_{n-1}, & P_i^{(n)} & = P_i^{(n)}, & P^{(-1)}   & = I_d, \\
  Q_{ℓₙ}^{(n)} & = Q_{n-1}^T a_n + q_{n-1}, & Q_i^{(n)} & = Q_i^{(n)}, & Q^{(-1)}_j & = 0,   \\
  p^{(n)}      & = P_{ℓₙ}^{(n-1)},            &           &              & p^{(-1)}_j & = 0,   \\
  p^{(n)}      & = P_{ℓₙ}^{(n-1)},            &           &              & q^{(-1)}   & = 1.
\end{align*}
where $i ≠ ℓ_n$.
What this sequence effectively does is reconstructing the lattice from an
initial solution vector $x ∈ ℝ^d$ and its coefficient vectors $a_n$.

\begin{lemma}[Wallis]
  \label{lem:mdcf-wallis}
  Let $x ∈ ℝ^d$, then
  \[
    [a₀; ℓ₁, a₁; …; ℓ_{n-1}, a_{n-1}; ℓ, x]
    = \left(P^{(n-1)} x + p^{(n-1)}\right) \oslash \left(Q^{(n-1)} x + q^{(n-1)} \right),
  \]
  where $u \oslash v$ denotes the element-wise division of two vectors.
\end{lemma}

\begin{proof}
  If $n = 0$, then
  \[
    [x] = x = (I_d x + 0) \oslash (0 x + 1).
  \]
  Suppose the lemma holds for $n ≥ 0$, then there exists matrices $P$ and $Q$
  as well as vectors $p$ and $q$ such that
  \begin{align*}
    y & = [a₀; ℓ₁, a₁; …; ℓ_{n-1}, a_{n-1}; ℓ, x]                              \\
      & = [a₀; ℓ₁, a₁; …; ℓ_{n-1}, a_{n-1} + \mathrm{pivot}_ℓ(x)]              \\
      & = (P (a + \mathrm{pivot}(x, ℓ)) + p) \oslash (Q (aₙ + \mathrm{pivot}(x, ℓ)) + q) \\
  \end{align*}
  We extend each fraction in the vector $y$ by $x_ℓ/x_ℓ$.
  The numerator has the following form
  \begin{align*}
    x_ℓ (P (a + \mathrm{pivot}(x, ℓ)) + p)
    & = x_ℓ (P a + P_ℓ \frac{1}{x_ℓ} + \sum_{i ≠ ℓ} P_i \frac{x_i}{x_ℓ} + p) \\
    & = \underbrace{(P a + p)}_{P_ℓ'} x_ℓ + \sum_{i ≠ ℓ} \underbrace{P_i}_{P_i'} x_i + \underbrace{P_ℓ}_{p'} \\
    & = P' x + p'.
  \end{align*}
  The proof for the denominator is analogous.
  Hence,
  \begin{align*}
    y & = (P(a + \mathrm{pivot}(x, ℓ)) + p) \oslash (Q(a + \mathrm{pivot}(x, ℓ)) + q) \\
      & = (P' x + p') \oslash (Q' x + q'). \qedhere
  \end{align*}
\end{proof}

\begin{proposition}
  Given $r ∈ ℝ$, let $x = (r¹, r², …, r^d)$.
  If the MDCF of $x$ is purely periodic, then $[ℚ(r) : ℚ] ≤ d + 1$.
\end{proposition}

\begin{proof}
  Suppose the algorithm is purely periodic on input $x$ with period $ℓ$.
  Let $y$ be the $ℓ$-th complete quotient of $x$, then $x = y$.
  By Lemma~\ref{lem:mdcf-wallis}, there are matrices $P, Q$ and vectors $p, q$ such that
  \[
    rⁱ = \frac{\sum_{j=1}^d P_{ij} rʲ + pᵢ}{\sum_{j=1}^d Q_{ij} rʲ + qᵢ}, \text{ for every } i ≤ d.
  \]
  Multiplying both sides with the denominator results in the polynomial equation
  \[
    \sum_{j=1}^d (Q_{ij} r^{i+j} - P_{ij} r^j) + r^i q_i - p_i = 0.
  \]
  For $i = 1$, the maximum degree of this polynomial is $d + 1$.
  Hence, $[ℚ(r) : ℚ] ≤ d + 1$.
\end{proof}

\begin{lemma}
  \todo[inline]{Some lemma which simplifies the system down to degree $d$.}
\end{lemma}

\begin{theorem}
  Given $r ∈ ℝ$, let $xᵢ = r^i$ for every $i ≤ d$.
  If the brute-force algorithm is eventually periodic for $x = (x₁, …, x_d)$,
  then $[ℚ(r) : ℚ] ≤ d + 1$.
\end{theorem}

\begin{proof}
  \[
    \frac{\sum_{j=1}^d P_{ij} f_j(r) + pᵢ}{\sum_{j=1}^d Q_{ij} f_j(r) + qᵢ}
    = \frac{\sum_{j=1}^d P_{ij} f_j(r) + pᵢ}{\sum_{j=1}^d Q_{ij} f_j(r) + qᵢ}
  \]
\end{proof}

\chapter{Experimental Analysis on Multi-Dimensional Continued Fractions}
\label{ch:implementation}

In the previous chapter, we have analyzed
In particular, we saw that most MDCFs converge,
and that any periodic MDCF consists of algebraic numbers.
The former solves the first part of Hermite's question, but the latter solves
only one direction of the second part.
The first part of this chapter will focus on the remaining direction:
whether every algebraic number admits a periodic MDCF.

We have already seen in Chapter~\ref{ch:fibonacci},
that the simplest periodic MDCFs can be considered a generalization of the
golden ratio.
The first part of this chapter extends this result by presenting further
examples of periodic MDCF for algebraic numbers.
These include a wide range of cube roots,
all of which appear to have periodic representations under one particular strategy.
While this does not amount to a proof, the evidence strongly supports
the possibility of a positive answer to the second part of Hermite’s question.

The second part of this chapter focuses on the approximation rate of MDCFs.
For ordinary continued fractions, the convergents are known to give
exceptionally good approximations to irrational numbers.
Here, we test whether MDCFs offer similar approximation behavior in higher dimensions.

\iffalse
% ==============================================================================
\section{Implementation details}
% ==============================================================================

% ==============================================================================
\begin{Python}[
    float=tbp,
    numbers=left,
    label={lst:bfs},
    caption={
      The implementation of the brute-force search for finding a periodic representation.
      The program iterates over all sequences with a maximum length of $N$
      until it finds a duplicate vector.
    }
  ]
def brute_force_search(x, N):
  d = len(x)
  indices = list(range(d))
  for n in range(N):
    for L in product(indices, repeat=n):
      y = x
      seen = {y: 0}
      for i in range(n):
        y = pivot(y, L[i])
        if y in seen:
          j = seen[y]
          start = L[:j]
          period = L[j:i+1]
          return start, period
        seen[y] = i + 1
\end{Python}
% ==============================================================================

% Brute-force search
The goal is to find a periodic MDCF for some algebraic number $α$.
For an MDCF, we require a sequence of indices $ℓ₁, ℓ₂, …$ which determine the
element to pivot with.
To find this sequence, different types of searches were constructed.

The program is an implementation of the generalized Euclidean algorithm
in Python and SageMath.
More specifically, only the pivot operation from the algorithm was implemented,
since this the actual part that is relevant for the construction of an MDCF.

The actual code for the search I implemented is shown in Listing~\ref{lst:bfs}.
The input to the algorithm is a vector $x$ containing algebraic numbers of
degree $≤ d+1$ and a maximum search depth $N$.
If possible, it outputs two index sequences of the start and period for a periodic MDCF,
which represents the original input vector.
To find this sequence, the algorithm uses a simple brute-force search over all
possible sequences $\{1,\dots,d\}^*$ with a maximum length of $N$.
We simply try every sequence of possible pivot indices in a breadth-first manner.
So we begin with all sequences of length $1$ and see if any vector occurs twice.
If not, then we continue with all sequences of length $2$ and check again if
any previous vector has occurred twice.
We continue this process indefinitely until we hopefully find a duplicate vector.
To keep track of duplicates, the dictionary \verb|seen| maps vectors to the index,
where they first occurred.

% ==============================================================================
\begin{Python}[
    float=tbp,
    numbers=left,
    caption={
      The implementation of the nondeterministic search.
      The search begins with the empty sequence and then queries the strategy
      for the next valid sequences.
      At the same time, it checks whether any vector has occurred twice
      and stops once it has found a duplicate.
    },
    label={lst:nondet-search},
  ]
def nondeterministic_search(x, N, strat):
  d = len(x)
  sequences = [[]]
  for n in range(N):
    new_sequences = []
    for L in sequences:
      y = x
      seen = {y: 0}
      for l in L:
        y = pivot(y, l)
        seen[y] = i + 1
      for l in strat(x, L):
        z = pivot(y, l)
        if z in seen:
          j = seen[z]
          start = L[:j]
          period = L[j:i+1]
          return start, period
        new_sequences.append(L + [l])
    sequences = new_sequences
\end{Python}
% ==============================================================================

% Nondeterministic search
The main goal of the brute-force search is to find a periodic representation for a cubic root at all.
The problem is that the search is quite expensive.
For a given maximum search depth, the search will take $O(d^n)$ steps,
so even for the lowest dimension $d = 2$, the search is already expensive.
The hope is that the sequences share something in common such that we can find
a more optimized search strategy, ideally one which can be decided without
trying every possible combination.
Therefore, two more types of searches were studied.
The first type is a deterministic search,
which only looks at one possible path in the tree.
For example, the minimum strategy would only look at one possible path.
The other type is a non-deterministic search,
which looks at multiple paths simultaneously.
For this type of search, the main point of interest was the approximation rate of the convergents
and whether convergents are the best rational approximations of the original input vector $x$,
i.e. whether they fulfill
\[
  \left|x_i - \frac{p_i^{(n)}}{q^{(n)}}\right| < \frac{1}{q^{(n)} \sqrt[d]{q^{(n)}}}, \text{ for every } i ≤ d
\]
at each step.
The idea would be that this could be one possible optimization to the brute-force search.
Instead of trying every possible candidate, we only choose paths which lead to good approximations.
For $d = 1$, this is already known by Lemma~\ref{lem:cf-approx},
but for higher dimensions it is not known whether the convergents are good
approximations, yet.

The strategy is given the initial input $x ∈ ℝ^d$ and a sequence of indices $L
∈ \{1, …, d\}^*$ and must return all valid indices which can be appended to the
current sequence to form a new sequence.
For the brute-force search, the strategy always outputs the entire set; any index is allowed.
The minimum strategy only outputs one index and its the one where $x^{(n)}$ has
the minimum fractional value, if $n$ is the length of the list $L$.
\[
  \texttt{strat} \colon ℝ^d → \mathcal P(\{1, …, d\}), x ↦ \{ℓ_1, …, ℓ_k\}.
\]

% Deterministic search
For the deterministic search, a strategy $s$ is a function $R^d → \{1, …, d\}$
which takes in the complete quotient $x^{(n)} ∈ ℝ^d$ of the input vector $x ∈ R^d$
and the number of iterations $n$.
The strategy outputs only a single index $ℓ$,
which is used to find the next complete quotient by $x^{(n+1)} =
\mathrm{pivot}_ℓ(x^{(n)})$.
For example, the minimum strategy would be defined as
\[
  \texttt{strat}(x, n) = \underset{\substack{ℓ ∈ \{1, …, d\} \\ \{x_ℓ\} ≠ 0}}{\text{arg min}} \{x_ℓ\}.
\]
The additional index $n$ is used for example in the Jacobi-Perron algorithm,
which chooses always the next index in the sequence.
So,
\[
  \texttt{strat}(x, n) = (n \bmod d) + 1.
\]

For the non-deterministic search, only the approximation criterion was tested,
but with different rates, i.e. whether for some constant $c ≥ 1$,
\[
  \left|x_i - \frac{p_i^{(n)}}{q^{(n)}}\right| < \frac{c}{q^{(n)} \sqrt[d]{q^{(n)}}}, \text{ for every } i ≤ d.
\]

In summary,
there are three different types of searches.
The first is the brute-force search, which is used to find a periodic MDCF at all.
The second is the nondeterministic search, which is used to see how well the
convergents approximate the original input vector.
The third is the deterministic search, which is used to compare the different
strategies.
\fi

\iffalse
% ==============================================================================
\section{Periodic MDCFs for Cube Roots}
% ==============================================================================

% TODO: Measure the times for each MDCF and list them!
For the first search, the cube roots $\sqrt[3]{2}$ to $\sqrt[3]{100}$ were tested.
The MDCFs for these roots is listed in Table~\ref{tbl:cubics}.
For cubic irrationals there are $O(2^n)$ possible sequences,
so the search is already quite expensive.
The search for the first 30 cube roots had a maximum search depth of $24$ and
this already took over two hours to complete.
Almost half of the time was spent on the root for $\sqrt[3]{29}$
and it did not find a representation for this root.
Instead, I used different strategies to find an MDCF for this root.

Regarding the representation of the roots,
there is no perceivable patterns between the roots and their length.
However, this is to be expected since continued fractions also follow no simple
pattern between square roots and their length.
What the MDCFs share in common is that their periods always have an even length
and the period always contains both indices;
that is, there is no periodic sequence consisting solely of one repeated index
(e.g., only ones or only twos).

Apart from the even length, there is one specific set of roots which have a
predetermined period length
and they are the roots with the shortest period, i.e. $\sqrt[3]{2},
\sqrt[3]{3}, \sqrt[3]{9}$ and $\sqrt[3]{28}$.
Out of all observed roots, they have the shortest period with only two indices.
The reason comes from a theorem proven by Bernstein \cite{Bernstein71}.
In his analysis of the Jacobi-Perron algorithm,
he has shown that Jacobi-Perron algorithm is periodic for any root of the form
% TODO: Is the period length always 1?
% TODO: Check the correct conditions. Specifically is c < D correct?
\[
  \sqrt[3]{D^3 + c}, \qquad \text{ where } c < D \text{ and } c|D.
\]
and that the period length is exactly $2$ in the case of cubic irrationals.
So other roots with a short periodic sequence are $\sqrt[3]{65}, \sqrt[3]{66}$,
for example.

\begin{table}[tbp]
  \caption{Representation of $ψ = \sqrt[3]{4}$ using the brute-force search.}
  \label{table:cube-root-4}
  \centering
  \footnotesize
  \begin{tabular}{lllllll}
  \uzlhline
  \uzlemph{$\ell$} & \uzlemph{$x_1$} & \uzlemph{$x_2$} & \uzlemph{$x_1$} & \uzlemph{$x_2$} & \uzlemph{$a_1$} & \uzlemph{$a_2$} \\
  \hline
  $0$ & $\psi$ & $\psi^{2}$ & $1.5874$ & $2.51984$ & $0$ & $1$ \\
  \hline
  \hline
  $0$ & $\frac{1}{4} \psi^{2}$ & $\psi - 1$ & $0.62996$ & $0.5874$ & $1$ & $0$ \\
  $0$ & $\psi - 1$ & $\psi^{2} - \psi$ & $0.5874$ & $0.93244$ & $1$ & $1$ \\
  $1$ & $\frac{1}{3} \psi^{2} + \frac{1}{3} \psi - \frac{2}{3}$ & $\psi - 1$ & $0.70241$ & $0.5874$ & $1$ & $1$ \\
  $0$ & $\frac{1}{3} \psi - \frac{1}{3}$ & $\frac{1}{3} \psi^{2} + \frac{1}{3} \psi - \frac{2}{3}$ & $0.1958$ & $0.70241$ & $5$ & $3$ \\
  $1$ & $\psi^{2} + \psi - 4$ & $\psi - 1$ & $0.10724$ & $0.5874$ & $0$ & $1$ \\
  $1$ & $-\frac{2}{3} \psi^{2} + \frac{1}{3} \psi + \frac{4}{3}$ & $\frac{1}{3} \psi^{2} + \frac{1}{3} \psi - \frac{2}{3}$ & $0.18257$ & $0.70241$ & $0$ & $1$ \\
  $1$ & $\frac{1}{2} \psi^{2} - 1$ & $\frac{1}{4} \psi^{2} + \frac{1}{2} \psi - 1$ & $0.25992$ & $0.42366$ & $0$ & $2$ \\
  $0$ & $-\frac{1}{5} \psi^{2} + \frac{1}{5} \psi + \frac{4}{5}$ & $\frac{2}{5} \psi^{2} + \frac{3}{5} \psi - \frac{8}{5}$ & $0.61351$ & $0.36038$ & $1$ & $0$ \\
  \uzlhline
\end{tabular}

\end{table}

Finding periodic MDCFs for higher dimensions is even more difficult than two dimensions.
There are now $O(d^N)$ possible sequences with a maximum depth of $N$.
So each search is exponentially more expensive than the cubic case.
Some of the easier ones to find were again the roots identified by Bernstein.
\fi

% ==============================================================================
\section{Comparison of More Efficient Strategies}
% ==============================================================================

The first part of the analysis is meant to find periodic MDCFs for cubic
irrationals.
For the construction, we use the $\mathrm{pivot}$ operation from generalized
Euclidean algorithm and some sequence of indices, which we pivot with.
The choice of our indices essentially defines a tree
with the initial input vector $x$ as the root
and subsequent nodes as complete quotients of one MDCF.
The edges in this tree are the indices $ℓ$, which we use to get from one
complete quotient $x^{(n)}$ to another $x^{(n+1)}$ by computing $x^{(n+1)} =
\mathrm{pivot}_ℓ(x^{(n)})$.
We have found a periodic representation if there is a path from the root to a
node such that the last node occurs twice on the path.

Initially I tried searching for the MDCF of cube roots
using a brute-force search.
For any irrational vector $x$,
I tried every possible sequence of indices,
which could be used for the construction of an MDCF.
Using this search, I was only able to find the cube roots from $\sqrt[3]{2}$
and $\sqrt[3]{32}$ with the exception of $\sqrt[3]{29}$.
The results are listed in Table~\ref{tbl:cubics}.
The MDCFs I found using this search are the shortest possible representation
for the number, when we measure the combined length of the preperiod and
period.
Since there are a total of $O(2^N)$ sequences,
the search took quite a lot of time.
Therefore, I proceeded to look into better strategies,
which could more easily find MDCFs for cube roots.

\begin{table}[tbp]
  \caption{
    The shortest periodic index sequences for cube roots found using the
    brute-force search algorithm. The maximum search depth was set to $20$ and
    only the sequence for $29$ was not found. The roots for $8$ and $27$ are
    omitted since they are perfect cubes.}
  \label{tbl:cubics}
  \centering
  \begin{minipage}{0.48\textwidth}
\footnotesize
\begin{tabular}{ll}
\uzlhline
\uzlemph{$x$} & \uzlemph{MDCF} \\ \hline
$\sqrt[3]{2}$ & $\left[
\begin{matrix} 1 \\ 1 \\ \end{matrix}\,\,
\overline{
\begin{matrix} 0 \\ 1 \\ \end{matrix}\,\,
\begin{matrix} 2 \\ 1 \\ \end{matrix}\,\,
}\right]$ \\
$\sqrt[3]{3}$ & $\left[
\begin{matrix} 1 \\ 2 \\ \end{matrix}\,\,
\begin{matrix} 2 \\ 0 \\ \end{matrix}\,\,
\overline{
\begin{matrix} 1 \\ 5 \\ \end{matrix}\,\,
\begin{matrix} 2 \\ 1 \\ \end{matrix}\,\,
}\right]$ \\
$\sqrt[3]{4}$ & $\left[
\begin{matrix} 1 \\ 2 \\ \end{matrix}\,\,
\overline{
\begin{matrix} 1 \\ 0 \\ \end{matrix}\,\,
\begin{matrix} 1 \\ 1 \\ \end{matrix}\,\,
\begin{matrix} 1 \\ 3 \\ \end{matrix}\,\,
\begin{matrix} 1 \\ 1 \\ \end{matrix}\,\,
\begin{matrix} 1 \\ 0 \\ \end{matrix}\,\,
\begin{matrix} 1 \\ 1 \\ \end{matrix}\,\,
\begin{matrix} 0 \\ 1 \\ \end{matrix}\,\,
\begin{matrix} 3 \\ 1 \\ \end{matrix}\,\,
}\right]$ \\
$\sqrt[3]{5}$ & $\left[
\begin{matrix} 1 \\ 2 \\ \end{matrix}\,\,
\overline{
\begin{matrix} 1 \\ 1 \\ \end{matrix}\,\,
\begin{matrix} 2 \\ 0 \\ \end{matrix}\,\,
\begin{matrix} 2 \\ 1 \\ \end{matrix}\,\,
\begin{matrix} 0 \\ 1 \\ \end{matrix}\,\,
\begin{matrix} 0 \\ 1 \\ \end{matrix}\,\,
\begin{matrix} 0 \\ 1 \\ \end{matrix}\,\,
\begin{matrix} 1 \\ 0 \\ \end{matrix}\,\,
\begin{matrix} 3 \\ 0 \\ \end{matrix}\,\,
}\right]$ \\
$\sqrt[3]{6}$ & $\left[
\begin{matrix} 1 \\ 3 \\ \end{matrix}\,\,
\overline{
\begin{matrix} 1 \\ 0 \\ \end{matrix}\,\,
\begin{matrix} 4 \\ 1 \\ \end{matrix}\,\,
\begin{matrix} 2 \\ 1 \\ \end{matrix}\,\,
\begin{matrix} 0 \\ 2 \\ \end{matrix}\,\,
\begin{matrix} 0 \\ 1 \\ \end{matrix}\,\,
\begin{matrix} 0 \\ 1 \\ \end{matrix}\,\,
\begin{matrix} 1 \\ 0 \\ \end{matrix}\,\,
\begin{matrix} 3 \\ 1 \\ \end{matrix}\,\,
}\right]$ \\
$\sqrt[3]{7}$ & $\left[
\begin{matrix} 1 \\ 3 \\ \end{matrix}\,\,
\overline{
\begin{matrix} 1 \\ 0 \\ \end{matrix}\,\,
\begin{matrix} 10 \\ 7 \\ \end{matrix}\,\,
\begin{matrix} 0 \\ 1 \\ \end{matrix}\,\,
\begin{matrix} 1 \\ 0 \\ \end{matrix}\,\,
\begin{matrix} 0 \\ 1 \\ \end{matrix}\,\,
\begin{matrix} 4 \\ 0 \\ \end{matrix}\,\,
}\right]$ \\
$\sqrt[3]{9}$ & $\left[
\begin{matrix} 2 \\ 4 \\ \end{matrix}\,\,
\begin{matrix} 12 \\ 4 \\ \end{matrix}\,\,
\overline{
\begin{matrix} 6 \\ 12 \\ \end{matrix}\,\,
\begin{matrix} 12 \\ 6 \\ \end{matrix}\,\,
}\right]$ \\
$\sqrt[3]{10}$ & $\left[
\begin{matrix} 2 \\ 4 \\ \end{matrix}\,\,
\overline{
\begin{matrix} 6 \\ 4 \\ \end{matrix}\,\,
\begin{matrix} 3 \\ 6 \\ \end{matrix}\,\,
\begin{matrix} 0 \\ 2 \\ \end{matrix}\,\,
\begin{matrix} 6 \\ 0 \\ \end{matrix}\,\,
}\right]$ \\
$\sqrt[3]{11}$ & $\left[
\begin{matrix} 2 \\ 4 \\ \end{matrix}\,\,
\overline{
\begin{matrix} 4 \\ 4 \\ \end{matrix}\,\,
\begin{matrix} 2 \\ 4 \\ \end{matrix}\,\,
\begin{matrix} 0 \\ 2 \\ \end{matrix}\,\,
\begin{matrix} 6 \\ 0 \\ \end{matrix}\,\,
}\right]$ \\
$\sqrt[3]{12}$ & $\left[
\begin{matrix} 2 \\ 5 \\ \end{matrix}\,\,
\overline{
\begin{matrix} 1 \\ 4 \\ \end{matrix}\,\,
\begin{matrix} 5 \\ 0 \\ \end{matrix}\,\,
\begin{matrix} 0 \\ 1 \\ \end{matrix}\,\,
\begin{matrix} 0 \\ 2 \\ \end{matrix}\,\,
\begin{matrix} 0 \\ 2 \\ \end{matrix}\,\,
\begin{matrix} 3 \\ 0 \\ \end{matrix}\,\,
\begin{matrix} 0 \\ 1 \\ \end{matrix}\,\,
\begin{matrix} 2 \\ 2 \\ \end{matrix}\,\,
\begin{matrix} 0 \\ 2 \\ \end{matrix}\,\,
\begin{matrix} 6 \\ 1 \\ \end{matrix}\,\,
}\right]$ \\
$\sqrt[3]{13}$ & $\left[
\begin{matrix} 2 \\ 5 \\ \end{matrix}\,\,
\begin{matrix} 2 \\ 1 \\ \end{matrix}\,\,
\overline{
\begin{matrix} 1 \\ 0 \\ \end{matrix}\,\,
\begin{matrix} 5 \\ 3 \\ \end{matrix}\,\,
\begin{matrix} 1 \\ 3 \\ \end{matrix}\,\,
\begin{matrix} 1 \\ 0 \\ \end{matrix}\,\,
\begin{matrix} 3 \\ 2 \\ \end{matrix}\,\,
\begin{matrix} 2 \\ 0 \\ \end{matrix}\,\,
}\right]$ \\
$\sqrt[3]{14}$ & $\left[
\begin{matrix} 2 \\ 5 \\ \end{matrix}\,\,
\begin{matrix} 2 \\ 1 \\ \end{matrix}\,\,
\begin{matrix} 0 \\ 1 \\ \end{matrix}\,\,
\begin{matrix} 2 \\ 0 \\ \end{matrix}\,\,
\overline{
\begin{matrix} 3 \\ 15 \\ \end{matrix}\,\,
\begin{matrix} 2 \\ 1 \\ \end{matrix}\,\,
\begin{matrix} 0 \\ 1 \\ \end{matrix}\,\,
\begin{matrix} 1 \\ 1 \\ \end{matrix}\,\,
}\right]$ \\
$\sqrt[3]{15}$ & $\left[
\begin{matrix} 2 \\ 6 \\ \end{matrix}\,\,
\begin{matrix} 2 \\ 0 \\ \end{matrix}\,\,
\begin{matrix} 6 \\ 1 \\ \end{matrix}\,\,
\overline{
\begin{matrix} 4 \\ 4 \\ \end{matrix}\,\,
\begin{matrix} 6 \\ 4 \\ \end{matrix}\,\,
\begin{matrix} 2 \\ 0 \\ \end{matrix}\,\,
\begin{matrix} 0 \\ 18 \\ \end{matrix}\,\,
\begin{matrix} 2 \\ 0 \\ \end{matrix}\,\,
\begin{matrix} 6 \\ 0 \\ \end{matrix}\,\,
}\right]$ \\
$\sqrt[3]{16}$ & $\left[
\begin{matrix} 2 \\ 6 \\ \end{matrix}\,\,
\overline{
\begin{matrix} 1 \\ 0 \\ \end{matrix}\,\,
\begin{matrix} 1 \\ 0 \\ \end{matrix}\,\,
\begin{matrix} 0 \\ 1 \\ \end{matrix}\,\,
\begin{matrix} 8 \\ 3 \\ \end{matrix}\,\,
\begin{matrix} 1 \\ 0 \\ \end{matrix}\,\,
\begin{matrix} 0 \\ 2 \\ \end{matrix}\,\,
\begin{matrix} 1 \\ 1 \\ \end{matrix}\,\,
\begin{matrix} 1 \\ 0 \\ \end{matrix}\,\,
\begin{matrix} 2 \\ 2 \\ \end{matrix}\,\,
\begin{matrix} 3 \\ 1 \\ \end{matrix}\,\,
\begin{matrix} 0 \\ 1 \\ \end{matrix}\,\,
\begin{matrix} 1 \\ 0 \\ \end{matrix}\,\,
\begin{matrix} 1 \\ 1 \\ \end{matrix}\,\,
\begin{matrix} 5 \\ 4 \\ \end{matrix}\,\,
}\right]$ \\
\uzlhline
\end{tabular}
\end{minipage}
\begin{minipage}{0.48\textwidth}
\footnotesize
\begin{tabular}{ll}
\uzlhline
\uzlemph{$x$} & \uzlemph{MDCF} \\ \hline
$\sqrt[3]{17}$ & $\left[
\begin{matrix} 2 \\ 6 \\ \end{matrix}\,\,
\begin{matrix} 1 \\ 1 \\ \end{matrix}\,\,
\begin{matrix} 1 \\ 0 \\ \end{matrix}\,\,
\overline{
\begin{matrix} 3 \\ 0 \\ \end{matrix}\,\,
\begin{matrix} 0 \\ 3 \\ \end{matrix}\,\,
\begin{matrix} 0 \\ 1 \\ \end{matrix}\,\,
\begin{matrix} 0 \\ 1 \\ \end{matrix}\,\,
\begin{matrix} 0 \\ 4 \\ \end{matrix}\,\,
\begin{matrix} 3 \\ 3 \\ \end{matrix}\,\,
\begin{matrix} 1 \\ 0 \\ \end{matrix}\,\,
\begin{matrix} 1 \\ 1 \\ \end{matrix}\,\,
}\right]$ \\
$\sqrt[3]{18}$ & $\left[
\begin{matrix} 2 \\ 6 \\ \end{matrix}\,\,
\begin{matrix} 1 \\ 1 \\ \end{matrix}\,\,
\begin{matrix} 1 \\ 0 \\ \end{matrix}\,\,
\begin{matrix} 1 \\ 1 \\ \end{matrix}\,\,
\begin{matrix} 1 \\ 0 \\ \end{matrix}\,\,
\begin{matrix} 1 \\ 0 \\ \end{matrix}\,\,
\overline{
\begin{matrix} 5 \\ 17 \\ \end{matrix}\,\,
\begin{matrix} 1 \\ 0 \\ \end{matrix}\,\,
\begin{matrix} 1 \\ 0 \\ \end{matrix}\,\,
\begin{matrix} 1 \\ 0 \\ \end{matrix}\,\,
\begin{matrix} 1 \\ 0 \\ \end{matrix}\,\,
\begin{matrix} 1 \\ 1 \\ \end{matrix}\,\,
}\right]$ \\
$\sqrt[3]{19}$ & $\left[
\begin{matrix} 2 \\ 7 \\ \end{matrix}\,\,
\begin{matrix} 1 \\ 0 \\ \end{matrix}\,\,
\begin{matrix} 2 \\ 0 \\ \end{matrix}\,\,
\overline{
\begin{matrix} 0 \\ 2 \\ \end{matrix}\,\,
\begin{matrix} 0 \\ 1 \\ \end{matrix}\,\,
\begin{matrix} 0 \\ 3 \\ \end{matrix}\,\,
\begin{matrix} 5 \\ 0 \\ \end{matrix}\,\,
\begin{matrix} 1 \\ 0 \\ \end{matrix}\,\,
\begin{matrix} 2 \\ 1 \\ \end{matrix}\,\,
}\right]$ \\
$\sqrt[3]{20}$ & $\left[
\begin{matrix} 2 \\ 7 \\ \end{matrix}\,\,
\overline{
\begin{matrix} 1 \\ 0 \\ \end{matrix}\,\,
\begin{matrix} 2 \\ 1 \\ \end{matrix}\,\,
\begin{matrix} 1 \\ 3 \\ \end{matrix}\,\,
\begin{matrix} 1 \\ 2 \\ \end{matrix}\,\,
\begin{matrix} 1 \\ 0 \\ \end{matrix}\,\,
\begin{matrix} 1 \\ 0 \\ \end{matrix}\,\,
\begin{matrix} 1 \\ 2 \\ \end{matrix}\,\,
\begin{matrix} 6 \\ 3 \\ \end{matrix}\,\,
}\right]$ \\
$\sqrt[3]{21}$ & $\left[
\begin{matrix} 2 \\ 7 \\ \end{matrix}\,\,
\begin{matrix} 1 \\ 0 \\ \end{matrix}\,\,
\begin{matrix} 3 \\ 2 \\ \end{matrix}\,\,
\overline{
\begin{matrix} 6 \\ 3 \\ \end{matrix}\,\,
\begin{matrix} 1 \\ 1 \\ \end{matrix}\,\,
\begin{matrix} 0 \\ 1 \\ \end{matrix}\,\,
\begin{matrix} 2 \\ 2 \\ \end{matrix}\,\,
\begin{matrix} 1 \\ 0 \\ \end{matrix}\,\,
\begin{matrix} 0 \\ 22 \\ \end{matrix}\,\,
\begin{matrix} 1 \\ 0 \\ \end{matrix}\,\,
\begin{matrix} 3 \\ 0 \\ \end{matrix}\,\,
}\right]$ \\
$\sqrt[3]{22}$ & $\left[
\begin{matrix} 2 \\ 7 \\ \end{matrix}\,\,
\begin{matrix} 1 \\ 1 \\ \end{matrix}\,\,
\begin{matrix} 4 \\ 0 \\ \end{matrix}\,\,
\begin{matrix} 0 \\ 4 \\ \end{matrix}\,\,
\begin{matrix} 4 \\ 0 \\ \end{matrix}\,\,
\begin{matrix} 1 \\ 0 \\ \end{matrix}\,\,
\overline{
\begin{matrix} 3 \\ 20 \\ \end{matrix}\,\,
\begin{matrix} 1 \\ 0 \\ \end{matrix}\,\,
\begin{matrix} 4 \\ 2 \\ \end{matrix}\,\,
\begin{matrix} 0 \\ 3 \\ \end{matrix}\,\,
\begin{matrix} 5 \\ 1 \\ \end{matrix}\,\,
\begin{matrix} 1 \\ 1 \\ \end{matrix}\,\,
}\right]$ \\
$\sqrt[3]{23}$ & $\left[
\begin{matrix} 2 \\ 8 \\ \end{matrix}\,\,
\overline{
\begin{matrix} 1 \\ 0 \\ \end{matrix}\,\,
\begin{matrix} 5 \\ 0 \\ \end{matrix}\,\,
\begin{matrix} 2 \\ 1 \\ \end{matrix}\,\,
\begin{matrix} 1 \\ 2 \\ \end{matrix}\,\,
\begin{matrix} 4 \\ 2 \\ \end{matrix}\,\,
\begin{matrix} 1 \\ 1 \\ \end{matrix}\,\,
\begin{matrix} 3 \\ 3 \\ \end{matrix}\,\,
\begin{matrix} 0 \\ 1 \\ \end{matrix}\,\,
\begin{matrix} 10 \\ 1 \\ \end{matrix}\,\,
\begin{matrix} 0 \\ 1 \\ \end{matrix}\,\,
\begin{matrix} 4 \\ 11 \\ \end{matrix}\,\,
\begin{matrix} 1 \\ 1 \\ \end{matrix}\,\,
\begin{matrix} 5 \\ 1 \\ \end{matrix}\,\,
\begin{matrix} 5 \\ 1 \\ \end{matrix}\,\,
\begin{matrix} 2 \\ 5 \\ \end{matrix}\,\,
\begin{matrix} 6 \\ 2 \\ \end{matrix}\,\,
\begin{matrix} 0 \\ 1 \\ \end{matrix}\,\,
\begin{matrix} 1 \\ 0 \\ \end{matrix}\,\,
\begin{matrix} 0 \\ 2 \\ \end{matrix}\,\,
\begin{matrix} 7 \\ 1 \\ \end{matrix}\,\,
}\right]$ \\
$\sqrt[3]{24}$ & $\left[
\begin{matrix} 2 \\ 8 \\ \end{matrix}\,\,
\overline{
\begin{matrix} 1 \\ 0 \\ \end{matrix}\,\,
\begin{matrix} 7 \\ 2 \\ \end{matrix}\,\,
\begin{matrix} 1 \\ 1 \\ \end{matrix}\,\,
\begin{matrix} 2 \\ 5 \\ \end{matrix}\,\,
\begin{matrix} 1 \\ 0 \\ \end{matrix}\,\,
\begin{matrix} 0 \\ 1 \\ \end{matrix}\,\,
\begin{matrix} 0 \\ 2 \\ \end{matrix}\,\,
\begin{matrix} 6 \\ 4 \\ \end{matrix}\,\,
}\right]$ \\
$\sqrt[3]{25}$ & $\left[
\begin{matrix} 2 \\ 8 \\ \end{matrix}\,\,
\overline{
\begin{matrix} 1 \\ 1 \\ \end{matrix}\,\,
\begin{matrix} 1 \\ 1 \\ \end{matrix}\,\,
\begin{matrix} 2 \\ 4 \\ \end{matrix}\,\,
\begin{matrix} 0 \\ 1 \\ \end{matrix}\,\,
\begin{matrix} 2 \\ 0 \\ \end{matrix}\,\,
\begin{matrix} 3 \\ 4 \\ \end{matrix}\,\,
\begin{matrix} 1 \\ 0 \\ \end{matrix}\,\,
\begin{matrix} 12 \\ 1 \\ \end{matrix}\,\,
\begin{matrix} 6 \\ 3 \\ \end{matrix}\,\,
\begin{matrix} 4 \\ 2 \\ \end{matrix}\,\,
\begin{matrix} 3 \\ 4 \\ \end{matrix}\,\,
\begin{matrix} 22 \\ 6 \\ \end{matrix}\,\,
\begin{matrix} 0 \\ 1 \\ \end{matrix}\,\,
\begin{matrix} 3 \\ 11 \\ \end{matrix}\,\,
\begin{matrix} 1 \\ 0 \\ \end{matrix}\,\,
\begin{matrix} 12 \\ 2 \\ \end{matrix}\,\,
\begin{matrix} 0 \\ 1 \\ \end{matrix}\,\,
\begin{matrix} 1 \\ 0 \\ \end{matrix}\,\,
\begin{matrix} 0 \\ 2 \\ \end{matrix}\,\,
\begin{matrix} 7 \\ 1 \\ \end{matrix}\,\,
}\right]$ \\
$\sqrt[3]{26}$ & $\left[
\begin{matrix} 2 \\ 8 \\ \end{matrix}\,\,
\begin{matrix} 1 \\ 0 \\ \end{matrix}\,\,
\begin{matrix} 25 \\ 20 \\ \end{matrix}\,\,
\overline{
\begin{matrix} 1 \\ 1 \\ \end{matrix}\,\,
\begin{matrix} 8 \\ 17 \\ \end{matrix}\,\,
\begin{matrix} 1 \\ 0 \\ \end{matrix}\,\,
\begin{matrix} 25 \\ 18 \\ \end{matrix}\,\,
}\right]$ \\
$\sqrt[3]{28}$ & $\left[
\begin{matrix} 3 \\ 9 \\ \end{matrix}\,\,
\begin{matrix} 27 \\ 6 \\ \end{matrix}\,\,
\overline{
\begin{matrix} 9 \\ 27 \\ \end{matrix}\,\,
\begin{matrix} 27 \\ 9 \\ \end{matrix}\,\,
}\right]$ \\
$\sqrt[3]{30}$ & $\left[
\begin{matrix} 3 \\ 9 \\ \end{matrix}\,\,
\overline{
\begin{matrix} 9 \\ 6 \\ \end{matrix}\,\,
\begin{matrix} 3 \\ 9 \\ \end{matrix}\,\,
\begin{matrix} 0 \\ 3 \\ \end{matrix}\,\,
\begin{matrix} 9 \\ 0 \\ \end{matrix}\,\,
}\right]$ \\
$\sqrt[3]{31}$ & $\left[
\begin{matrix} 3 \\ 9 \\ \end{matrix}\,\,
\begin{matrix} 0 \\ 1 \\ \end{matrix}\,\,
\begin{matrix} 1 \\ 6 \\ \end{matrix}\,\,
\begin{matrix} 13 \\ 8 \\ \end{matrix}\,\,
\overline{
\begin{matrix} 1 \\ 0 \\ \end{matrix}\,\,
\begin{matrix} 5 \\ 9 \\ \end{matrix}\,\,
\begin{matrix} 0 \\ 7 \\ \end{matrix}\,\,
\begin{matrix} 29 \\ 2 \\ \end{matrix}\,\,
\begin{matrix} 2 \\ 7 \\ \end{matrix}\,\,
\begin{matrix} 14 \\ 1 \\ \end{matrix}\,\,
}\right]$ \\
$\sqrt[3]{32}$ & $\left[
\begin{matrix} 3 \\ 10 \\ \end{matrix}\,\,
\overline{
\begin{matrix} 2 \\ 12 \\ \end{matrix}\,\,
\begin{matrix} 4 \\ 2 \\ \end{matrix}\,\,
\begin{matrix} 1 \\ 1 \\ \end{matrix}\,\,
\begin{matrix} 2 \\ 44 \\ \end{matrix}\,\,
\begin{matrix} 1 \\ 0 \\ \end{matrix}\,\,
\begin{matrix} 4 \\ 1 \\ \end{matrix}\,\,
\begin{matrix} 6 \\ 1 \\ \end{matrix}\,\,
\begin{matrix} 6 \\ 4 \\ \end{matrix}\,\,
\begin{matrix} 0 \\ 2 \\ \end{matrix}\,\,
\begin{matrix} 2 \\ 0 \\ \end{matrix}\,\,
\begin{matrix} 2 \\ 1 \\ \end{matrix}\,\,
\begin{matrix} 8 \\ 9 \\ \end{matrix}\,\,
\begin{matrix} 3 \\ 3 \\ \end{matrix}\,\,
\begin{matrix} 9 \\ 1 \\ \end{matrix}\,\,
}\right]$ \\
\uzlhline
\end{tabular}
\end{minipage}

\end{table}

The code used for the comparison is shown in Listing~\ref{lst:det-search}.
It is a search on the tree guided by a specific strategy.
Each strategy is given the current input vector $x^{(n)}$ beginning with $x^{(0)} = x$.
It would then output one index $ℓ$ and the search would continue with
$x^{(1)} = \mathrm{pivot}_ℓ(x^{(0)})$.
The search would continue until we find a duplicate vector $x^{(n)} = x^{(m)}$
with $n < m$ on the path determined by the strategy.
Since they are the same, $x^{(n)}$ would mark the beginning of the period.

% ==============================================================================
\begin{Python}[
    float=tbp,
    numbers=left,
    caption={
      The implementation of the search for periodic MDCFs.
      The strategy \texttt{strat} outputs a single index $ℓ$, which is used
      for pivoting.
      The search stops once a duplicate vector $x$ has been found and the
      program returns the preperiod and period once found.
    },
    label={lst:det-search},
  ]
def search(x, N, strat):
  seen = {x: 0}
  for n in range(N):
    l = strat(x)
    x = pivot(x, l)
    if x in seen:
      j = seen[x]
      start = L[:j]
      period = L[j:i+1]
      return start, period
    else:
      seen[y] = i + 1
\end{Python}
% ==============================================================================

In summary, the following deterministic strategies were tried:
\begin{itemize}
  \item $\textbf{Min}, \textbf{Max}$: Choosing the minimum and maximum fractional value, respectively.
    These are the strategies which have been analyzed in Chapter~\ref{ch:fibonacci}.
  \item $\textbf{JPA}$: The Jacobi-Perron algorithm,
    which chooses indices in a fixed order.
    Specifically, it chooses the indices $1, 2, …, d$, and repeats this sequence indefinitely.
  \item $\textbf{JPA}'$: A modification of the Jacobi-Perron algorithm introduced by Podsypanin \cite{Podsypanin77}.
    Given a vector $x = (x₁, x₂)$ it chooses the index
    \[
      ℓ =
      \begin{cases}
        1, & \text{ if } x₁ > x₂, \\
        2, & \text{ if } x₁ < x₂.
      \end{cases}
    \]
    For higher dimensions, the algorithm chooses the largest element in $x$.
  \item $\textbf{TY}$:
    The algorithm of Tamura and Yasutomi \cite{Tamura09},
    which is based on the idea of the modified JPA.
    Given vector $x = (x₁, x₂)$, the algorithm chooses the index
    \[
      ℓ =
      \begin{cases}
        1, & \text{ if } \frac{x₁}{\sqrt{|N(x₁)|}} > \frac{x₂}{\sqrt{|N(x₂)|}}, \\
        2, & \text{ if } \frac{x₁}{\sqrt{|N(x₁)|}} < \frac{x₂}{\sqrt{|N(x₂)|}}.
      \end{cases}
    \]
    Again, the algorithm chooses the largest element in each iteration.
    However, it scales down each element by the square root of its norm.
    For higher dimensions, the algorithm chooses
    \[
      ℓ = \argmax_i \frac{x_i}{\sqrt[d+1]{|N(x_i)|}}.
    \]
  \item $\textbf{CC}$: Choosing the closest convergent.
    Out of the $d$ possible indices,
    we choose the one which produces the closest convergent $r^{(n)}$,
    which means that it minimizes the distance to the original input vector.
    This is measured either using the Euclidean norm $\|x - r^{(n)}\|_2$ or using the maximum norm $\|x - r^{(n)}\|_{\infty}$.
\end{itemize}
% TODO: Mention that the Tamura and Yasutomi algorithm was tested by the authors themselves for both cubic and quadratic cases.
% TODO: For how many steps did we run the construction?

Table~\ref{tbl:comparison} lists the results for this section.
The clear winner is the algorithm by Tamura and Yatusomi.
It has found a periodic representation for every cubic irrational I tested.

Since Tamura and Yasutomi's algorithm worked so well,
I have also tested it on fifth, sixth and seventh roots
from $\sqrt[d]{2}$ up to $\sqrt[d]{200}$.
After fourth roots, the algorithm no longer returns a periodic MDCF for all roots.
In fact, I have only found periodic MDCFs for the following roots:
\begin{itemize}
  \item Fifth roots:
    \sqrt[5]{2}, \sqrt[5]{7}, \sqrt[5]{11}, \sqrt[5]{13}, \sqrt[5]{19},
    \sqrt[5]{25}, \sqrt[5]{29}, \sqrt[5]{31}, \sqrt[5]{33}, \sqrt[5]{59},
    \sqrt[5]{82}, \sqrt[5]{123}, \sqrt[5]{152}.
  \item Sixth roots: \sqrt[6]{18}, \sqrt[6]{65}, \sqrt[6]{66},\sqrt[6]{198}.
  \item Seventh roots: \sqrt[6]{2}.
\end{itemize}

% ==============================================================================
\section{Usage in Simultaneous Approximation}
% ==============================================================================

The continued fractions play an important role in Diophantine approximation,
where the goal is to approximate real numbers using rational numbers.
In Lemma~\vref{lem:cf-approx}, we have already seen that the convergents
$pₙ/qₙ$ of a continued fraction $x$ approximate the represented number $x$
particularly well.
More specifically, that every convergent satisfies the bound
\[
  \left|α - \frac{pₙ}{qₙ}\right| < \frac{1}{qₙ^2}.
\]
The Lemma also follows from Dirichlet's approximation theorem.

For its multidimensional counterpart,
the question is whether they approximate the vector particularly well.
Approximating a vector instead of a single number is also known as simultaneous
Diophantine approximation.
Given an irrational vector $(α₁, …, α_d)$, the task is to find a good
rational approximation $(p₁/q, …, p_d/q)$ for every number $α_i$ at once.
Using the simultaneous version of Dirichlet's approximation theorem \cite{Schmidt80},
one can show that there are infinitely many rational vectors $(p₁/q, …, p_d/q)$,
which satisfy
\begin{equation}
  \label{eq:sim-approx}
  \left|α_i - \frac{p_i}{q}\right| ≤ \frac{1}{q^{1 + 1/d}}
  \quad
  \text{ for every } i ∈ \{1, …, d\}.
\end{equation}
Such vectors will be called \emph{good rational approximations} of $(α₁, …, α_d)$.
The idea would be that the convergents of MDCFs are such good rational approximations.
However, if that is not the case, there is still a possibility that some path
during the construction has good rational approximations.
From strongest to weakest, I have analyzed the following questions:
\begin{enumerate}
  \item Do all paths have convergents which are good rational approximations?
  \item Is there a path where all convergents are good rational approximations?
  \item Is there a path where infinitely many convergents are good rational approximations?
\end{enumerate}

To answer these question,
I have implemented a breadth-first search over the construction of an MDCF for the vector $x$.
The search begins with the root $x^{(0)} = x$ and it expands a node $x^{(n)} ∈ ℝ^d$
by adding the vector $\mathrm{pivot}_ℓ(x^{(n)})$ for every $ℓ ∈ \{1, …, d\}$ to the queue.
However, it only expands a node $x^{(n)}$
if its convergent vector $r^{(n)}$ is a good approximation,
i.e. satisfies the bound from Equation~\ref{eq:sim-approx}.
Otherwise, the node is considered a leaf.
The search terminates either after a maximum number of steps is reached or if
there are no more nodes in the queue.
The latter of which would show that there are examples, where no path with the
given approximation bound exists.
This covers the first and second questions.
For the third question, I have implemented a brute-force search,
which simply tries all possible sequences to test for the approximation rate.
The goal would be to see how many convergents satisfy the approximation bound.

Figure~\ref{fig:results-approx} shows the results of this analysis.

% TODO: When repeating the period multiple times, does this actually keep the
% approximation bound intact or does it violate it after some point?
For most cubic roots from the previous test,
there are MDCFs which satisfy the approximation bound.
However, they are notably different MDCFs than those found in the brute-force
search.
So some cubic roots can have different MDCFs
and not all of them must necessarily be good simultaneous approximations.

The most surprising result of this analysis is that there are some cubic roots
for which there exists no path which satisfies the approximation bound of Equation~\ref{eq:sim-approx}.
Perron already suggested that his algorithm does not satisfy this bound \cite{Perron07},
so a simple strategy ought to violate the bound at some point.
However, it turns out that no strategy can keep the approximation bound at each step.
One example is the vector $(\sqrt[3]{5}, \sqrt[3]{25})$.
In this case, the number of convergents starts out growing but quickly drops
after only a few number of iterations.

Since not all cubic roots can be approximated well using MDCFs,
I weakened the bound to allow all convergents which satisfy
\[
  \left|x_i - \frac{p_i}{q}\right| < \frac{c}{q^{1 + 1/d}} \qquad \text{ for every } i ≤ d,
\]
where $c$ is some constant independent of $n$.
The specific constant for each root is listed in Table~\ref{tbl:approx-const}.
% TODO: Make the actual table
\begin{table}[tbp]
  \centering
  \begin{tabular}{cc}
    \uzlhline
    Root & Constant \\
    \hline
    TODO & TODO \\
    \uzlhline
  \end{tabular}
  \caption{Approximation constants for the cubic roots.}
  \label{tbl:approx-const}
\end{table}

% TODO: Add plot for growth

% TODO: Add table for numbers which admit good approximations

% TODO: What about transcendental numbers?

\chapter{Conclusion}
\label{ch:conclusion}


\begin{bibtex-entries}
@article{Murru15,
  title     = {On the periodic writing of cubic irrationals and a generalization of R{\'e}dei functions},
  author    = {Murru, Nadir},
  journal   = {International Journal of Number Theory},
  volume    = {11},
  number    = {03},
  pages     = {779--799},
  year      = {2015},
  publisher = {World Scientific}
}

@book{Bernstein06,
  title     = {The Jacobi-Perron algorithm: Its theory and application},
  author    = {Bernstein, Leon},
  year      = {1971},
  publisher = {Springer},
}

@article{Perron07,
  title     = {Grundlagen f{\"u}r eine Theorie des Jacobischen Kettenbruchalgorithmus},
  author    = {Perron, Oskar},
  journal   = {Mathematische Annalen},
  volume    = {64},
  number    = {1},
  pages     = {1--76},
  year      = {1907},
  publisher = {Springer}
}

@article{Jacobi68,
  title   = {Allgemeine Theorie der kettenbruch{\"a}hnlichen Algorithmen, in welchen jede Zahl aus drei vorhergehenden gebildet wird.},
  author  = {Jacobi, CGJ},
  journal = {Journal f{\"u}r die reine und angewandte Mathematik},
  volume  = {69},
  pages   = {29--64},
  year    = {1868}
}

@article{Karpenkov2024,
  title     = {On a periodic Jacobi--Perron type algorithm},
  author    = {Karpenkov, Oleg},
  journal   = {Monatshefte f{\"u}r Mathematik},
  volume    = {205},
  number    = {3},
  pages     = {531--601},
  year      = {2024},
  publisher = {Springer}
}

@book{Lame1844,
  title     = {Note sur la limite du nombre des divisions dans la recherche du plus grand commun diviseur entre deux nombres entier},
  author    = {Lam{\'e}, Gabriel},
  year      = {1844},
  publisher = {Bachelier}
}

@misc{Gupta00,
  title         = {Bifurcating Continued Fractions},
  author        = {Ashok Kumar Gupta and Ashok Kumar Mittal},
  year          = {2000},
  eprint        = {math/0002227},
  archivePrefix = {arXiv},
  primaryClass  = {math.GM},
  url           = {https://arxiv.org/abs/math/0002227},
}

@article{Northshield11,
	author    = {Sam Northshield},
	journal   = {The American Mathematical Monthly},
	number    = {2},
	pages     = {pp. 171--175},
	publisher = {Taylor & Francis, Ltd., Mathematical Association of America},
	title     = {A Short Proof and Generalization of Lagrange’s Theorem on Continued Fractions},
	volume    = {118},
	year      = {2011}
}

@article{Lagrange1770,
  title={Additions au m{\'e}moire sur la r{\'e}solution des {\'e}quations num{\'e}riques},
  author={Lagrange, Joseph-Louis},
  journal={M{\'e}m. Berl},
  volume={24},
  year={1770}
}
\end{bibtex-entries}

\end{document}
