\documentclass[english,version-2020-11]{uzl-thesis}


\UzLThesisSetup{
  Logo-Dateiname        = {uzl-thesis-logo-itcs.pdf},
  Verfasst              = {am}{Institut für Theoretische Informatik},
  Titel auf Deutsch     = {
    Vorlage für die \LaTeX-Klasse »uzl-thesis« zur Nutzung bei
    Bachelor-~und Masterarbeiten an der  Universität~zu~Lübeck
  },
  Titel auf Englisch    = {
    Template for the \LaTeX\ Class “uzl-thesis” for
    Bachelor's and Master's Theses Written at the University~of~Lübeck
  },
  Autor                 = {Daniel Knaack},
  Betreuerin            = {Prof. Dr. Petra Wichtig-Wichtig},
  Mit Unterstützung von = {Harry Hilfreich},
  Masterarbeit,
  Studiengang           = {Informatik},
  Datum                 = {1. Januar 2021},
  Abstract              = {TODO},
  Zusammenfassung       = {TODO},
  Numerische Bibliographie,
}

\UzLStyle{alegrya modern design}

\begin{document}

\chapter{Bounds}

\section{Determinant}

Assumptions:
\begin{itemize}
  \item The algorithm always chooses the same vector from $C$ until it is removed.
  \item The algorithm always chooses the same index $\ell$ until $x_\ell$ is integral.
  \item The first fractional $x_\ell$ is chosen.
\end{itemize}

\begin{lemma}
  \label{lemma:half}
  If $x_\ell = 1/2$, then $x_\ell$ is integral in the next iteration.
\end{lemma}

\begin{proof}
  Follows from the update rule:
  \[
    x_\ell' = 1/x_\ell = 2. \qedhere
  \]
\end{proof}

\begin{lemma}
  There is a vector $c$ and matrix $B$
  such that $\det B' = \left(\frac12\right)^k \det B$ in $k \le d$,
  where $B'$ is the resulting basis from applying the algorithm with $B$ and $c$.
\end{lemma}

\begin{proof}
  We choose $c = (1, \dots, 1)$ and $B_i = 2^i e_i$ where $e_i$ is the $i$th unit vector.
  In the first iteration, $x_i = 2^{-i}$.
  By Lemma~\ref{lemma:half}, the solution for the first coordinate is integral
  in the next iteration.
  By the update rule, if $x_j = 2^{i - j}$ for $j > i$ in the $i$th iteration,
  then in the next iteration, we have
  \[
    x_i' = \frac{1}{x_j} = 2 \quad \text{ and } \quad x_{j}' = -\frac{x_j}{x_i} = \frac{2^{i-j}}{1/2} = -2^{i + 1 - j}. \qedhere
  \]
\end{proof}

\begin{lemma}
  There is no vector $c$ and no matrix $B$
  such that $\det B' > \left(\frac12\right)^k \det B$ in $k \le d$ steps,
  where $B'$ is the resulting basis from applying the algorithm with $B$ and $c$.
\end{lemma}

\begin{lemma}
  There is no vector $c$ and no matrix $B$
  such that $\det B' \ge \left(\frac12\right)^k \det B$ in $k > d$ steps,
  where $B'$ is the resulting basis from applying the algorithm with $B$ and $c$.
\end{lemma}

\section{Worst-Case}

In one dimension, the worst case input consists of Fibonacci numbers.

\[\begin{aligned}
  x_1[0] & = a_1/b_1                       \\
  x_1[1] & = b_1/(a_1 \bmod b_1)              && = b_1 / r_1[0]      \\
  x_1[2] & = r_1[0]/(b \bmod r_1)       && = r_1[0] / r_1[1] \\
  x_1[3] & = r_1[1]/(r_1 \bmod r_1[1])  && = r_1[1] / r_1[2] \\
         & \vdots
\end{aligned}\]

\begin{align*}
  a_1    & = b_1 + r_1[0]      \\
  b_1    & = r_1[0]+ r_1[1]  \\
  r_1[0] & = r_1[1] + r_1[2] \\
  r_1[1] & = r_1[2] + r_1[3] \\
         & \vdots
\end{align*}
Setting the last two $r_1[t-1]$ and $r_1[t]$ to $1$ gives us the Fibonacci numbers.

For the other indices:
\[\begin{aligned}
  x_1[0] & = a_2/(b_2 \cdot b_1)                                    \\
  x_1[1] & = -(a_2 \bmod (b_2 \cdot b_1)) / (b_2 \cdot r_1[0])        &  & = -r_2[0] / (b_2 \cdot r_1[0]) \\
  x_1[2] & = -(r_2[0] \bmod (b_2 \cdot r_1[0])) / (b_2 \cdot r_1[1])  &  & = -r_2[1] / (b_2 \cdot r_1[1]) \\
  x_1[3] & = -(r_2[1] \bmod (b_2 \cdot r_1[1])) / (b_2 \cdot r_1[2])  &  & = -r_2[2] / (b_2 \cdot r_1[2]) \\
         & \vdots
\end{aligned}\]

\begin{align*}
  a_2    & = b_2 b_1 - r_2[0]      \\
  r_2[0] & = b_2 r_1[0] - r_2[1] \\
  r_2[1] & = b_2 r_1[1] - r_2[2] \\
  r_2[2] & = b_2 r_1[2] - r_2[3] \\
         & \vdots
\end{align*}

\begin{align*}
  a_2 & = r_2[t] + b_2 \sum_{i=0}^t (-1)^{t-i} F_i
\end{align*}

\chapter{Implementation}

\section{Using an Out-of-the-Box Solver}

\begin{itemize}
  \item Algorithm technically only requires one fractional bit to determine how to round.
  \item Problem: Solving a system of linear equations requires exact fractional solutions.
    Using floating-point numbers leads to inaccurate results.
  \item Could be solved by using an iterative method? However, does a method
    exist, which is accurate up to one fractional bit?
\end{itemize}

\section{Custom Implementation}

\begin{itemize}
  \item Uses custom \texttt{ratio} datatype which implements rational numbers.
  \item Linear system solver implemented using Gaussian elimination and pivoting.
  \item Only the basic algorithm is implemented, so far.
\end{itemize}

\begin{bibtex-entries}
\end{bibtex-entries}

\end{document}
