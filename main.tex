\documentclass[english,version-2020-11]{uzl-thesis}

\UzLThesisSetup{
  Masterarbeit,
  Logo-Dateiname        = {uzl-thesis-logo-itcs.pdf},
  Verfasst              = {am}{Institut für Theoretische Informatik},
  Titel auf Deutsch     = {Über elementare Eigenschaften einer mehrdimensionalen Verallgemeinerung des euklidischen Algorithmus},
  Titel auf Englisch    = {On Elementary Properties of a Multi-Dimensional Generalization of the Euclidean Algorithm},
  Autor                 = {Daniel Knaack},
  Betreuerin            = {Prof. Dr. Kim-Manuel Klein},
  Studiengang           = {Informatik},
  Datum                 = {18. Juni 2024},
  Abstract              = {TODO},
  Zusammenfassung       = {TODO},
  Numerische Bibliographie,
}

\UzLStyle{pagella contrast design}
\usetikzlibrary{decorations.pathreplacing, calligraphy}

\newcommand\N{{\mathbb N}}
\newcommand\Z{{\mathbb Z}}
\newcommand\Q{{\mathbb Q}}
\DeclareMathOperator*{\argmin}{arg\,min}

\begin{document}

% ==============================================================================

\chapter{Introduction}

% ==============================================================================

\begin{problem}[Hermite's Question]
  Is there a representation of the real numbers as a sequence of integers such
  that the sequence is eventually periodic if and only if the real number is a
  cubic irrational?
\end{problem}

% TODO: Fix definition of eventually periodic
More formally, we are looking for a function which maps any real number $r$ to
a sequence $(a_n)_{n \in \N}$ such that $a_{k+n} = a_{\ell+n}$ for all $n \ge k$ and
some $k \ne \ell$ if and only if there is a cubic polynomial $p$ such that $p(r) = 0$.
Importantly, the function must map \emph{all} real numbers to a sequence.
Nadir Murru \cite{Murru15} has already shown that there exists a mapping from all
cubic irrationals to eventually periodic sequences.
However, the author was unable to derive a sequence for all real numbers.

% ==============================================================================

\chapter{Preliminaries}

% ==============================================================================

\section{Analysis of the Rational Euclidean Algorithm}

The Euclidean algorithm can be seen as operating on a single rational number $x
= a/b \in \Q$ and returning the greatest common divisor between $a$ and $b$.

% TODO
\begin{Pseudocode}
while  do
  $(a, b) \gets (b \bmod a, a)$
end
\end{Pseudocode}

The size of the input is measured by the size of the denominator.
Since there might be multiple elements with the same denominator but different
numerators which take the same number of steps, we look at the element
with the smallest numerator.

\begin{lemma}
  For the Euclidean Algorithm, $F(n) / F(n+1)$ is the worst-case input,
  i.e. there exists no other input $a / b$ such that $b < F(n + 1)$ or
  $a < F(n)$ which requires more than $n$ steps.
\end{lemma}

\begin{proof}
  Let $X_n$ be the set of inputs, which require $n$ steps in the Euclidean Algorithm.
  For $n = 1$, we have $1/2$ which obviously requires one step of the algorithm.
  Assume $x' = \frac{a}{b}$ is the element with the smallest denominator, which
  requires $n$ steps.
  Then $x = \frac{b}{a+b}$ is the element with the smallest denominator,
  which requires $n + 1$ steps.

  First, $x$ requires $n + 1$ steps since applying the algorithm on this input
  yields $x'$.
  Next, we need to show that $x'$ is the element with the smallest denominator.
  For a contradiction, assume that there is an element $y = c/d \in X_{n+1}$
  with $d < a + b$ or $c < b$.
  Applying the algorithm on $y$ yields $(d \bmod c) / c$,
  from which it follows that $c \ge b$.
  Otherwise, it would contradict our assumption that $a/b$ is the element with
  smallest denominator.
  By the same argument $d \bmod c \ge a$ and therefore $d - c \ge a$.
  Putting both inequalities together leads to
  \[
    d < a + b \le d - c + c = d,
  \]
  which is a contradiction.
\end{proof}

\section{Continued Fraction}

\begin{definition}
  A continued fraction for a (potentially infinite) list of numbers $a_0, a_1,
  \dots \in \Z$ is defined as
  \begin{align*}
    [a_0; a_1, \dots] = a_0 + \frac{1}{[a_1; a_2, \dots]}
  \end{align*}
\end{definition}

\begin{definition}
  The $n$-th convergent of a continued fraction $[a_0; a_1, \dots]$
  is defined as the finite continued fraction $[a_0; a_1, \dots, a_n]$.
\end{definition}

\begin{lemma}
  For every $n \in \N$ there exists two polynomials $p_n$ and $q_n$ such that the
  $n$-th convergent of $[a_0; a_1, \dots]$ can be represented by the quotient
  of the two polynomials:
  \[
    [a_0; a_1, \dots, a_n] = \frac{p_n(a_0, a_1, \dots, a_n)}{q_n(a_0, a_1, \dots, a_n)}
  \]
\end{lemma}

\begin{proof}
  For $n = 0$, we have $p_0(a_0) = a_0$ and $q_0(a_0) = 1$.
  By induction, it follows that
  \[
    [a_0; a_1, \dots, a_n] = a_0 + \frac{1}{[a_1; a_2 \dots, a_n]} = a_0 + \frac{q_{n-1}(a_1, \dots, a_n)}{p_{n-1}(a_1, \dots, a_n)}.
  \]
  From this, we can derive the polynomials $p_n$ and $q_n$:
  \begin{align*}
    p_n(a_0, \dots, a_n) & = a_0 p_{n-1}(a_1, \dots, a_n) + p_{n-2}(a_2, \dots, a_n) \\
    q_n(a_0, \dots, a_n) & = p_{n-1}(a_1, \dots, a_n). \qedhere
  \end{align*}
\end{proof}

\begin{definition}
  The canonical structure of the $n$-th convergent $[a_0; a_1, \dots, a_n]$ is defined as
  \[
    [a_0; a_1, \dots, a_n] = \frac{p_n}{q_n} = \frac{p_n(a_0, \dots, a_n)}{p_{n-1}(a_1, \dots, a_n)}.
  \]
\end{definition}

\begin{lemma}
  The canonical structure of the continued fractions satisfies the recurrence
  \[
    \begin{matrix}
      p_n = a_n p_{n-1} + p_{n-2}, & p_{-1} = 1, & p_0 = a_0 \\
      q_n = a_n q_{n-1} + q_{n-2}, & q_{-1} = 0, & q_0 = 1
    \end{matrix}
  \]
\end{lemma}

\begin{lemma}
  For every positive $x \in \Q$, there are unique
  $a_0, a_1, \dots, a_n \in \N \setminus \{0\}$ with $a_n \ne 1$
  such that
  \[
    x = [a_0; a_1, \dots, a_n].
  \]
\end{lemma}

\begin{proof}

\end{proof}

\begin{lemma}
  If $[a_0; a_1, \dots, a_n]$ is the unique continued fraction for $a/b$,
  then the Euclidean algorithm requires $n$ steps on input $(a, b)$.
\end{lemma}

\begin{lemma}
  The smallest continued fraction which requires $n$ steps is $[1; 1, 1, \dots, 1]$.
\end{lemma}

\section{Generalized Euclidean Algorithm}

\begin{Pseudocode}
solve $Bx = c$
while $x$ is not integral do
  find $x_\ell$ which is not integral
  $c \gets B_\ell$
  $B_\ell \gets Bx \bmod Π(B)$
  solve $Bx = c$
end
\end{Pseudocode}

\begin{Pseudocode}
solve $Bx = c$
while $x$ is not integral do
  find $x_\ell$ which is not integral
  $c \gets B_\ell$
  $B_\ell \gets Bx \bmod Π(B)$

  for $i$ in $1..{\le} \ell$ do
    if $i = \ell$ then
      $x_i' \gets 1 / \{x_\ell\}$
    else
      $x_i' \gets \{x_i\} / \{x_\ell\}$
    end
  end
  $x \gets x'$
end
\end{Pseudocode}

\begin{align*}
  x_i' = \begin{cases}
    1 / x_l, & \text{ if } i = l, \\
    x_i / x_l & \text{ otherwise.}
  \end{cases}
\end{align*}

\section{Algebraic Numbers}

\begin{definition}
  A complex number $\alpha$ is an algebraic number if $\alpha$ is the root of some monic polynomial with rational coefficients
  \[
    p(\alpha) = 0, \quad p(x) = x^n + a_{n-1} x^{n-1} + \dots a_1 x + a_0, \quad a_0, a_1, \dots, a_{n-1} \in \Q.
  \]
\end{definition}

\begin{definition}[Ring]
  A ring is a set $R$ with two binary operations $+$ (addition) and $\cdot$ (multiplication)
  satisfying the following conditions:
  \begin{enumerate}
    \item $(R, +)$ is a commutative group.
    \item $(R, \cdot)$ is a monoid, i.e. $\cdot$ is associative and there exists an identity element $1 \in R$.
    \item Multiplication is distributive with respect to addition:
      \begin{align*}
        a \cdot (b + c) = (a \cdot b) + (a \cdot c), & \quad \text{for all } a, b, c \in R \\
        (b + c) \cdot a = (b \cdot a) + (c \cdot a), & \quad \text{for all } a, b, c \in R
      \end{align*}
  \end{enumerate}
  A ring is commutative if multiplication is commutative.
\end{definition}

\begin{definition}[Ideal]
  % TODO: Add note about left and right-sided ideals?
  Given a commutative ring $R$, an ideal $I$ is a subset of $R$ satisfying
  \begin{enumerate}
    \item $(I, +)$ is a subgroup of $(R, +)$
    \item For every $r \in R$ and every $x \in I$, the $r \cdot x \in I$.
  \end{enumerate}
\end{definition}

\begin{definition}[Quotient Ring]
  For an ideal $I$ over a ring $R$, the equivalence relation $\sim$ is defined as
  \[
    a \sim b \quad \text{if and only if} \quad a - b \in I.
  \]
  % TODO: Define the equivalence class of an element.
  The set of equivalence classes is denoted by $R/I$ and forms another ring with
  \begin{align*}
    (a + I) + (b + I) = (a + b) + I. \\
    (a + I) \cdot (b + I) = (a \cdot b) + I.
  \end{align*}
\end{definition}

\begin{example}
  Let $\Q[X]$ be the ring of polynomials in the variable $X$ with only rational
  coefficients and $I = (X^2 - X - 1)$ an ideal consisting of all multiples of $X^2 - X - 1$.
  Then, the operations of the quotient ring $\Q[X]/(X^2 - X - 1)$ are
  \begin{align*}
    (a_0 + a_1 X) + (b_0 + b_1 X)
    & = (a_0 + b_0) + (a_1 + b_1) X \\
    (a_0 + a_1 X) \cdot (b_0 + b_1 X)
    & = (a_0 b_0) + (a_1 b_0 + a_0 b_1) X + (a_1 b_1) X^2 \\
    & = (a_0 b_0 + a_1 b_1) + (a_1 b_0 + a_0 b_1 + a_1 b_1) X
  \end{align*}
\end{example}

The quotient ring $\Q[X]/(X^2 + X + 1)$ can also be seen as a extension field $\Q(\alpha)$ over $Q$,
where $\alpha$ is an algebraic number satisfying for the respective polynomial.
Hence, $\Q[X]/(X^2 + X + 1)$ forms a field.

\begin{definition}[Simple Extension]
  Given an algebraic number $\alpha$ of degree $d$,
  a simple extension of the rational numbers with $\alpha$ is the set
  \[
    \Q(\alpha) = \{\sum_{i=0}^d \alpha^i q_i \mid q_i \in \Q\}.
  \]
\end{definition}

% ==============================================================================

\chapter{Choosing the Minimum}

% ==============================================================================

We begin with the simplest strategy: Choosing the minimum in each iteration.

\section{Bound on the Decrease of the Determinant}

\begin{proposition}
  The determinant decreases by a factor of at least $\max\{1/3,1/(d + 1)\}$ over two iterations
  using the minimum strategy.
\end{proposition}

\begin{proof}
  Let $x ∈ [0, 1)^d$.
  We assume WLOG the values $x₁, x₂, \dots, x_d$ are ordered in increasing order.
  Suppose $x₁ \{x_i / x₁\} ≥ 1/3$ for every $i ≤ d$ and $x₁ \{1 / x₁\} ≥ 1/3$.
  Then, over two iterations we have a decrease of
  \[
    x₁ \left\{\frac{1}{x₁}\right\} ≤ 1 - x₁ ≤ \frac{1}{2}. \qedhere
  \]
\end{proof}

\section{Golden Ratio}

The goal is to find an input $x ∈ [0, 1]^d$,
where the decrease stays constant in each iteration and is maximal.
This input cannot be rational since it would eventually lead to one element in $x$ being zero.
Therefore, our input must consist of at least one algebraic number.

First, assume that in the input $x$ the value $x₁$ is the smallest and $x₂$ the second smallest.
After pivoting, $x₂'$ has to be the smallest element in $x' = \mathrm{pivot}(x, 1)$ unless $x₁ < 1/2$.
However, in that case the decrease is already smaller in the first iteration.
Therefore, we can assume that $x₂'$ is the smallest element in the next iteration.
For the decrease to remain constant, it follows that $x₁ = x₂'$.
Furthermore, we must have $x₂ = x₃'$.
Extending this to all elements leads to the following equations:
\[
  x₁ = \frac{x₂}{x₁} - 1,
  \quad x₂ = \frac{x₃}{x₁} - 1,
  \quad \dots,
  \quad x_{d-1} = \frac{x_d}{x₁} - 1,
  \quad x_d = \frac{1}{x₁} - 1.
\]
These can be rearranged such that $x_{i + 1} = x₁ (x_i + 1) = x₁^{i+1} + x₁^i + \dots + x₁$
and substituting $x_d$ in the last equation gives us
\[
  x₁^d + x₁^{d-1} + \dots + x₁ = \frac{1}{x₁} - 1
  \iff
  x₁^{d+1} + x₁^{d-1} + \dots + x₁ - 1 = 0.
\]

\begin{lemma}
  The polynomial $p(x) = x^{d+1} + x^d + \dots + x - 1$ has only one positive real root.
\end{lemma}

\begin{corollary}
  If $ψ$ is the positive root of $p(x)$, then $ψ^i + ψ^{i-1} + \dots + ψ < 1$ for $i ≤ d$.
\end{corollary}

In the following lemma, we use the fact that if $p(ψ) = 0$, then
\[
  ψ^{-1} = ψ^d + ψ^{d-1} + \dots + ψ + 1.
\]

\begin{lemma}
  Let $ψ$ be the positive real root of $p(x) = x^{d+1} + \dots + x - 1$
  and let $x$ be a $d$-dimensional vector with $x_i = ψ^i + \dots + ψ$.
  Then, the decrease of the determinant under $x$ remains constant over every iteration.
\end{lemma}

\begin{proof}
  Because $ψ$ is positive, $x$ is already ordered in increasing order.
  Hence, we pivot with $x₁$ first.
  Let $x' = \mathrm{pivot}(x, 1)$,
  then
  \begin{align*}
    x₁'
    & = \left\{\frac{1}{x₁}\right\}
    = \left\{\frac1{ψ}\right\}
    = \left\{ψ^d + ψ^{d-1} + \dots + ψ\right\} = x_d, \\
    x_{i+1}'
    & = \left\{\frac{x_{i+1}}{x₁}\right\}
    = \left\{\frac{ψ^{i+1} + ψ^i + \dots + ψ}{ψ}\right\}
    = \left\{ψ^i + ψ^{i-1} + \dots + ψ + 1\right\}
    = x_i.
  \end{align*}
  Because we choose the minimum, $x'$ yields the same decrease in the next iteration.
\end{proof}

\section{Fibonacci Numbers}

\[
  F(n + d + 1) = F(n + d) + \dots + F(n + 1) + F(n).
\]

\section{Multi-Dimensional Continued Fractions}

\begin{align*}
  y_d = \frac{1}{x_1} - a_d,
  \quad y_i = \frac{x_{i+1}}{x_1} - a_1
\end{align*}

\begin{align*}
  x_1 = \frac{1}{a_d + y_d},
  \quad x_{i+1} = \frac{a_i + y_i}{a_d + y_d}
\end{align*}

\begin{align*}
  K_1(a^{(1)}, a^{(2)}, \dots) & = \frac{1}{a^{(1)}_d + K_d(a^{(2)}, a^{(3)}, \dots)} \\
  K_{i+1}(a^{(1)}, a^{(2)}, \dots) & =
    \frac{a^{(1)}_i + K_i(a^{(2)}, a^{(3)}, \dots)}
         {a^{(1)}_d + K_d(a^{(2)}, a^{(3)}, \dots)} \\
\end{align*}

\section{Comparison to the Jacobi-Perron Algorithm}

The algorithm in this form, where we always choose the same pivot,
is equivalent to the Jacobi-Perron algorithm, which was first proposed
by Jacobi \cite{Jacobi68} and later analyzed by Perron \cite{Perron07}.
In modern times, the algorithm was picked up again by Bernstein \cite{Bernstein06}.
Each of them has analyzed the algorithm in the context of Hermite's Question.

Charles Hermite initially posed a question to Jacobi, whether there exists a
representation of the real numbers as an infinite sequence of natural numbers
such that the sequence is eventually periodic if and only if the number is a
cubic irrational. The question remains unanswered with the latest attempt
\cite{Murru15} showing that there is an infinite sequence which is eventually
periodic for every cubic irrational.

% ==============================================================================

\chapter{Choosing the Closest Pair of Neighbors}

% ==============================================================================

We use the following strategy:

\begin{enumerate}
  \item Sort $x$ in increasing order.
  \item Find the index $ℓ$ which minimizes $\{x_{ℓ+1}/x_ℓ\}$.
  \item Choose $ℓ$ in the first iteration and $ℓ + 1$ in the second iteration.
\end{enumerate}

\section{Bounds on the Decrease of the Determinant}

\begin{theorem}
  The determinant decreases by a factor of at least $d+1$ over two iterations
  using the closest neighbor strategy.
\end{theorem}

\begin{proof}
  Let $x ∈ ℚ^d$ with $0 < x_i < 1$ for every $1 ≤ i ≤ d$.
  We assume $x$ is already given in increasing order.
  We additionally define $x_0 = 0$ and $x_{d+1} = 1$.
  Suppose that pivoting with any $ℓ ∈ \{0, 1, \dots, d\}$
  decreases the determinant slower than $(d+1)$ over two iterations,
  i.e.
  \[
    % TODO: Handle ℓ = 0 case, (just a formality)
    x_1 > \frac{1}{d+1},
    \quad \text{ and } \quad
    x_\ell \left\{\frac{x_{\ell+1}}{x_\ell}\right\} > \frac{1}{d+1}, \quad \text{ for all } 1 ≤ ℓ ≤ d.
  \]
  Because $x_{\ell+1} > x_\ell$, we have
  \[
    x₁ - x₀ > \frac{1}{d+1},
    \quad \text{ and } \quad
    x_{\ell+1} - x_\ell ≥ x_\ell \left\{\frac{x_{\ell+1}}{x_\ell}\right\} > \frac{1}{d+1}.
  \]

  The inequality tells us, that the distance between all neighbors
  (including $1$ and $0$) must be greater than $1/(d+1)$.
  Or equivalently, we must space our $d$ variables on the unit interval such that they
  divide it into $d+1$ intervals and each interval is larger than $1/(d+1)$.

  {\begin{center}
    \begin{tikzpicture}
      \def\one{0.8 \textwidth}
      \draw (0, 0) -- (\one, 0);

      \draw (0, -0.2) -- (0, 0.2) node[above] {$x₀$};
      \draw (\one, -0.2) -- (\one, 0.2) node[above] {$x_{d+1}$};
      \draw ({3.5 * \one / 5}, 0.2) node[above] {$\dots$};

      \foreach \x/\y in {1,...,3,4/d}
        \draw ({\x * \one / 5}, -0.2) -- ({\x * \one / 5}, 0.2) node[above] {$x_{\y}$};
      \foreach \x in {1,...,3,5}
        \draw[decorate, line width=1.25pt, decoration={calligraphic brace, mirror}]
          ({(\x - 1) * \one / 5 + 1}, -0.3) -- node[below] {$> \frac{1}{d+1}$} ({\x * \one / 5 - 1}, -0.3);
    \end{tikzpicture}
  \end{center}}

  However, no such values can exist.
  The maximum size we can achieve is exactly $1/(d+1)$ by evenly
  spacing the values.
  Moving one value increases the size of one interval, but decreases the size
  of the other interval.
  Therefore, we must have a decrease of at least $(d+1)$ over two iterations.
\end{proof}

\section{Golden Ratio and Fibonacci Numbers}

To find a generalization of the golden ratio for this strategy,
we first establish an equation for the decrease of the determinant.
The decrease must stay the same in each iteration.
Hence, we should choose values for $x₁, \dots, x_d$ such that the decrease
stays the same for all possible combinations.
\[
  x_1 = \frac{x_2}{x_1} - 1 = \frac{x_3}{x_2} - 1 = \dots = \frac{x_d}{x_{d-1}} = \frac{1}{x_d} - 1.
\]
Solving the equation for $x₁$ yields the following polynomial:
\[
  x₁^{d+1} + x₁ - 1 = 0.
\]

\begin{itemize}
  \item \textbf{Polynomial}: \[x^{d+1} + x - 1 = 0.\]
  \item \textbf{Algebraic Solution Vector}: \[x_i = \psi^{d+1-i}.\]
  \item \textbf{Linear Recurrence}: \[F(n + 1) = F(n) + F(n - d).\]
  \item \textbf{Rational Solution Vector}: \[x_i = \frac{F(n)}{F(n-d+i)}.\]
  \item \textbf{Matrix}:
    \[\left(\begin{array}{cccc|c}
      F(n)   & 0      & \dots  & 0      & F(n - d) \\
        0    & F(n)   & \dots  & 0      & F(n - d + 1) \\
      \vdots & \vdots & \ddots & \vdots & \vdots   \\
        0    & 0      & \dots  & F(n)   & F(n - 1) \\
    \end{array}\right)\]
\end{itemize}

The algebraic solution falls apart after two iterations when pivoting with the
smallest element, i.e. $x_1$.

% ==============================================================================

\chapter{Notes on the Determinant}

% ==============================================================================

\section{Generalized Fibonacci Sequence}

Dividing two consecutive Fibonacci numbers approaches the golden ratio as $n$ increases.
The golden ratio is a solution to the equation $x^2 - x - 1 = 0$.
For higher dimensions, we will encounter similar polynomial equations with higher degree.
The goal of this section is to generalize the relationship between linear
recurrences like the Fibonacci sequence with their respective golden ratio.

Since the algorithm expects $x_i$ between $0$ and $1$, we have to use $1/\phi$ instead of $\phi$.
The value $1/\phi$ is the root of another polynomial $x^2 + x - 1$, which is the reciprocal
of the polynomial $x^2 - x - 1$.

\begin{definition}
  Given a polynomial $p = \sum_{i=0}^n a_i x^n$, its \emph{reciprocal polynomial},
  denoted as $p^*$, is defined as
  \[
    p^*(x) = x^n p(x^{-1}) = a_n + a_{n-1} x + \dots + a_0 x^n.
  \]
\end{definition}

\begin{example}
  The reciprocal of the golden ratio polynomial $x^2 - x - 1$ is
  \begin{align*}
    x^2 + x - 1,
  \end{align*}
  which has a root at $\psi = 1/\phi$.
\end{example}

\begin{lemma}
  Let $p$ be a polynomial. Then, $a$ is a root of $p$ if and only if $a^{-1}$ is a root of $p^*$.
\end{lemma}

\begin{proof}
  Let $a$ be a root of $p$. It follows $p^*(a^{-1}) = a^n p(a) = a^n \cdot 0 = 0$.
  Now let $a^{-1}$ be a root of $p^*$. Then $p^*(a^{-1}) = a^n p(a) = 0$.
  By construction, $a$ cannot be zero and therefore we must have $p(a) = 0$.
\end{proof}

\begin{definition}
  A linear recurrence with coefficients~$a_0, \dots, a_d$ is an equation of the
  following form:
  \[
    F(n + 1) = a_0 F(n - d) + a_1 F(n - d + 1) + \dots + a_{d-1} F(n - 1) + a_d F(n).
  \]
\end{definition}

\begin{definition}
  Given a linear recurrence~$F$ with coefficients~$a_0, \dots, a_d$, its
  \emph{characteristic polynomial} $p_F$ is defined as
  \[
    p_F(x) = x^{d+1} - a_d x^d - a_{d-1} x^{d-1} - \dots - a_1 x - a_0.
  \]
\end{definition}

\begin{example}
  The linear recurrence which has $p_d^*(x)$ as its characteristic polynomial is:
  \begin{align*}
    F(n) = F(n - 1) + (-1)^{d+1} F(n - d) - \sum_{i=2}^{d - 1} (-1)^{i+1} 2 F(n - i).
  \end{align*}
\end{example}

% TODO: Show that this converges! We're still missing the convergence criteria
\begin{lemma}
  Let $F$ be a linear recurrence with coefficients $a_0, \dots, a_d$.
  If its characteristic polynomial has a real positive root $\phi$
  and the sequence $F(n+1)/F(n)$ is converging, then
  \[
    \lim_{n \to \infty} \frac{F(n + 1)}{F(n)} = \phi.
  \]
\end{lemma}

\begin{proof}
  By assumption, the ratios approach some limit $L$. It follows:
  \[
    L
    = \lim_{n \to \infty} \frac{F(n + 1)}{F(n)}
    = \lim_{n \to \infty} a_0 + \sum_{i = 1}^d \frac{a_i F(n - d + i)}{F(n)}
  \]
  Each term in the sum can be rewritten as a product of consecutive terms:
  \[
    \frac{F(n - 1 - d + i)}{F(n - 1)}
    = \frac{F(n - 1 - d + i)}{F(n - d + i)} \frac{F(n - d + i)}{F(n - d + i + 1)} \dots \frac{F(n - 2)}{F(n-1)}.
  \]
  Hence, each term in the sum approaches $L^{-i}$,
  which results in the following polynomial:
  \[
    L = a_0 + \sum_{i = 1}^d a_{d - i} L^{-i},
    \quad \text{or equivalently} \quad
    L^{d+1} = a_0 + a_1 L + \dots + a_d L^d,
  \]
  which directly corresponds to a root of its characteristic polynomial.
\end{proof}

\begin{corollary}
  Under the same conditions, $\lim_{n \to \infty} F(n + i) / F(n) = \phi^i$ for $i > 1$.
\end{corollary}

Given, the linear recurrence $F(n + d + 1) = F(n) + a_1 F(n - 1) + \dots + a_d F(n + d)$,
the value for $x_i^{(n)}$ in the solution $x^{(n)}$ is
\begin{equation}
  \label{eq:general-solution}
  x_i^{(n)} = \frac{F(n - i) + \sum_{k=1}^{i-1} a_{d-k} F(n - k)}{F(n)}.
\end{equation}

\begin{lemma}
  Pivoting with the first element of $x^{(n)}$ using the modified update rule yields the vector
  \[
    x' = (x^{(n-1)}_2, x^{(n-1)}_3, \dots, x^{(n-1)}_d, x^{(n-1)}_1).
  \]
\end{lemma}

\begin{proof}
  For $x_1$, we get
  \[
    \begin{aligned}
      \frac{1}{x_1}
      & = \frac{F(n)}{F(n - 1)} \\
      & = a_d + \frac{F(n - d - 1) + a_1 F(n - d - 2) + \dots + a_{d-1} F(n - 2)}{F(n - 1)} \\
      & = a_d + x^{(n-1)}_d.
    \end{aligned}
  \]
  For the other values, we get
  \begin{align*}
    \frac{x_i}{x_1}
    & = \frac{F(n - i) + \sum_{k=1}^{i-1} a_{d-k} F(n - k)}{F(n)} \frac{F(n)}{F(n - 1)} \\
    & = a_{d-1} + \frac{F(n - i) + \sum_{k=2}^{i-1} a_{d-k} F(n - k)}{F(n-1)} \\
    & = x^{(n-1)}_{i+1} \qedhere
  \end{align*}
\end{proof}

\begin{corollary}
  Running the generalized Euclidean algorithm with $x^{(n)}$ as the solution to
  the linear system $B x = c$ requires $n$ steps, if $x^{(n)}_1$ is the
  smallest element in $x^{(n)}$.
\end{corollary}

Of course, the Euclidean algorithm receives a matrix $B \in \Z^{d \times d}$
and vector $c \in \Z^d$ as its input and not the solution $x$ itself.
However, we can construct a very simple linear system $B^{(n)} x = c^{(n)}$,
where $x^{(n)}$ is the solution, in the following way:
\[
  B^{(n)} =
  \begin{pmatrix}
    F(n) & 0 & \dots & 0 & 0 \\
    0 & F(n) & \dots & 0 & 0 \\
    \vdots & \vdots & \ddots & \vdots & \vdots \\
    0 & 0 & \dots & F(n) & 0 \\
    0 & 0 & \dots & 0 & F(n) \\
  \end{pmatrix},
\]
\[
  c^{(n)} =
  \begin{pmatrix}
    F(n - 1) \\
    F(n - 2) + a_{d-1} F(n - 1) \\
    \vdots \\
    F(n - d + 1) + \sum_{k=1}^{d-2} a_{d-k} F(n - k) \\
    F(n - d) + \sum_{k=1}^{d-1} a_{d-k} F(n - k) \\
  \end{pmatrix}.
\]

\begin{lemma}
  The vector $x^{(n)}$ from Equation~\ref{eq:general-solution} is a solution to the
  linear system $B^{(n)} x = c^{(n)}$.
\end{lemma}

\begin{proof}
  Follows directly from the definition of $x^{(n)}$.
\end{proof}

\section{Pivoting the Same Element}

In one dimension, a solution is bad when the value $x_1$ is close to $1$.
However, in the next iteration $x_1$ would be close to $0$ and therefore
would lead to a greater decrease in the determinant.
An initial guess would be that the solution lies exactly between $0$ and $1$ at $1/2$.
However, this solution is also good as after just one step we have achieved integrality.

The solution is to choose a value $x_1$ such that its value in the first
iteration is the same as its value in the second iteration.
This means we are trying to solve the following equation:
\[
  x_1 = 1/x_1 - 1.
\]
This gives us the polynomial $x^2 + x - 1$ where the root is the inverse of the
golden ratio.

% TODO: Explore other strategies. Can we generalize this to every possible strategy?
In just two dimensions, we already have an additional degree of freedom.
Since the solution $x$ consists of two values, we can choose which index we pivot with.
We will choose the value with the smallest fractional value as this decreases
the determinant the fastest.
We assume w.l.o.g. that our values are sorted in increasing order.
This means we pivot with $x_1$ first.

% TODO: What about just choosing the golden ratio for every x_i?
% TODO: Are odd dimensions easier since the decrease is not worse? We could
% choose two values to be the same, since the ratio for odd dimensions is
% smaller than the previous dimension
Just like in one dimension we want to choose $x_1$ carefully such that the
determinant decreases the same amount in every iteration.
A simple solution would be to fill both values with the golden ratio.
In this case, it does not matter which index we choose as each one decreases
the determinant the same amount.
However, in two dimensions we can make this decrease slightly worse
by choosing different values for $x_1$ and $x_2$ such that we swap the pivot
in each iteration.

Assuming the value for $x_1$ is already chosen where would $x_2$ be placed?
Ideally, it would be close to the right side of $x_1$, because then the new
value $-x_2 / x_1$ would be close to $-1$ and the fractional value would be
close to $1$.
Since the decrease must stay the same in each iteration, this gives us the
following two equations:
\begin{align*}
  x_1 = 2 - x_2 / x_1 \\
  x_2 = 1 / x_1 - 1,\\
\end{align*}
which leads to the following polynomial:
\begin{align*}
  x_1^3 - 2x_1^2 - x_1 + 1 = 0
\end{align*}

Let $\phi_2$ be the root of this polynomial.
If we choose $x_1 = \phi_2$ and $x_2 = 2\phi_2 - \phi_2^2$,
then the values in the second iteration are
\[\begin{aligned}
  x_1' & = 1 / x_1 - 1   &  & = 1 / \phi_2 - 1                    &  & = 2\phi_2 + \phi_2^2 &  & = x_2, \\
  x_2' & = 2 - x_2 / x_1 &  & = 2 - (2\phi_2 + \phi_2^2) / \phi_2 &  & = \phi_2             &  & = x_1
\end{aligned}\]

We can extend this to $d$ dimensions using the following equations:
\begin{align*}
  x_1 = 2 - x_2 / x_1 \\
  x_2 = 2 - x_3 / x_1 \\
  x_3 = 2 - x_4 / x_1 \\
  \vdots \\
  x_{d-1} = 2 - x_d / x_1 \\
  x_d = 1 / x_1 - 1 \\
\end{align*}

These leads us to two polynomials depending on the dimension.
\begin{itemize}
  \item For even $d$: $x - 1 + 2 x^2 - 2 x^3 + \dots + x^{d+1}$
  \item For odd $d$: $1 - x - 2 x^2 + 2 x^3 - 2 x^4 \dots + x^{d+1}$
\end{itemize}
or in one equation $p_d(x) := 1 - x - (-x)^{d+1} - \sum_{i = 2}^{d} 2 (-x)^{i} = 0$.
Let $\phi_d$ be the root of this polynomial.
Then, we choose as our solution $x = (x_1, \dots, x_d)$ with
\begin{equation}
  \label{eq:rotate-solution}
  x_1 = \phi_d, \quad x_i = 2 - x_1 x_{i-1} = -(-\phi_d)^i + \sum_{k=0}^{i-1} 2 (-\phi_d)^i.
\end{equation}

\begin{example}
  For $d = 5$, we get the following values:

  \begin{tikzpicture}[scale=10]
    \draw (0.5820216424559055, -0.01) -- node[above] {$x_1$} (0.5820216424559055, 0.01);
    \draw (0.7181491667223714, -0.01) -- node[above] {$x_2$} (0.7181491667223714, 0.01);
    \draw (0.7661126076135922, -0.01) -- node[above] {$x_3$} (0.7661126076135922, 0.01);
    \draw (0.6837042616132034, -0.01) -- node[above] {$x_4$} (0.6837042616132034, 0.01);
    \draw (0.8252940926247403, -0.01) -- node[above] {$x_5$} (0.8252940926247403, 0.01);
    \draw[->] (0, 0) -- (1, 0);
  \end{tikzpicture}
\end{example}

\begin{itemize}
  \item \textbf{Polynomial}:
    \[
      1 - x - (-x)^{d+1} - \sum_{i = 2}^{d} 2 (-x)^{i} = 0.
    \]
  \item \textbf{Algebraic Solution Vector}:
    \[
      x_i = 2x + (-x)^{i+1} + \sum_{k=2}^{i} (-x)^{k}.
    \]
  \item \textbf{Linear Recurrence}:
    \[
      F(n) = F(n - 1) + (-1)^{d+1} F(n - d) - \sum_{i=2}^{d - 1} (-1)^{i+1} 2 F(n - i).
    \]
  \item \textbf{Rational Solution Vector}:
    \[
      x_1 = \frac{F(n - 1)}{F(n)}, x_i = \frac{2 F(n-1) + (-1)^{i+1} F(n - i) + \sum_{k=2}^i (-1)^k F(n-k)}{F(n)}.
    \]
  \item \textbf{Matrix}:
    \[\left(\begin{array}{cccc|c}
      F(n)   & 0      & \dots  & 0      & F(n - 1) \\
        0    & F(n)   & \dots  & 0      & 2 F(n - 1) - F(n - 2) \\
      \vdots & \vdots & \ddots & \vdots & \vdots   \\
        0    & 0      & \dots  & F(n)   & 2 F(n-1) + (-1)^{d+1} F(n - d) + \sum_{k=2}^d (-1)^k F(n-k) \\
    \end{array}\right)\]
\end{itemize}

\begin{example}
  \begin{align*}
    \left\{ \frac{1}{x_1} \right\}
    & = \left\{ \frac{F(n - 1) + 2 F(n - 2) - 2 F(n - 3) + F(n - 4)}{F(n - 1)} \right\} \\
    & = \frac{2 F(n - 2) - 2 F(n - 3) + F(n - 4)}{F(n - 1)} \\
    \left\{ -\frac{x_2}{x_1} \right\}
    & = \left\{ -\frac{2 F(n - 1) - F(n - 2)}{F(n)} \frac{F(n)}{F(n - 1)} \right\} \\
    & = \frac{F(n - 2)}{F(n - 1)} \\
    \left\{ -\frac{x_3}{x_1} \right\}
    & = \left\{ -\frac{2 F(n - 1) - 2 F(n - 2) + F(n - 3)}{F(n)} \frac{F(n)}{F(n - 1)} \right\} \\
    & = \frac{2 F(n - 2) - F(n - 3)}{F(n - 1)}
  \end{align*}
\end{example}

\section{An Idempotent Solution for Two Dimensions}

Let $\vec x = (\psi, 1 - \psi)$, where $\psi$ is the only real positive root of $x^2 + x - 1$.
For this solution, it does not matter which element we choose to pivot with, it will always
yield the same result.

Pivoting with $x_1$:
\begin{align*}
  \frac{1}{x_1} - 1 & = \frac{1}{\psi} - 1 = \psi, \\
  1 - \frac{x_2}{x_1} & = 1 - \frac{1 - \psi}{\psi} = 1 - \psi.
\end{align*}

Pivoting with $x_2$:
\begin{align*}
  \frac{1}{x_2} - 2 & = \frac{1}{1 - \psi} - 2 = \psi, \\
  1 - \frac{x_1}{x_2} & = 1 - \frac{\psi}{1 - \psi} = 1 - \psi.
\end{align*}

\begin{align*}
  \left(
  \begin{array}{cc|c}
    F(n) & 0    & F(n-1) \\
    0    & F(n) & F(n) - F(n-1) \\
  \end{array}
  \right)
\end{align*}

\section{Bound over Two Iterations}

\begin{proposition}
  In one dimension, the determinant decreases by a factor of at least $1/2$ over
  two iterations.
\end{proposition}

\begin{proof}
  Let $x₁ ∈ ℚ$ be the input to the 1-dimensional algorithm.
  We assume $x₁ > 1/2$ as the other case already proves the assumption.
  Then, $1/x₁ < 2$ and we have
  $$
    x₁ \left\{\frac{1}{x₁}\right\}
    = x₁ \left(\frac{1}{x₁} - 1\right)
    = 1 - x₁ < 1/2.
    $$
  Therefore, over two iterations we decrease by at least $1/2$.
\end{proof}

How can we derive the golden ratio from this?

What can we expect for higher dimensions?
This depends on the strategy of our pivot element $x_\ell$.
There are different strategies we can follow like choosing the minimum in each iteration, for example.

\section{Golden Ratio}

\begin{align*}
  ψ(ψ + 1)^d - 1 = 0,
\end{align*}

Reflecting the polynomial gives us:
\begin{align*}
  φ^{d+1} - (φ + 1)^d = 0.
\end{align*}
Using the binomial theorem:
\begin{align*}
  φ^{d+1} = ∑_{i=0}^d \binom{d}{i} φ^i,
\end{align*}
which leads directly to the Fibonacci numbers for this ratio:
\begin{align*}
  F(n + d + 1) = ∑_{i=0}^d \binom{d}{i} F(n + i).
\end{align*}

We choose $x₁ = ψ$ and $x_i = x₁ (1 + x₁)^{i-1} = ψ (1 + ψ)^{i-1}$.

\begin{align*}
  \text{pivot}(\text{pivot}(x, ℓ₁), ℓ₂) = \left(x_{σ(1)}, \dots, x_{σ(d)}\right).
\end{align*}

From the proof, it follows that $1/(d+1)$ should be the worst-case.
However, using $1/(d+1)$ exactly only requires one iteration until the values are integral.

\begin{align*}
  ψ^{d+1} + ∑_{k=0}^d k ψ^{d-k} = 1.
\end{align*}

Is this the optimal one?

\begin{theorem}
  There is no vector $x ∈ (0, 1)^d$
  such that any pivot operation yields the same decrease
  over every iteration for $d ≥ 2$.
\end{theorem}

\begin{proof}
  Suppose there is such an $x$.
  Then, the distance between neighbors is the same between all elements of $x$.
  This vector is unique.
\end{proof}

% ==============================================================================

\chapter{Notes on Hermite's Question}

% ==============================================================================

\begin{problem}[Hermite's Question]
  Is there a representation of the real numbers as a sequence of integers such
  that the sequence is eventually periodic if and only if the real number is a
  cubic irrational?
\end{problem}

More formally: Does there exists a function $f$ which maps any $α ∈ ℝ$ to an
infinite sequence $(a_n)_{n ∈ ℕ}$ of natural numbers such that $a_n$ is
eventually periodic if and only if $x$ is a cubic irrational?

For any $d$th irrational number $x ∈ ℝ$, we can find a unique irreducible polynomial $p(x) ∈ ℚ[x]$
which has $x$ as its root and is monic.
Furthermore, it must have $d$ distinct roots $α₁, α₂, \dots, α_d ∈ ℂ$,
where one of them is $x$.

\section{Karpenkov's HAPD Algorithm}

See \cite{Karpenkov2024}.
Given a monic irreducible polynomial $p(x)$ over $ℚ$ with integer coefficients and roots $α₁, \dots, α_d ∈ \overline{Q}$,
we can find algebraic vectors $v_1, \dots, v_d$ for each root $α_i$:
\[
  v_i = (q_0(α_i), q_1(α_i), q_2(α_i), \dots, q_{d-1}(α_i)),
\]
where the polynomials $q_i$ must have degree $i$, respectively.
The choice of $q_i$ does not matter as long as they have degree $i$.
Furthermore, we can find a rational matrix $C(p) ∈ ℚ^{d × d}$ such that $v_1, \dots, v_d$ are the eigenvectors of $M$.
Let $p(x) = x^d + a_{d-1} x^{d-1} + \dots + a₁ x^1 + a₀$, then
\[
  C(p) =
  \begin{pmatrix}
    0 & 0 & \dots & 0 & -a₀ \\
    1 & 0 & \dots & 0 & -a₁ \\
    0 & 1 & \dots & 0 & -a₂ \\
    \vdots & \vdots & \ddots & \vdots & \vdots \\
    0 & 0 & \dots & 1 & -a_{d-1} \\
  \end{pmatrix},
\]
By multiplying all elements of this matrix with the least common multiple of the denominators in $C(p)$,
we can turn it into an integer matrix.
The matrix $C(p)$ is also called the \emph{companion matrix} of $p$.

The matrix naturally gives rise to a linear recurrence of the form
\[
  F(n + d) = -a_{d-1} F(n + d - 1) - \dots - a_1 F(n + 1) - a_0 F(n).
\]
For example, when $d = 2$ and $a_0 = a_1 = -1$, the linear recurrence
corresponds to the Fibonacci sequence.

\begin{definition}
  The \emph{Markov-Davenport characteristic} of a vector $x ∈ ℝ^3$ with respect to vectors $u, v, w ∈ ℝ^3$
  is defined as
  \[
    \det
    \begin{pmatrix}
      x_1 & v_1 & w_1 \\
      x_2 & v_2 & w_2 \\
      x_3 & v_3 & w_3 \\
    \end{pmatrix}
    · \det
    \begin{pmatrix}
      u_1 & x_1 & w_1 \\
      u_2 & x_2 & w_2 \\
      u_3 & x_3 & w_3 \\
    \end{pmatrix}
    · \det
    \begin{pmatrix}
      u_1 & v_1 & x_1 \\
      u_2 & v_2 & x_2 \\
      u_3 & v_3 & x_3 \\
    \end{pmatrix}
  \]
\end{definition}

% ==============================================================================

\chapter{Notes on the Implementation}

% ==============================================================================

\section{Using an Out-of-the-Box Solver}

\begin{itemize}
  \item Algorithm technically only requires one fractional bit to determine how to round.
  \item Problem: Solving a system of linear equations requires exact fractional solutions.
    Using floating-point numbers leads to inaccurate results.
  \item Could be solved by using an iterative method? However, does a method
    exist, which is accurate up to one fractional bit?
\end{itemize}

\section{Custom Implementation}

\begin{itemize}
  \item Uses custom \texttt{ratio} datatype which implements rational numbers.
  \item Linear system solver implemented using Gaussian elimination and pivoting.
  \item Only the basic algorithm is implemented, so far.
\end{itemize}

% ==============================================================================

\begin{bibtex-entries}
@article{Murru15,
  title     = {On the periodic writing of cubic irrationals and a generalization of R{\'e}dei functions},
  author    = {Murru, Nadir},
  journal   = {International Journal of Number Theory},
  volume    = {11},
  number    = {03},
  pages     = {779--799},
  year      = {2015},
  publisher = {World Scientific}
}

@book{Bernstein06,
  title     = {The Jacobi-Perron algorithm: Its theory and application},
  author    = {Bernstein, Leon},
  volume    = {207},
  year      = {2006},
  publisher = {Springer}
}

@article{Perron07,
  title     = {Grundlagen f{\"u}r eine Theorie des Jacobischen Kettenbruchalgorithmus},
  author    = {Perron, Oskar},
  journal   = {Mathematische Annalen},
  volume    = {64},
  number    = {1},
  pages     = {1--76},
  year      = {1907},
  publisher = {Springer}
}

@article{Jacobi68,
  title   = {Allgemeine Theorie der kettenbruch{\"a}hnlichen Algorithmen, in welchen jede Zahl aus drei vorhergehenden gebildet wird.},
  author  = {Jacobi, CGJ},
  journal = {Journal f{\"u}r die reine und angewandte Mathematik},
  volume  = {69},
  pages   = {29--64},
  year    = {1868}
}

@article{Karpenkov2024,
  title     = {On a periodic Jacobi--Perron type algorithm},
  author    = {Karpenkov, Oleg},
  journal   = {Monatshefte f{\"u}r Mathematik},
  volume    = {205},
  number    = {3},
  pages     = {531--601},
  year      = {2024},
  publisher = {Springer}
}
\end{bibtex-entries}

\end{document}
