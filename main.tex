\documentclass[english,version-2020-11]{uzl-thesis}

\UzLThesisSetup{
  Masterarbeit,
  Logo-Dateiname        = {uzl-thesis-logo-itcs.pdf},
  Verfasst              = {am}{Institut für Theoretische Informatik},
  Titel auf Deutsch     = {Über elementare Eigenschaften einer mehrdimensionalen Verallgemeinerung des euklidischen Algorithmus},
  Titel auf Englisch    = {On Elementary Properties of a Multi-Dimensional Generalization of the Euclidean Algorithm},
  Autor                 = {Daniel Knaack},
  Betreuerin            = {Prof. Dr. Kim-Manuel Klein},
  Studiengang           = {Informatik},
  Datum                 = {18. Juni 2025},
  Abstract              = {
    A number is a quadratic irrational if and only if it can be represented by
    a periodic continued fraction.
    In 1839, Charles Hermite asked Jacobi,
    whether a similar representation for algebraic numbers of higher degree exists.
    Although the question was posed nearly two centuries ago,
    no known representation has been found.
    Previous work has focused on individual algorithms,
    but none of them have resolved the question.
    In contrast, this thesis considers a broader class of multidimensional
    continued fractions, which unifies many of these algorithms under a single
    framework.

    The basis of these continued fractions is a generalization of the Euclidean algorithm,
    mirroring the already established connection between ordinary continued fractions and
    the Euclidean algorithm.
    I show that all periodic multidimensional continued fractions represent
    algebraic numbers and that a subset of them always converges.
    Moreover, I provide an experimental analysis comparing several algorithms
    and examining their effectiveness in simultaneous Diophantine approximation.

    % TODO: Maybe we should move this into contributions?
    %The primary application of continued fractions has been the approximation
    %of real numbers by rationals.
    %The assumption is that multidimensional continued fractions serve a similar
    %role in the approximation of real vectors.
    %Despite this, I have found specific cases,
    %where the multidimensional continued fractions are unable to approximate a
    %real vector continuously.
  },
  Zusammenfassung       = {
    Im Jahr 1839 stellte Charles Hermite die Frage, ob jede reelle Zahl durch eine
    Folge ganzer Zahlen dargestellt werden kann, die genau dann periodisch ist,
    wenn die dargestellte Zahl eine kubisch irrationale Zahl ist.
    Während Kettenbrüche dieses Problem für quadratische Irrationale lösen,
    ist bis heute keine analoge Darstellung für algebraische Zahlen höheren Grades bekannt.
    Diese Arbeit untersucht einen neuen Zugang zu Hermites Frage durch die Konstruktion
    multidimensionaler Kettenbrüche (MDCF), die gewöhnliche Kettenbrüche mithilfe
    einer Verallgemeinerung des euklidischen Algorithmus auf höhere Dimensionen erweitern.

    Es wird gezeigt, dass die meisten multidimensionalen Kettenbrüche konvergieren
    und dass jede periodische Darstellung eine algebraische Zahl beschreiben muss,
    deren Grad nach oben beschränkt ist.
    Diese Ergebnisse beantworten eine Richtung von Hermites Frage in höheren Dimensionen.
    Eine experimentelle Untersuchung ergänzt die theoretischen Resultate und liefert
    numerische Hinweise darauf, dass eine breite Klasse algebraischer Zahlen --
    einschließlich vieler kubischer Wurzeln -- zu periodischen MDCFs führt.
  },
  Numerische Bibliographie,
}

\UzLStyle{pagella contrast design}
\addbibresource{refs.bib}

\usepackage{standalone}
\usepackage{todonotes}
\usepackage{pgf-pie}
\usepackage{pgfplots}
\pgfplotsset{compat=1.18}

\usetikzlibrary{decorations.pathreplacing, calligraphy, intersections, backgrounds, graphdrawing, graphs, calc, 3d}
\usepgflibrary{shadings}
\usegdlibrary{trees}

\newcommand\N{{\mathbb N}}
\newcommand\Z{{\mathbb Z}}
\newcommand\Q{{\mathbb Q}}
\DeclareMathOperator*{\argmin}{arg\,min}
\DeclareMathOperator*{\argmax}{arg\,max}
\DeclareMathOperator*{\lcm}{lcm}
\newcommand\floor[1]{\left\lfloor#1\right\rfloor}
\newcommand\ceil[1]{\left\lceil#1\right\rceil}

\begin{document}

\chapter{Introduction}
\label{ch:intro}

In 1839, Charles Hermite wrote a letter to Jacobi~\cite{Hermite50} about the
representation of real numbers.
He asked, whether there exists a representation of the real numbers as a
sequence of integers, which is periodic if and only if the represented number
is a cubic irrational, i.e. the root of a cubic polynomial.
Although he posed the question nearly two centuries ago,
it remains unanswered to this day.

The standard way to represent numbers is through decimal notation.
A number is represented as a sequence of digits, which begins with the integer
part and is followed by a (potentially infinite) sequence of digits for the
fractional part.
If the decimal expansion of a number is finite, then the number is clearly rational.
Furthermore, if the decimal expansion is periodic, then the number is rational, too.
The same behavior occurs for continued fractions and quadratic irrationals.
Continued fractions are fractions of the form
\[
  a₀ + \cfrac{1}{a₁ + \cfrac{1}{a₂ + \cfrac{1}{⋱}}},
\]
where $a₀, a₁, a₂, …$ are integers.
Any real number has a continued fraction expansion,
which can be constructed using the Euclidean algorithm.
If that continued fraction is periodic, then the number must be a quadratic irrational.
More importantly, the converse is also true:
If a number is a quadratic irrational,
then its continued fraction must be eventually periodic.
Naturally, we can ask whether we can extend this and
find a periodic representation for cubic irrationals.
However, no such representation exists as of yet.

The interest in a generalization to cubic irrationals
comes from the effectiveness of continued fractions in related fields.
The primary example is Diophantine approximation, where the goal is to
approximate real numbers using rational numbers as closely as possible.
It turns out that the best rational approximations are exactly those provided
by continued fractions.
A generalization of continued fractions could serve a similar role in the field
of simultaneous Diophantine approximation, where the goal is to approximate
multiple real numbers at once with a single rational vector.

Hermite's original question only applies to cubic irrationals,
but it can be easily generalized to any algebraic number:
Does there exist a representation of the real numbers as a sequence of integers
that is periodic if and only if the represented number is a algebraic number of
degree $d$?
There are two parts to this question.
The first is the representation of real numbers as a sequence of integers.
Each finite subsequence of the representation should give us a rational number,
which approaches the represented number as the sequence grows larger.
The second part is about the periodicity of the integer sequence.
In the original question, the sequence should repeat after some point if and
only if the represented is a cubic irrational.
The general question asks whether a periodic sequence exists for any algebraic
number with a certain degree.

% ==============================================================================
\section{Background}
\label{sec:jacobi-perron}
% ==============================================================================

Since Hermite originally posed his question to Jacobi, it was he who first
attempted to answer it.
He developed an algorithm \cite{Jacobi68} inspired by the Euclidean algorithm,
which calculates the greatest common divisor of three numbers instead of two.
At each step,
the algorithm chooses the smallest number and uses it to divide the other two.
In the next triple, the other two numbers are replaced by their remainder.
This process is continued until the greatest common divisor is found.
Later, Oskar Perron generalized Jacobi's method to arbitrary dimensions \cite{Perron07},
resulting in what is now called the Jacobi–Perron Algorithm (JPA).
His algorithm is essentially a generalization of the Euclidean algorithm to $n$ numbers.
At each step, he still chooses the smallest element at each iteration.

Continued fractions are typically constructed using the Gauss transformation,
which is defined as
\begin{align*}
  T(x) = \frac{1}{x - \floor{x}}.
\end{align*}
This transformation is applied repeatedly until $T^m(x) = T^n(x)$ for $m ≠ n$,
in which case the continued fraction is periodic.
The Jacobi-Perron uses a similar transform,
which takes a vector $x = (x₁, …, x_d)$ as input and calculates
\begin{align*}
  T(x₁, …, x_d) =
  \left(
  \frac{x_2 - \floor{x_2}}{x_1 - \floor{x_1}},
  \frac{x_3 - \floor{x_3}}{x_1 - \floor{x_1}},
  …,
  \frac{x_d - \floor{x_d}}{x_1 - \floor{x_1}},
  \frac{1}{x_1 - \floor{x_1}}
  \right)
\end{align*}
Again, this transformation is applied repeatedly until $T^m(x) = T^n(x)$ for $m ≠ n$,
at which point it becomes periodic.
However, Perron has only shown one direction:
If the algorithm becomes periodic, then $x$ is an algebraic number with degree $d+1$.
In an attempt to find an algorithm, which solves both directions,
previous work often replaces this transformation with a different one,
which can be considered JPA-type algorithms.
For example, a different transformation could use the second smallest number.
They usually consider only one particular transformation and iterate this until a period has been found.
Despite numerous different transformations being proposed,
Hermite's question remains open.

% ==============================================================================
\section{Contributions of this Thesis}
\label{sec:contributions}
% ==============================================================================

% The other algorithms focus on a single path, whereby the only use their own
% transformation function to find a periodic representation.
This thesis extends previous work on multidimensional continued fractions (MCFs).
The JPA usually only considers the smallest number in each iteration,
however there are algorithms based on the JPA which choose a different number,
like choosing the largest number \cite{Podsypanin77} or choosing some number
based on a cost function \cite{Tamura09}.
In this thesis, I present a whole class of MCFs,
which encompass all JPA-type algorithms.
The main contributions regarding these MCFs are as follows:
\begin{enumerate}
  \item \textbf{Convergence}:
    I establish sufficient conditions under which the proposed class of MCFs
    algorithms converges.
    This generalizes known convergence results for the JPA to a wider family of
    transformations and solves the first part of Hermite's question.
  \item \textbf{Algebraicity from Periodicity}:
    I prove that if a multidimensional continued fraction expansion becomes
    eventually periodic, then the original input vector consists of algebraic
    numbers.
    This solves one direction for the second part of Hermite's question.
    It still leaves open the question whether algebraic numbers always lead to
    periodic MCFs.
\end{enumerate}

In addition to these theoretical results,
I have performed an experimental analysis on the MCFs.
The aim of this analysis was to see which strategies could be used
to construct periodic MCFs for algebraic numbers
and I provide a comparison of existing JPA-type algorithms,
which can be used for constructing MCFs of cubic and quartic irrationals.
One strategy shows particular promise in the construction,
since it has provided a periodic MCF for any cubic and quartic irrational I tested.

The second type of analysis was focused on the application of MCFs.
Since ordinary continued fractions can be used for the best rational approximations
of a single real numbers, the idea for MCFs would be that they provide the best
rational approximations of real vectors.
With my analysis, I show that not all convergents lead to good rational approximations.
I give one specific example where no MCF produces good rational
approximations of a vector at all times.

The basis of these MCFs is a generalization of the Euclidean algorithm from
Klein and Reuter \cite{Klein24}.
The initial aim was to analyze the algorithm on its worst-case performance
and whether there exists some generalization of Fibonacci numbers,
which represent the worst-case for the classical Euclidean algorithm.
As such, a side result of this thesis is a proof that such Fibonacci numbers do
exist at least for one strategy and, more importantly, that they represent the
worst-case for this strategy.
Using these numbers, I derive the multidimensional analogue of the golden
ratio, which can be seen as one of the simplest example of a periodic MCF.

In summary, there are three main contributions:
\begin{enumerate}
  \item The multidimensional continued fractions,
    including a proof of their convergence and
    how they lead to algebraic numbers.
  \item An experimental analysis of the multidimensional continued fractions
    on cubic irrationals as well as their application in simultaneous
    Diophantine approximation.
  \item Worst-case bounds for the generalized Euclidean algorithm
    using higher-order Fibonacci numbers and multidimensional golden ratios.
\end{enumerate}

% ==============================================================================
\section{Related Work}
\label{sec:related-work}
% ==============================================================================

As previously mentioned,
one of the first algorithms studied for this problem is the JPA.
The algorithm was analyzed by Bernstein \cite{Bernstein71} again,
who identified explicit classes of cubic irrationals for which the JPA
becomes periodic \cite{Bernstein64A, Bernstein65, Bernstein64B}.
However, numerical computations by Elsner and Hasse \cite{Elsner67} have shown
that the JPA seems to fail for certain cube roots,
casting doubt on whether the algorithm can actually solve Hermite’s problem.
As a result, numerous alternative algorithms have since been proposed
\cite{Assaf05, Hendy81, Schweiger00, Schweiger13}.

One such alternative is the family of bifurcating or ternary continued
fractions \cite{Gupta00},
which extend the classical continued fractions to two dimensions.
Using this representation, Murru has constructed periodic expansions for all
cubic irrationals \cite{Murru15}.
While this addresses one half of Hermite’s question --
showing that every cubic irrational can be represented periodically --
it does not provide a representation for arbitrary real numbers,
and thus, it does not provide a full answer to Hermite's question.

More recently, Karpenkov has proposed two new algorithms \cite{Karpenkov21, Karpenkov24}.
The first is called the $\sin^2$-algorithm and he has shown that this algorithm
becomes periodic for every totally real cubic irrational -- that is any roof of
a cubic polynomial with three real roots.
The second algorithm is called the HAPD algorithm \cite{Karpenkov24} and he
conjectures that this algorithm is periodic for all cubic irrationals,
although this conjecture remains unproven.

Apart from the algorithmic approach,
there has also been significant effort in a geometric generalization of
continued fractions.
Felix Klein famously interpreted the convergents of a continued fractions as
points on integer lattices \cite{Klein95}.
His interpretation has lead to a geometric proof of Lagrange’s theorem, which
will also be presented in this thesis following the work of Korkina~\cite{Korkina96}.
Arnol'd has suggested a generalization of this interpretation to higher
dimensions \cite{Arnold98} and it was conjectured that they satisfy a
multidimensional equivalent of Lagrange's theorem,
which shows that every quadratic irrational has an eventually periodic
continued fraction.
This conjecture was eventually proven by German \cite{German08},
which indicates that there could be a connection between the multidimensional
continued fractions and algebraic numbers.

Beyond multidimensional continued fractions, there have been alternative
generalizations of classical number-theoretic functions.
One notable example is the Minkowski question-mark function $?(x)$,
which maps a quadratic irrational $x$ to a rational number.
Efforts to define higher-dimensional analogues of this function aim to mirror
the relationship between algebraic numbers and their representations.
There has been an extension of this function to two dimensions \cite{Beaver04},
but this has not solved Hermite's question, yet.

% ==============================================================================
\section{Structure of this Thesis}
\label{sec:structure}
% ==============================================================================

Chapter~\ref{ch:preliminaries} introduces the necessary background for this thesis,
which mainly includes algebraic number theory and lattice theory.
Chapter~\ref{ch:quadratic} goes through the case of quadratic irrationals and continued fractions.
It presents a geometric interpretation of continued fractions based on Klein polygons,
which is used in the subsequent proof of Lagrange's theorem.
Chapter~\ref{ch:generalized-euclidean} introduces the generalized Euclidean algorithm.
In Chapter~\ref{ch:fibonacci}, the generalized Euclidean algorithm is analyzed
for worst-case performance under two different strategies.
Each results in a different generalization of Fibonacci numbers and its own definition of a golden ratio.
This golden ratio is the first case of a periodic representation of an algebraic number.
On the basis of this result, Chapter~\ref{ch:mdcf} generalizes the continued fractions to higher dimensions
and presents the two main theoretical results for thesis: Convergence and periodicity.
Chapter~\ref{ch:implementation} analyzes the second part of Hermite's problem,
whether multidimensional continued fractions of algebraic numbers are always periodic.
I present examples of such continued fractions for cubic irrationals and I
compare different strategies for constructing these continued fractions.

\chapter{Preliminaries}

In this thesis, we will consider real algebraic numbers which are roots of polynomials
\[
  p(x) = a_{d+1} x^{d+1} + a_d x^d + \dots + a_1 x + a_0
\]
with $aᵢ ∈ ℤ$ and $p$ is irreducible.

\begin{itemize}
  \item Quadratic fields as vector spaces
  \item Algebraic field extensions
  \item Monic polynomials, irreducible polynomials
  \item Rational expressions, polynomial rings
\end{itemize}

A rational expression of a number $α$ is a fraction of two polynomials, i.e.
\[
  \frac{p(α)}{q(α)} = \frac{pₙ x^n + \dots + p₁ x + p₀}{qₙ x^n + \dots + q₁ x + q₀}.
\]
The set of all rational expressions of an algebraic number $α$ is denoted as $ℚ(α)$.

\chapter{Continued Fractions}
\label{ch:quadratic}

% TODO: Find a source for this chapter
We begin with Hermite's question for quadratic irrationals.
A \emph{quadratic irrational} is any real number
that is a root of a polynomial with degree $2$.
The main goal of this chapter is to show that the continued fraction expansion
of a number is periodic if and only if the number is a quadratic irrational.
Traditionally, this result is proven using the algebraic properties of the
continued fraction and the minimal polynomial of the quadratic irrational.
In contrast, this chapter takes a geometric approach based on Klein polygons.
These polygons visualize the continued fractions as points in a two-dimensional
integer lattice and reduce the proof to a simple geometric argument based on unimodular matrices.

% ==============================================================================
\section{Definition and Basic Properties}
\label{sec:cf-def}
% ==============================================================================

Although continued fractions are typically defined over a sequence of
positive integers, we extend the definition to allow for real numbers.
This will be useful for the theorems which follow.
A continued fraction will be denoted more compactly as $[a₀; a₁, …]$.

\begin{definition}
  \label{def:cont-frac}
  Given a sequence $(aₙ)_{n≥0}$ of real numbers, the finite continued
  fractions over this sequence are defined inductively as
  \[
    [a₀] = a₀, \qquad
    [a₀; a₁, …, aₙ] = a₀ + \frac{1}{[a₁; a₂, …, aₙ]}.
  \]
  The infinite continued fraction $[a₀; a₁, a₂, …]$ is then defined if the limit
  \[
    r = \lim_{n → ∞} [a₀; a₁, …, aₙ]
  \]
  exists.
\end{definition}

In a finite continued fraction $[a₀; a₁, …, aₙ]$,
we begin at the first coefficient $a₀$ and
calculate the continued fraction of its tail $[a₁; a₂, …, aₙ]$.
However, this works in reverse, too.
We can begin at the last coefficient $aₙ$
and merge it with the previous coefficient
using the same approach as in the definition.

\begin{lemma}
  \label{lem:cf-nesting}
  Let $a₀, a₁, …, aₙ, x ∈ ℝ$, then
  \[
    [a₀; a₁, …, aₙ, x] = [a₀; a₁, …, a_n + 1/x]
  \]
\end{lemma}

\begin{proof}
  If $n = 0$, then
  \[
    [a₀; x] = a₀ + \frac{1}{[x]} = a₀ + \frac{1}{x} = [a₀ + 1/x].
  \]
  Suppose the lemma is true for some $n ≥ 0$, then
  \begin{align*}
    [a₀; a₁, …, aₙ, x]
    & = a₀ + \frac{1}{[a₁; a₂, …, aₙ, x]} \\
    & = a₀ + \frac{1}{[a₁; a₂, …, aₙ + 1/x]} \\
    & = [a₀; a₁, …, aₙ, x]. \qedhere
  \end{align*}
\end{proof}

For an infinite continued fraction,
it is necessary to show that the finite continued fractions $[a₀; a₁, …, aₙ]$ are converging.
These continued fractions are called \emph{convergents},
specifically the finite continued fraction $[a₀; a₁, …, aₙ]$
is called the \emph{$n$-th convergent} of $[a₀; a₁, …]$.
If $[a₀; a₁, …]$ is a continued fraction with integer coefficients,
then each convergent produces a unique rational $p/q$.
This representation is called the \emph{canonical representation} of the convergent.
We calculate this representation inductively as follows:

The canonical representation of the first convergent is simply
\[
  [a₀] = a₀.
\]
Suppose that the $(n-1)$-th convergent of $[a₁; a₂, …]$ is $p'/q'$,
then the $n$-th convergent of the continued fractions $[a₀; a₁, …]$ is
\[
  [a₀; a₁, …, aₙ]
  = a₀ + \frac{1}{[a₁; a₂, …, aₙ]}
  = a₀ + \frac{q'}{p'}
  = \frac{p' a₀ + q'}{p'}.
\]
Thus, the canonical representation $p/q$ of $[a₀; a₁, …, aₙ]$ is defined as:
\begin{align*}
  p = p' a₀ + q', \quad q = p'.
\end{align*}

Of course, this requires the convergent of a different continued fraction.
Instead, we use a more direct way to calculate the canonical representation $pₙ/qₙ$
from the previous convergents $p_{n-1}/q_{n-1}$ and $p_{n-2}/q_{n-2}$
of the same continued fraction.
We use the following recurrence to calculate $p_n/q_n$:
\begin{align*}
  p_n & = p_{n-1} a_n + p_{n-2}, & p_{-1} & = 1, & p_{-2} & = 0, \\
  q_n & = q_{n-1} a_n + q_{n-2}, & q_{-1} & = 1, & q_{-2} & = 0.
\end{align*}

\begin{lemma}
  \label{lem:cf-wallis}
  Let $x ∈ ℝ$, then
  \[
    [a₀; a₁, …, a_{n-1}, x] = \frac{pₙ}{qₙ} = \frac{p_{n-1} x + p_{n-2}}{q_{n-1} x + q_{n-2}}.
  \]
\end{lemma}

\begin{proof}
  If $n = 0$, then
  \[
    [x] = x = \frac{1x + 0}{0x + 1} = \frac{p_{-1} x + p_{-2}}{q_{-1} x + q_{-2}}.
  \]
  Suppose, the lemma is true for $n ≥ 0$.
  By Lemma~\ref{lem:cf-nesting}, we have
  \begin{align*}
    [a₀; a₁, …, a_{n-1}, x]
    & = [a₀; a₁, …, a_{n-1} + 1/x].
  \end{align*}
  From the induction hypothesis, it follows that
  \begin{align*}
    [a₀; a₁, …, a_{n-1} + 1/x]
    & = \frac{p_{n - 2} (aₙ + 1/x) + p_{n-3}}{q_{n-2} (aₙ + 1/x) + q_{n-3}} \\
    & = \frac{x (p_{n-2} aₙ + p_{n-3}) + p_{n-2}}{x (q_{n-2} aₙ + q_{n-3}) + q_{n-2}} \\
    & = \frac{p_{n-1} x + p_{n-2}}{q_{n-1} x + q_{n-2}}. \qedhere
  \end{align*}
\end{proof}

Thus, we can use this simple recurrence to calculate the canonical
representation of a convergent.
This representation is unique if the fraction $pₙ/qₙ$ cannot be reduced any further.
In other words, $pₙ$ and $qₙ$ have a greatest common divisor of $1$.
By Bezout's identity, $pₙ$ and $qₙ$ have a greatest common divisor of $1$
if and only if there are integers $a, b$ such that $apₙ + bqₙ = 1$.
Choosing $a = q_{n-1}$ and $b = p_{n-1}$ solves the identity
as shown by the following lemma.

\begin{lemma}
  \label{lem:cf-det}
  For every $n ≥ 0$, we have $p_{n-1} q_{n-2} - q_{n-1} p_{n-2} = (-1)^n$.
\end{lemma}

\begin{proof}
  For $n = 0$, we have
  \[
    p_{-1} q_{-2} - q_{-1} p_{-2} = 1 - 0 = 1.
  \]
  Suppose that the lemma holds for $n ≥ 0$.
  By Lemma~\ref{lem:cf-wallis},
  \begin{align*}
    p_{n-1} q_{n-2} - q_{n-1} p_{n-2}
    & = (p_{n-2} a_{n-1} + p_{n-3}) q_{n-2} - (q_{n-2} a_{n-1} + q_{n-3}) p_{n-2} \\
    & = p_{n-3} q_{n-2} - q_{n-3} p_{n-2}.
  \intertext{By the induction hypothesis,}
    p_{n-1} q_{n-2} - q_{n-1} p_{n-2}
    & = -(p_{n-2} q_{n-3} - q_{n-2} p_{n-3})
      = -(-1)^{n-1}
      = (-1)^n. \qedhere
  \end{align*}
\end{proof}

% ==============================================================================
\section{Construction Using the Euclidean Algorithm}
\label{sec:cf-construction}
% ==============================================================================

% TODO: Should we jump out of the gate with this problem?
The definition only tells us how to calculate the value of a continued fraction,
not how to construct a continued fraction.
However, the first part of Hermite's question requires an explicit representation of every real number.
Thus, we need to construct a unique continued fraction $[a₀; a₁, …]$ for every
real number $x$ such that $x = [a₀; a₁, …]$.

So far, we have allowed continued fractions with real coefficients.
For the construction, we allow only integer coefficients.
More specifically, we allow only continued fractions $[a₀; a₁, a₂, …]$
where the first coefficient $a₀$ can be any integer but the remaining ones have to be positive.
These constraints do not guarantee a unique representation alone.
Consider $x = 3/2$, with the current requirements there are two possible representations:
\[
  x = [1; 1, 1] = 1 + \cfrac{1}{1 + \cfrac{1}{1}} \qquad \text{ or } \qquad x = [1; 2] = 1 + \cfrac{1}{2}.
\]
The issue is that we can always split the last coefficient in the continued fraction.
In general, if $x = [a₀; a₁, …, aₙ]$, then also $x = [a₀; a₁, …, aₙ - 1, 1]$.
Therefore, we additionally require that in a finite continued fraction the last value is never $1$.

We begin with the representation of rational numbers.
We use the Euclidean algorithm to construct the continued fraction for a number $x ∈ ℚ$.
More specifically, if $x = p/q$, then we run the algorithm with the input pair $(p, q)$.
At each step, we keep track of the quotient,
which will be used as a coefficient in the continued fraction.
So if we calculate $p = a₀q + r$ in the first division step, then the quotient $a₀$ will be
the first coefficient in the continued fraction $[a₀; a₁, …, aₙ]$ for $p/q$.
%This is also the reason why we allow $a₀$ to be negative.
%If the fraction $p/q$ is negative, then $a₀$ is negative, too.

\begin{example}
  \label{ex:euclidean-cf}
  Consider $x = 13/5$.
  The Euclidean algorithm computes
  \begin{align*}
    13 & = 2 · 5 + 3 \\
     5 & = 1 · 3 + 2 \\
     3 & = 1 · 2 + 1 \\
     2 & = 2 · 1 + 0.
  \end{align*}
  The quotient in each line correspond directly to the continued fraction of $13/5$:
  \[
    \frac{13}{5}
    = [2; 1, 1, 2]
    = 2 + \cfrac{1}{1 + \cfrac{1}{1 + \cfrac{1}{2}}}
    = 2 + \cfrac{1}{1 + \cfrac{2}{3}}
    = 2 + \cfrac{3}{5}
    = \frac{13}{5}.
  \]
\end{example}

\begin{lemma}
  \label{lem:cf-rat}
  Every rational number has a finite continued fraction.
\end{lemma}

\begin{proof}
  Let $p/q$ be the reduced fraction of a rational number, i.e. $\gcd(p, q) = 1$.
  We proceed by induction over the number of steps when running the
  Euclidean algorithm on $(p, q)$.
  Suppose only one step is required, then $p = a₀ q + 0$.
  Because $\gcd(p, q) = 1$, the fraction $p/q$ must be an integer.
  Hence, the continued fraction $[a₀]$ represents $p/q$ correctly.

  Next, suppose that there exists a valid continued fraction for any rational
  number which requires $n$ steps in the Euclidean algorithm.
  Let $p = a₀ q + r$ be the first step of the Euclidean algorithm
  and suppose $(p, q)$ requires $n+1$ steps.
  Then, $(q, r)$ requires $n$ steps and by induction there exists a continued
  fraction $[a₁; a₂, …, aₙ]$ for $q/r$.
  Using this fact, we can construct the continued fraction for $p/q$, since
  \[
    \frac{p}{q}
    = a₀ + \frac{r}{q}
    = a₀ + \frac{1}{\frac{r}{q}}
    = a₀ + \frac{1}{[a₁; a₂, …, aₙ]}
    = [a₀; a₁, …, aₙ].
  \]
  Therefore, $[a₀; a₁, …, aₙ]$ is a valid continued fraction for $p/q$.
\end{proof}

For rational numbers,
the application of the Euclidean is straightforward.
We simply run the algorithm on the pair $(p, q)$ to construct a continued fraction for $p/q$.
However, if the number is irrational, then the application is not so straightforward.
The issue is that it is not clear how we can split up an irrational number into
two inputs $(a, b)$ for the Euclidean algorithm.

% TODO: We could maybe rewrite this as peeling of the first coefficient using the Gauss transform
Instead, we take a different approach.
We use the real division with remainder operation from
Example~\vref{ex:real-divmod}, but between iterations we rescale the input.
\iffalse
The idea is to look at the ratios between the two inputs of the Euclidean
algorithm.
Suppose $a = qb + r$.
Then, the quotient is the integer part of the ratio $a/b$ and $r/b$ is the
fractional part of $a/b$.
In the next iteration,
we could calculate $b = q'r + r'$
\fi
For an irrational number $x₀ ∈ ℝ$,
we begin the algorithm with the input $(x₀, 1)$ and
we split $x$ into its integer and fractional part according to the division
operation,
i.e.
\[
  x₀ = \floor{x₀} · 1 + \{x₀\}
\]
Then, the quotient is $\floor{x₀}$ and the remainder is $\{x₀\}$.
Normally, we would continue with $(1, \{x\})$,
but instead we multiply each number by $1/\{x\}$ such that
we have the pair $(x₁, 1) = (1/\{x\}, 1)$, where the second number is $1$ again.
Thus, the quotient in the next iteration is the integer part again
and the remainder is the fractional part.
We repeat this with
\[
  x_{n+1} = \frac{1}{\{x_n\}}
\]
for each $n ≥ 0$ until $x_n = 0$.
At each step, we keep track of the integer part $aₙ = \floor{xₙ}$ and store it
as one coefficient in the continued fraction $[a₀; a₁, …]$.

This transformation works, because it removes the first coefficient of $[a₀; a₁, …]$,
and applying it $n$ times removes $n$ coefficients.
Suppose we already know that the continued fraction of $x$ is $[a₀; a₁, …]$.
Its fractional part is entirely determined by $[a₁; a₂, …]$,
since $a₁$ is a positive integer and $1/(a₁ + y)$ is less than $1$ for any positive $y$.
Thus, $a₀$ determines the integer part of the continued fraction.
If we subtract $a₀$ from $x$ and then calculate its inverse, then
\[
  \frac{1}{[a₀; a₁, …] - a₀} = \frac{1}{[0; a₁, …]} = \frac{1}{\frac{1}{[a₁; a₂, …]}} = [a₁; a₂, …].
\]
Thus, we map $x₀ = [a₀; a₁, …]$ to $x₁ = [a₁; a₂, …]$.
If we repeat this $n$ times, then we remove $n$ coefficients from $x$ such that we have $xₙ = [aₙ; a_{n+1}, …]$.
The values $xₙ$ are also called \emph{complete quotients} of $x$,
since $xₙ = [a_n; a_{n+1}, …]$.

\begin{example}
  \label{ex:cf-rat}
  Consider $x = 13/5$, again.
  The new algorithm computes
  \begin{align*}
    \frac{13}{5} = 2 + \frac{3}{5}
    \quad \Rightarrow \quad
    \frac{5}{3} = 1 + \frac{2}{3}
    \quad \Rightarrow \quad
    \frac{3}{2} = 1 + \frac{1}{2}
    \quad \Rightarrow \quad
    \frac{2}{1} = 2 + 0.
  \end{align*}
  Thus, we get the same continued fraction $[2; 1, 1, 2]$ for $13/5$
  as in Example~\ref{ex:euclidean-cf}.
\end{example}

\begin{example}
  \label{ex:cf-irrat}
  Consider $x = \sqrt{2}$, which has an integer part of $1$.
  The first iteration of the algorithm results in
  \[
    \sqrt{2} = 1 + (\sqrt{2} - 1).
  \]
  The inverse of $\sqrt{2} - 1$ is $\sqrt{2} + 1$,
  since $(\sqrt{2} - 1)(\sqrt{2} + 1) = 2 - 1 = 1$.
  Thus, the next input pair is $(\sqrt{2} + 1, 1)$ and the algorithm computes
  \[
    \sqrt{2} = 2 + (\sqrt{2} - 1).
  \]
  After this step,
  the algorithm cycles.
  Therefore, the continued fraction for $\sqrt{2}$ is $[1; 2, 2, …]$.
\end{example}

% TODO: Explain that we need to show that it converges
The remaining step is to show that the constructed continued fraction of a
number $x ∈ ℝ$ actually produces the correct representation.
By definition, $x = [a₀; a₁, …]$ if and only if
\[
  x = \lim_{n → ∞} [a₀; a₁, …, aₙ]
\]
and the limit exists.
In other words, the convergents must approach the number $x$ as $n$ increases.
Thus, the next step is proving that convergents approach the number $x$.
Lemma~\ref{lem:cf-det} already hints at the convergence,
because dividing the equation by $q_n q_{n-1}$ results in
\[
  \frac{p_n}{q_n} - \frac{p_{n-1}}{q_{n-1}} = \frac{(-1)^{n+1}}{q_n q_{n-1}}.
\]
Since $aₙ > 0$ for $n ≥ 1$, the denominators $q_n$ and $q_{n-1}$ are always
increasing after some point.
Consequently, the distance between consecutive convergents is decreasing.
But we need to show that the convergents actually approach the number $x$.
For the proof, we use the fact that the algorithm always guarantees $xₙ - aₙ < 1$.

\begin{lemma}
  \label{lem:cf-approx}
  Let $x ∈ ℝ$ and let $[a₀; a₁, …]$ be its continued fraction expansion.
  If $pₙ/qₙ$ is the $n$-th convergent, then
  \[
    \left| x - \frac{pₙ}{qₙ} \right| < \frac{1}{qₙ^2}.
  \]
\end{lemma}

% TODO: Have we proven that (p_{n-1} q_n - p_n q_{n-1}) = (-1)^n yet?
\begin{proof}
  Let $x_n = [a_n; a_{n+1}, …]$.
  By Lemma~\ref{lem:cf-nesting}, $x = [a₀; a₁, …, a_{n-1}, x_n]$ and $a_n = \floor{x_n}$.
  Using the linear recurrence from Lemma~\ref{lem:cf-wallis},
  we can represent both $x$ as well as the convergent $p_n/q_n$ using
  $p_{n-1}/q_{n-1}$ and $p_{n-2}/q_{n-2}$ together with the coefficients $x_n$
  and $a_n$, respectively.
  Hence,
  \begin{align*}
    \left| x - \frac{pₙ}{qₙ} \right|
    & = \left| \frac{x_n p_{n-1} + p_{n-2}}{x_n q_{n-1} + q_{n-2}} - \frac{a_n p_{n-1} + p_{n-2}}{a_n q_{n-1} + q_{n-2}} \right| \\
    & = \left| \frac{(x_n p_{n-1} + p_{n-2})(a_n q_{n-1} + q_{n-2}) - (x_n q_{n-1} + q_{n-2})(a_n p_{n-1} + p_{n-2})}{(x_n q_{n-1} + q_{n-2})(a_n q_{n-1} + q_{n-2})} \right| \\
    & = \left| \frac{(p_{n-1} q_{n-2} - q_{n-1} p_{n-2})(x_n - a_n)}{((x_n - a_n) q_{n-1} + q_n) q_n} \right|.
  \end{align*}
  From Lemma~\ref{lem:cf-det} it follows that
  \begin{align*}
    \left| x - \frac{pₙ}{qₙ} \right|
    & = \Biggl| \frac{(-1)^{n+1} \overbrace{(x_n - a_n)}^{≤ 1}}{q_n^2 + \underbrace{(x_n - a_n)}_{≥ 0} q_{n-1} q_n} \Biggr| < \frac{1}{q_n^2}. \qedhere
  \end{align*}
\end{proof}

% TODO: Mention that we can also prove the other direction
We now have a continued fraction for both rational number and irrational numbers.
What remains to be shown is that the continued fractions are unique.

\begin{theorem}
  \label{thm:irrat-cf}
  Every real number $x$ has a unique continued fraction.
\end{theorem}

\begin{proof}
  From previous considerations, it follows that for every number $x ∈ ℝ$,
  there exists a continued fraction $[a₀; a₁, …]$ such that $[a₀; a₁, …] = x$.
  Now, suppose there is a different continued fraction $[b₀; b₁, …]$ with $[b₀; b₁, …] = x$.
  Because $[0; a₁, a₂, …]$ and $[0; b₁, b₂, …]$ both lie between $0$ and $1$,
  the continued fractions must share the same first coefficient $a₀ = b₀$.
  Otherwise, they would not have the same integer part and represent different numbers.
  By induction, suppose that the first $n ≥ 0$ terms are equal.
  Then, the continued fractions $[a_{n+1}; a_{n+2}, …]$ and $[b_{n+1}; b_{n+2}, …]$ must be equal.
  But by the same argument, we have $a_{n+1} = b_{n+1}$.
  Therefore, the continued fraction $[a₀; a₁, …]$ for $x$ is unique.
\end{proof}

% ==============================================================================
\section{Geometrical Interpretation Based on Klein Polygons}
\label{sec:cf-geometry}
% ==============================================================================

The geometry of continued fraction was first analyzed by Klein \cite{Klein95}.
He viewed the convergents $p_n/q_n$ of a continued fraction $[a₀; a₁, …]$ not
as rational numbers on a one-dimensional number line,
but as a convergent \emph{vector} $bₙ = (pₙ, qₙ)$ in the two-dimensional integer lattice $ℤ^2$.
Klein has shown many geometrical analogues of important theorems for continued fraction
using these vectors.
However, we will focus on a small subset, which will be useful for the
continued fractions of quadratic irrationals.

\begin{figure}[tb]
  \centering
  \includestandalone{figures/klein-polygon}
  \caption{
    The convergents $pₙ/qₙ$ of $\sqrt{2}$ as vectors $bₙ = (pₙ, qₙ)^⊤$.
    The even and odd convergents form two different polygonal chains which
    approach the irrational line given by $x/y = \sqrt{2}$.
  }
  \label{fig:klein-polygon}
\end{figure}

As an example, Figure~\ref{fig:klein-polygon} shows the convergent vectors of $\sqrt{2}$.
The figure clearly shows that there are two distinct polygonal chains:
One formed by the even convergents and the other formed by the odd convergents.
These chains both appear to approach a line where $x/y = \sqrt{2}$.
The phenomenon can be explained by the convergence as proven in Lemma~\ref{lem:cf-approx}.
If each convergent $(pₙ, qₙ)$ is scaled down by its second coordinate,
then the resulting vector is $(pₙ/qₙ, 1)$.
This vector lies within a distance of at most $1/qₙ^2$ from the vector $(\sqrt{2}, 1)$.
Consequently, scaling the vector back up results in a maximum distance of $1/qₙ$,
which decreases as $n$ increases.
Thus, convergence in this geometrical interpretation means that the vectors are
approaching a line spanned by the vector $(x, 1)$.

Convergence is one property that can be interpreted geometrically.
The definition of continued fractions itself admits a geometrical interpretation.
Suppose we have the vector $\tilde b_{n-1} = (\tilde p_{n-1}, \tilde q_{n-1})^⊤$
for the continued fraction $[a₁; a₂, …, aₙ]$.
Then,
\[
  \frac{pₙ}{qₙ}
  = \frac{a₀ \tilde p_{n-1} + \tilde q_{n-1}}{\tilde p_{n-1}}
  \iff
  \binom{pₙ}{qₙ}
  = \binom{a₀ \tilde p_{n-1} + \tilde q_{n-1}}{\tilde p_{n-1}}
\]
What is more important is that this operation is linear.
We can represent it as the product of two matrices $R$ and $S$.
The first matrix $R$ calculates the reciprocal by swapping the two coordinates.
The second matrix $S$ skews the vector by its second coordinate.
These matrices are defined as follows:
\[
  S^a =
  \begin{pmatrix}
    1 & 1 \\
    0 & 1 \\
  \end{pmatrix}^a
  =
  \begin{pmatrix}
    1 & a \\
    0 & 1 \\
  \end{pmatrix},
  \quad
  R =
  \begin{pmatrix}
    0 & 1 \\
    1 & 0 \\
  \end{pmatrix}.
\]
The vector $b_n$ can then be calculated according to
\[
  \binom{pₙ}{qₙ}
  = S^{a₀} R \binom{\tilde p_{n-1}}{q_{n-1}}
  = S^{a₀} \binom{\tilde q_{n-1}}{\tilde p_{n-1}}
  = \binom{a₀ \tilde p_{n-1} + \tilde q_{n-1}}{\tilde p_{n-1}}.
\]

For the continued fractions, we quickly moved from the definition to Lemma~\ref{lem:cf-wallis},
which tells us how to calculate the convergent $pₙ/qₙ$ using $p_{n-1}/q_{n-1}$ and $p_{n-2}/q_{n-2}$.
Naturally, we can extend this to our geometrical interpretation, too.
In fact, it is just a linear operation:
\begin{align*}
  b_n =
  \begin{pmatrix}
    p_n \\ q_n
  \end{pmatrix}
  =
  \begin{pmatrix}
    p_{n-1} a_n + p_{n-2} \\ q_{n-1} a_n + q_{n-2}
  \end{pmatrix}
  =
  \begin{pmatrix}
    p_{n-1} \\ q_{n-1}
  \end{pmatrix}
  a_n
  +
  \begin{pmatrix}
    p_{n-2} \\ q_{n-2}
  \end{pmatrix}
  = b_{n-1} a_n + b_{n-2}.
\end{align*}

% TODO: Can we add a figure on the side here
The geometry also explains the choice of the coefficient $aₙ$.
If we replace $aₙ$ in the equation with a variable $t ≥ 0$,
then $b_{n-1} t + b_{n-2}$ forms a ray starting at $b_{n-2}$ and going in the
direction of $b_{n-1}$.
This ray must intersect the line generated by $(x, 1)$ at some point.
In fact, it intersects the line at $t = xₙ$, since
\[
  λ
  \begin{pmatrix}
    x \\
    1 \\
  \end{pmatrix}
  =
  b_{n-1} x_n + b_{n-2}
  \iff
  \frac{λ x}{λ} = x = \frac{p_{n-1} x_n + p_{n-2}}{q_{n-1} x_n + q_{n-2}}.
\]
Since $aₙ = \floor{xₙ}$, we choose $aₙ$ as the last integer point before the intersection point.

For the first new lemma, we return to what we saw in
Figure~\ref{fig:klein-polygon} and show that even and odd convergent lie on
different sides of the line spanned by $(x, 1)$.
In terms of rational numbers,
this means that the convergents alternate between being less than $x$ and being
greater than $x$.

\begin{lemma}
  \label{lem:klein-conv}
  Even and odd convergent lie on different sides of the line spanned by $(x, 1)$.
\end{lemma}

\begin{proof}
  We proceed via induction on $n$.
  For the first two vectors $b_{-2}$ and $b_{-1}$, this is obviously true.
  Now suppose, that $b_{n-1}$ and $b_{n-2}$ lie on different sides.
  The next vector $b_n$ can be calculated according to
  \[
    b_n = b_{n-1} a_n + b_{n-2}.
  \]
  Furthermore, we can find an intersection with the line spanned $(x, 1)$ using
  the complete quotient $xₙ$ of $x$:
  \[
    λ
    \begin{pmatrix}
      x \\
      1 \\
    \end{pmatrix}
    = b_{n-1} xₙ + b_{n-2}.
  \]
  By the induction hypothesis, the vector $b_{n-2}$ is on a different side than $b_{n-1}$.
  However, $a_n$ is smaller than $x_n$, so $b_n = b_{n-1} a_n + b_{n-2}$ never crosses the line.
  Therefore, $b_n$ is on the same side as $b_{n-2}$.
\end{proof}

If we combine the previous two vectors in a matrix $B_n = \begin{pmatrix}
  b_{n-1} & b_{n-2} \\
\end{pmatrix}$,
then we can calculate the next matrix by multiplying with the matrices $S$ and $R$, since
\[
  B_n S^{a_n} R =
  \begin{pmatrix}
    p_{n-1} & p_{n-2} \\
    q_{n-1} & q_{n-2} \\
  \end{pmatrix}
  \begin{pmatrix}
    a_n & 1 \\
    1   & 0 \\
  \end{pmatrix}
  =
  \begin{pmatrix}
    p_{n-1} a_n + p_{n-2} & p_{n-1} \\
    q_{n-1} a_n + q_{n-2} & q_{n-1} \\
  \end{pmatrix}
  =
  B_n.
\]
Because $\det(S^a) = \det(S)^a = 1$ and $\det(R) = -1$, it is straightforward to
see that $B_n$ always has determinant $±1$.
Geometrically, this means that there is no integer point inside the
parallelogram spanned by $b_{n-1}$ and $b_{n-2}$.
This is the geometrical analogue of Lemma~\ref{lem:cf-det},
because $\det(B_n) = p_{n-1} q_{n-2} - p_{n-2} q_{n-1}$.

\begin{lemma}
  \label{lem:klein-empty}
  Between the polygonal chains of the even and odd convergent vectors,
  there exists no nonzero integer point.
\end{lemma}

\begin{proof}
  We can cover the area between the chains
  using parallelograms made up of the points $b_n$, $b_{n-1}, b_{n-2}$, and a suitable fourth point.
  Each parallelogram contains exactly
  \begin{align*}
    \left|\det\begin{pmatrix}
      b_n - b_{n-2} & b_{n-1} - b_{n-2}
    \end{pmatrix}\right|
    & = \left|\det\begin{pmatrix}
      a_n b_{n-1} & b_{n-1} - b_{n-2}
    \end{pmatrix}\right| \\
    & = \left|\det\begin{pmatrix}
      a_n b_{n-1} & -b_{n-2}
    \end{pmatrix}\right| \\
    & = a_n
  \end{align*}
  integer points.
  However, these points all lie on the line between $b_n$ and $b_{n-2}$.
  Therefore, the area between the chains cannot contain any integer point.
\end{proof}

\begin{definition}
  \label{def:klein-polygon}
  Let $v₁, v₂ ∈ ℝ^2$ be two linearly independent vectors
  and let $C$ be the cone in $ℝ^2$ spanned by the vectors $v₁, v₂$, i.e.
  \[
    C = \{\, λ₁ v₁ + λ₂ v₂ \mid λ₁, λ₂ ≥ 0 \,\}.
  \]
  The \emph{Klein polygon} $K$ spanned by $v₁, v₂$ is defined as the convex hull
  of all nonzero integer points inside the cone, i.e.
  \[
    K = \mathrm{conv}(C ∩ ℤ^2 \setminus \{(0, 0)\}).
  \]
  The \emph{boundary} of a Klein polygon is denoted as $Π(K)$.
\end{definition}

The even and odd convergent vectors are part of two such Klein polygons.
The even convergents are part of the Klein polygon $K_0$ spanned by $(0, 1)$ and $(x, 1)$,
and the odd convergents are part of the Klein polygon $K_1$ spanned by $(x, 1)$ and $(1, 0)$.
This is shown in the following theorem.

\begin{theorem}
  The Klein polygons $K_0$ and $K_1$ are equal to the
  convex hull of the even and odd convergents, respectively.
\end{theorem}

\begin{proof}
  Suppose there is a point in the Klein polygons,
  which is not in the convex hull of either the even or odd convergents.
  This means that the point must lie between the two polygonal chains,
  which is impossible by Lemma~\ref{lem:klein-empty}.
  Now suppose there is a point in the convex hull of the convergents,
  but not in one of the Klein polygons.
  Then, there must be one convergent outside the Klein polygon,
  which is closer to the line $(x, 1)$.
  However, this integer point is still inside the cone,
  so it is also contained in the Klein polygon.
\end{proof}

\begin{corollary}
  The vertices of $K_0$ and $K_1$ are equal to the
  even and odd convergents, respectively.
\end{corollary}

% ==============================================================================
\section{Quadratic Irrationals and Periodicity}
\label{sec:cf-quadratic}
% ==============================================================================

% TODO: Write introduction for this section
This section is about the second part of Hermite's question for quadratic
irrationals, i.e. whether the continued fraction of a number is periodic if and only
if the number is a quadratic irrational.
Formally, we call a continued fraction $[a₀; a₁, …]$ \emph{periodic}
if there exists a starting index $k₀ ≥ 0$ and a period $ℓ ≥ 1$ such that $aₖ = a_{k+ℓ}$ for every $k ≥ k₀$.
A continued fraction is called \emph{purely periodic} if $k₀ = 0$,
i.e. the period starts immediately.
For a periodic continued fraction starting at $K$ with length $ℓ$,
we will denote it as $[a₀; a₁, …, a_{k₀-1}, \overline{a_{k₀}, …, a_{k₀+ℓ}}]$.
This is similar to how in decimal notation, we denote a period with a bar over the digits,
e.g. $1/3 = 0.\overline{3}$.
In a continued fraction, we similarly denote the period with a line over the
coefficients that are infinitely repeated.
For example, $\sqrt{2} = [1; \overline{2}]$.

For the proof, we have to show two directions.
The first is that every periodic continued fraction is a quadratic irrational
and the second is that every quadratic irrational has a periodic continued fraction.
We begin with the former and we will use the geometry from the previous section
in the proof.

\begin{theorem}
  If the continued fraction of $x ∈ ℝ$ is periodic, then $x$ is a quadratic irrational.
\end{theorem}

\begin{proof}
  Let $x$ be a purely periodic continued fraction $[a₀; a₁, …]$.
  Then, there exists a complete quotient such that $x = x_ℓ = [a_ℓ; a_{ℓ+1}, …]$ for some $ℓ ≥ 1$.
  For each pair of convergent vectors $b_{n-1}, b_{n-2}$, we can use the
  complete quotient $x_ℓ$ to find an intersection point with the line $(x, 1)$
  and since $x_ℓ = x$, we can also use $x$ itself.
  Stated differently, there exists a scalar $λ ∈ ℝ$ such that
  \[
    B_ℓ
    \begin{pmatrix}
      x \\
      1 \\
    \end{pmatrix}
    =
    b_{ℓ-1} x + b_{ℓ-2}
    = λ
    \begin{pmatrix}
      x \\
      1 \\
    \end{pmatrix}
  \]
  Therefore, $(x, 1)^⊤$ is an eigenvector of the matrix $B_ℓ$
  and $λ$ is the eigenvalue.
  Because $B$ is a 2-by-2 matrix,
  the eigenvalue can only be a quadratic irrational.
  The eigenvector is a solution to the linear system $(B_ℓ - λ I) (x, 1)^⊤ = 0$,
  where the coefficients come from the field $ℚ(λ)$.
  Therefore, $x$ must be a quadratic irrational.

  We proceed with the eventually-periodic case.
  Suppose $x$ is periodic only after some index $k ≥ 0$,
  then $xₖ = [a_k; a_{k+1}, …]$ is a quadratic irrational.
  By Lemma~\ref{lem:cf-wallis},
  \[
    x = \frac{p_{k-1} x_k + p_{k-2}}{q_{k-1} x_k + q_{k-2}}.
  \]
  Because $x$ is a rational expression of $x_k$ with integer coefficients,
  it lives in the same field as $x_k$.
  Therefore, $x$ is a quadratic irrational, too.
\end{proof}

% TODO: Figure for shift of Klein polygon
The converse was originally proven by Lagrange \cite{Lagrange70}.
Here, a proof by Korkina \cite{Korkina96} is presented,
which uses the geometrical interpretation of Klein to show periodicity.
The idea is that for quadratic irrationals,
there always exists a unimodular matrix $U$ which shifts the corner points of a
Klein polygon along its boundary and preserves volume.
Because the volume between $b_n$ and $b_{n-2}$ directly corresponds to
the coefficients $a_0, a_1, a_2, …$ of the continued fraction,
they must repeat at some point.

\begin{figure}[tb]
  \centering
  \includestandalone{figures/full-klein-polygon}
  \caption{
    The four Klein polygons spanned by $±(\sqrt{2}, 1)$ and $±(-\sqrt{2}, 1)$.
  }
  \label{fig:full-klein-polygon}
\end{figure}

For the proof, we not only consider the Klein polygon in the positive quadrant,
but in all four quadrants.
Instead of the vectors $(1, 0)$ and $(0, 1)$, we use the conjugate
$\overline{x}$ and the vector $(\overline{x}, 1)$.
There are now four different Klein polygons each spanned by $±(x, 1)$
and $±(-\overline{x}, 1)$.
An example is shown in Figure~\ref{fig:full-klein-polygon}.
The idea behind the proof is that the matrix $U$ does not change any of the four
Klein polygons, but it is not the identity and therefore it must move the
points along the boundary.

\begin{theorem}
  If $x$ is a quadratic irrational,
  then its continued fraction is periodic.
\end{theorem}

\begin{proof}
  Let $\overline{x}$ denote the conjugate of $x$.
  By Theorem~\ref{thm:unimodular-algebraic}, we can find a unimodular matrix $U$ for $x$,
  which has $(x, 1)^⊤$ and $(\overline{x}, 1)^⊤$ as eigenvectors.
  Applying $U$ on the Klein polygon, we have
  \begin{align*}
    UK
    = U \mathrm{conv}(C ∩ ℤ^2 \setminus \{(0, 0)\})
    = \mathrm{conv}(UC ∩ Uℤ^2 \setminus \{(0, 0)\}).
  \end{align*}
  The matrix $U$ is unimodular, so by Lemma~\ref{lem:unimodular} we have $Uℤ^2 = ℤ^2$.
  Let $v ∈ C$.
  By definition of $C$,
  there exist coefficients $c₁, c₂ ≥ 0$ such that $v = c₁ (x, 1)^⊤ + c₂ (\overline{x}, 1)^⊤$.
  Because the vectors of $C$ are eigenvectors of $U$, we have
  \begin{align*}
    U v
    = c₁
    U\begin{pmatrix}
      x \\
      1 \\
    \end{pmatrix}
    +
    c₂ U\begin{pmatrix}
      x \\
      1 \\
    \end{pmatrix}
    = c₁ λ₁ \begin{pmatrix}
      x \\
      1 \\
    \end{pmatrix}
    + c₂ λ₂ \begin{pmatrix}
      x \\
      1 \\
    \end{pmatrix}
  \end{align*}
  Furthermore, the eigenvalues must be positive.
  If not, then we simply apply $U$ twice.
  Thus, $UC = C$ and $K$ must be invariant under $U$.

  Similarly, the boundary $Π(K)$ must also be invariant under this transformation.
  Because $U$ is a linear transformation which preserves the orientation of
  vectors and is not the identity, it must shift the integer points along the
  boundary.
  We assume that it shifts it in the positive direction,
  i.e. $B_{n+k} = U B_n$ for some $k ≥ 1$.
  If not, then we can choose $U^{-1}$ to shift them in the positive direction.
  Because $\det(U) = 1$, the matrix also preserves volume and
  \[
    a_{n+k}
    = \det\begin{pmatrix}
      b_{n+k} & b_{n+k-2}
    \end{pmatrix}
    = \det(U) \det\begin{pmatrix}
      b_n & b_{n-2}
    \end{pmatrix}
    = a_n.
  \]
  Hence, the continued fraction is periodic after some point.
\end{proof}

\begin{example}
  Consider the continued fraction $[1; \overline{2}]$ of $\sqrt{2}$.
  Even if we only know the first few terms of the continued fraction, i.e.
  $\sqrt{2} ≈ [1; 2, 2]$, we can still deduce that it must be periodic.
  We use the matrix representation to calculate the third convergent
  \[
    B_3 = B_2 · SSR = B_1 · SSR · SSR = SR · SSR · SSR = SRS · SRS · SR
  \]
  The first six matrices in this product are $U = SRSSRS = (SRS)^2$
  and
  \[
    U = (SRS)^2 =
    \left(
    \begin{pmatrix}
      1 & 1 \\
      0 & 1 \\
    \end{pmatrix}
    ·
    \begin{pmatrix}
      1 & 0 \\
      0 & 1 \\
    \end{pmatrix}
    ·
    \begin{pmatrix}
      1 & 1 \\
      0 & 1 \\
    \end{pmatrix}
    \right)^2
    =
    \begin{pmatrix}
      1 & 2 \\
      1 & 1 \\
    \end{pmatrix}^2
    =
    \begin{pmatrix}
      3 & 4 \\
      2 & 3 \\
    \end{pmatrix}
    .
  \]
  This matrix is the mutliplication matrix for $3 + 2\sqrt{2}$.
  In Example~\vref{ex:sqrt2-unit},
  we have already seen that this matrix has $(±\sqrt{2}, 1)$ as eigenvectors.
  Therefore, if $B_2$ and $B_3$ are convergent matrices,
  then so are $U^n B_2$ and $U^n B_3$ for all $n ≥ 1$.
  Thus, the continued fraction of $\sqrt{2}$ is periodic.
\end{example}

\chapter{The Generalized Euclidean Algorithm}

In this chapter, we look at the generalized version of the Euclidean algorithm \cite{Klein24}.
While the original algorithm works on numbers,
the generalized version works with vectors.
More specifically, the generalized version works on lattices.
For this chapter, we proceed analogously to the Euclidean algorithm.
First, we look at how the generalized algorithm works and then use it to find a
higher-dimensional analogue to the Fibonacci numbers and the golden ratio.
Using the golden ratio, we can naturally extend this generalized algorithm to
the real numbers, just like the original single-dimensional algorithm.

\section{Basics of Lattice Theory}

\begin{figure}[b]
  \centering
  \includestandalone{figures/lattice}
  \caption{A two-dimensional lattice with vectors $B_1 = (2, 1)$ and $B_2 = (1, 3)$.}
\end{figure}

\begin{itemize}
  \item Vector space as the linear combination over a basis
  \item Lattices as an integral combination over a basis
\end{itemize}

\begin{definition}
  Given a basis $B ∈ ℤ^{d × n}$, the \emph{lattice} over the basis $B$ is defined as
  \[
    \mathcal{L}(B) = \left\{\, B₁z₁ + \dots + B_n z_n \mid z_1, \dots, z_n ∈ ℤ^d \,\right\}.
  \]
  The \emph{rank} of $\mathcal{L}(B)$ is $n$ and its \emph{dimension} is $d$.
  If $n = d$, then $\mathcal{L}(B)$ is a \emph{full rank} lattice.
\end{definition}

\begin{problem}[Lattice Basis Reduction]~
  \begin{itemize}
    \item \textbf{Input}: A matrix $A ∈ ℤ^{d × n}$ with $\text{rank}(A) = d$.
    \item \textbf{Output}: A matrix $B ∈ ℤ^{d × d}$ with $\mathcal{L}(B) = \mathcal{L}(A)$.
  \end{itemize}
\end{problem}

In this thesis, I only consider the case for one additional vector, i.e. $n = d + 1$.

% TODO: Example for an over-defined basis and what the reduced basis is.
\begin{example}
  Consider $A = \begin{pmatrix}
    2 & 1 & 3 \\
    1 & 3 & 4 \\
  \end{pmatrix}$.
  The matrix $B = \begin{pmatrix}
    2 & 1 \\
    1 & 3 \\
  \end{pmatrix}$
  spans the same lattice,
  since $A_3 = A_1 + A_2$.
  Therefore, $B$ would is the reduced basis of $\mathcal L(A)$.
\end{example}

% TODO: Another example which shows that you can't just take a submatrix of the
% original matrix.

\begin{definition}
  The \emph{fundamental parallelepiped} of a lattice $\mathcal{L}(B)$ with $B ∈ ℤ^{d × n}$ is defined as
  \[
    Π(B) = \left\{\, B₁ x₁ + \dots + B_n x_n \mid x_1, \dots, x_n ∈ [0, 1) \,\right\}
  \]
\end{definition}

A useful fact about the fundamental parallelepiped of a lattice $\mathcal L(B)$ is that
if $B$ is a square integer matrix,
then the volume of the parallelepiped $Π(B)$ and
the number of integer points $ℤ^n$ contained in $Π(B)$ is determined by $\mathrm{det}(B)$,
i.e.
\[
  \mathrm{vol}(Π(B)) = |Π(B) ∩ ℤ^n| = |\det(B)|.
\]

\section{Description of the Algorithm}

In the previous example,
we saw that we could represent the last column vector as an integral
combination of the previous two,
which allows us to reduce the basis for the lattice to only those two column vectors.
However, in general it is not as easy as this.
Consider the matrix $A = ?$.
In this case, $A_3 = ? + ?$, which is clearly not an integral combination.
So $A' = ?$ does not span the same lattice as $A$.

Each point $x ∈ ℝ^d$ can be represented as a combination of a lattice point $z
∈ \mathcal{L}(B)$ and a point in the fundamental parallelepiped $r ∈ Π(B)$.
Specifically,
\[
  x = z + r = B\floor{x} + B\{x\}.
\]

\[
  x \pmod{Π(B)} := x - B\floor{x}.
\]

\begin{Pseudocode}[float=tb,caption={The Generalized Euclidean Algorithm \cite{Klein24}.}]
solve $Bx = c$
while $x$ is not integral do
  find $x_ℓ$ which is not integral
  $c ← B_ℓ$
  $B_ℓ ← B\{x\}$
  solve $Bx = c$
end
\end{Pseudocode}

The algorithm requires solving a linear system in each iteration.
However, we do not have to do this in every iteration.
We only have to do this in the first iteration and in the following iterations
we simply update this solution from the old solution.
If $x = (x₁, …, x_d)$ is the solution in the previous iteration,
then $x' = (x₁', …, x_d')$ with
\begin{align*}
  x_i' =
  \begin{cases}
    \frac{1}{\{x_ℓ\}},  & \text{ if } i = ℓ, \\
    -\frac{\{x_i\}}{\{x_ℓ\}} & \text{ otherwise,}
  \end{cases}
\end{align*}
is the solution in the next iteration.
This update rule follows from
\[
  B_ℓ \{x_ℓ\} + \sum_{i ≠ ℓ} B_i \{x_i\} = r
  \iff
  r - \sum_{i ≠ ℓ} B_i \{x_i\} = B_ℓ \{x_ℓ\}
  \iff
  r \frac{1}{\{x_ℓ\}} - \sum_{i ≠ ℓ} B_i \frac{\{x_i\}}{\{x_ℓ\}} = B_ℓ.
\]

% TODO: Should we add a citation for Northshield and explain that continued
% fractions map positive to positive values which seems to be a fundamental
% requirement for the continued fractions to be periodic?

% I think a better wording would be, that the update rule makes the negation
% visible, which is not optimal. The update rule itself doesn't negate the
% variables, even without the update rule we would still have negated
% variables, since the update rule is just an improvement of the original
% algorithm.

Although the update rule speeds up the algorithm considerably, it is not
optimal for the analysis in the following sections.
The rule flips the sign of all elements inside the solution vector in each
iteration.
Instead, I propose a slight modification to the generalized algorithm which
maps each $xᵢ$ to another positive value.
After we replace $B_ℓ$ with $c$, we flip the signs of all vectors $B_i$ with $i ≠ ℓ$.
This leads to the modified update rule, where the values $x_i$ for $i ≠ ℓ$ are
no longer negated:
\begin{align*}
  x_i' =
  \begin{cases}
    \frac{1}{\{x_ℓ\}},  & \text{ if } i = ℓ, \\
    \frac{\{x_i\}}{\{x_ℓ\}} & \text{ otherwise.}
  \end{cases}
\end{align*}
By $\mathrm{pivot}_ℓ(x) = x'$, we denote this modified update rule.
The modified algorithm can be seen in Listing~\ref{lst:modified-generalized-euclidean}.
In the algorithm, first $B_ℓ$ is flipped and then the whole matrix $B$ is flipped,
This is the same as only flipping the vectors $B_i$ for $i ≠ ℓ$.

\begin{Pseudocode}[float=tb, caption={The Modified Algorithm.}, label={lst:modified-generalized-euclidean}]
solve $Bx = c$
while $x$ is not integral do
  find index $ℓ$ for which $x_ℓ$ is not integral
  $c ← B_ℓ$
  $B_ℓ ← -B\{x\}$
  $B ← -B$
  $x ← \mathrm{pivot}_ℓ(x)$
end
\end{Pseudocode}

\begin{lemma}
  The determinant of $B$ decreases by $\{x_ℓ\}$ in each iteration.
\end{lemma}

\begin{theorem}
  The generalized Euclidean algorithm terminates in at most $\det(B)$ steps.
\end{theorem}

\begin{figure}[t]
  \centering
  \includestandalone{figures/pivot-choice}
  \caption{
    Different choices for the remainder of vector $c$. The original algorithm
    always uses $r$ as the remainder, but the modified update rule would also consider $r'$.}
\end{figure}

\section{Higher-Order Fibonacci Numbers and Their Golden Ratios}

We have already seen that the worst-case input for the one-dimensional
Euclidean algorithm are the Fibonacci numbers.
Plugging in two Fibonacci numbers always requires at least n steps for the
algorithm.
This leads to the question whether such numbers also exist in higher dimensions
and what their golden ratios look like.

% I would like a better explanation of what the goal of this section is

The answer to this question does not lead to a single definition of the
Fibonacci numbers per dimension, but many different types of Fibonacci numbers.
The generalized Euclidean algorithm allows for one additional degree of freedom
by choosing which column vector we swap with.
Therefore, there cannot be just one definitive Fibonacci number, there are
different ones depending on the choice of our swap/pivot.

The first strategy one may think of is to choose the index ell with the
smallest fractional value.
After all, this gives us the largest decrease in the determinant over one
iteration.
So locally, this would be the highest decrease we can achieve. But what would a
Fibonacci number look like in this case?

First, we are indeed talking about just numbers and not vectors.
The Euclidean algorithm takes one-dimensional integers as input, so one would
expect that the generalized algorithm would take Fibonacci vectors as input.
But the numbers of steps that the algorithm requires only depends on the
initial solution vector x.
In fact, the entire execution of the algorithm only depends on this vector.
There are infinitely many possible linear systems which have the same solution
vector x, and hence there would be infinitely many "Fibonacci vectors".
Therefore, we look instead only for the numbers inside the rational vector.

% TODO: Actually explain how we derive the Fibonacci numbers for this strategy.

The Fibonacci numbers for this strategy are defined as
\begin{align*}
  F(0) = F(1) = ⋯ = F(d) = 1, \qquad F(n + 1) = F(n) + F(n - 1) + ⋯ + F(n - d).
\end{align*}
We choose our initial input as
\begin{align*}
  B & =
  \begin{pmatrix}
    F(n + 1) & 0        & ⋯ & 0 \\
    0        & F(n + 1) & ⋯ & 0 \\
    ⋮        & ⋮        & ⋱ & ⋮ \\
    0 & 0 & ⋯ & F(n + 1) \\
  \end{pmatrix}, \\
  c & =
  \begin{pmatrix}
    F(n) \\
    F(n) + F(n - 1) \\
    ⋮ \\
    F(n) + F(n - 1) + ⋯ + F(n + 1 - d) \\
  \end{pmatrix},
\end{align*}
which gives us the solution vector
\[
  x =
  \left(
    \frac{F(n)}{F(n + 1)},\;
    \frac{F(n) + F(n - 1)}{F(n + 1)},\;
    ⋯,\;
    \frac{F(n) + F(n - 1) + ⋯ + F(n + 1 - d)}{F(n + 1)}\;
  \right).
\]
The smallest fractional value in this vector is $x₁$, so we pivot with $ℓ = 1$ first.
\begin{align*}
  \{x₁'\}
  & = \left\{\frac{1}{\{x₁\}}\right\}
  = \left\{\frac{F(n + 1)}{F(n)}\right\}
  = \frac{F(n - 1) + ⋯ + F(n - d)}{F(n)}.
\end{align*}
This gives us the value for $x_d$, if we would have started with $F(n)$ instead of $F(n+1)$.
The other values in our input vector $x$ are calculated as follows:
\begin{align*}
  \{xᵢ'\}
  & = \left\{\frac{\{xᵢ\}}{\{x₁\}}\right\} \\
  & = \left\{\frac{F(n) + F(n - 1) + ⋯ + F(n + 1 - i)}{F(n + 1)} · \frac{F(n + 1)}{F(n)}\right\} \\
  & = \frac{F(n - 1) + ⋯ + F(n - d)}{F(n)}.
\end{align*}
So in the next iteration the value $x_i$ has the same value as $x_{i+1}$ if we
would have started with $F(n)$.
Therefore, in the next iteration the smallest fractional value must be $x_2$.
We continue this until we have cycled through all variables.

Another example would be the opposite strategy: Choosing the maximum in each iteration.
In this case, the Fibonacci numbers are defined as
\begin{align*}
  F(0) = \dots = F(d) = 1, \qquad F(n + 1) = F(n) + F(n - d).
\end{align*}

The golden ratios for each definition is the limit of consecutive Fibonacci numbers
\[
  φ = \lim_{n → ∞} \frac{F(n + 1)}{F(n)},
\]
where $F(n)$ can be any of the previously defined Fibonacci numbers.

\section{Extension to Real Numbers}

\chapter{Higher-Order Fibonacci Numbers and their Golden Ratios}
\label{ch:fibonacci}

% TODO: Add reference to Fibonacci numbers in the continued fraction chapter
One of the simplest periodic continued fraction is $[1; \overline{1}]$.
This fraction evaluates to the golden ratio and its convergents are ratios of
consecutive Fibonacci numbers.
Naturally, we can consider the simplest vectors $x ∈ ℝ^d$ for the generalized
Euclidean algorithm, where the integer part in each iteration is $1$.
Such vectors can be seen as a generalization of the golden
ratio to higher dimensions and their approximation can be seen as a
generalization of Fibonacci numbers.
However, there does not exist a single definitive golden ratio,
Since the generalized algorithm allows an additional freedom in the choice of the pivot.
So there are actually multiple different possible definitions.
Some of these are well known, like the supergolden ratio or the plastic ratio,
which are both roots of cubic polynomials.

Since the definition of the golden ratios depends entirely on the choice of our
pivot element $x_ℓ$, we consider two strategies for choosing $x_ℓ$ in this
chapter: First choosing the smallest fractional value $\{x_ℓ\}$ and second
choosing the largest fractional value $\{x_ℓ\}$.
Both of these strategies have a respective golden ratio for which the algorithm
becomes periodic and they each have a corresponding linear recurrence, which
can be seen as a generalization of the Fibonacci numbers.
In the end, we will derive a general definition of higher-order Fibonacci
sequences and show that the generalized Euclidean algorithm is periodic for
their golden ratios.

% ==============================================================================
\section{The Classical Case: Fibonacci Numbers and the Golden Ratio}
% ==============================================================================

We revisit the classical case of Fibonacci numbers and the golden ratio
to understand their relationship with the Euclidean algorithm.
Fibonacci numbers are a recursive sequence.
Starting with the initial conditions $F(0) = 1$ and $F(1) = 1$,
the remaining terms are calculated using the formula
\[
  F(n) = F(n-1) + F(n-2).
\]
The first terms of this sequence are $1, 1, 2, 3, 5, 8, 13, 21, …$

Their relationship with the Euclidean algorithm becomes apparent when analyzing
the runtime of the algorithm.
In fact, the analysis was actually one of the first practical application of
the Fibonacci numbers.
Lamé \cite{Lame44} used the Fibonacci numbers to show that the number of
iterations in the Euclidean algorithm is bounded by the number of digits in the
input.
The proof is considered by some as the beginning of computational
complexity theory.

\begin{theorem}
  \label{thm:lame}
  If the Euclidean algorithm requires $n$ steps for $(a, b) ∈ ℤ_{> 0}^2$ with $a < b$,
  then $a ≥ F(n+2)$ and $b ≥ F(n+1)$.
\end{theorem}

\begin{proof}
  For $n = 0$, we have $F(2) = 2$ and $F(2) = 1$
  and they are the smallest pair of positive integers which require one step of
  the Euclidean algorithm.
  In fact, they are the smallest pair which satisfy $a < b$.
  So any pair, which takes only one step, must satisfy $a ≥ F(2)$ and $b ≥ F(1)$.
  Suppose that the theorem holds for some $n ≥ 0$.
  Let $(a, b)$ be an input which requires $n+1$ steps.
  Then, $a = qb + r$ for some integers $q, r ≥ 1$.
  For the pair $(b, r)$, the algorithm requires $n$ steps, so $b ≥ F(n+2)$ and
  $r ≥ F(n+1)$ by induction.
  But then
  \[
    a = qb + r ≥ F(n+2) + F(n+1) = F(n+3).
  \]
  Therefore, the theorem also holds for $n+1$.
\end{proof}

The theorem also follows easily using the continued fraction of the ratios
between Fibonacci numbers.
Given a finite continued fraction $[a₀; a₁, …, a_n]$, the Euclidean algorithm
requires $n$ steps for any input $(a, b)$ with the same ratio $a/b$ as the
continued fraction.
Since $[1; 1, …, 1] = F(n+1)/F(n)$ for some $n ≥ 0$, the Euclidean algorithm
takes at least $n$ steps for two consecutive Fibonacci numbers.
Furthermore, Fibonacci numbers are the smallest numbers which require $n$ steps
since we can only increase the coefficients of the continued fraction.

The continued fractions also show the connection between the Fibonacci numbers
and the golden ratio.
The convergents of the continued fraction $[1; \overline{1}]$ are ratios
between consecutive Fibonacci numbers $F(n+1)/F(n)$ and as $n$ approaches
infinity the convergents tend towards the golden ratio $φ$.

\iffalse
\begin{example}
  Consider $a = 13$ and $b = 8$.
  The algorithm proceeds as follows:
  \[
    \begin{array}{rclcrcl}
      13/8 & = & 1 & · & 5/8 & + & 3/8 \\
       5/8 & = & 1 & · & 3/8 & + & 2/8 \\
       3/8 & = & 1 & · & 2/8 & + & 1/8 \\
       2/8 & = & 2 & · & 1/8 & + & 0.
    \end{array}
  \]
\end{example}
\fi

\begin{theorem}
  $\lim_{n → ∞} \frac{F(n+1)}{F(n)} = φ$.
\end{theorem}

\begin{proof}
  First, to show that the continued fraction $[1; 1, …, 1]$ actually evaluates
  to the ratio of consecutive Fibonacci numbers, we use the formula from
  Lemma~\ref{lem:cf-wallis}:
  \[
    p_n = a_n p_{n-1} + p_{n-2}, q_n = a_n q_{n-1} + q_{n-2}.
  \]
  Because $a_n = 1$ for every $n ≥ 0$ and $p_{-1} = 1, p_{-2} = 0$,
  the Fibonacci sequence and the sequence $p_n$ are the same.
  This is also true for $q_n$, but $q_{-1} = 0$ and $q_{-2} = 1$, so the sequence lags one step behind $p_n$.
  Therefore, the convergents are actually $F(n+1)/F(n)$.
  The convergence is already given by Lemma~\vref{lem:cf-approx},
  so all that remains to be shown is that they actually converge to the golden ratio $φ$.
  Suppose they converge to some different limit $φ'$
  \[
    φ' = \lim_{n → ∞} \frac{F(n+1)}{F(n)} = \lim_{n → ∞} 1 + \frac{F(n-1)}{F(n)} = 1 + \frac{1}{φ'}.
  \]
  Multiplying this equation by $φ'$ results in the defining polynomial of the golden ratio $φ$.
  Because the ratios are always positive, we must have $φ' = φ$.
\end{proof}

% TODO: Metallic means

\begin{example}
  Let $φ$ be the golden ratio,
  which satisfies $φ^2 - φ - 1 = 0$
  and has an approximate value of $1.618$.
  The algorithm proceeds as follows:
  \[
    \begin{array}{rclcrcl}
      φ & = & 1 & · & 1   & + & φ - 1 \\
         1 & = & 1 & · & (φ - 1) & + & (2 - φ) \\
       φ - 1 & = & 1 & · & (2 - φ) & + & (2φ - 3) \\
       2 - φ & = & 1 & · & (2φ - 3) & + & (5 - 3φ) \\
       & \vdots &
    \end{array}
  \]
\end{example}

In summary,
the defining property of Fibonacci numbers with respect to the Euclidean
algorithm is that they represent the worst-case for the algorithm.
At each step, they have a quotient of $1$, so they decrease slower than any other pair of integers.

% ==============================================================================
\section{Higher-Order Fibonacci Numbers for the Minimum Strategy}
% ==============================================================================

% TODO: There should be a discussion on what it even means to have a Fibonacci number in higher dimensions
% TODO: We should also discuss the thing about idempotence under the choice of the pivot.
% TODO: Should we have a discussion about why we choose this particular linear system.
% TODO: The discussion about the linear system should go in the generalized Euclidean algorithm chapter

For the generalisation of Fibonacci numbers with respect to the generalized
Euclidean algorithm,
we must find a sequence which similarly has a quotient of $1$ at each iteration.
However, the problem is that no single definitive sequence can exist.
With the generalized Euclidean algorithm we have an additional choice with the
element we choose to pivot with.
Therefore, multiple different notions of Fibonacci numbers exist for this
algorithm.

% I would like a better explanation of what the goal of this section is
We begin by deriving the Fibonacci numbers for one particular strategy.
Recall that the determinant decreases by $\{x_ℓ\}$ in each iteration.
So the first strategy one may think of is to choose the index $ℓ$ with the
smallest fractional value.
After all, this gives us the largest decrease in the determinant over one iteration.
Locally, this would be the highest decrease we can achieve.
So we begin with this strategy and see how the Fibonacci numbers are defined.

% TODO: Actually explain how we derive the Fibonacci numbers for this strategy.
% TODO: Should we have a table for the Fibonacci numbers?
% TODO: What about adding references to their OEIS number?

% TODO: Maybe it's better to show the numbers in the ratios, since we explain
% that they approach some golden ratio anyways
\begin{table}[tbp]
  \caption{The first 10 $d$-bonacci numbers with $d = 1, …, 5$ and their golden ratios.}
  \label{tbl:min-fibonacci}
  \centering
  \begin{tabular}{cc|cccccccccc}
\uzlhline
\uzlemph{$d$} & \uzlemph{$φ$} & \uzlemph{$F(0)$} & \uzlemph{$F(1)$} & \uzlemph{$F(2)$} & \uzlemph{$F(3)$} & \uzlemph{$F(4)$} & \uzlemph{$F(5)$} & \uzlemph{$F(6)$} & \uzlemph{$F(7)$} & \uzlemph{$F(8)$} & \uzlemph{$F(9)$} \\
\hline
$1$ & $1.61803$ & 1 & 2 & 3 & 5 & 8 & 13 & 21 & 34 & 55 & 89 \\
$2$ & $1.83929$ & 1 & 1 & 3 & 5 & 9 & 17 & 31 & 57 & 105 & 193 \\
$3$ & $1.92756$ & 1 & 1 & 1 & 4 & 7 & 13 & 25 & 49 & 94 & 181 \\
$4$ & $1.96595$ & 1 & 1 & 1 & 1 & 5 & 9 & 17 & 33 & 65 & 129 \\
$5$ & $1.98358$ & 1 & 1 & 1 & 1 & 1 & 6 & 11 & 21 & 41 & 81 \\
$6$ & $1.99198$ & 1 & 1 & 1 & 1 & 1 & 1 & 7 & 13 & 25 & 49 \\
\uzlhline
\end{tabular}

\end{table}

\begin{definition}
  The Fibonacci numbers for this strategy are defined as
  \begin{enumerate}
    \item $F(0) = F(1) = ⋯ = F(d) = 1$.
    \item $F(n + 1) = F(n) + F(n - 1) + ⋯ + F(n - d)$.
  \end{enumerate}
  For $d = 2, 3$ and $4$, they are also referred to as the Tribonacci, Tetranacci
  and Pentanacci numbers.
  In general, they are known as the $d$-bonacci numbers.
\end{definition}

For these numbers, we use the solution vector
\[
  x =
  \left(
    \frac{F(n)}{F(n + 1)},\;
    \frac{F(n) + F(n - 1)}{F(n + 1)},\;
    ⋯,\;
    \frac{F(n) + F(n - 1) + ⋯ + F(n + 1 - d)}{F(n + 1)}\;
  \right).
\]
The smallest fractional value in this vector is $x₁$, so we pivot with $ℓ = 1$ first.
\begin{align*}
  \{x₁'\}
  & = \left\{\frac{1}{\{x₁\}}\right\}
  = \left\{\frac{F(n + 1)}{F(n)}\right\}
  = \frac{F(n - 1) + ⋯ + F(n - d)}{F(n)}.
\end{align*}
This gives us the value for $x_d$, if we would have started with $F(n)$ instead of $F(n+1)$.
The other values in our input vector $x$ are calculated as follows:
\begin{align*}
  \{xᵢ'\}
  & = \left\{\frac{\{xᵢ\}}{\{x₁\}}\right\} \\
  & = \left\{\frac{F(n) + F(n - 1) + ⋯ + F(n + 1 - i)}{F(n + 1)} · \frac{F(n + 1)}{F(n)}\right\} \\
  & = \frac{F(n - 1) + ⋯ + F(n - d)}{F(n)}.
\end{align*}
So in the next iteration the value $x_i$ has the same value as $x_{i+1}$ if we
would have started with $F(n)$.
Therefore, in the next iteration the smallest fractional value must be $x_2$.
We continue this until we have cycled through all variables.

Another example would be the opposite strategy: Choosing the maximum in each iteration.
In this case, the Fibonacci numbers are defined as
\begin{align*}
  F(0) = \dots = F(d) = 1, \qquad F(n + 1) = F(n) + F(n - d).
\end{align*}

The golden ratios for these numbers is the limit
\[
  φ = \lim_{n → ∞} \frac{F(n + 1)}{F(n)}.
\]

% TODO: There should be a note, that we're taking the actual minimum, so we
% stop once we see any zero. Because at that point, we're cycling.

The following lemma shows that the $\mathrm{pivot}$ operation can also be
viewed as a multi-dimensional division with remainder operation, where we have
a set of integers $p₁, …, p_d$ and divide them by one element $p_ℓ$.

\begin{lemma}
  \label{lem:divmod}
  % TODO: This should basically prove that pivot is just division with
  % remainder
  Let $x ∈ ℚ^d$ and $x' = \mathrm{pivot}_ℓ(x)$ for some index $ℓ$.
  Suppose $x = (p₁/q, p₂/q, …, p_d/q)$ and $x' = (p₁'/q', p₂'/q', …, p_d'/q')$
  with each having at least one irreducible fraction.
  Furthermore, let $x_{(k)} = p_{(k)}/q$ and $x_{(k)}' = p_{(k)}'/q'$ denote
  the $k$-th smallest element in each vector, respectively.
  Then,
  \[
    q ≥ p_{(d)}',
    \qquad
    p_ℓ ≥ a_{(d)} p_{(d)}' + q',
    \qquad
    p_i ≥ a_{(i)} p_{(d)}' + p_{(i)}'
  \]
  for every $i ≤ d - 1$ and $a = \floor{x}$.
\end{lemma}

\begin{proof}
  % TODO: Reference the rule
  By the update rule, we have
  \[
    \frac{p_ℓ'}{q'} = \frac{1}{\frac{p_ℓ}{q} - a_ℓ} = \frac{q}{p_ℓ - a_ℓ q}
  \]
  Therefore, $q = p_ℓ'$ and $p_ℓ = a_ℓ q + q' = a_ℓ p_ℓ' + q'$.
  For the remaining elements $x_i = p_i/q$ with $i ≠ ℓ$, we have
  \[
    \frac{p_i'}{q'} = \frac{\frac{p_i}{q} - a_i}{\frac{p_ℓ}{q} - a_ℓ} = \frac{p_i - a_i q}{p_ℓ - a_ℓ q}.
  \]
  Therefore, $p_i = p_i' + a_i q = p_i' + a_i p_ℓ'$.
\end{proof}

To show that the $d$-bonacci numbers are the worst-case for the minimum
strategy, we first need a few requirements.
The first is that the vector $x$ is strictly positive,
which can be easily solved by ignoring the first iteration.
If $x$ contains any negative values, then applying one step of the algorithm
produces a strictly positive vector.
The second requirement is that the values in $x$ are all different.
The reasoning behind this requirement is that one could easily check if there
are two elements $x_i, x_j$ with $x_i = x_j$.
Pivoting with either $ℓ = i$ or $ℓ = j$ removes the other element from the
vector, since the element will be zero in the next iteration.
With these requirements, we can show that the $d$-bonacci numbers are the worst-case.

\begin{theorem}
  Let $x = (p₁/q, …, p_d/q) ∈ ℚ^d$
  and let $x_{(i)} = p_{(i)}/q$ denote the $i$-th smallest element.
  If the algorithm requires at least $n ≥ 0$ steps for $x$ when taking the
  minimum fractional value at each step, then
  \[
    q ≥ F(n),
    \qquad
    p_{(i)} ≥ \sum_{k = 0}^i F(n - k),
    \text{ for every } i ≤ d.
  \]
\end{theorem}

\begin{proof}
  For $n = d$, we have
  \begin{align*}
    q ≥ F(d) = 1,
    \qquad
    p_{(i)} ≥ \sum_{k=0}^i F(d - k) = i + 1 \text{ for every } i ≤ d.
  \end{align*}
  These bounds are the smallest bounds we can have, since any smaller set of
  integers does not even satisfy $0 < q < p_{σ(1)} < ⋯ < p_{σ(d)}$.
  Furthermore, any other set of numbers requires at least $0$ steps, so
  the theorem holds for $n = d$.

  % TODO: Fix indices
  Suppose the algorithm requires $n$ steps for some input $x' = (p₁'/q', …, p_d'/q')$.
  The particular order of the values in $x'$ does not matter for the strategy.
  Hence, we can assume without loss of generality that $q' < p_2' < p_3' < … < p_d' < p_1'$.
  By induction, $q' ≥ F(n)$ and because $p_1'$ is the largest value,
  \begin{align*}
    p_{σ(1)}' & ≥ \sum_{k=0}^d F(n - k) = F(n + 1).
  \end{align*}
  Similarly, because all other values with $i ≠ 1$ are shifted one place to the
  left, we have
  \begin{align*}
    p_{(i)}' & ≥ \sum_{k=0}^i F(n - k) = \sum_{k=1}^{i+1} F(n + 1 - k) \text{ for } 2 ≤ i ≤ d.
  \end{align*}
  Let $x ∈ ℚ^d$ such that $x' = \mathrm{pivot}_ℓ(x)$ for some index $ℓ$ and $a = \floor{x}$.
  From Lemma~\ref{lem:divmod}, it follows that
  \begin{align*}
    q
    & ≥ p_{(d)}' ≥ F(n+1), \\
    p_{(1)}
    & ≥ a_{(1)} p_{(d)}' + q' ≥ F(n + 1) + F(n) = \sum_{k=0}^1 F(n + 1 - k) \\
    p_{(i)}
    & ≥ a_{(i)} p_{(d)}' + p_{(i)}' ≥ F(n + 1) + ∑_{k=0}^i F(n - k) = ∑_{k=0}^{i+1} F(n + 1 - k).
  \end{align*}
  The permutation $π$ which orders $x$ might not be the same as $σ$.
  However, if two elements are swapped, then we can easily swap the bounds, too.
  Therefore, the theorem also holds for $n+1$.
\end{proof}

% ==============================================================================
\section{Higher-Order Fibonacci Numbers for the Maximum Strategy}
% ==============================================================================

Choose the maximum fractional value, but never the actual maximum value.
The reason behind the latter is that the maximum value is always the previous
pivot element.
Therefore, if we choose the same maximum element, we could end up choosing the
same index over and over again.
Hence, we remove this index from the consideration and only look at the maximum
fractional value without the actual maximum.

If we have a vector $x ∈ ℚ$, then pivoting with this strategy must result in a
vector $x'$, where exactly one value is greater than $1$.
Because $\{x_ℓ\}$ is the maximum, every other element has a smaller fractional
value and $\{x_i\}/\{x_ℓ\}$ is smaller than $1$.
For the analysis, we assume the same for the initial vector $x$.
If $x$ does not satisfy these conditions, then we can iterate once and it has
to satisfy them.

The worst-case for this strategy are called maximum Fibonacci numbers and they
are defined by the following recurrence relation:
\begin{align*}
  F(n) =
  \begin{cases}
    1, & \text{ if } n ≤ 0, \\
    F(n - 1) + F(n - 1 - d), & \text{ if } n > 0.
  \end{cases}
\end{align*}
The first few terms of this sequences for different dimensions are listed in
Table~\ref{tbl:max-fibonacci}.

\begin{table}[tbp]
  \caption{The first 10 Fibonacci numbers for $d = 1, …, 5$ and their respective golden ratio.}
  \label{tbl:max-fibonacci}
  \centering
  \begin{tabular}{cc|cccccccccc}
\uzlhline
\uzlemph{$d$} & \uzlemph{$φ$} & \uzlemph{$F(0)$} & \uzlemph{$F(1)$} & \uzlemph{$F(2)$} & \uzlemph{$F(3)$} & \uzlemph{$F(4)$} & \uzlemph{$F(5)$} & \uzlemph{$F(6)$} & \uzlemph{$F(7)$} & \uzlemph{$F(8)$} & \uzlemph{$F(9)$} \\
\hline
$1$ & $1.61803$ & 1 & 2 & 3 & 5 & 8 & 13 & 21 & 34 & 55 & 89 \\
$2$ & $1.46557$ & 1 & 1 & 2 & 3 & 4 & 6 & 9 & 13 & 19 & 28 \\
$3$ & $1.37971$ & 1 & 1 & 1 & 2 & 3 & 4 & 5 & 7 & 10 & 14 \\
$4$ & $1.32143$ & 1 & 1 & 1 & 1 & 2 & 3 & 4 & 5 & 6 & 8 \\
$5$ & $1.29577$ & 1 & 1 & 1 & 1 & 1 & 2 & 3 & 4 & 5 & 6 \\
$6$ & $1.23256$ & 1 & 1 & 1 & 1 & 1 & 1 & 2 & 3 & 4 & 5 \\
\uzlhline
\end{tabular}

\end{table}

To construct the solution vector $x$ for these numbers, we choose a consecutive
ratio of two Fibonacci numbers for each $x_i$, where the index $i$ determines
the distance between the two.
More specifically, we choose the vector
\begin{align*}
  x = \left(
    \frac{F(n-1)}{F(n)},
    \frac{F(n-2)}{F(n)},
    …,
    \frac{F(n-(d-1)}{F(n)},
    \frac{F(n-d) + F(n)}{F(n)} \right).
\end{align*}

This time, we pivot with the largest fractional value, which is $x_1$.
Let $a = \floor{x}$, then the next vector is $x' = (x₁', …, x_d')$ where
the first element is
\begin{align*}
  x_1'
  = \frac{1}{\left\{\frac{F(n-1)}{F(n)}\right\}}
  = \frac{F(n)}{F(n-1)}
  = \frac{F(n-1) + F(n-1-d)}{F(n)},
\end{align*}
the elements with $i ≤ d - 1$ are
\begin{align*}
  x_i'
  = \frac{\frac{F(n-i)}{F(n)}}{\frac{F(n-1)}{F(n)}}
  = \frac{F(n-i)}{F(n-1)}
\end{align*}
and the last element is
\begin{align*}
  x_d'
  = \frac{\frac{F(n-d) + F(n)}{F(n)} - 1}{\frac{F(n-1)}{F(n)}}
  = \frac{F(n-d)}{F(n-1)}.
\end{align*}

\begin{theorem}
  Suppose the Euclidean algorithm requires at least $n$ steps for $x = (p₁/q, …, p_d/q)$
  with $0 < p₁ < ⋯ < p_{d-1} < q < p_d$ when taking the maximum fractional value at each step.
  Then, there exists a permutation $σ$ such that
  \[
    q ≥ F(n+d),
    \qquad
    p_{σ(d)} ≥ F(n + d) + F(n),
    \qquad
    p_{σ(i)} ≥ F(n + i)
    \quad
    \text{ for every } i ≤ d - 1.
  \]
\end{theorem}

\begin{proof}
  For $n = 0$, we have
  \[
    \left(\frac{F(2)}{F(1)}, \frac{F(3)}{F(1)}, …, \frac{F(d+1)}{F(1)} \right)
    = \left(\frac{2}{1}, \frac{3}{1}, …, \frac{d+1}{1} \right).
  \]
  This is the first positive input, which satisfies the ordering and it requires only one step.
  Therefore, any other input that requires one step must be larger.
  Now suppose that the theorem holds for any $n ≥ 0$.
  Let $x ∈ ℚ^d$ and $a = \floor{x}$.
  Suppose that $x' = \mathrm{pivot}_ℓ(x) = (p₁'/q', …, p_d'/q')$ requires at
  least $n$ steps, where $ℓ$ is chosen according to the strategy.
  By induction, there exists a permutation $σ$ such that
  \begin{align*}
    q'        & ≥ F(n),                     \\
    p_{σ(d)}' & ≥ F(n) + F(n - d) = F(n+1), \\
    p_{σ(i)}' & ≥ F(n - i)
    \text{ for every } i ≤ d - 1.
  \end{align*}
  We assume without loss of generality that $σ(i) = i$ and that it orders the elements in increasing order.
  Because $p_{σ(d)}$ is the largest element in $x'$, we have $ℓ = σ(d) = d$.
  However, $x_ℓ$ cannot be the largest element in $x$ because of the strategy.
  Therefore, $a_ℓ = 0$ and there exists another element $x_k$ with $k ≠ ℓ$ and $a_k ≥ 1$.
  Futhermore, that element must come before $x_ℓ$ since $ℓ = d$.
  We can now bound $x$ using $x'$ and Lemma~\ref{lem:divmod}:
  \begin{align*}
    q   & = p_ℓ' ≥ F(n+1), \\
    p_ℓ & = a_ℓ p_ℓ' + q' = q' ≥ F(n) ≥ F((n+1) - 1), \\
    p_k & = a_k p_ℓ' + p_k' ≥ p_ℓ' + p_k' ≥ F(n + 1) + F(n + 1 - (k + 1)), \\
    p_i & = a_k p_ℓ' + p_i' ≥ p_i' ≥ F(n - i) = F((n+1) - (i+1)).
  \end{align*}
  where $i < ℓ = d$ and $i ≠ k$.
  We're missing the bound for $F(n+1) + F(n+1 - d)$ and $F(n + 1 - k)$.
  The former follows from the bound of $p_k$.
  Because $k < ℓ$, we have
  \begin{align*}
    p_k & ≥ F(n + 1) + F(n + 1 - (k + 1)) ≥ F(n+1) + F(n+1 - d).
  \end{align*}
\end{proof}

% ==============================================================================
\section{Combining both Strategies into the Minimax Strategy}
% ==============================================================================

We choose two indices $k, ℓ$ at once such that they minimize $\{\{x_k\}/\{x_ℓ\}\}$.
We pivot with $k$ first to get $x'$, then with $ℓ$ to get $x''$.
It follows that:

\begin{itemize}
  \item $1/\{\{x_ℓ\}/\{x_k\}\}$ is the largest element in $x''$
  \item $\{\{x_ℓ\}/\{x_k\}\}$ is the smallest fractional value in $x'$
  \item $1/\{x_k\}$ is the largest element in $x'$
\end{itemize}

For this strategy, the Fibonacci numbers are defined as
\begin{align*}
  F_0(n) & = F_1(n-1) + F_1(n-2) + ⋯ + F_1(n-d-1), \\
  F_1(n) & = F_0(n-1) + F_0(n-d-1).
\end{align*}

\begin{theorem}
  If the algorithm requires $n ≥ 0$ steps with the minimax strategy,
  then
  \[
    q ≥ F(),
    \qquad
    p_{(i)} ≥ F().
  \]
\end{theorem}

\begin{proof}
  Suppose the algorithm requires $n$ steps for $x''$.
  By induction,
  \[
    q'' ≥ F(),
    \qquad
    p_{(i)}'' ≥ F().
  \]
  By Lemma~\ref{lem:divmod},
  \begin{align*}
    q' & = p_ℓ'' ≥ ?, \\
    p_ℓ' & = a_ℓ' p_ℓ'' + q'' ≥ q'' ≥ ?, \\
    p_i' & = a_i' p_ℓ'' + p_i'' ≥ p_i'' ≥ ?, \\
    p_k' & = a_k' p_ℓ'' + p_k'' ≥ p_ℓ'' + p_k'' ≥ ?.
  \end{align*}
  Again by Lemma~\ref{lem:divmod},
  \begin{align*}
    q & = p_k' ≥ F(n + 1), \\
    p_ℓ & = a_k' p_k' + q' ≥ p_k' + q' ≥ ?, \\
    p_i & = a_i' p_k' + p_i' ≥ p_k' + p_i' ≥ ?.
  \end{align*}
\end{proof}

\begin{definition}
  A linear recurrence with coefficients~$a_0, \dots, a_d ≥ 0$ is an equation of the
  following form:
  \[
    F(n + 1) = a_0 F(n - d) + a_1 F(n - d + 1) + \dots + a_{d-1} F(n - 1) + a_d F(n).
  \]
\end{definition}

\iffalse % determinant
The golden ratio for this sequence also shows up for a different strategy.
When we choose two indices $ℓ$ and $ℓ'$ such that they minimize the ratio
$\{x_{ℓ'}\}/\{x_{ℓ}\}$, then we can bound the decrease of the determinant over
two iterations using the golden ratio $φ$.

% TODO: I feel like this was using $ψ = φ^{-1}$ instead of $φ$.
\begin{theorem}
  The determinant decreases by at least $φ^{d+1}$ over two iterations with $d ≥ 2$.
\end{theorem}

\begin{proof}
  We assume without loss of generality that the vector is sorted in increasing order and
  $0 ≤ x_i ≤ 1$ for every $i ≤ d$.
  For a contradiction, assume the algorithm yields a smaller decrease than $φ^{d+1}$ on input $x$.
  We must have $x_i > φ^{d+1-i}$ for every $i = 1, \dots, d$.
  For the first value, we have $x₁ > φ^d$, because otherwise we have a total decrease of
  \[
    x₁ \left\{ \frac{x₂}{x₁} \right\} ≤ φ^{d-1} - φ^d = φ^{2d+1} < φ^d.
  \]
  Suppose $x_i > φ^{d+1-i}$ and $x_{i+1} ≤ φ^{d-i}$.
  Then we can achieve a total decrease of
  \[
    x_i · \left\{ \frac{x_{i+1}}{x_i} \right\} ≤ x_{i+1} - x_i < φ^{d-i} - φ^{d+1-i} = φ^{2d+1-i} < φ^d.
  \]
  It follows that $x_i > φ^{d+1-i}$ for every $i ≤ d$.
  But then
  \[
    x_d \left\{ \frac{1}{x_d} \right\} ≤ 1 - x_d < 1 - φ = φ^{d+1}.
  \]
  Hence, we achieve a decrease of at least $φ^{d+1}$ over two iterations.
\end{proof}

The bound for this strategy is tight.
We can construct an input that moves arbitrarily close to this bound.
Formally, for every sufficiently small $ε > 0$, we can find an input $x$ which
achieves a decrease of at most $φ^{d+1} - ε$.
The idea is to choose $x_d$ to be just over $φ$ and all other variables as a multiple of $x_d$
such that we have to choose $x_d$.
But choosing $x_d$ only gives us a decrease of $φ^{d+1} - ε$ in total.

% TODO: Explain what sufficiently small means
\begin{theorem}
  For every (sufficiently small) $ε > 0$,
  there exists an input $x ∈ ℝ^d$ with $d ≥ 2$,
  which achieves a decrease in the determinant of exactly $φ^{d+1} - ε$ over two
  iterations.
\end{theorem}

\begin{proof}
  We choose $x_i = φ^{d+1-i} (1 + ε)$ for $i = 1, \dots, d$.
  The strategy chooses between
  \[
    \left\{ \frac{x_{i+1}}{x_i} \right\}
    = \frac{φ^{d-i} (1 + ε)}{φ^{d+1-i} (1 + ε)} - 1 = \frac{1}{φ} - 1,
    \text{ and }
    \left\{ \frac{1}{x_d} \right\}
    = \frac{1}{φ + ε} - 1
  \]
  Clearly, the strategy chooses $x_d$ since its ratio $\{1/x_d\}$ is the smallest.
  Therefore, the total decrease over two iterations is
  \[
    (φ + ε) \left\{ \frac{1}{φ + ε} \right\} = 1 - φ - ε = φ^{d+1} - ε.
    \qedhere
  \]
\end{proof}
\fi % determinant

% begin periodicity
Given, the linear recurrence $F(n + d + 1) = F(n) + a_1 F(n - 1) + \dots + a_d F(n + d)$,
the value for $x_i^{(n)}$ in the solution $x^{(n)}$ is
\begin{equation}
  \label{eq:general-solution}
  x_i^{(n)} = \frac{F(n - i) + \sum_{k=1}^{i-1} a_{d-k} F(n - k)}{F(n)}.
\end{equation}

\begin{lemma}
  Pivoting with the first element of $x^{(n)}$ using the modified update rule yields the vector
  \[
    x' = (x^{(n-1)}_2, x^{(n-1)}_3, \dots, x^{(n-1)}_d, x^{(n-1)}_1).
  \]
\end{lemma}

\begin{proof}
  For $x_1$, we get
  \[
    \begin{aligned}
      \frac{1}{x_1}
      & = \frac{F(n)}{F(n - 1)} \\
      & = a_d + \frac{F(n - d - 1) + a_1 F(n - d - 2) + \dots + a_{d-1} F(n - 2)}{F(n - 1)} \\
      & = a_d + x^{(n-1)}_d.
    \end{aligned}
  \]
  For the other values, we get
  \begin{align*}
    \frac{x_i}{x_1}
    & = \frac{F(n - i) + \sum_{k=1}^{i-1} a_{d-k} F(n - k)}{F(n)} \frac{F(n)}{F(n - 1)} \\
    & = a_{d-1} + \frac{F(n - i) + \sum_{k=2}^{i-1} a_{d-k} F(n - k)}{F(n-1)} \\
    & = x^{(n-1)}_{i+1} \qedhere
  \end{align*}
\end{proof}

\begin{corollary}
  Running the generalized Euclidean algorithm with $x^{(n)}$ as the solution to
  the linear system $B x = c$ requires $n$ steps, if $x^{(n)}_1$ is the
  smallest element in $x^{(n)}$.
\end{corollary}

Of course, the Euclidean algorithm receives a matrix $B \in \Z^{d \times d}$
and vector $c \in \Z^d$ as its input and not the solution $x$ itself.
However, we can construct a very simple linear system $B^{(n)} x = c^{(n)}$,
where $x^{(n)}$ is the solution, in the following way:
\[
  B^{(n)} = F(n) I_d, \qquad c^{(n)}_k = \sum_{i=0}^k a_{d-i} F(n - i) \text{ for } k ≤ d.
\]
% end periodicity

% ==============================================================================
\section{Multi-Dimensional Golden Ratios and Metallic Means}
% ==============================================================================

% begin convergence
Dividing two consecutive Fibonacci numbers approaches the golden ratio as $n$ increases.
The golden ratio is a solution to the equation $x^2 - x - 1 = 0$.
For higher dimensions, we will encounter similar polynomial equations with higher degree.
The goal of this section is to generalize the relationship between linear
recurrences like the Fibonacci sequence with their respective golden ratio.

\begin{definition}
  The \emph{characteristic polynomial} of a linear recurrence~$F$ with
  nonnegative integer coefficients~$a_0, \dots, a_d$ is defined as
  \[
    p_F(x) = x^{d+1} - a_d x^d - a_{d-1} x^{d-1} - \dots - a_1 x - a_0.
  \]
\end{definition}

% TODO: Why is the characteristic polynomial defined as such, any reason?

\begin{example}
  The characteristic polynomial of the $d$-bonacci sequence is
  \[
    x^{d+1} - x^d - ⋯ - x - 1.
  \]
\end{example}

The root can be considered a generalization of the golden ratio.
If we weaken the requirement that each coefficient $a_i$ is either $0$ or $1$ and
instead allow any nonnegative number $a_i ∈ ℤ_{≥ 0}$, then this can be considered a
generalization of the metallic means.

The question for the higher-order Fibonacci sequence is whether they behave the
same way the classical Fibonacci numbers do.
Most importantly, whether their ratios
\[
  r_n
  := \frac{F(n+1)}{F(n)}
  = \frac{a_0 F(n - d)}{F(n)} + \frac{a₁ F(n - d + 1)}{F(n)} + ⋯ + \frac{a_{d-1} F(n-1)}{F(n)} + 1.
\]
are actually converging to the only positive real root of their characteristic
polynomial $p_F$ or not.
In the following, we will show that this is indeed the case.
First, we rewrite the terms of the equation such that each ratio in the sum can
be rewritten as a product of consecutive ratios $r_{n-i}$:
\begin{align*}
  \frac{F(n - d + i)}{F(n)}
  & = \frac{F(n - d + i + 1)}{F(n - d + i)} \frac{F(n - d + i + 2)}{F(n - d + i + 1)} \dots \frac{F(n-1)}{F(n)} \\
  & = \frac{1}{r_{n - d + i}} · \frac{1}{r_{n - d + i + 1}} · \dots · \frac{1}{r_{n-1}}.
\end{align*}
So we can calculate the ratio $r_n$ using the previous ratios $r_{n-1}, r_{n-2}, …, r_{n-d}$ as follows:
\[
  r_n = 1 + \frac{a_d}{r_{n-1}} + \frac{a_{d-1}}{r_{n-1} r_{n-2}} + ⋯ + \frac{a₀}{r_{n-1} r_{n-2} \dots r_{n-d}}.
\]
Using this equation, we can show that the ratios must be bounded.

\begin{lemma}
  The ratios $r_n$ are bounded between $1$ and $d+1$.
\end{lemma}

\begin{proof}
  The first $d - 1$ ratios are all $1$.
  The ratio $r_d$ is equal to
  \[
    \frac{F(d+1)}{F(d)} = \frac{F(0) + F(1) + ⋯ + F(d)}{F(d)} = \frac{1 + 1 + ⋯ + 1}{1} = d + 1,
  \]
  which clearly satisfies the bounds of this lemma.
  By induction, suppose that the previous ratios $r_{n-1}, r_{n-2}, …, r_{n-d}$
  all satisfy the bound between $1$ and $d+1$.
  From previous consideration, we can reformulate the ratio $r_n$ as follows:
  \[
    r_n = 1 + \frac{a₁}{r_{n-1}} + \frac{a₂}{r_{n-1} r_{n-2}} + \dots + \frac{a_d}{r_{n-1} r_{n-2} \dots r_{n-d}}.
  \]
  Since $r_{n-i} ≤ d+1$, we can bound $r_n$ from below by
  \[
    r_n ≥ 1 + \frac{a₁}{d+1} + \frac{a₂}{(d+1)^2} + \dots + \frac{a_d}{(d+1)^d} ≥ 1
  \]
  and since $r_{n-i} ≥ 1$, we can bound $r_n$ from above by
  \[
    r_n ≤ 1 + \frac{a₁}{1} + \frac{a₂}{1} + \dots + \frac{a_d}{1} ≤ d+1.
  \]
  Hence, $1 ≤ r_n ≤ d+1$ for every $n ≥ 0$.
\end{proof}

\begin{figure}[tbp]
  \centering
  \includestandalone{figures/fibonacci-convergence}
  \caption{
    Illustration of the convergence proof for ratios of Fibonacci numbers.
    The points represent the ratios $r_n$.
    The sequences $s_n$ and $t_n$ are the minimum and maximum from a block of
    consecutive ratios, respectively.
    Both sequences converge towards the same limit $φ$, so the ratios converge
    towards this limit, too.
  }
  \label{fig:fibonacci-convergence}
\end{figure}

Showing that the sequence is monotone would be enough to show the convergence.
However, the sequence is clearly not monotone.
Even the ratios of the original Fibonacci sequence alternate between increasing
and decreasing.
So we cannot possibly prove that the higher-order sequences are monotone.
Instead, we bound the sequence between two other sequences $s_n$ and $t_n$
and show that the two sequences are converging to the same limit.
From the squeeze theorem, it follows that $r_n$ converges to the same limit.
The main idea is illustrated in Figure~\ref{fig:fibonacci-convergence}.
The sequences are
\[
  s_n = \min\{r_n, r_{n-1}, …, r_{n-d} \}, \qquad t_n = \max\{r_n, r_{n-1}, …, r_{n-d}\}
\]
Because $r_n$ clearly lies between the two sequences, so it must converge to the same limit as $s_n$ and $t_n$.
For these sequences, we already know from the previous lemma that they are bounded,
so it only remains to be shown that they are monotone,
where $s_n$ is always increasing and $t_n$ is decreasing.

\begin{lemma}
  The sequences $s_n$ and $t_n$ are monotone.
\end{lemma}

\begin{proof}
  Each ratio can be represented as a convex combination of the previous ratios, i.e.
  \[
    r_{n+1} = λ₀ r_n + λ₁ r_{n-1} + \dots + λ_d r_{n-d}
  \]
  using $λ_i = F(n - i) / F(n + 1)$.
  To show that this is indeed a convex combination, all coefficients $λ_i$
  need to be nonnegative and $λ₀ + λ₁ + \dots + \lambda_d = 1$.
  The former follows from the fact that Fibonacci numbers are always increasing,
  while the latter follows simply from the definition of the Fibonacci numbers.
  Because $s_n$ is the minimum and $t_n$ the maximum of $r_n, r_{n-1}, …, r_{n-d}$,
  we can bound the next maximum by
  \[
    t_{n+1} ≤ r_{n+1} = λ₀ r_n + λ₁ r_{n-1} + \dots + λ_d r_{n-d} ≤ λ₀ t_n + λ₁ t_n + ⋯ + λ_d t_n = t_n.
  \]
  and the next minimum by
  \[
    s_{n+1} ≥ r_{n+1} = λ₀ r_n + λ₁ r_{n-1} + \dots + λ_d r_{n-d} ≥ λ₀ s_n + λ₁ s_n + ⋯ + λ_d s_n = s_n.
  \]
  Therefore, $s_n ≤ s_{n+1} ≤ t_{n+1} ≤ t_n$.
\end{proof}

\begin{lemma}
  The sequences $s_n$ and $t_n$ are converging to the same limit.
\end{lemma}

% TODO: Finish this proof
\begin{proof}
  For a contradiction, suppose there exists some $δ > 0$ such that $t_n - s_n > δ$ for every $n ≥ 0$.
  Out of the previous ratios, there must be one ratio $r_k$ exactly equal to $t_n$.
  Therefore,
  \begin{align*}
    s_{n+1} ≥ r_n & = λ₀ r_{n-d-1} + λ₁ r_{n-d} + ⋯ + λ_d r_{n-1} \\
                  & ≥ λ_k t_n + \sum_{i ≠ k} λ_i s_n \\
                  & = λ_k t_n + (1 - λ_k) s_n = s_n + λ_k (t_n - s_n) ≥ s_n + λ_k δ.
  \end{align*}
  We have
  \[
    λ_k = \frac{F(n+k)}{F(n+d+1)} = \frac{1}{r_{n+k} r_{n+k+1} \dots r_{n+d+1}} ≥ \frac{1}{(d+1)^{d+1-k}}.
  \]
  Hence, $λ_k$ is always greater than some constant $c > 0$ for every $k ≥ 0$.
  But then
  \[
    s_{n+i} ≥ s_{n+i-1} + c δ ≥ s_{n+i-2} + 2c δ ≥ \dots ≥ s_n + i c δ
  \]
  and $s_{n+i}$ would always increase as $i$ approaches infinity,
  which contradicts the previous lemma.
  Therefore, $δ = 0$ and it follows that $s_n$ and $t_n$ are approaching the same limit.
\end{proof}

% TODO: Show that this converges! We're still missing the convergence criteria
\begin{theorem}
  Let $F$ be a linear recurrence with coefficients $a_0, \dots, a_d ≥ 0$
  and let $φ$ be the real positive root of its characteristic polynomial.
  Then,
  \[
    \lim_{n \to \infty} \frac{F(n + 1)}{F(n)} = φ.
  \]
\end{theorem}

\begin{proof}
  By the previous lemma, the ratios $r_n$ approach some limit $r$. It follows:
  \[
    r
    = \lim_{n → ∞} r_n
    = \lim_{n → ∞} 1 + \frac{a_d}{r_{n-1}} + \frac{a_{d-1}}{r_{n-1} r_{n-2}} + ⋯ + \frac{a₀}{r_{n-1} r_{n-2} \dots r_{n-d}}.
  \]
  Hence, each denominator in the sum approaches $r^i$,
  which results in the following polynomial:
  \[
    r = a₀ + \sum_{i = 1}^d \frac{a_{d - i}}{r^i}
    \iff
    r^{d+1} = a₀ + a₁ r + \dots + a_d r^d,
  \]
  which directly corresponds to a root of its characteristic polynomial.
  Furthermore, the ratios are always positive, so $r = φ$.
\end{proof}

\begin{corollary}
  For the same recurrence $F$, $\lim_{n → ∞} F(n + i) / F(n) = φ^i$ for $i > 1$.
\end{corollary}
%end{convergence}

%begin{periodicity}
The previous theorem leads us straight to an algebraic solution vector,
for which the generalized Euclidean algorithm is periodic.
We simply use the previous vector $x^{(n)}$ defined for the Fibonacci sequence
and increase $n$ towards infinity:
\begin{align*}
  x_i
  & = \lim_{n → ∞} x_i^{(n)} \\
  & = \lim_{n → ∞} \frac{F(n - i) + \sum_{k=1}^{i-1} a_{d-k} F(n - k)}{F(n)} \\
  & = φ^{d-i} + \sum_{k=1}^{i-1} a_{d-k} φ^{d-k}.
\end{align*}

\begin{theorem}
  The generalized Euclidean algorithm is periodic for $x = (φ, φ^2, …, φ^d)$,
  where $φ$ is the positive root of the polynomial
  \[
    x^{d+1} - a_{d} x^d - ⋯ - a_1 x - a₀
  \]
  with $a_0, a_1, …, a_d ∈ ℤ_{≥ 0}$.
\end{theorem}

\begin{proof}
  We pivot with $ℓ = 1, 2, …, d$ twice.
  In the first round, we get a vector $x'$ with
  \[
    x_i' = ...
  \]
  Once we have the vector $x'$, applying the generalized Euclidean algorithm
  will produce the vector $x''$ with
  \[
    x_ℓ'' = …, x_i'' = …
  \]
  So the algorithm just rotates the elements one spot to the right.
  Applying the algorithm $d$ times with the indices $1, 2, …, d$
  amounts to a complete rotation where each value ends up where it started.
  Therefore, we have reached $x'$ again and the algorithm becomes periodic.
\end{proof}
%end{periodicity}

%begin{inverse golden ratio}
\iffalse
\begin{definition}
  Given a polynomial $p = \sum_{i=0}^n a_i x^n$, its \emph{reciprocal polynomial},
  denoted as $p^*$, is defined as
  \[
    p^*(x) = x^n p(x^{-1}) = a_n + a_{n-1} x + \dots + a_0 x^n.
  \]
\end{definition}

\begin{example}
  The reciprocal of the golden ratio polynomial $x^2 - x - 1$ is
  \begin{align*}
    x^2 + x - 1,
  \end{align*}
  which has a root at $ψ = 1/φ$.
\end{example}

\begin{lemma}
  Let $p$ be a polynomial. Then, $a$ is a root of $p$ if and only if $a^{-1}$ is a root of $p^*$.
\end{lemma}

\begin{proof}
  Let $a$ be a root of $p$. It follows $p^*(a^{-1}) = a^n p(a) = a^n \cdot 0 = 0$.
  Now let $a^{-1}$ be a root of $p^*$. Then $p^*(a^{-1}) = a^n p(a) = 0$.
  By construction, $a$ cannot be zero and therefore we must have $p(a) = 0$.
\end{proof}
\fi
%end{inverse golden ratio}

% ==============================================================================
\iffalse
\section{Metallic Ratios}
% ==============================================================================

The result from the previous section can be generalized to an even larger class
of polynomials by allowing $a_1, a_2, …, a_d ∈ \Z_{≥ 0}$ instead of only binary
values.
The roots of these polynomials can be seen as a generalization of the metallic ratios.
Indeed, for the one-dimensional case the polynomials are
\[
  x^2 - a₁ x - 1,
\]
which characterizes all possible metallic ratios, like the silver ratio ($a₁ = 2$) or
the bronze ratio ($a₁ = 3$).
The generalization considers higher-degree polynomials of the form
\[
  x^{d+1} - a_d x^d - \dots - a_1 x - 1
\]
with nonnegative integer coefficients $a_1, a_2, …, a_d$.
Each of these polynomials has a corresponding linear recurrence,
which can be reconstructed by the characteristic polynomial.
For the linear recurrence,
the ratios of consecutive terms approaches the real positive root of the
polynomial again.
\fi

\chapter{Multi-Dimensional Continued Fractions}
\label{ch:mdcf}

For quadratic irrationals, we used continued fractions to represent them as a
periodic sequence of integers.
The continued fractions were constructed using the Euclidean algorithm.
Naturally, we can generalize continued fractions to higher dimensions using the
generalized Euclidean algorithm.
This leads to a concept of multi-dimensional continued fractions, which could
potentially lead to an answer of Hermite's question.
Since it is based on the generalized Euclidean algorithm,
it extends previous generalizations of continued fractions based on the
Jacobi-Perron algorithm \cite{Gupta00}.

This chapter introduces the concept of MDCFs and discusses their potential as
an answer to Hermite's question.
We begin by defining what they are and deriving many properties similar to
continued fractions.
Subsequently, we look at numbers represented by periodic MDCFs and whether they
must be algebraic numbers of degree $d+1$.
Finally, we discuss the geometry behind the multi-dimensional continued fractions.
The other direction -- Whether algebraic numbers of degree $d+1$ admit periodic
MDCFs -- will be discussed in the next chapter.

% ==============================================================================
\section{Construction using the Generalized Euclidean Algorithm}
% ==============================================================================

To derive the continued fraction for a real number, we have used the Euclidean algorithm.
In particular, we looked at the ratio $a/b$ between the two inputs $a$ and $b$,
and we have used the integer part of that ratio as a coefficient in the
continued fraction.
Iterating the Euclidean algorithm then gave us a unique representation for
every real number.
The Euclidean algorithm can also be viewed as the generalized algorithm when $d = 1$.
In this case, the ratio $a/b$ represents the solution $x$ of the linear system $bx = a$.
So the integer part of the solution $x$ also represents a coefficient in the
continued fraction.
We can then use the pivot operation to iterate over this solution and gain the
rest of the coefficients.

The multi-dimensional continued fraction can now be easily derived from the
solution vector $x$ in higher dimensions.
Specifically, given a vector $x ∈ ℝ^d$, we take the integer part of each
element $\floor{x}$ as a coefficient of the multi-dimensional continued
fraction.
Then, we get the subsequent coefficients by iterating over this vector using
the pivot operation.
This gives us a top-down construction of MDCFs, but typically continued
fractions are defined in a bottom-up fashion.
For a bottom-up definition of the MDCFs, we simply reverse the pivot operation.
The inverse operation can be derived as follows:
Let $x ∈ ℝ^d$ and $a = \floor{x}$.
If $x' = \mathrm{pivot}_ℓ(x)$ for a given index $ℓ$, then
\[
  \begin{array}{lcrlcr}
    \displaystyle x_i' & = & \displaystyle \frac{x_i - a_i}{x_ℓ - a_ℓ}, &
    \displaystyle x_ℓ' & = & \displaystyle \frac{x_i - a_i}{x_ℓ - a_ℓ} \\[1em]
  \end{array}
\]
for all $i ≠ ℓ$.
For the inverse operation, we have to derive every element of $x$ from $x'$,
which can be done by the following equations:
\[
  \begin{array}{lcrlcr}
    \displaystyle x_i & = & a_i + \displaystyle \frac{x_i'}{x_ℓ'}, &
    \displaystyle x_ℓ & = & a_ℓ + \displaystyle \frac{1}{x_ℓ'}
  \end{array}
\]
This allows us to calculate the previous vector $x$ from the next vector $x'$,
if we know the integer vector $a$.
For the inverse function, we do not need to store the index,
since in the next vector $x_ℓ' = 1/\{x_ℓ\}$ is always the largest element.
So we can derive $ℓ$ from the vector by its largest element.
By $\mathrm{pivot}^{-1}$ we denote this inverse function.
We have
\[
  \mathrm{pivot}_ℓ(x) = x' \iff \mathrm{pivot}^{-1}(a, x') = x.
\]
This operation leads directly to a definition of MDCFs.

Although, we initially used integer vectors $a ∈ ℤ^d$ for the definition of the inverse operation,
we can also look at what happens when $a$ is not necessarily an integer vector.
For subsequent lemmas, we will need a definition of MDCFs which allows rational
or even real vectors as coefficients.

\begin{definition}
  Given a sequence of $d$-dimensional real vectors $\{r^{(n)}\}_{n ≥ 0}$,
  the \emph{$d$-dimensional continued fraction} $[r^{(0)}; r^{(1)}, …]$ is defined as
  \[
    [r^{(0)}; ℓ₁, r^{(1)}; …] = \lim_{n → ∞} [r^{(0)}; ℓ₁, r^{(1)}; …; ℓₙ, r^{(n)}],
  \]
  where the finite continued fractions $[r^{(0)}; r^{(1)}, …, r^{(n)}]$
  are defined inductively as
  \[
    [r^{(0)}] = r^{(0)},
    \qquad
    [r^{(0)}, r^{(1)}, …, r^{(n)}]
    = \mathrm{pivot}^{-1}\big(r^{(0)}, [r^{(1)}, r^{(2)}, …, r^{(n)}]\big).
  \]
\end{definition}

For the representation to be correct, we require $\max_i r_i^{(n)} ≠ 0$ for $n ≥ 1$.
This is similar to the continued fractions, where only the first value could be zero,
while all subsequent values had to be positive.
For the multi-dimensional counterpart we only require that the pivot element is not zero.
The other values in the vector $r^{(n)}$ can assume any non-negative value.

The terminology from one-dimensional continued fractions naturally carry over to its
multi-dimensional counterpart.
Given an infinite MDCF representation~$[a₀; ℓ_1, a_1; …]$ of a vector $x ∈ ℝ^d$, we define the following:

\begin{itemize}
  \item The \emph{$k$-th convergent} of $x$ is the finite MDCF $[r^{(0)}; r^{(1)}, …, r^{(n)}]$.
  \item The \emph{$k$-th complete quotient} is the MDCF $[r^{(n)}; r^{(n+1)}, …]$.
  \item The MDCF is \emph{eventually periodic} if there exists an index~$n₀ ≥ 0$
    and a period~$k ≥ 1$ such that $aₙ = a_{n+k}$ and $ℓₙ = ℓ_{n+k}$
    for every $n ≥ n₀$.
    The MDCF is \emph{purely periodic} if $n₀ = 0$.
\end{itemize}

For the analysis of MDCFs,
we begin with two lemmas which generalize Lemma~\ref{lem:cf-nesting} and
Lemma~\vref{lem:cf-wallis}.
The first allows us to merge the last two coefficients of an MDCF.
The second gives us a formula for calculating the convergents.

\begin{lemma}
  \label{lem:mdcf-nesting}
  Let $x ∈ ℝ^d$, then
  \[
    [r^{(0)}; r^{(1)}, …, r^{(n)}, x]
    = [r^{(0)}; r^{(1)}, …, r^{(n-1)}, \mathrm{pivot}^{-1}(r^{(n)}, x)]
  \]
\end{lemma}

\begin{proof}
  If $n = 0$, then by definition,
  \[
    [r^{(0)}; x] = \mathrm{pivot}^{-1}(r^{(0)}, [x]) = \mathrm{pivot}^{-1}(r^{(0)}, x) = [\mathrm{pivot}^{-1}(r^{(0)}, x)].
  \]
  Suppose the lemma holds for any $n ≥ 0$, then
  \begin{align*}
    [r^{(0)}; r^{(1)}, …, r^{(n+1)}, x]
    & = \mathrm{pivot}^{-1}(r^{(0)}, [r^{(1)}; r^{(2)}, …, r^{(n+1)}, x]) \\
    & = \mathrm{pivot}^{-1}(r^{(0)}, [r^{(1)}; r^{(2)}, …, r^{(n)}, \mathrm{pivot}^{-1}(r^{(n)}, x)] \\
    & = [r^{(0)}; r^{(1)}, …, r^{(n)}, \mathrm{pivot}^{-1}(r^{(n+1)}, x)]. \qedhere
  \end{align*}
\end{proof}

% TODO: Explain how to derive the sequences
The second lemma is more complicated.
In the lemma for continued fractions,
we defined the convergents of a continued fraction~$pₙ/qₙ$
using a fractional transformation based on the previous two
terms~$p_{n-1}/q_{n-1}$ and $p_{n-2}/q_{n-2}$.
For MDCFs, we can similarly derive a recursive formula to derive the values of
the convergent vector $(p₁/q, \dots, p_d/q)$ using the previous convergents.
Deriving the sequence is more involved than the one-dimensional case since
there is an additional index $ℓ$ at each step.

The sequence is essentially the generalized Euclidean algorithm, but in reverse.
Just like the MDCF is the application of the inverse pivot operation.
In the algorithm, we begin with a basis $B ∈ ℤ^{d×d}$ and a vector $c ∈ ℤ^d$.
First, we find a solution $x$ to $Bx = c$ and then we exchange $c$ with $B_ℓ$
and $B_ℓ$ with the remainder $B\{x\} = Bx - Ba_i$.
We continue this process until $x$ is entirely integral.
For the sequence, we reverse this process.
We begin with an already reduced basis $B'$ and a vector $c'$, which is an
integral combination of $B'$.
Then, we exchange $B'_{ℓ}$ with $c'$ and $c'$ with $B' a_i + B'_{ℓ}$.
We continue this process until $B x = c$.

In total, there are two types of sequences:
A sequence of $d+1$ vectors $P_0^{(n)}, P_1^{(n)}, …, P_d^{(n)} ∈ ℤ^{d+1}$ and a sequence
of $d+1$ scalars $Q_0^{(n)}, Q_1^{(n)}, …, Q_d^{(n)} ∈ ℤ$,
which represent the denominators of the convergents.
The sequences begin with
\begin{align*}
  P_0^{(-1)} & = 0, & P_i^{(-1)} & = e_i, & P_0^{(-1)} = 0 \\
  Q_0^{(-1)} & = 1, & Q_i^{(-1)} & = 0,   & P_0^{(-1)} = 1
\end{align*}
where $1 ≤ i ≤ d$ and the sequences are updated according to the following
recurrences:
\begin{align*}
  P_ℓ^{(n)} & = P_0^{(n-1)} + P_1^{(n-1)} r_1^{(n)} + ⋯ + P_d^{(n-1)} r_d^{(n)}, &
  P_i^{(n)} & = P_i^{(n-1)}, &
  P_0^{(n)} & = P_{ℓₙ}^{(n-1)} \\
  Q_ℓ^{(n)} & = Q_0^{(n-1)} + Q_1^{(n-1)} r_1^{(n)} + ⋯ + Q_d^{(n-1)} r_d^{(n)}, &
  Q_i^{(n)} & = Q_i^{(n-1)}, &
  Q_0^{(n)} & = Q_{ℓₙ}^{(n-1)},
\end{align*}
where $i ≠ ℓ$ and $ℓ$ is the index of the maximum element in the complete
quotient $x^{(n)}$. % TODO: n or n - 1?
What this sequence effectively does is reconstructing the lattice from an
initial solution vector $x ∈ ℝ^d$ and its coefficient vectors $a_n$.

Modelling this back to the Euclidean algorithm,
the vectors $P_1^{(n)}, …, P_d^{(n)}$ represent the basis and $P_0^{(n)}$ is
the additional vector, which is reduced during a run of the algorithm.
This matches the behavior of the algorithm in reverse.
We begin with the vector $P_0^{(-1)} = 0$ since the Euclidean algorithm
would end with a zero vector.
Then, the formula calculates a new vector using an integral combination of the
old vectors and stores it in $P_{ℓ_n}^{(n)}$.
This corresponds directly to the modulo and exchange operation of the
generalized Euclidean algorithm.
As the vectors $P_0^{(n)}, P_1^{(n)}, …, P_d^{(n)}$ would grow infinitely, the
values $Q_0^{(n)}, Q_1^{(n)}, …, Q_d^{(n)}$ scale the vectors back down and as
such they ensure that they converge to a limit as $n$ increases.

% TODO: Ensure that this is correct for the first index!
\begin{lemma}
  \label{lem:mdcf-wallis}
  Let $r^{(0)}, r^{(1)}, …, r^{(n-1)}, x ∈ ℝ^d$, then
  \[
    [r^{(0)}; r^{(1)}, …, r^{(n-1)}, x]
    = \frac{P_0^{(n-1)} + P_1^{(n-1)} x_1 + ⋯ + P_d^{(n-1)} x_d}{Q_0^{(n-1)} + Q_1^{(n-1)} x_1 + ⋯ + Q_d^{(n-1)} x_d}.
  \]
\end{lemma}

\begin{proof}
  If $n = 0$, then
  \[
    [x]
    = x
    = \frac{x}{1}
    = \frac{0 + e₁ x₁ + ⋯ + e_d x_d}{1 + 0 x₁ + ⋯ + 0 x_d}
    = \frac{P₀^{(0)} + P₁^{(0)} x₁ + ⋯ + P_d^{(0)} x_d}{Q_0^{(0)} + Q_1^{(0)} x₁ + ⋯ + Q_d^{(0)} x_d}.
  \]
  Suppose the lemma holds for $n ≥ 0$,
  we show that it also holds for $n+1$.
  From the previous lemma, it follows that
  \begin{align*}
    [r^{(0)}; r^{(1)}; …, r^{(n)}, x] & = [r^{(0)}; r^{(1)}, …, r^{(n-1)}, \mathrm{pivot}^{-1}(r^{(n)}, x)].
  \end{align*}
  Let $y = \mathrm{pivot}^{-1}(r^{(n)}, x)$ and $ℓ = \argmax_i x_i$.
  By the induction hypothesis,
  \begin{align*}
    [r^{(0)}; r^{(1)}; …, r^{(n)}, x] & = \frac{x_ℓ}{x_ℓ} · \frac{P_0^{(n-1)} + P_1^{(n-1)} y_1 + ⋯ + P_d^{(n-1)} y_d}{Q_0^{(n-1)} + Q_1^{(n-1)} y_1 + ⋯ + Q_d^{(n-1)} y_d},
  \end{align*}
  where the fraction is expanded with $x_ℓ/x_ℓ$ such that the numerator and denominator are each multiplied by $x_ℓ$.
  The numerator can then be simplified as follows:
  \begin{align*}
    & \hphantom{{} = {}} x_ℓ \left( P_0^{(n-1)} + P_ℓ^{(n-1)} y_ℓ + \sum_{i ≠ ℓ, i ≠ 0} P_i^{(n-1)} y_i \right) \\
    & = x_ℓ \left( P_0^{(n-1)} + P_ℓ^{(n-1)} \left( r_ℓ^{(n)} + \frac{1}{x_ℓ} \right) + \sum_{i ≠ ℓ, i ≠ 0} P_i^{(n-1)} \left(r_i^{(n)} + \frac{x_i}{x_ℓ} \right) \right) \\
    & = P_0^{(n-1)} x_ℓ + P_ℓ^{(n-1)} r_ℓ^{(n)} x_ℓ + P_ℓ^{(n-1)} + \sum_{i ≠ 0, i ≠ ℓ} P_i^{(n-1)} r_i^{(n)} x_ℓ + P_i^{(n-1)} x_i \\
    & = \underbrace{\left( P_0^{(n-1)} + P_ℓ^{(n-1)} r_ℓ^{(n)} + \sum_{i ≠ 0, i ≠ ℓ} P_i^{(n-1)} r_i^{(n)} \right)}_{P_ℓ^{(n)}} x_ℓ
      + \underbrace{P_ℓ^{(n-1)}}_{P_0^{(n)}}
      + \sum_{i ≠ 0, i ≠ ℓ} \underbrace{P_i^{(n)}}_{P_i^{(n)}} x_i \\
    & = P_0^{(n)} + P_1^{(n)} x_1 + ⋯ + P_d^{(n)} x_d.
  \end{align*}
  The simplification for the denominator is identical.
  Therefore,
  \[
    [r^{(0)}; r^{(1)}, …, r^{(n)}, x]
    = \frac{P_0^{(n)} + P_1^{(n)} x_1 + ⋯ + P_d^{(n)} x_d}{Q_0^{(n)} + Q_1^{(n)} x_1 + ⋯ + Q_d^{(n)} x_d}.
    \qedhere
  \]
\end{proof}

% ==============================================================================
\section{Convergence of Multi-Dimensional Continued Fractions}
% ==============================================================================

% TODO: Maybe we should call r^{(n)} the primary convergent and the other ones
% secondary convergents. Something to better make the distinction.
% TODO: I don't like how we say that the requirement is from Perron, when it's
% not really from him, but just based on his version.
% TODO: We should mention somewhere that a^{(n)} is always positive.
% TODO: The indices here are wrong...
In the following, we assume that we have an MDCF $[a^{(0)}, a^{(1)}, a^{(2)}
…]$ with only positive integer coefficients $a^{(1)}, a^{(2)}$ and one
(potentially negative) integer vector $a^{(0)}$.
Furthermore, we assume that this is an MDCF constructed using the generalized
Euclidean algorithm for some irrational vector $x = (x₁, …, x_d)$.
In the following, $x^{(k)}$ denotes the $k$-th complete quotient of $x$
and the $n$-th convergent is denoted as
\[
  r^{(n)}
  = (r_1^{(n)}, …, r_d^{(n)})
  = \left( \frac{P_{1ℓ_n}^{(n)}}{Q_{ℓ_n}^{(n)}}, \dots, \frac{P_{dℓ_n}^{(n)}}{Q_{ℓ_n}^{(n)}} \right).
\]
The aim of this section is to show that the convergents actually live up to
their name and converge towards the vector $x$ as $n$ increases.
The proof is based on Perron's proof for the convergence of his
algorithm \cite{Perron07}.
For the proof we need two more requirements:
The first comes from Perron's original proof.
In his version, he required that there exists some constant $c > 0$ such that
\[
  0 < \frac{1}{a_{ℓ}^{(n)}} ≤ c
  \quad \text{ and } \quad
  0 ≤ \frac{a_i^{(n)}}{a_{ℓ}^{(n)}} ≤ c \quad \text{ for every } i ≠ ℓ = \argmax_i x_i^{(n)}.
\]
In addition to his requirement, we also need that every element is the maximum infintely often,
or equivalently the associated index sequence contains every index infinitely often.
This is already given in the Jacobi-Perron algorithm,
but not in the generalized Euclidean algorithm.
So we need this additional requirement for the convergence.

The vector $r^{(n)}$ is only one possible convergent.
The other vectors $P_i^{(n)}/Q_i^{(n)}$ with $i ≠ ℓ$ are also convergent
vectors, since $i = \argmax_i x_i^{(m)}$ at some index $m < n$ if $n$ is chosen large enough.
The first step for the proof is to show that each convergent $r^{(n)}$ is a
convex combination of these vectors.
This shows that the vector $x$ lies in the convex hull of the convergents.

% TODO: We should probably add superscripts to the lambdas as they depend on n.
\begin{lemma}
  \label{lem:conv-conv}
  There exists $λ₀, λ₁, …, λ_d ≥ 0$ with $λ₀ + λ₁ + ⋯ + λ_d = 1$ such that
  \[
    r_i^{(n)} = λ₀ \frac{P_{i0}^{(n-1)}}{Q_0^{(n-1)}} + λ₁ \frac{P_{id}^{(n-1)}}{Q_1^{(n-1)}} + ⋯ + λ_d \frac{P_{id}^{(n-1)}}{Q_1^{(n-1)}}.
  \]
\end{lemma}

\begin{proof}
  We choose $n ≥ 0$ such that every index in $\{1, …, d\}$ has occurred at least once.
  Let $ℓ = \argmax_i x_i^{(n)}$.
  By Lemma~\ref{lem:mdcf-wallis}, we can calculate $P_{iℓ}^{(n)}$ using the previous values as follows:
  \begin{align*}
    P_{iℓ}^{(n)} = \sum_{k = 1}^d P_{ik}^{(n-1)} a_k^{(n)} + P_{i0}^{(n-1)}.
  \end{align*}
  Dividing by $Q_ℓ^{(n)}$ gives us
  \begin{align*}
    \frac{P_{iℓ}^{(n)}}{Q_ℓ^{(n)}} = \sum_{k = 1}^d \frac{P_{ik}^{(n-1)}}{Q_ℓ^{(n)}} a_k^{(n)} + \frac{P_{i0}^{(n-1)}}{Q_ℓ^{(n)}}.
  \end{align*}
  Using $λ_k = a_k^{(n)} \frac{Q_k^{(n-1)}}{Q_ℓ^{(n)}}$ for $1 ≤ k ≤ d$ and $λ₀ = \frac{Q_0^{(n-1)}}{Q_ℓ^{(n)}}$,
  we can reformulate the convergent as the convex combination for this lemma.
  Because of the requirement that every index has occurred at least once,
  the denominators $Q_k^{(n-1)}$ of the convex combination cannot be zero.
  Therefore, this is a valid representation of the convergent $r_i^{(n)}$.
  For the coefficients $λ_i$, we require $λ₀ + λ₁ + ⋯ + λ_d = 1$ and $0 ≤ λᵢ ≤ 1$ for every index $i$.
  The first property follows from the definition of $Q_ℓ^{(n)}$:
  \[
    Q_ℓ^{(n)} = Q_0^{(n-1)} + Q_1^{(n-1)} a_1^{(n)} + ⋯ + Q_d^{(n-1)} a_d^{(n)}
  \]
  which is equivalent to
  \[
    1 = \frac{Q_0^{(n-1)}}{Q_ℓ^{(n)}} + \frac{Q_1^{(n-1)}}{Q_ℓ^{(n)}} a_1^{(n)} + ⋯ + \frac{Q_d^{(n-1)}}{Q_ℓ^{(n)}} a_d^{(n)} = λ₀ + λ₁ + ⋯ + λ_d.
  \]
  The second property follows from the fact that $a^{(n)}$ is non-negative.
  Therefore, $Q_ℓ^{(n)}$ is always less than or equal to $a_i^{(n)}
  Q_i^{(n-1)}$ for every index $i ≤ d$ and $λ_i$ is always bounded
  between $0$ and $1$.
\end{proof}

To prove the convergence, we will apply the squeeze theorem, again.
We use the two sequences $s^{(n)} = (s_1^{(n)}, …, s_d^{(n)})$ and $t^{(n)} = (t_1^{(n)}, …, t_d^{(n)})$ with
\[
  s_i^{(n)} = \min\left\{\frac{P_{i0}^{(n)}}{Q_0^{(n)}}, \frac{P_{i1}^{(n)}}{Q_1^{(n)}}, …, \frac{P_{id}^{(n)}}{Q_d^{(n)}}\right\}
\]
and
\[
  t_i^{(n)} = \max\left\{\frac{P_{i0}^{(n)}}{Q_0^{(n)}}, \frac{P_{i1}^{(n)}}{Q_1^{(n)}}, …, \frac{P_{id}^{(n)}}{Q_d^{(n)}}\right\}.
\]
The convergent $r^{(n)}$ clearly lies between the sequences $s^{(n)}$ and $t^{(n)}$.
So if $s^{(n)}$ and $t^{(n)}$ converge to the same limit, then $r^{(n)}$ also
converges to the same limit.
For the convergence of $s^{(n)}$ and $t^{(n)}$, we show that they are monotone
and bounded.
By the monotone-convergence theorem, they must converge.

\begin{lemma}
  The sequences $s_i^{(n)}$ and $t_i^{(n)}$ are monotone and bounded.
\end{lemma}

\begin{proof}
  We can bound the new convergent $r^{(n)}$ from below using $s^{(n-1)}$.
  For the previous iteration, we already have $P_{ij}^{(n-1)}/Q_j^{(n-1)} ≥ s_i^{(n-1)}$
  for every $0 ≤ j ≤ d$ by construction of the sequence~$s_i^{(n-1)}$.
  Using Lemma~\ref{lem:conv-conv},
  \begin{align*}
    \frac{P_{iℓ}^{(n)}}{Q_ℓ^{(n)}}
    & = λ₀ \frac{P_{i0}^{(n-1)}}{Q_0^{(n-1)}} + λ₁ \frac{P_{i1}^{(n-1)}}{Q_1^{(n-1)}} + ⋯ + λ_d \frac{P_{id}^{(n-1)}}{Q_d^{(n-1)}} \\
    & ≥ λ₀ s_i^{(n-1)} + λ₁ s_i^{(n-1)} + ⋯ + λ_d s_i^{(n-1)} = s_i^{(n-1)}.
  \end{align*}
  The bound for the other convergents $r_j^{(n)}$ with $j ≠ ℓ$ is even easier since
  \[
    \frac{P_{ij}^{(n)}}{Q_j^{(n)}} = \frac{P_{ij}^{(n-1)}}{Q_j^{(n-1)}} ≥ s_i^{(n-1)}.
  \]
  Using the same argument, we can bound each convergent from above by $t_i^{(n-1)}$.
  Therefore, in each iteration the convergent $r_i^{(n)}$ is
  bounded between the previous minimum~$s_i^{(n-1)}$ and the previous maximum~$t_i^{(n-1)}$,
  or equivalently $s_i^{(n-1)} ≤ s_i^{(n)} ≤ r_i^{(n)} ≤ t_i^{(n)} ≤ t_i^{(n-1)}$.
\end{proof}

The previous lemma only shows that the sequences are converging,
not that they are converging to the same limit.
To show the second part, we first need this crucial lemma
about the coefficients $λ₀, λ₁, …, λ_d$ in the convex combination from Lemma~\ref{lem:conv-conv}.
In particular, there always exists one $λ_k$ which is neither exactly $0$ nor $1$.
% TODO: Is this not trivial?? Yes, an we should reformulate this statement to
% make it clear that this is constant. It always moves by a constant amount
Stated differently, the vector $x$ never lies exactly on a convergent.
This is also the crucial property, that Perron has shown for the convergence of
his algorithm.

\begin{lemma}
  \label{lem:lambda-pos}
  There exists one index $ℓ$ with $λ_ℓ > c$ for some positive constant $c$.
\end{lemma}

% TODO: Show that it is not equal to 1!
\begin{proof}
  We choose $k = ℓ_{n-1}$.
  Using the initial requirements for the coefficients $a_i^{(n)}$ and the fact
  that $Q_k^{(n-1)} ≥ Q_i^{(n-1)} = Q_i^{(n-2)}$ for every $i ≠ k$,
  it follows that
  \[
    \frac{λ_i}{λ_k} = \frac{a_i^{(n)}}{a_k^{(n)}} · \frac{Q_i^{(n-1)}}{Q_k^{(n-1)}} ≤ c · 1,
  \]
  or equivalently $λᵢ ≤ c λₖ$. From this, it follows that
  \[
    1 = λ₀ + λ₁ + ⋯ + λ_d ≤ λ_k (1 + dc).
  \]
  Therefore, $λₖ ≥ 1/(1 + dc) > 0$.
\end{proof}

\begin{figure}[tbp]
  \centering
  \includegraphics{build/convergence}
  \caption{
    Illustration of the proof for the convergence of MDCFs.
    The points $r^{(i)}, r^{(j)}$ and $r^{(k)}$ represent the convergents and $x$ is
    the vector, which is approximated by them.
    If $r^{(k)}$ is the convergent of the current iteration,
    then either $r^{(i)}$ or $r^{(j)}$ has to be moved in the next iteration.
    However, neither can go in the red triangle,
    since $λ_k > 1/dc$.
  }
  \label{fig:convergence}
\end{figure}

The two sequences $s^{(n)}$ and $t^{(n)}$ wrap the convex hull in an
axis-aligned bounding box.
If the sequences converge to the same limit,
then that means that the bounding box shrinks to a point.
Therefore, the convergent $r^{(n)}$, which is inside this box, converges to the
same point.
However, the box could also approach some non-empty box, i.e.
\[
  \lim_{n → ∞} s_i^{(n)} < \lim_{n → ∞} t_i^{(n)}.
\]
But the previous lemma contradicts this assumption,
because each convergent moves by a constant percentage towards this vector.
Therefore, we can always find a point at which the convergents step over this
non-empty box and every convergent is inside the box.
This is the main idea behind the following lemma.

% TODO: Should we add the requirements, like every index occurs infinitely
% often (but the correct version)?
\begin{lemma}
  \label{lem:min-max-conv}
  The minimum $s_i^{(n)}$ and maximum $t_i^{(n)}$ are converging to the same
  limit.
\end{lemma}

\begin{proof}
  Let $s_i = \lim_{n → ∞} s_i^{(n)}$
  and $t_i = \lim_{n → ∞} t_i^{(n)}$ for every $i ≤ d$.
  Suppose $s_i < t_i$.
  For the minimum~$s_i$, there must exist an $ε > 0$ and an index $N ≥ 0$ such
  that for every $n ≥ N$,
  \[
    \frac{P_{ij}^{(n)}}{Q_j^{(n)}} ≥ s_i^{(n)} > s_i - ε.
  \]
  Let $ℓ' = \argmax_i x_i^{(n-1)}$.
  Then, $r_i^{(n-1)} = P_{iℓ'}^{(n-1)}/Q_{ℓ'}^{(n-1)}$ and
  %$Q_i^{(n-1)} = Q_i^{(n-2)}$ and $Q_k^{(n-1)} = Q_0^{(n-2)} + Q_1^{(n-2)} a_1^{(n-1)} + ⋯ + Q_d^{(n-2)} a_d^{(n-1)}$. % TODO: Why was this here?
  \begin{align*}
    r_i^{(n)}
    & = λ₀ \frac{P_{i0}^{(n-1)}}{Q_0^{(n-1)}} + λ₁ \frac{P_{i1}^{(n-1)}}{Q_1^{(n-1)}} + ⋯ + λ_d \frac{P_{id}^{(n-1)}}{Q_d^{(n-1)}} \\
    & > \lambda_{ℓ'} \frac{P_{iℓ'}^{(n-1)}}{Q_{ℓ'}^{(n-1)}} + \sum_{i ≠ k} λ_i (s_i - ε) \\
    & = \lambda_{ℓ'} r_i^{(n-1)} + (1 - λ_{ℓ'}) (s_i - ε),
  \end{align*}
  or equivalently
  \begin{align*}
    r_i^{(n)} - s_i > λ_{ℓ'} \left( r_i^{(n-1)} - s_i \right) - ε.
  \end{align*}
  Lemma~\ref{lem:lambda-pos} tells us that there is a positive constant $M$
  that bounds $λ_{ℓ'}$ from below.
  It follows that
  \begin{align*}
    r_i^{(n)} - s_i > \frac{1}{M} \left( r_i^{(n-1)} - s_i \right) - ε
  \end{align*}
  for some constant $M > 0$.
  Advancing this inequality from $n$ to $n+1, n+2, …, n+k$ results in the new
  inequality
  \begin{align*}
    r_i^{(n+k)} - s_i > \frac{1}{M^k} \left( r_i^{(n-1)} - s_i \right) - ε\left( \frac{1}{M} + \frac{1}{M^2} + ⋯ + \frac{1}{M^{k-1}} \right).
  \end{align*}
  Since every index occurs infinitely often,
  there must be one index $n$ where $r_i^{(n-1)} ≤ t_i$
  and another index $n+k$ where $r_i^{(n+k)} ≥ s_i$.
  \begin{align*}
    0 > \frac{1}{M^k} \left( t_i - s_i \right) - ε\left( \frac{1}{M} + \frac{1}{M^2} + ⋯ + \frac{1}{M^{k-1}} \right).
  \end{align*}
  But $ε$ can be chosen arbitrarily small,
  which is a contradiction to the fact that $t_i - s_i > 0$.
  Therefore, $t_i$ and $s_i$ converge to the same limit.
\end{proof}

% TODO: Fix statement, i.e. ℓₙ no longer exists...
\begin{theorem}
  \label{thm:mdcf-conv}
  If for every $n ≥ 0$ there exists a constant $c ≥ 0$ such that
  \[
    0 < \frac{1}{a^{(n)}_ℓ} ≤ c, \qquad 0 ≤ \frac{a^{(n)}_i}{a^{(n)}_ℓ} ≤ c.
  \]
  for every $i ≠ ℓ = \argmax_i x_i^{(n)}$ and every index occurs infinitely
  often in the sequence $ℓ$, then the sequence $r^{(n)}$ converges to $x$.
\end{theorem}

\begin{proof}
  Let $x^{(n)}$ denote the $n$-th complete quotient of $x$.
  By Lemma~\ref{lem:mdcf-wallis}, we can represent each element in $x$ as
  \[
    x_i = \frac{P_{i0}^{(n-1)} + P_{i1}^{(n-1)} x_1^{(n)} + ⋯ + P_{id}^{(n-1)} x_d^{(n)}}{Q_{i0}^{(n-1)} + Q_{i1}^{(n-1)} x_1^{(n)} + ⋯ + Q_{id}^{(n-1)} x_d^{(n)}}.
  \]
  Using a similar argument as Lemma~\ref{lem:conv-conv}, we can represent this
  as a convex combination
  \[
    x_i = λ₀ \frac{P_{i0}^{(n-1)}}{Q_0^{(n-1)}}  + λ₁ \frac{P_{i1}^{(n-1)}}{Q_1^{(n-1)}} + λ_d \frac{P_{id}^{(n-1)}}{Q_d^{(n-1)}}.
  \]
  with $λ_i = x_i^{(n)} \frac{Q_i^{(n-1)}}{Q_ℓ^{(n)}}$.
  From the previous lemma, we know that $r^{(n)}$ converges to some limit $x'$.
  Since every index occurs infinitely often,
  we can find an index $m ≤ n - 1$ such that $ℓ_m = j$ and
  \[
    \frac{P_{ij}^{(n-1)}}{Q_j^{(n-1)}} = \frac{P_{ij}^{(m)}}{Q_j^{(m)}} = r_i^{(m)},
  \]
  or equivalently we can find an index where the term is a convergent.
  As $n$ increases there are infinitely many such convergents,
  so each term converges to $x_i'$.
  Thus,
  \[
    x_i
    = \lim_{n → ∞} λ₀ \frac{P_{i0}^{(n-1)}}{Q_0^{(n-1)}}  + λ₁ \frac{P_{i1}^{(n-1)}}{Q_1^{(n-1)}} + λ_d \frac{P_{id}^{(n-1)}}{Q_d^{(n-1)}}
    = λ₀ x_i' + λ₁ x_i' + ⋯ + λ_d x_i'
    = x_i'
  \]
  for every $i ≤ d$.
\end{proof}

% TODO: Explain that the theorem always applies since we choose the maximum
One application of this theorem are periodic MDCFs.
Since they only have a finite set of a coefficients they easily satisfy the
first requirement of the theorem.

\begin{corollary}
  Periodic MDCFs converge if the period contains every index.
\end{corollary}

% ==============================================================================
\section{The Geometry behind Multi-Dimensional Continued Fractions}
% ==============================================================================

In the geometrical interpretation of continued fractions,
each convergent $pₙ/qₙ$ was represented as a two-dimensional vector $(pₙ, qₙ)$.
These vectors approached an irrational line spanned by the vector $(1, α)$ where
$α$ is some irrational number.
For the generalization of this interpretation to multi-dimensional continued
fractions, we represent each convergent, which is a $d$-dimensional rational vector,
as a $(d+1)$-dimensional integer vector.
Specifically, given a vector $r = (p₁/q₁, …, p_d/q_d) ∈ ℚ^d$, first we find a
common denominator $(p₁'/q, …, p_d'/q_d)$ and then we map it to the vector
$\hat r = (q, p₁', …, p_d') ∈ ℤ^{d+1}$.
We can go back from the integer vector to a rational vector by dividing each
coordinate with the first and removing the first coordinate from the integer
vector:
Given $(x₀, x₁, …, x_d) ∈ ℤ^{d+1}$, we map it back to $(x₁/x₀, …, x_d/x₀) ∈ ℚ^d$.

In this space, the representation for a particular convergent is not unique,
there can be multiple integer vectors representing the same convergent.
For example, if we have a vector $\hat r ∈ ℤ^d$ for a convergent,
then we can multiply with some scalar $λ ∈ ℤ$ and get a new vector $c' = λ c$
which represents the same convergent.
This is because the scalar is eliminated when mapping back to the rational vector:
\[
  λ (c₀, c₁, …, c_d)
  ↦ \left(\frac{λ c₁}{λ c₀}, …, \frac{λ c_d}{λ c_0} \right)
  = \left(\frac{c₁}{c₀}, …, \frac{c_d}{c_0} \right).
\]
Therefore, two nonzero vectors $a, b ∈ ℝ^{d+1}$ are equivalent,
denoted as $a \sim b$, if $a = λ b$ for some scalar $λ ∈ ℝ$.
This relation then defines the equivalence class of an element $a ∈ ℝ^{d+1}$ as
\[
  [a] = \mathrm{span}(a) = \{ λ a \mid λ ∈ ℝ, λ ≠ 0 \}.
\]
Formally, this is known as a real projective space, denoted as $\mathbb{RP}^{d+1}$.
It is the set of equivalence classes $ℝ^{d+1} \setminus \{0\}$ defined by the
equivalence relation $\sim$.
An element $x$ of this space is denoted as $[x₀, x₁, …, x_d]$,
where the square brackets indicate that this element is an equivalence
class.
In summary, we have the following mappings from $ℝ^d$ to $\mathbb{RP}^{d+1}$
and vice-versa:

\begin{center}
  \begin{tikzpicture}
    \matrix[
      column sep=2cm,
      nodes={text width=3cm, align=center},
    ] {
      \node (L0) {$\mathbb{R}^d$}; &
      \node (R0) {$\mathbb{RP}^{d+1}$}; \\
      \node (L1) {$(x₁, …, x_d)$}; &
      \node (R1) {$[1, x₁, …, x_d]$}; \\
      \node (L2) {$(x₁/x₀, …, x_d/x₀)$}; &
      \node (R2) {$[x₀, x₁, …, x_d]$}; \\
    };

    \draw[->] (L1) -- node[above] {} (R1);
    \draw[<-] (L2) -- node[above] {} (R2);
  \end{tikzpicture}
\end{center}

For example, consider the vector $[1, x₁, x₂]$ and suppose that $0 ≤ x₁, x₂ < 1$.
A pivot operation with $ℓ = 1$ would result in the vector $[1, 1/x₁, x₂/x₁]$.
This vector is equivalent to $[x₁, 1, x₂]$.
Therefore, we can reformulate this operation as a coordinate swap of $x_ℓ$ with
the new coordinate $x₀$:
\[
  \begin{bmatrix}
    0 & 1 & 0 \\
    1 & 0 & 0 \\
    0 & 0 & 1 \\
  \end{bmatrix}
  ·
  \begin{bmatrix} 1 \\ x₁ \\ x₂ \\ \end{bmatrix}
  =
  \begin{bmatrix} x₁ \\ 1 \\ x₂ \\ \end{bmatrix}
  =
  \begin{bmatrix} 1 \\ 1/x₁ \\ x₂/x₂ \\ \end{bmatrix}.
\]
If $x₁$ and $x₂$ are greater than $1$,
then we first have to subtract the integer part from the vector $x$.
In the projective space, this is equivalent to a series of skew operations:
\[
  \begin{bmatrix}
    1 & 0 & 0 \\
    -\floor{x₁} & 1 & 0 \\
    0 & 0 & 1 \\
  \end{bmatrix}
  ·
  \begin{bmatrix}
    1 & 0 & 0 \\
    0 & 1 & 0 \\
    -\floor{x₂} & 0 & 1 \\
  \end{bmatrix}
  ·
  \begin{bmatrix} 1 \\ x₁ \\ x₂ \\ \end{bmatrix}
  =
  \begin{bmatrix} 1 \\ x₁ - \floor{x₁} \\ x₂ - \floor{x₂} \\ \end{bmatrix}.
\]
Importantly, each of those matrices has
determinant $±1$.
Therefore, the whole operation can be easily reversed by inverting the matrix.
This is the equivalent of the inverse pivot operation in
the projective space $\mathbb{RP}^{d+1}$.
Using this notion, we can reformulate the MDCF as a series of matrix multiplications.
In the following $T(a)$ denotes the translation by a vector $a ∈ ℝ^d$
and the matrix $S(ℓ)$ denotes the swap of $x_ℓ$ with $x_0$.
The definition of the MDCFs can then be written as
\[
  [a^{(0)}] = \hat a^{(0)}, \qquad
  [a^{(0)}; a^{(1)}, …, a^{(n)}] = T(a₀) · S(ℓ) · [a^{(1)}; a^{(2)}, …, a^{(n)}],
\]
where $ℓ = \argmax_i [a^{(1)}; a^{(2)}, …, a^{(n)}]_i$ and $\hat a^{(0)} = [1, a_1^{(0)}, …, a_d^{(0)}]$.

We can also use the projective space to dramatically simplify Lemma~\ref{lem:mdcf-wallis}.
Instead of two different sequences $P_i^{(n)}$ and $Q_i^{(n)}$, we simplify it to a single matrix sequence $\{B^{(n)}\}_{n ≥ 0}$,
where one matrix $B^{(n)}$ consists of the column vectors $B₀^{(n)}, B₁^{(n)}, …, B_d^{(n)}$.
We begin with the identity matrix, $B^{(0)} = I_d$, and update the matrix according to
\begin{align*}
  B_{ℓₙ}^{(n)} = B^{(n-1)} \hat a^{(n)},
  \qquad B_i^{(n)} = B_i^{(n-1)},
  \qquad B_0^{(n)} = B_{ℓₙ}^{(n-1)},
\end{align*}
where $\hat a^{(n)} = (1, a_1^{(n)}, …, a_d^{(n)})$.
By construction, $B^{(n)}$ is the combined matrix of the original sequences
$P_i^{(n)}$ and $Q_i^{(n)}$ defined for Lemma~\vref{lem:mdcf-wallis}:
\[
  B^{(n)} = \begin{pmatrix}
    Q_0^{(n)} & Q_1^{(n)} & ⋯ & Q_d^{(n)} \\
    P_0^{(n)} & P_1^{(n)} & ⋯ & P_d^{(n)} \\
  \end{pmatrix}.
\]
In fact, we can prove an equivalent statement from this lemma using the new sequence.

\begin{lemma}
  \label{lem:mdcf-wallis'}
  Let $x ∈ ℝ^d$ and $\hat x = (1, x₁, …, x_d)$, then
  \[
    [a^{(0)}; ℓ₁, a^{(1)}; …; ℓ_{n-1}, a^{(n-1)}; ℓ, x] = [B^{(n-1)} \hat x]
  \]
\end{lemma}

% TODO
\begin{proof}
  By construction of $B^{(n-1)}$, we have
  \begin{align*}
    B^{(n-1)} \hat x
    & = B_0^{(n-1)} \hat x_0 + B_1^{(n-1)} \hat x_1 + ⋯ + B_d^{(n-1)} x_d \\
    & =
    \begin{pmatrix}
      P_0^{(n-1)} \\
      Q_0^{(n-1)} \\
    \end{pmatrix} \hat x_0
    + \begin{pmatrix}
      P_1^{(n-1)} \\
      Q_1^{(n-1)} \\
    \end{pmatrix} \hat x_1
    + ⋯ + \begin{pmatrix}
      P_d^{(n-1)} \\
      Q_d^{(n-1)} \\
    \end{pmatrix} \hat x_d \\
    & = \begin{pmatrix}
      P_0^{(n-1)} \hat x_0 + P_1^{(n-1)} \hat x_1 + ⋯ + P_d^{(n-1)} \hat x_d \\
      Q_0^{(n-1)} \hat x_0 + Q_1^{(n-1)} \hat x_1 + ⋯ + Q_d^{(n-1)} \hat x_d \\
    \end{pmatrix}.
  \end{align*}
  Projecting this vector back to $ℚ^d$ results exactly in the vector
  \[
    \frac{P_0^{(n-1)} + P_1^{(n-1)} x_1 + ⋯ + P_d^{(n-1)} x_d}{Q_0^{(n-1)} + Q_1^{(n-1)} x_1 + ⋯ + Q_d^{(n-1)} x_d}
  \]
  from
  Lemma~\ref{lem:mdcf-wallis}.
\end{proof}

% ==============================================================================
\section{Periodic Multi-Dimensional Continued Fractions}
% ==============================================================================

The proof for periodic MDCFs is based on the same theorem for the Jacobi-Perron
algorithm, originally proven by Perron \cite{Perron07}.
We begin with the purely periodic case.
The idea behind this proof is that in a purely periodic MDCF of a vector $x ∈ ℝ^d$,
the vector itself is an eigenvector for one of the matrices $B^{(k)}$.
Furthermore, the elements of this eigenvector can only be algebraic numbers with degree $≤ d+1$.

\begin{lemma}
  \label{lem:mdcf-purely-periodic}
  If there exists a purely periodic MDCF for $x ∈ ℝ^d$,
  then $xᵢ$ is an algebraic number of degree $≤ d+1$ for every $i ≤ d$.
\end{lemma}

% TODO: Should we use x ≡ y or [x] = [y]?
\begin{proof}
  If the MDCF is purely periodic, then there is some index $n ≥ 1$ such that $x = x^{(n)}$.
  Let $\hat x = [1, x₁, …, x_d]$ and $\hat x^{(n)} = [1, x_1^{(n)}, …, x_d^{(n)}]$.
  By Lemma~\ref{lem:mdcf-wallis'},
  \[
    [\hat x] = [B^{(n)} \hat x^{(n)}] = [B^{(n)} \hat x] \iff λ \hat x = B^{(n)} \hat x,
  \]
  for some nonzero $λ ∈ ℝ$.
  Therefore, we are looking for an eigenvector $\hat x$ and an eigenvalue $λ$ of $B^{(n)}$.
  The characteristic polynomial $\det(B^{(n)} - λ I)$ can have a degree of at most $d+1$,
  therefore the eigenvalue $λ$ is an algebraic number of degree $d+1$.
  For the eigenvector $\hat x$, we have to find a nontrivial solution to the
  homogeneous linear system
  \[
    (B^{(n)} - λ I) \hat x = 0.
  \]
  Each coefficient in this linear system is either an integer or $λ$ and is
  therefore contained in the field $ℚ(λ)$.
  Hence, the elements of $\hat x$ must come from the same
  field $ℚ(λ)$ and they must be algebraic numbers with the same degree.

  Finally, the eigenvector $\hat x$ has to be projected back from homogeneous coordinates $x$.
  Since $xᵢ = \hat xᵢ / \hat x₀$ and the values $\hat xᵢ$ and $\hat x₀$ are members of the field $ℚ(λ)$,
  the projected value $xᵢ$ must also be contained in the same field.
  Therefore, each element in $x$ is an algebraic number of degree $≤ d+1$.
\end{proof}

\begin{theorem}
  \label{thm:mdcf-periodic}
  If there exists a periodic MDCF for a given vector $x ∈ ℝ^d$,
  then each $x_i$ is a root of some polynomial with degree $≤ d+1$ for every $i ≤ d$.
\end{theorem}

\begin{proof}
  Given such a MDCF for $x$, let $x^{(k)}$ denote the $k$-th complete quotient
  of this fraction.
  Suppose that the MDCF is periodic after $K ≥ 0$ with period $ℓ ≥ 0$, i.e.
  $x^{(k)} = x^{(k+ℓ)}$ for every $k ≥ K$.
  By Lemma~\ref{lem:mdcf-wallis'},
  \[
    [\hat x] = [B^{(k)} \hat x^{(k)}],
  \]
  which means that very element in $x$ can be represented as a linear rational
  expression of $x^{(k)}$:
  \[
    x_i = \frac{∑_{j=1}^d B_{ij}^{(k)} x_j^{(k)} + B_{i0}^{(k)}}{\sum_{j=1}^d B_{0j}^{(k)} x_j^{(k)} + B_{00}^{(k)}},
  \]
  % TODO: I think we should be more precise here since the elements could be
  % inside different fields. They are not, but this sentence does not indicate
  % that.
  From the previous lemma, it follows that the elements of $\hat x^{(k)}$
  are contained in some field $ℚ(λ)$, where $λ$ is an algebraic number of degree $≤ d+1$.
  Furthermore, $B^{(k)}$ consists solely of integers.
  Therefore, every element in $x$ is contained in the same field $ℚ(λ)$
  and they must also be an algebraic numbers of degree $≤ d+1$.
\end{proof}

% TODO: There should be a discussion about the potential for answering
% Hermite's question here...

\chapter{Experimental Analysis of the Periodicity}

\section{Finding a Periodic Representation through Brute Force}

% TODO: Maybe replace this with the pseudocode instead?
\begin{minipage}{0.48\textwidth}
\begin{Python}[basicstyle=\small\ttfamily, frame={}]
def brute_force_search(xs, max_depth):
    d = len(xs)
    for L in sequences(d, max_depth):
        ys = tuple(xs)
        index = {ys: 0}
        for i, l in enumerate(L):
            ys = pivot(ys, l)
            if ys in index:
                j = index[ys]
                start = L[:j]
                period = L[j:i+1]
                return start, period
            else:
                index[ys] = i + 1
\end{Python}
\end{minipage}
\vrule width 1pt
\hfill
\begin{minipage}{0.48\textwidth}
\begin{Python}[basicstyle=\small\ttfamily, frame={}]
def sequences(base, max_digits):
    seq = []
    while True:
        carry = 1
        for i in range(len(seq) - 1, -1, -1):
            seq[i] += carry
            if seq[i] == base:
                seq[i] = 0
                carry = 1
            else:
                carry = 0
                break
        if carry:
            if len(seq) < max_digits:
                seq.insert(0, 1)
            else:
                break
        yield seq[1:]
\end{Python}
\end{minipage}

\section{Results for Cubic Irrationals}

\begin{example}
  The sequences for the prime numbers are:
  \begin{itemize}
    \item $2^{1/3}$: $0\overline{10}$.
    \item $3^{1/3}$: $01\overline{01}$.
    \item $5^{1/3}$: $0\overline{00111000}$.
    \item $7^{1/3}$: $0\overline{010100}$.
    \item $11^{1/3}$: $0\overline{1100}$.
    \item $13^{1/3}$: $00\overline{010000}$.
    \item $17^{1/3}$: $000\overline{11110000}$.
    \item Another choice for $5^{1/3}$ with shorter period: $00110\overline{101010}$.
  \end{itemize}
\end{example}

\begin{remark}
  The first number that has a leading $1$ as the pivot is $12$,
  which has the pivot sequence $1\overline{0111011101}$.
  However, there are other possible choices where it does not have a leading $0$.
\end{remark}

\begin{table}[t]
  \caption{Representation of $ψ = \sqrt[3]{4}$ using the brute-force search.}
  \label{table:cube-root-4}
  \centering
  \begin{tabular}{lllllll}
  \uzlhline
  \uzlemph{$\ell$} & \uzlemph{$x_1$} & \uzlemph{$x_2$} & \uzlemph{$x_1$} & \uzlemph{$x_2$} & \uzlemph{$a_1$} & \uzlemph{$a_2$} \\
	\hline
  $0$ & $\psi$ & $\psi^{2}$ & $1.5874$ & $2.51984$ & $1$ & $2$ \\
  $0$ & $\frac{1}{3} \psi^{2} + \frac{1}{3} \psi + \frac{1}{3}$ & $-\frac{1}{3} \psi^{2} + \frac{2}{3} \psi + \frac{2}{3}$ & $1.70241$ & $0.88499$ & $1$ & $0$ \\
  $1$ & $\frac{1}{4} \psi^{2} + \frac{1}{2} \psi$ & $\frac{1}{2} \psi^{2}$ & $1.42366$ & $1.25992$ & $1$ & $1$ \\
  $0$ & $\frac{1}{4} \psi^{2} + 1$ & $\frac{1}{2} \psi^{2} + \psi + 1$ & $1.62996$ & $3.84732$ & $1$ & $3$ \\
  $0$ & $\psi$ & $\psi^{2} - 2 \psi + 2$ & $1.5874$ & $1.34504$ & $1$ & $1$ \\
  $1$ & $\frac{1}{3} \psi^{2} + \frac{1}{3} \psi + \frac{1}{3}$ & $\psi - 1$ & $1.70241$ & $0.5874$ & $1$ & $0$ \\
  $1$ & $\frac{1}{3} \psi + \frac{2}{3}$ & $\frac{1}{3} \psi^{2} + \frac{1}{3} \psi + \frac{1}{3}$ & $1.1958$ & $1.70241$ & $1$ & $1$ \\
  $0$ & $\frac{1}{12} \psi^{2} - \frac{1}{6} \psi + \frac{1}{3}$ & $\frac{1}{4} \psi^{2} + \frac{1}{2} \psi$ & $0.27875$ & $1.42366$ & $0$ & $1$ \\
  $0$ & $\psi + 2$ & $\psi^{2} - 1$ & $3.5874$ & $1.51984$ & $3$ & $1$ \\
  $0$ & $\frac{1}{3} \psi^{2} + \frac{1}{3} \psi + \frac{1}{3}$ & $-\frac{1}{3} \psi^{2} + \frac{2}{3} \psi + \frac{2}{3}$ & $1.70241$ & $0.88499$ & $1$ & $0$ \\
  \uzlhline
\end{tabular}

\end{table}

\begin{table}[t]
  \caption{Period Length of the first $28$ numbers.}
  \centering
  \begin{tabular}{ll}
  \uzlhline
  \uzlemph{$n^3$} & \uzlemph{Period Length of $n^3$} \\
  \hline
  2 & 2 \\
  3 & 2 \\
  4 & 8 \\
  5 & 8 \\
  6 & 8 \\
  7 & 6 \\
  9 & 2 \\
  10 & 4 \\
  11 & 4 \\
  12 & 10 \\
  13 & 6 \\
  14 & 4 \\
  15 & 6 \\
  16 & 14 \\
  17 & 8 \\
  18 & 6 \\
  19 & 6 \\
  20 & 8 \\
  21 & 8 \\
  22 & 6 \\
  23 & 20 \\
  24 & 8 \\
  25 & 20 \\
  26 & 4 \\
  28 & 2 \\
  \uzlhline
\end{tabular}

\end{table}

\section{Results for Quartic and Higher-Degree Irrationals}

\section{Evaluation of More Efficient Search Strategies}

We can likely rule out alternating pivot sequences since this corresponds to
the Jacobi-Perron algorithm and for this algorithm it is conjectured
\cite{Karpenkov21} that the algorithm is not periodic for $\sqrt[3]{4}$.
However, the brute-force algorithm is periodic for this input (see Table~\ref{table:cube-root-4}).

The choice of the initial input also matters.
We generally choose $x = (α, q(α))$, where $q$ is some polynomial of degree $2$.
But different choices of $q$ produce different sequences.
For example, $(\sqrt[3]{2}, \sqrt[3]{4})$ produces a different sequence of coefficients than $(\sqrt[3]{2}, \sqrt[3]{6})$.
Even though both inputs represent the same number $\sqrt[3]{2}$.
Choosing $q(α) = α^2 - α$ makes the sequence purely periodic for $\sqrt[3]{3}$ and $\sqrt[3]{4}$.

\chapter{Conclusion and Outlook}
\label{ch:conclusion}

The main objective of this thesis
was to investigate Hermite's question
on a new representation of real numbers,
which is periodic if and only if the number is algebraic.
The basis for this representation has been a generalization of Euclidean
algorithm by \citeauthor{Klein24}.
Specifically, I have used a small subroutine from the algorithm,
which was used to update the old solution vector instead of solving a linear
system in each iteration.
This subroutine was then used as the basis of the multidimensional continued fractions.
Although these were not enough to fully solve Hermite's question,
we have seen that many concepts from the Euclidean algorithm naturally carry
over to its generalization.

Beginning with the Fibonacci numbers,
we saw that different strategies lead to different definitions of Fibonacci numbers.
The first example of such numbers was for the minimum strategy,
which chooses the smallest fractional value.
For this strategy, the $d$-bonacci numbers represent the worst case,
just like the Fibonacci numbers represent the worst case for the classical
Euclidean algorithm.
The second strategy improves upon this strategy by minimizing the ratio between two values over two iterations.
For this strategy, its golden ratio and the corresponding Fibonacci numbers
already came up when bounding the decrease of the determinant over just two iterations.
Although the golden ratio is not periodic under the strategy itself,
we have seen that the maximum strategy leads to periodicity.
These two roots were the first examples of periodic inputs for the generalized Euclidean algorithm.

These examples led to the development of multidimensional continued fractions
as presented in Chapter~\ref{ch:mdcf}.
They are essentially the generalized Euclidean algorithm, but in reverse.
Since the algorithm allows us to choose a pivot index at each iteration,
there does not exist a unique multidimensional continued fraction for every vector.
However, this choice provides more flexibility.
Instead of focusing on a single algorithm, the multidimensional continued
fractions unify many Jacobi-Perron-type algorithms under one framework.
Additionally, they share many properties with continued fractions.
For example, the linear recurrence for continued fractions proven in Lemma~\ref{lem:cf-wallis}
has an equivalent sequence proven in Lemma~\ref{lem:mdcf-wallis} for
multidimensional continued fractions.

The convergence of multidimensional continued fractions was proven in Section~\ref{sec:mcf-convergence}.
The proof is considerably more complex than for continued fractions.
However, it still shows that the multidimensional continued fractions are
converging under any algorithm, which chooses all possible pivot indices and
chooses these indices not too far apart.
This covers algorithms like the classical Jacobi-Perron algorithm,
but also other algorithms, which choose the pivot indices based on a fixed list.
Most importantly, it establishes the first part of Hermite's question.
It shows that many algorithms lead to a representation of the real numbers
using the multidimensional continued fractions.

While the first part of Hermite's question has been solved,
the second part is only partially solved.
Whereas the first part only asks of a representation of the real numbers,
the second part states that the repersentation should be periodic if and only
if the number is a cubic irrational (or algebraic number, in general).
The first direction has been proven in Section~\ref{sec:mcf-periodic}.
The proof is based on eigenvectors and shows that if the MCF of an irrational vector $x$ is
purely periodic, then $x$ is one of the eigenvectors in the convergent matrix.
% TODO: We have to define the convergent matrix

Theorem~\ref{thm:unimodular-algebraic} suggests that we can always find a
unimodular matrix with $x$ as one of its eigenvectors,
thereby indicating that we should be able to find a periodic MCF.
However, I was not able to show this in the scope of this thesis.
Nevertheless, the analysis in Chapter~\ref{ch:implementation}
strongly supports the conjecture.
In particular, the algorithm by \citeauthor{Tamura09} has shown promise for
finding periodic representation of cubic and quartic irrationals.
The authors have already proven that their algorithm is periodic for certain
classes of cubic polynomials.
Future research into this algorithm could possibly lead to a periodic sequence
for all cubic and quartic irrationals.

In the case of continued fractions,
we were able to show that they are periodic only if the number is a quadratic irrational.
The proof uses Klein polygons, which are preserved under a specific unimodular transformation
and we can find such a transformation for every quadratic irrational.
We can find a similar transformation for all algebraic numbers, but for the
multidimensional continued fractions we are missing the connection between them
and Klein polytopes.
Section~\ref{sec:mdcf-geometry} already provides the necessary foundation for this.
Furthermore, there already exists a generalization of Lagrange's theorem by
\citeauthor{German08}~\cite{German08}.

% TODO: Add note about how approximation doesn't work well with the multi-dimensional continued fractions


\end{document}
