\documentclass[english,version-2020-11]{uzl-thesis}


\UzLThesisSetup{
  Masterarbeit,
  Logo-Dateiname        = {uzl-thesis-logo-itcs.pdf},
  Verfasst              = {am}{Institut für Theoretische Informatik},
  Titel auf Deutsch     = {Über elementare Eigenschaften einer mehrdimensionalen Verallgemeinerung des euklidischen Algorithmus},
  Titel auf Englisch    = {On Elementary Properties of a Multi-Dimensional Generalization of the Euclidean Algorithm},
  Autor                 = {Daniel Knaack},
  Betreuerin            = {Prof. Dr. Kim-Manuel Klein},
  Studiengang           = {Informatik},
  Datum                 = {18. Juni 2024},
  Abstract              = {TODO},
  Zusammenfassung       = {TODO},
  Numerische Bibliographie,
}

\UzLStyle{alegrya modern design}

\begin{document}

\chapter{Preliminaries}

\section{Generalized Euclidean Algorithm}

\begin{Pseudocode}
solve $Bx = c$
swap $B_l$ and $c$
...
\end{Pseudocode}

\chapter{Bounds}

Assumption:
\[
  x_1 = \frac{x_2}{x_1} - 1 = \dots = \frac{x_{d}}{x_{d-1}} - 1 = \frac{1}{x_d} - 1,
\]
which leads to the following polynomials:
\begin{align}
  \label{eq:solution}
  x_d^{d+1} + x_d - 1 = 0 \quad \text{ and } \quad x_i = x_d^{d - i + 1}.
\end{align}

In one dimension, the root of the polynomial $\phi_1$ is the \emph{inverse of the golden ratio}.
In two dimension, the root of the polynomial $\phi_2$ is the \emph{inverse of the supergolden ratio}.
Hence, we have to substitute $x$ with the inverse of $y$ to get the actual golden ratio:

\begin{align*}
  (1/y)^{d+1} + 1/y - 1       & = 0 \quad \text{(Expand second term with $y^d$)} \\
  1/y^{d+1} + y^d/y^{d+1} - 1 & = 0 \quad \text{(Multiply with $y^d$)}           \\
  1 + y^d - y^{d+1}           & = 0 \quad \text{(Multiply with $-1$)}            \\
  y^{d+1} - y^d - 1           & = 0                                              \\
\end{align*}

Let $\phi_d$ be the root of this polynomial.
For $d = 1$, the root $\phi_1$ is the \emph{golden ratio}.
For $d = 2$, the root $\phi_2$ is the \emph{supergolden ratio}.

The golden ratio can be approximated using the Fibonacci sequence,
the super golden ratio can be approximated using Narayana's cow sequence:

\begin{align*}
  F(0) = F(1) = 1,        & \quad F(n) = F(n - 1) + F(n - 2), n \ge 2 \\
  N(0) = N(1) = N(2) = 1, & \quad N(n) = N(n - 1) + N(n - 3), n \ge 3
\end{align*}

From the Narayana sequence, we can naturally generalize the sequence to higher dimensions:

\[
  F_d(0) = 0, F_d(1) = \dots = F_d(d) = 1, \quad F_k(n) = F_d(n - 1) + F_d(n - d - 1).
\]

\begin{table}[t]
  \caption{The first 18 Fibonacci numbers of orders one to five.}
  \begin{tabular}{c|cccccccccccccccccccc}
    $n$     & 0 & 1 & 2 & 3 & 4 & 5 & 6 & 7  & 8  & 9  & 10 & 11 & 12  & 13  & 14  & 15  & 16  & 17   \\ %& 18   & 19   \\
    \hline
    $d = 1$ & 0 & 1 & 1 & 2 & 3 & 5 & 8 & 13 & 21 & 34 & 55 & 89 & 144 & 233 & 377 & 610 & 987 & 1597 \\ %& 2584 & 4181 \\
    $d = 2$ & 0 & 1 & 1 & 1 & 2 & 3 & 4 & 6  & 9  & 13 & 19 & 28 & 41  & 60  & 88  & 129 & 189 & 277  \\ %& 406  & 595  \\
    $d = 3$ & 0 & 1 & 1 & 1 & 1 & 2 & 3 & 4  & 5  & 7  & 10 & 14 & 19  & 26  & 36  & 50  & 69  & 95   \\ %& 131  & 181  \\
    $d = 4$ & 0 & 1 & 1 & 1 & 1 & 1 & 2 & 3  & 4  & 5  & 6  & 8  & 11  & 15  & 20  & 26  & 34  & 45   \\ %& 60   & 80   \\
    $d = 5$ & 0 & 1 & 1 & 1 & 1 & 1 & 1 & 2  & 3  & 4  & 5  & 6  & 7   & 9   & 12  & 16  & 21  & 27   \\ %& 34   & 43   \\
  \end{tabular}
\end{table}

Any two consecutive terms can be used to approximate the corresponding ratio, i.e. $\lim_{n \to \infty} F_d(n) / F_d(n + 1) = \phi_d$.

Going back to the Euclidean algorithm,
an example matrix which leads to the solution given by Equation~\ref{eq:solution} is:
\begin{align*}
  \left(\begin{array}{ccccc|c}
    a^d    & 0       & \cdots & 0      & 0      & b^d     \\
    0      & a^{d-1} & \cdots & 0      & 0      & b^{d-1} \\
    \vdots & \vdots  & \ddots & \vdots & \vdots & \vdots  \\
    0      & 0       & \cdots & a^2    & 0      & b^2     \\
    0      & 0       & \cdots & 0      & a      & b       \\
  \end{array}\right),
\end{align*}
where $a = F_d(n)$ and $b = F_d(n + 1)$.

\begin{definition}
  The $d$-th golden ratio $\phi^{(d)}$ is the real positive root of $x^{d+1} - x^d - 1$.
\end{definition}

\begin{definition}
  The $d$-Fibonacci numbers are defined as follows:
  \begin{itemize}
    \item $F^{(d)}_0 = F^{(d)}_1 = \dots = F^{(d)}_{d-1} = 1$,
    \item $F^{(d)}_{n} = F^{(d)}_{n-1} + F^{(d)}_{n-d-1}$ for $n \ge d$.
  \end{itemize}
\end{definition}

\begin{lemma}
  The sequence $R^{(d)}_{n+d}$ can be reformulated as follows:
  \begin{align*}
    R^{(d)}_{n+d}
    & = 1 + \frac{1}{R^{(d)}_{n+d-1} R^{(d)}_{n+d-2} \dots R^{(d)}_n}
  \end{align*}
\end{lemma}

\begin{proof}
  \begin{align*}
    R^{(d)}_{n+d}
    & = \frac{F^{(d)}_{n+d+1}}{F^{(d)}_{n+d}}
    = \frac{F^{(d)}_{n+d} + F^{(d)}_{n}}{F^{(d)}_{n+d}}
    = 1 + \frac{F^{(d)}_{n}}{F^{(d)}_{n+d}} \\
    & = 1 + \frac{F^{(d)}_{n}}{R^{(d)}_{n+d-1} F^{(d)}_{n+d-1}}
    = 1 + \frac{F^{(d)}_{n}}{R^{(d)}_{n+d-1} R^{(d)}_{n+d-2} F^{(d)}_{n+d-2}} \\
    & \mathrel{\vdots} \\
    & = 1 + \frac{F^{(d)}_{n}}{R^{(d)}_{n+d-1} R^{(d)}_{n+d-2} \dots R^{(d)}_n F^{(d)}_n} \\
    & = 1 + \frac{1}{R^{(d)}_{n+d-1} R^{(d)}_{n+d-2} \dots R^{(d)}_n}
    \qedhere
  \end{align*}
\end{proof}

\begin{lemma}
  The sequence $R^{(d)}_n = F^{(d)}_{n+1} / F^{(d)}_{n}$ is bounded between $1$ and $2$.
\end{lemma}

\begin{proof}
  For all $n < d - 1$ this is obviously the case.
  Per induction, assume that $R^{(d)}_n, R^{(d)}_{n+1}, \dots, R^{(d)}_{n+d-1}$ are bounded.
  From the previous lemma, we know
  \begin{align*}
    R^{(d)}_{n+d}
    & = 1 + \frac{1}{R^{(d)}_{n+d-1} R^{(d)}_{n+d-2} \dots R^{(d)}_n}
  \end{align*}
  By the induction hypothesis, each ratio is bounded between $1$ and $2$.
  Therefore, $R^{(d)}_{n+d}$ is equally bounded between $1$ and $2$.
\end{proof}

\begin{lemma}
  The sequence $R^{(d)}_n$ is monotonically increasing.
\end{lemma}

\begin{proof}

\end{proof}

\begin{theorem}
  The sequence $R^{(d)}_n$ converges to $\phi^{(d)}$.
\end{theorem}

\begin{proof}
  From the previous two lemmas, we know that the sequence converges.
  Therefore, there exists some $L$ such that
  \[
    \lim_{n \to \infty} R^{(d)}_n = \lim_{n \to \infty} 1 + \frac{1}{R^{(d)}_{n-1} R^{(d)}_{n-2} \dots R^{(d)}_{n-d}} = L.
  \]
  As $n$ approaches infinity, all ratios approach some limit $L$.
  Hence, $L = 1 + 1/L^d$ or equivalently $L^{d+1} - L^d - 1 = 0$,
  which fits the definition of the golden ratio $\phi^{(d)}$.
\end{proof}

\chapter{Implementation}

\section{Using an Out-of-the-Box Solver}

\begin{itemize}
  \item Algorithm technically only requires one fractional bit to determine how to round.
  \item Problem: Solving a system of linear equations requires exact fractional solutions.
    Using floating-point numbers leads to inaccurate results.
  \item Could be solved by using an iterative method? However, does a method
    exist, which is accurate up to one fractional bit?
\end{itemize}

\section{Custom Implementation}

\begin{itemize}
  \item Uses custom \texttt{ratio} datatype which implements rational numbers.
  \item Linear system solver implemented using Gaussian elimination and pivoting.
  \item Only the basic algorithm is implemented, so far.
\end{itemize}

\begin{bibtex-entries}
\end{bibtex-entries}

\end{document}
